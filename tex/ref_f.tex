%       Copyright (c) 2001-2002 by INRIA and SDTools, All Rights Reserved.
%       Use under OpenFEM trademark.html license and LGPL.txt library license
%       $Revision: 1.13 $  $Date: 2010/04/02 16:17:57 $

%------------------------------------------------------------------------------
%------------------------------------------------------------------------------
%------------------------------------------------------------------------------
\rtop{fecom}{fecom}

\noindent UI command function for the visualization of 3-D deformation plots.

\rsyntax\begin{verbatim}
fecom
fecom CommandString
fecom('CommandString',AdditionalArgument)
\end{verbatim}\nlvs

\rmain{Description}

The non current {\sl SDT 3} version of this function is included in OpenFEM. Use the {\tt help} \ts{fecom} command to get help.


\rmain{See also} % - - - - - - - - - - - - - - - - - -

\noindent \feplot, \femesh

%----------------------------------------------------------------------------
%       Copyright (c) 2001-2014 by INRIA and SDTools, All Rights Reserved.
%       Use under OpenFEM trademark.html license and LGPL.txt library license
%       $Revision: 1.100 $  $Date: 2019/02/15 17:39:27 $

%-----------------------------------------------------------------------------
\rtop{femesh}{femesh}

Finite element mesh handling utilities.\index{global variable}\index{global variable}

\rsyntax\begin{verbatim}
femesh CommandString
femesh('CommandString')
[out,out1] = femesh('CommandString',in1,in2)
\end{verbatim}

\rmain{Description}

You should use \feutil\ function that provides equivalent commands to {\tt femesh} but using model data structure.

{\tt femesh} provides a number of tools for mesh creation and manipulation. {\tt femesh} uses global variables to define the proper object of which to apply a command. {\tt femesh} uses the following {\sl standard global variables} which are declared as global in your workspace when you call \femesh\index{FEnode}\index{FEelt}

\lvs\noindent\begin{tabular}{@{}p{.15\textwidth}@{}p{.85\textwidth}@{}}
\rz{\tt FEnode} & main set of nodes \\
\rz{\tt FEn0}   & selected set of nodes \\
\rz{\tt FEn1}   & alternate set of nodes \\
\rz{\tt FEelt}  & main finite element model description matrix \\
\rz{\tt FEel0}  & selected finite element model description matrix \\
\rz{\tt FEel1}  & alternate finite element model description matrix \\
\end{tabular}


By default, {\tt femesh} automatically uses base workspace definitions of the standard global variables (even if they are not declared as global). When using the standard global variables within functions, you should always declare them as global at the beginning of your function. If you don't declare them as global modifications that you perform will not be taken into account, unless you call {\tt femesh} from your function which will declare the variables as global there too. The only thing that you should avoid is to use {\tt clear} (instead of {\tt clear global}) within a function and then reinitialize the variable to something non-zero. In such cases the global variable is used and a warning is passed.

\noindent Available {\tt femesh} commands are

%  - - - - - - - - - - - - - - - - - - - - - - - - - - - - - - - - - - -
\ruic{femesh}{;}{}

\noindent {\sl Command chaining.} Commands with no input (other than the command) or output argument, can be chained using a call of the form {\tt femesh(';Com1;Com2')}. \commode\ is then used for command parsing.

%  - - - - - - - - - - - - - - - - - - - - - - - - - - - - - - - - - - -
\ruic{femesh}{Add}{ FEel{\ti i} FEel{\ti j}, AddSel}

\noindent {\sl Combine two FE model description matrices.} The characters \tsi{i} and \tsi{j} can specify any of the main \ts{t}, selected \ts{0} and alternate \ts{1} finite element model description matrices.  The elements in the model matrix {\tt FEel}{\ti j} are appended to those of {\tt FEel}{\ti i}.

\noindent\ts{AddSel} is equivalent to \ts{AddFEeltFEel0} which adds the selection {\tt FEel0} to the main model {\tt FEelt}.

This is an example of the creation of {\tt FEelt} using 2 selections ({\tt FEel0} and {\tt FEel1}) 

%begindoc
\begin{verbatim}
femesh('Reset');
femesh('Testquad4');                    % one quad4 created
femesh('Divide',[0 .1 .2 1],[0 .3 1]);  % divisions
FEel0=FEel0(1:end-1,:);                 % suppress 1 element in FEel0
femesh('AddSel');                       % add FEel0 into FEelt
FEel1=[Inf abs('tria3');9 10 12  1 1 0];% create FEel1 
femesh('Add FEelt FEel1');              % add FEel1 into FEelt
femesh PlotElt                          % plot FEelt
\end{verbatim}%enddoc


%  - - - - - - - - - - - - - - - - - - - - - - - - - - - - - - - - - - -
\ruic{femesh}{AddNode}{ [,New] [, From i] [,epsl {\ti val}]}

\noindent {\sl Combine, append }(without/with new) {\tt FEn0} to {\tt FEnode}.  Additional uses of {\tt AddNode} are provided using the format

{\tt [AllNode,ind]=femesh('AddNode',OldNode,NewNode);}

{\sl which combines} {\tt NewNode} {\sl to} {\tt OldNode}.  
\ts{AddNode} finds nodes in {\tt NewNode} that coincide with nodes in {\tt OldNode} and appends other nodes to form {\tt AllNode}.  {\tt ind} gives the indices of the {\tt NewNode} nodes in the {\tt AllNode} matrix.

{\tt NewNode} can be specified as a matrix with three columns giving {\tt xyz} coordinates. The minimal distance below which two nodes are considered identical is given by \sdtdef\ {\tt epsl} (default {\tt 1e-6}).

{\tt [AllNode,ind]=femesh('AddNode From 10000',OldNode,NewNode);} gives node numbers starting at 10000 for nodes in {\tt NewNode} that are not in {\tt OldNode}.

\begin{SDT}
SDT uses an optimized algorithm available in {\tt feutilb}. See \ltr{feutil}{AddNode} for more details.
\end{SDT}

%  - - - - - - - - - - - - - - - - - - - - - - - - - - - - - - - - - - -
\ruic{femesh}{AddTest}{ [,-EGID {\ti i}][,{\ti NodeShift},Merge,Combine]}\index{wire-frame plots}

{\sl Combine test and analysis models}. When combining test and analysis models you typically want to overlay a detailed finite element mesh with a coarse wire-frame representation of the test configuration. These models coming from different origins you will want combine the two models in {\tt FEelt}.

By default the node sets are considered to be disjoint. New nodes are added starting from \texline {\tt max(FEnode(:,1))+1} or from {\tt \tsi{NodeShift}+1} if the argument is specified. Thus {\tt femesh('addtest {\tsi NodeShift}',TNode,TElt)} adds test nodes {\tt TNode} to {\tt FEnode} while adding {\tt NodeShift} to their initial identification number. The same {\tt NodeShift} is added to node numbers in {\tt TElt} which is appended to {\tt FEelt}. {\tt TElt} can be a wire frame matrix read with \ufread.

With \ts{merge} it is assumed that some nodes are common but their numbering is not coherent. {\tt femesh('addtest merge',NewNode,NewElt)} can also be used to merge to FEM models. Non coincident nodes (as defined by the \ts{AddNode} command) are added to {\tt FEnode} and {\tt NewElt} is renumbered according to the new {\tt FEnode}. \ts{Merge-Edge} is used to force mid-side nodes to be common if the end nodes are.

With \ts{combine} it is assumed that some nodes are common and their numbering is coherent. Nodes with new {\tt NodeId} values are added to {\tt FEnode} while common {\tt NodeId} values are assumed to be located at the same positions.

You can specify an {\tt EGID} value for the elements that are added using \ts{AddTest -EGID -1}. In particular negative {\tt EGID} values are display groups so that they will be ignored in model assembly operations.

The combined models can then be used to create the test/analysis correlation using \fesens. An application is given in the {\tt gartte} demo, where a procedure to match initially different test and FE coordinate frames is outlined.

%  - - - - - - - - - - - - - - - - - - - - - - - - - - - - - - - - - - -
\ruic{femesh}{Divide}{ {\ti div1 div2 div3}}

\noindent {\sl Mesh refinement by division of elements.} \ts{Divide} applies to all groups in {\tt FEel0}.

See equivalent \ltr{feutil}{Divide} command.

%begindoc
\begin{verbatim}
% Example 1 : beam1
femesh('Reset');
femesh(';Testbeam1;Divide 3;PlotEl0'); % divide by 3
fecom TextNode

% Example 2 : you may create a command string
number=3;
st=sprintf(';Testbeam1;Divide %f;PlotEl0',number);
femesh('Reset');
femesh(st);
fecom TextNode

% Example 3 : you may use uneven division
femesh('Reset');femesh('testquad4'); % one quad4 created
femesh('DivideElt',[0 .1 .2 1],[0 .3 1]); 
femesh PlotEl0

\end{verbatim}%enddoc

%  - - - - - - - - - - - - - - - - - - - - - - - - - - - - - - - - - - -
\ruic{femesh}{DivideInGroups}{}

Finds groups of {\tt FEel0} elements that are not connected (no common node) and places each of these groups in a single element group. 

%begindoc
\begin{verbatim}
femesh('Reset');femesh('testquad4'); % one quad4 created
femesh('RepeatSel 2 0 0 1'); % 2 quad4 in the same group
femesh('DivideInGroups');    % 2 quad4 in 2 groups
\end{verbatim}%enddoc

%  - - - - - - - - - - - - - - - - - - - - - - - - - - - - - - - - - - -
\ruic{femesh}{DivideGroup}{ {\ti i ElementSelectors}}

Divides a single group \tsi{i} of {\tt FEelt} in two element groups. The first new element group is defined based on the element selectors (see \ser{findelt}).

%  - - - - - - - - - - - - - - - - - - - - - - - - - - - - - - - - - - -
\ruic{femesh}{Extrude}{ {\ti nRep tx ty tz}}

\noindent {\sl Extrusion}.  Nodes, lines or surfaces that are currently selected (put in {\tt FEel0}) are extruded \tsi{nRep} times with global translations \tsi{tx ty tz}.  

You can create irregular extrusion giving a second argument (positions of the sections for an axis such that {\tt tx ty tz} is the unit vector).

See \ltr{feutil}{Extrude} for more details.

%begindoc
\begin{verbatim}
% Example 1 : beam
femesh('Reset');
femesh('Testbeam1'); % one beam1 created
femesh(';Extrude 2 1 0 0;PlotEl0'); % 2 extrusions in x direction

% Example 2 : you may create the command string
number=2;step=[1 0 0];
st=sprintf(';Testbeam1;Extrude %f %f %f %f',[number step]);
femesh('Reset');
femesh(st);  femesh PlotEl0

% Example 3 : you may use uneven extrusions in z direction
femesh('Reset'); femesh('Testquad4')
femesh('Extrude 0 0 0 1', [0 .1 .2 .5 1]); % 
% 0 0 0 1        :  1 extrusion in z direction
% [0 .1 .2 .5 1] :  where extrusions are made
femesh PlotEl0
\end{verbatim}%enddoc


%  - - - - - - - - - - - - - - - - - - - - - - - - - - - - - - - - - - -
\ruic{femesh}{FindElt}{ {\ti ElementSelectors}}
\index{element!selection}

\noindent {\sl Find elements} based on a number of selectors described in \ser{findelt}. The calling format is 

{\tt [ind,elt] = femesh('FindElt withnode 1:10')} 

where {\tt ind} gives the row numbers of the elements (but not the header rows except for unique superelements which are only associated to a header row) and {\tt elt} (optional) the associated element description matrix. \ts{FindEl0} applies to elements in {\tt FEel0}.

When operators are accepted, equality and inequality operators can be used. Thus {\tt group\verb+~=+[3 7]} or {\tt pro < 5} are acceptable commands. See also \lts{femesh}{Sel}\ts{Elt}, \lts{femesh}{Remove}\ts{Elt} and \lts{femesh}{DivideGroup}, the {\tt gartfe} demo, \fecom\ selections.


%  - - - - - - - - - - - - - - - - - - - - - - - - - - - - - - - - - - -
\ruic{femesh}{FindNode}{ \tsi{Selectors}}
\index{node!selection}

\noindent {\sl Find node numbers} based on a number of selectors listed in \ser{findnode}. 

Different selectors can be chained using the logical operations \ts{\&} (finds nodes that verify both conditions), \ts{|} (finds nodes that verify one or both conditions). Condition combinations are always evaluated from left to right (parentheses are not accepted).

\noindent Output arguments are the numbers {\tt NodeID} of the selected nodes and the selected nodes {\tt node} as a second optional output argument. 

\noindent As an example you can show node numbers on the right half of the {\tt z==0} plane using the commands

\noindent{\tt fecom('TextNode',femesh('FindNode z==0 \& x>0'))}

Following example puts markers on selected nodes
%begindoc
\begin{verbatim}
model=demosdt('demo ubeam'); femesh(model); % load U-Beam model
fecom('ShowNodeMark',femesh('FindNode z>1.25'),'color','r')
fecom('ShowNodeMark',femesh('FindNode x>0.2*z|x<-0.2*z'),...
      'color','g','marker','o')
\end{verbatim}%enddoc


Note that you can give numeric arguments to the command as additional \femesh\ arguments. Thus the command above could also have been written 

{\tt fecom('TextNode',femesh('FindNode z== \& x>=',0,0)))}

 See also the {\tt gartfe} demo.


%  - - - - - - - - - - - - - - - - - - - - - - - - - - - - - - - - - - -
\ruic{femesh}{Info}{ [ ,FEel\tsi{i}, Node\tsi{i}]}

\noindent {\sl Information on global variables}.  \ts{Info} by itself gives information on all variables. The additional arguments {\tt FEelt} ...  can be used to specify any of the main {\tt t}, selected {\tt 0} and alternate {\tt 1} finite element model description matrices. \ts{InfoNode}\tsi{i} gives information about all elements that are connected to node \tsi{i}. To get information in {\tt FEelt} and in {\tt FEnode}, you may write

{\tt femesh('InfoElt')} or {\tt femesh('InfoNode')}
 
%  - - - - - - - - - - - - - - - - - - - - - - - - - - - - - - - - - - -
\ruic{femesh}{Join}{ [,el0] [group \tsi{i}, \tsi{EName}]}

\noindent {\sl Join the} {\sl groups} \tsi{i} or all the groups of type \tsi{EName}. \ts{JoinAll} joins all the groups that have the same element name. By default this operation is applied to {\tt FEelt} but you can apply it to {\tt FEel0} by adding the \ts{el0} option to the command. Note that with the selection by group number, you can only join groups of the same type (with the same element name).

%begindoc
\begin{verbatim}
femesh('Reset'); femesh(';Test2bay;PlotElt');
% Join using group ID
femesh('InfoElt');   % 2 groups at this step
femesh JoinGroup1:2  % 1 group now
% Join using element name
femesh('Reset'); femesh('Test2bay;PlotElt');
femesh Joinbeam1     % 1 group now
\end{verbatim}%enddoc

\ruic{femesh}{Model}{ [,0]} % - - - - - - - - - - - - - - - - - - - 

{\tt model=femesh('Model')} returns the FEM structure (see~\ser{model}) with fields {\tt model.Node=FEnode} and {\tt model.Elt=FEelt} as well as other fields that may be stored in the {\tt FE} variable that is persistent in \femesh. {\tt model=femesh('Model0')} uses {\tt model.Elt=FEel0}.

%  - - - - - - - - - - - - - - - - - - - - - - - - - - - - - - - - - - -
\ruic{femesh}{ObjectBeamLine}{ {\ti i}, ObjectMass {\ti i}}

\noindent {\sl Create a group of }\beam\ {\sl elements}.  The node numbers \tsi{i} define a series of nodes that form a continuous beam (for discontinuities use {\tt 0}), that is placed in {\tt FEel0} as a single group of \beam\ elements.

For example \femesh{\tt ('ObjectBeamLine 1:3 0 4 5')} creates a group of three \beam\ elements between nodes {\tt 1 2}, {\tt 2 3}, and {\tt 4 5}.

An alternate call is {\tt femesh('ObjectBeamLine',ind)} where {\tt ind} is a vector containing the node numbers. You can also specify a element name other than {\tt beam1} and properties to be placed in columns 3 and more using {\tt femesh('ObjectBeamLine -EltName',ind,prop)}.

{\tt femesh('ObjectMass 1:3')} creates a group of concentrated \mass\ elements at the declared nodes.

%begindoc
\begin{verbatim}
femesh('Reset')
FEnode = [1 0 0 0  0  0 0;   2 0 0 0  0  0 .15; ... 
          3 0 0 0 .4  1 .176;4 0 0 0 .4 .9 .176];
prop=[100 100 1.1 0 0]; % MatId ProId nx ny nz
femesh('ObjectBeamLine',1:4,prop);femesh('AddSel');
%or femesh(';ObjectBeamLine 1 2 0 2 3 0 3 4;AddSel');
% or femesh('ObjectBeamLine',1:4);
femesh('ObjectMass',3,[1.1 1.1 1.1])
femesh AddSel
femesh PlotElt; fecom TextNode
\end{verbatim}%enddoc

%  - - - - - - - - - - - - - - - - - - - - - - - - - - - - - - - - - - -
\ruic{femesh}{ObjectHoleInPlate}{}

\noindent {\sl Create a} \quada\ {\sl mesh of a hole in a plate.} The format is {\tt 'ObjectHoleInPlate {\ti N0 N1 N2 r1 r2 ND1 ND2 NQ}'}. See \ltr{feutil}{ObjectHoleInPlate} for more details.

%begindoc
\begin{verbatim}
FEnode = [1 0 0 0  0 0 0; 2 0 0 0  1 0 0; 3 0 0 0  0 2 0];
femesh('ObjectHoleInPlate 1 2 3 .5 .5 3 4 4');
femesh('Divide 3 4'); % 3 divisions around, 4 divisions along radii
femesh PlotEl0
% You could also use the call
FEnode = [1 0 0 0  0 0 0;  2 0 0 0  1 0 0; 3 0 0 0  0 2 0];
%   n1 n2 n3 r1 r2 nd1 nd2 nq
r1=[ 1  2  3 .5 .5  3   4   4];
st=sprintf('ObjectHoleInPlate %f %f %f %f %f %f %f %f',r1);
femesh(st); femesh('PlotEl0')
\end{verbatim}%enddoc


%  - - - - - - - - - - - - - - - - - - - - - - - - - - - - - - - - - - -
\ruic{femesh}{ObjectHoleInBlock}{}

\noindent {\sl Create a} \hexah\ {\sl mesh of a hole in a rectangular block.} The format is {\tt 'ObjectHoleInBlock {\ti x0 y0 z0  nx1 ny1 nz1  nx3 ny3 nz3 dim1 dim2 dim3 r nd1 nd2 nd3 ndr}'}. See \ltr{feutil}{ObjectHoleInBlock} for more details.

%begindoc
\begin{verbatim}
femesh('Reset')
femesh('ObjectHoleInBlock 0 0 0  1 0 0  0 1 1  2 3 3 .7  8 8 3 2')
femesh('PlotEl0') 
\end{verbatim}%enddoc

%  - - - - - - - - - - - - - - - - - - - - - - - - - - - - - - - - - - -
\ruic{femesh}{Object}{[Quad,Beam,Hexa] {\ti MatId ProId}}

\noindent {\sl Create or add a model} containing {\tt quad4} {\sl elements}. The user must define a rectangular domain delimited by four nodes and the division in each direction. The result is a regular mesh. 

For example femesh{\tt ('ObjectQuad 10 11',nodes,4,2)} returns model with 4 and 2 divisions in each direction with a {\tt MatId} 10 and a {\tt ProId} 11.

%begindoc
\begin{verbatim}
femesh('reset');
node = [0  0  0; 2  0  0; 2  3  0; 0  3  0];
femesh('Objectquad 1 1',node,4,3); % creates model 
femesh('AddSel');femesh('PlotElt')

node = [3  0  0; 5  0  0; 5  2  0; 3  2  0];
femesh('Objectquad 2 3',node,3,2); % matid=2, proid=3
femesh('AddSel');femesh('PlotElt');femesh Info
\end{verbatim}%enddoc

Divisions may be specified using a vector between {\tt [0,1]} :
%begindoc
\begin{verbatim}
node = [0  0  0; 2  0  0; 2  3  0; 0  3  0];
femesh('Objectquad 1 1',node,[0 .2 .6 1],linspace(0,1,10)); 
femesh('PlotEl0');
\end{verbatim}%enddoc


Other supported object topologies are beams and hexahedrons. For example
%begindoc
\begin{verbatim}
femesh('Reset')
node = [0  0  0; 2  0  0;1  3  0; 1  3  1];
femesh('Objectbeam 3 10',node(1:2,:),4); % creates model
femesh('AddSel');
femesh('Objecthexa 4 11',node,3,2,5); % creates model 
femesh('AddSel');
femesh PlotElt; femesh Info
\end{verbatim}%enddoc


%  - - - - - - - - - - - - - - - - - - - - - - - - - - - - - - - - - - -
\ruic{femesh}{}{Object [Arc, Annulus, Circle,Cylinder,Disk]}

Build selected object in {\tt FEel0}. See \ltr{feutil}{Object} for a list of available objects. For example:

%begindoc
\begin{verbatim}
femesh('Reset')
femesh(';ObjectArc 0 0 0 1 0 0 0 1 0 30 1;AddSel');
femesh(';ObjectArc 0 0 0 1 0 0 0 1 0 30 1;AddSel');
femesh(';ObjectCircle 1 1 1 2 0 0 1 30;AddSel');
femesh(';ObjectCircle 1 1 3 2 0 0 1 30;AddSel');
femesh(';ObjectCylinder 0 0 0  0 0 4 2 10 20;AddSel');
femesh(';ObjectDisk 0 0 0 3 0 0 1 10 3;AddSel');
femesh(';ObjectAnnulus 0 0 0 2 3 0 0 1 10 3;AddSel');
femesh('PlotElt')
\end{verbatim}%enddoc

%  - - - - - - - - - - - - - - - - - - - - - - - - - - - - - - - - - - -
\ruic{femesh}{Optim}{ [Model, NodeNum, EltCheck]}

\noindent \ts{OptimModel} removes nodes unused in {\tt FEelt} {\sl from} {\tt FEnode}.

\ts{OptimNodeNum} does a permutation of nodes in {\tt FEnode} such that the expected matrix bandwidth is smaller. This is only useful to export models, since here DOF renumbering is performed by \femk.

\ts{OptimEltCheck} attempts to fix geometry pathologies (warped elements) in {\tt quad4}, {\tt hexa8} and {\tt penta6} elements.

%  - - - - - - - - - - - - - - - - - - - - - - - - - - - - - - - - - - -
\ruic{femesh}{Orient}{, Orient {\ti i} [ , n {\ti nx ny nz}]}

{\sl Orient elements}.  For volumes and 2-D elements which have a defined orientation, {\tt femesh('Orient')} calls element functions with standard material properties to determine negative volume orientation and permute nodes if needed. This is in particular needed when generating models via \ts{Extrude} or \ts{Divide} operations which do not necessarily result in appropriate orientation (see \integrules). When elements are too distorted, you may have a locally negative volume. A warning about {\tt warped} volumes is then passed. You should then correct your mesh. Note that for 2D meshes you need to use 2D topology holders \qfourp, {\tt t3p, ...}.

{\sl Orient normal of shell elements.} For plate/shell elements (elements with parents of type {\tt quad4}, {\tt quadb} or {\tt tria3}) in groups \tsi{i} of {\tt FEelt}, this command computes the local normal and checks whether it is directed towards the node located at \tsi{nx ny nz}. If not, the element nodes are permuted so that a proper orientation is achieved. A \ts{-neg} option can be added at the end of the command to force orientation away rather than towards the nearest node.

{\tt femesh('Orient {\ti i}',node)} can also be used to specify a list of orientation nodes. For each element, the closest node in {\tt node}  is then used for the orientation. {\tt node} can be a standard 7 column node matrix or just have 3 columns with global positions.

For example

%begindoc
\begin{verbatim}
% Init example
femesh('Reset'); femesh(';Testquad4;Divide 2 3;')
FEelt=FEel0; femesh('DivideGroup1 withnode1'); 
% Orient elements in group 2 away from [0 0 -1]
femesh('Orient 2 n 0 0 -1 -neg');
\end{verbatim}%enddoc

%  - - - - - - - - - - - - - - - - - - - - - - - - - - - - - - - - - - -
\ruic{femesh}{Plot}{ [Elt, El0]}

\noindent {\sl Plot selected model.} {\tt PlotElt} calls \feplot\ to initialize a plot of the model contained in {\tt FEelt}. {\tt PlotEl0} does the same for {\tt FEel0}. This command is really just the declaration of a new model using  {\tt feplot('InitModel',femesh('Model'))}.

Once the plot initialized you can modify it using \feplot\ and \fecom. 

%  - - - - - - - - - - - - - - - - - - - - - - - - - - - - - - - - - - -
\ruic{femesh}{Lin2quad}{, Quad2Lin, Quad2Tria, etc.}

\noindent {\sl Basic element type transformations.}

Element type transformation are applied to elements in {\tt FEel0}. See \ltr{feutil}{Lin2Quad} fore more details and a list of transformations.

%begindoc
\begin{verbatim}
% create 4 quad4 
femesh(';Testquad4;Divide 2 3'); 
femesh('Quad2Tria'); % conversion
femesh PlotEl0
% create a quad, transform to triangles, divide each triangle in 4
femesh(';Testquad4;Quad2Tria;Divide2;PlotEl0;Info'); 
% lin2quad example:
femesh('Reset'); femesh('Testhexa8');
femesh('Lin2Quad epsl .01');
femesh('Info')
\end{verbatim}%enddoc

%  - - - - - - - - - - - - - - - - - - - - - - - - - - - - - - - - - - -
\ruic{femesh}{RefineBeam}{ {\ti l}}

\noindent {\sl Mesh refinement.} This function searches {\tt FEel0} for beam elements and divides elements so that no element is longer than \tsi{l}.

%  - - - - - - - - - - - - - - - - - - - - - - - - - - - - - - - - - - -
\ruic{femesh}{Remove}{[Elt,El0] {\ti ElementSelectors}}

\noindent {\sl Element removal.} This function searches {\tt FEelt} or {\tt FEel0} for elements which verify certain properties selected by \hyperlink{findelt}{{\ti ElementSelectors}} and removes these elements from the model description matrix. A sample call would be

%begindoc
\begin{verbatim}
% create 4 quad4 
femesh('Reset'); femesh(';Testquad4;Divide 2 3'); 
femesh('RemoveEl0 WithNode 1')
femesh PlotEl0
\end{verbatim}%enddoc


%  - - - - - - - - - - - - - - - - - - - - - - - - - - - - - - - - - - -
\ruic{femesh}{RepeatSel}{ {\ti nITE tx ty tz}}

\noindent {\sl Element group translation/duplication.} \ts{RepeatSel} repeats the selected elements ({\tt FEel0}) \tsi{nITE} times with global axis translations \tsi{tx ty tz} between each repetition of the group. If needed, new nodes are added to {\tt FEnode}. An example is treated in the {\tt d\_truss} demo. 

%begindoc
\begin{verbatim}
femesh('Reset'); femesh(';Testquad4;Divide 2 3'); 
femesh(';RepeatSel 3 2 0 0'); % 3 repetitions, translation x=2
femesh PlotEl0
% alternate call:
%                                        number, direction
% femesh(sprintf(';repeatsel %f %f %f %f', 3,    [2 0 0]))
\end{verbatim}%enddoc


%  - - - - - - - - - - - - - - - - - - - - - - - - - - - - - - - - - - -
\ruic{femesh}{Rev}{ {\ti nDiv OrigID Ang nx ny nz}}

\noindent {\sl Revolution} of selected elements in {\tt FEel0}. See \ltr{feutil}{Rev} for more details.
For example:

%begindoc
\begin{verbatim}
FEnode = [1 0 0 0  .2 0   0; 2 0 0 0  .5 1 0; ...  
          3 0 0 0  .5 1.5 0; 4 0 0 0  .3 2 0];
femesh('ObjectBeamLine',1:4);
femesh('Divide 3')
femesh('Rev 40 o 0 0 0 360 0 1 0');
femesh PlotEl0
fecom(';Triax;View 3;ShowPatch')
% An alternate calling format would be
%     divi origin angle direct
%r1 = [40  0 0 0  360   0 1 0];
%femesh(sprintf('Rev %f o %f %f %f %f %f %f %f',r1))
\end{verbatim}%enddoc


%  - - - - - - - - - - - - - - - - - - - - - - - - - - - - - - - - - - -
\ruic{femesh}{RotateSel}{ {\ti OrigID Ang nx ny nz}}

\noindent {\sl Rotation.} The selected elements {\tt FEel0} are rotated by the angle \tsi{Ang} (degrees) around an axis passing trough the node of number \tsi{OrigID} (or the origin of the global coordinate system) and of direction {\tt [}\tsi{nx ny nz}{\tt ]} (the default is the {\tt z} axis {\tt [0 0 1]}). The origin can also be specified by the {\sl xyz }values preceded by an \ts{o}

{\tt femesh('RotateSel o 2.0 2.0 2.0 \ \ \  90  1 0 0')}

This is an example of the rotation of {\tt FEel0} 

%begindoc
\begin{verbatim}
femesh('Reset');
femesh(';Testquad4;Divide 2 3'); 
% center is node 1, angle 30, aound axis z
%                                       Center angle dir
st=sprintf(';RotateSel %f %f %f %f %f',[1      30   0 0 1]);
femesh(st);  femesh PlotEl0
fecom(';Triax;TextNode'); axis on
\end{verbatim}%enddoc


%  - - - - - - - - - - - - - - - - - - - - - - - - - - - - - - - - - - -
\ruic{femesh}{Sel}{ [Elt,El0] {\ti ElementSelectors}}

\noindent {\sl Element selection}. \ts{SelElt} places in the selected model {\tt FEel0} elements of {\tt FEelt} that verify certain conditions. You can also select elements within {\tt FEel0} with the \ts{SelEl0} command. Available element selection commands are described under the \ts{FindElt} command and~\ser{findelt}. 

{\tt femesh('SelElt  {\ti ElementSelectors}')}.

%  - - - - - - - - - - - - - - - - - - - - - - - - - - - - - - - - - - -
\ruic{femesh}{SelGroup}{ {\ti i}, SelNode {\ti i}}

\noindent {\sl Element group selection}. The element group \tsi{i} of {\tt FEelt} is placed in {\tt FEel0} (selected model). \ts{SelGroup}\tsi{i} is equivalent to \ts{SelEltGroup}\tsi{i}.

{\sl Node selection}. The node(s) \tsi{i} of {\tt FEnode} are placed in {\tt FEn0} (selected nodes). 

%  - - - - - - - - - - - - - - - - - - - - - - - - - - - - - - - - - - -
\ruic{femesh}{SetGroup}{ [{\ti i},{\ti name}] [Mat {\ti j}, Pro {\ti k}, EGID {\ti e}, Name {\ti s}]}

\noindent {\sl Set properties of a group.} For group(s) of {\tt FEelt} selector by number \tsi{i}, name \tsi{name}, or \ts{all} you can modify the material property identifier \tsi{j}, the element property identifier \tsi{k} of all elements and/or the element group identifier \tsi{e} or name \tsi{s}. For example

\begin{verbatim}
 femesh('SetGroup1:3 pro 4')
 femesh('SetGroup rigid name celas') 
\end{verbatim}


If you know the column of a set of element rows that you want to modify, calls of the form {\tt FEelt(femesh('FindElt{\ti Selectors}'),{\ti Column})= {\ti Value}} can also be used.

%begindoc
\begin{verbatim}
 model=femesh('Testubeamplot');
 FEelt(femesh('FindEltwithnode {x==-.5}'),9)=2;
 femesh PlotElt; 
 cf.sel={'groupall','colordatamat'};
\end{verbatim}%enddoc


You can also use {\tt femesh('set groupa 1:3 pro 4')} to modify properties in {\tt FEel0}.

%  - - - - - - - - - - - - - - - - - - - - - - - - - - - - - - - - - - -
\ruic{femesh}{SymSel}{ {\ti OrigID nx ny nz}}

\noindent {\sl Plane symmetry.  }\ts{SymSel} replaces elements in {\tt FEel0} by elements symmetric with respect to a plane going through the node of number \tsi{OrigID} (node {\tt 0} is taken to be the origin of the global coordinate system) and normal to the vector {\tt [}\tsi{nx ny nz}{\tt ]}. If needed, new nodes are added to {\tt FEnode}.  
Related commands are \lts{femesh}{TransSel}, \lts{femesh}{RotateSel} and \lts{femesh}{RepeatSel}.

%  - - - - - - - - - - - - - - - - - - - - - - - - - - - - - - - - - - -
\ruic{femesh}{Test}{}

Some unique element model examples. See list with {\tt femesh('TestList')}.
For example a simple cube model can be created using\\
%begindoc
\begin{verbatim}
model=femesh('TestHexa8'); % hexa8 test element
\end{verbatim}%enddoc

\begin{SDTDEV}
\color{red}
Command {\tt femesh('TestUseLegacy',1)} can be used to enable the legacy formulation. This is done in {\tt basic\_elt\_test} test.
\color{black}
\end{SDTDEV}

%  - - - - - - - - - - - - - - - - - - - - - - - - - - - - - - - - - - -
\ruic{femesh}{TransSel}{ {\ti tx ty tz}}

\noindent {\sl Translation of the selected element groups}.  \ts{TransSel} replaces elements of {\tt FEel0} by their translation of a vector {\tt [}\tsi{tx ty tz}{\tt ]} (in global coordinates).  If needed, new nodes are added to {\tt FEnode}.  Related commands are \lts{femesh}{SymSel}, \lts{femesh}{RotateSel} and \lts{femesh}{RepeatSel}.

%begindoc
\begin{verbatim}
femesh('Reset');
femesh(';Testquad4;Divide 2 3;AddSel'); 
femesh(';TransSel 3 1 0;AddSel'); % Translation of [3 1 0]
femesh PlotElt
fecom(';Triax;TextNode')
\end{verbatim}%enddoc


%  - - - - - - - - - - - - - - - - - - - - - - - - - - - - - - - - - - -
\ruic{femesh}{UnJoin}{ \tsi{Gp1 Gp2}}

{\sl Duplicate nodes which are common to two groups.} To allow the creation of interfaces with partial coupling of nodal degrees of freedom, \ts{UnJoin} determines which nodes are common to the element groups \tsi{Gp1} and \tsi{Gp2} of {\tt FEelt}, duplicates them and changes the node numbers in \tsi{Gp2} to correspond to the duplicate set of nodes. In the following call with output arguments, the columns of the matrix {\tt InterNode} give the numbers of the interface nodes in each group {\tt InterNode = femesh('UnJoin 1 2')}.

%begindoc
\begin{verbatim}
 femesh('Reset'); femesh('Test2bay');
 femesh('FindNode group1 & group2') % nodes 3 4 are common
 femesh('UnJoin 1 2');
 femesh('FindNode group1 & group2') % no longer any common node
\end{verbatim}%enddoc

A more general call allows to separate nodes that are common to two sets of elements \texline {\tt femesh('UnJoin',{\ti 'Selection1'},{\ti 'Selection2'})}. Elements in \tsi{Selection1} are left unchanged while nodes in \tsi{Selection2} that are also in \tsi{Selection1} are duplicated.



%  - - - - - - - - - - - - - - - - - - - - - - - - - - - - - - - - - - -
\rmain{See also}

\noindent \femk, \fecom, \feplot, \ser{fem}, demos {\tt gartfe},  {\tt d\_ubeam}, {\tt beambar} ... 














%       Copyright (c) 2001-2019 by INRIA and SDTools, All Rights Reserved.
%       Use under OpenFEM trademark.html license and LGPL.txt library license
%       $Revision: 1.101 $  $Date: 2025/02/12 11:13:25 $

%-----------------------------------------------------------------------------
\rtop{feutil}{feutil}

Finite element mesh handling utilities.\index{global variable}\index{global variable}

\rsyntax
\begin{verbatim}
[out,out1] = feutil('CommandString',model,...)
\end{verbatim}

\rmain{Description}

{\tt feutil} provides a number of tools for mesh creation and manipulation. 

Some commands return the model structure whereas some others return only the element matrix.
To mesh a complex structure one can mesh each subpart in a different model structure (model, mo1, ...) and combine each part using \lts{feutil}{AddTest} command. 
\begin{SDT}
To handle complex model combination (not only meshes but whole models with materials, bases, ...), one can use the \lts{feutilb}{CombineModel} command.
\end{SDT}

\noindent Available {\tt feutil} commands are

%  - - - - - - - - - - - - - - - - - - - - - - - - - - - - - - - - - - -
\ruic{feutil}{Advanced}{}
Advanced command with non trivial input/output formats or detailed options are listed under \khref{feutila}{feutila}.

%  - - - - - - - - - - - - - - - - - - - - - - - - - - - - - - - - - - -
\ruic{feutil}{AddElt}{}
{\tt model.Elt=feutil('AddElt',model.Elt,'EltName',data)}

This command can be used to add new elements to a model. \ts{EltName} gives the element name used to fill the header. {\tt data} describes elements to add (one row per element).

Command option \ts{-newId} forces new {\tt EltId} following the maximum {\tt EltId} of {\tt model.Elt} to be assigned to the added elements, using this command generates a second output providing the {\tt EltId} convertion list {\tt [oldId newId;...]} for the added elements.

Following example adds {\tt celas} elements to the basis of a simple cube model.
%begindoc
\begin{verbatim}
% Adding elements to a model 
femesh('Reset'); model=femesh('Testhexa8'); % simple cube model
data=[1 0 123 0 0 1 1e3; 2 0 123 0 0 1 1e3;
      3 0 123 0 0 1 1e3; 4 0 123 0 0 1 1e3]; % n1 n2 dof1 dof2 EltId ProId k
model.Elt=feutil('AddElt',model.Elt,'celas',data);
cf=feplot(model);

% Cleanup eltid for model
[eltid,model.Elt]=feutil('EltIdFix;',model);
el1=[1 4 123 0 1 0 1000];
[model.Elt,r1]=feutil('AddElt-newId',model.Elt,'celas',el1);
\end{verbatim}%enddoc

%  - - - - - - - - - - - - - - - - - - - - - - - - - - - - - - - - - - -
\ruic{feutil}{AddNode}{[,New] [, From \tsi{i}] [,epsl \tsi{val}]}

{\tt [AllNode,ind]=feutil('AddNode',{\ti OldNode},{\ti NewNode});}

\noindent {\sl Combine} (without command option \ts{New}) or {\sl append} (with command option \ts{New}) {\ti NewNode} to {\ti OldNode}. 
Without command option \ts{New}, \ts{AddNode} combines {\ti NewNode} to {\ti OldNode}: it finds nodes in {\ti NewNode} that coincide with nodes in {\ti OldNode} and appends other nodes to form {\tt AllNode}. 
With command option \ts{New}, \ts{AddNode} simply appends {\ti NewNode} to {\ti OldNode}.

{\tt AllNode} is the new node matrix with added nodes.
{\tt ind} (optional) gives the indices of the {\ti NewNode} nodes in the {\tt AllNode} matrix.

{\ti NewNode} can be specified as a matrix with three columns giving {\tt xyz} coordinates. The minimal distance below which two nodes are considered identical is given by \sdtdef\ \tsi{epsl} (default {\tt 1e-6}).

{\tt [AllNode,ind]=feutil('AddNode From 10000',{\ti OldNode},{\ti NewNode});} gives node numbers starting at 10000 for nodes in {\ti NewNode} that are not in {\ti OldNode}.

\begin{SDT}
SDT uses an optimized algorithm available in \feutilb\ . 

By default, nodes that repeated in {\ti NewNode} are coalesced onto the same node (a single new node is added). If there is not need for that coalescence, you can get faster results with \ts{AddNode-nocoal}.  

{\tt ind=feutilb('AddNode -near epsl \tsi{value}',{\ti n1},{\ti n2});} returns a sparse matrix with non zero values in a given colum indicating of {\ti n1} nodes that are within {\tt epsl} of each {\ti n2} node (rows/columns correspond to {\tt n2/n1} node numbers). 

{\tt id=feutilb('AddNode -nearest epsl \tsi{value}',{\ti n1},{\ti xyz});} returns vector giving the nearest {\tt {\ti n1} NodeId}  to each {\ti xyz} node the search area being limited to {\tt epsl}. When specified with a 7 column {\ti n2}, the result is {\tt sparse(n2(:,1),1,n1\_index)}. For fine meshes the algorithm can use a lot of memory. If {\ti n2} is not too large it is then preferable to use an \ts{AddNode} command with a tolerance sufficient for a match {\tt [n3,ind]=feutil('AddNode epsl \tsi{value}',{\ti n1},{\ti n2});id=n3(ind,1)}.
\end{SDT}

%  - - - - - - - - - - - - - - - - - - - - - - - - - - - - - - - - - - -
\ruic{feutil}{AddSet}{[NodeId, EltId, FaceId, EdgeId]}

Command \ts{AddSet} packages the generation of sets in an SDT model. Depending on the type of set several command options can apply.

\begin{itemize}
\item {\tt model=feutil('AddSetNodeId',model,'name','FindNodeString')} adds the selection \texline \hyperlink{findnode}{\ts{FindNodeString}} as a \ltt{set} of nodes \ts{name} to {\tt model}. \hyperlink{findnode}{\ts{FindNodeString}} can be replaced by a column vector of {\tt NodeId}.

\item Syntax is the same for \ts{AddSetEltId} with a \hyperlink{findelt}{\ts{FindEltString}} selection. \hyperlink{findelt}{\ts{FindEltString}} can be replaced by a column vector of {\tt EltId}. Command option \ts{FromInd} allows providing element indices instead of IDs.

\item For faces with \htr{feutil}{AddSetFaceId}, the element selection argument \hyperlink{findelt}{\ts{FindEltString}} must result in the generation of a face selection. One can use the \ts{SelFace} token in the \hyperlink{findelt}{\ts{FindEltString}} to this purpose. As an alternative, one can directly provide an element matrix resulting from a \ts{SelFace} selection, or a 2 column list of respectively {\tt EltId} and Face identifiers.
For face identifier conversion to other code conventions, one can use command option \ts{@}\tsi{fun} to obtain a set with a {\tt ConvFcn} set to \tsi{fun}, see \ltt{set} for more details.

\item For generation of {\tt EdgeId} sets, the element selection argument \hyperlink{findelt}{\ts{FindEltString}} must result in the generation of an edge selection. One can use the \ts{SelEdge} token in the \hyperlink{findelt}{\ts{FindEltString}} to this purpose. As an alternative, one can directly provide an element matrix resulting from a \ts{SelEdge} selection, or a 2 column list of respectively {\tt EltId} and Edge identifiers. Support for edge identifier conversion and {\tt setname} selection is not provided yet.
\end{itemize}

The option \ts{-id }\tsi{value} can be added to the command to specify a set ID.

\begin{SDT}
By default the generated set erases any previously existing set with the same name, regardless of the type. Command option \ts{New} alters this behavior by incrementing the set name. One can use the command second output to recover the new name.
\end{SDT}

\vs

Command option \ts{-Append} allows generation of a \ltt{meta-set}. The meta-set is an agglomeration of several sets of possibly various types, see \ltt{set} for more information.
\begin{itemize}
\item The base syntax requires providing the meta-set name and the set name. \texline {\tt model=feutil('AddSetEltId -Append',model,'name','FindEltString','subname')} will thus add the elements found as a sub set named {\tt subname} of meta-set {\tt name}. {\tt subname} can be a 1x2 cell array {\tt \{subname,subgroup\}} providing the set name and the set {\tt subgroup} it belongs to. By default {\tt subgroup} is set to the set type.

\item Generation of a meta-set gathering all base sets in the model is possible by omitting {\tt subname} and the {\tt FindEltString}.
\end{itemize}

\vs

By default command \ts{AddSet} returns the model as a first output and possibly the set data structure in a second output. Command option \ts{-get} alters this behavior returning the data set structure without adding it to the model. For {\tt FaceId} or {\tt EdgeId} sets, command option \ts{-get} can output the elements selected by the \hyperlink{findelt}{\ts{FindEltString}}.

Following example defines a set of each type on the {\tt ubeam} model:
%begindoc
\begin{verbatim}
% Defining node elements or face sets in a model
model=demosdt('demo ubeam'); 
% Add a set of NodeId, and recover set data structure
[model,data]=feutil('AddSetNodeId',model,'nodeset','z==1');
% Add a set of EltId 
model=feutil('AddSetEltId -id18',model,'eltset','WithNode{z==0}');
% Generate a set of EltId without model addition
data=feutil('AddSetEltId -id18 -get',model,'eltset','WithNode{z==0}');
% Generate a set of FaceId
model=feutil('AddSetFaceId',model,'faceset','SelFace & InNode{z==0}');
% Generate a set of FaceId without model addition
[data1,elt]=feutil('AddSetFaceId -get',model,'faceset','SelFace & WithNode{z==0}');

% Sample visalization commands
cf=feplot; % get feplot handle
[elt,ind]=feutil('FindElt setname  eltset',model); % FindElt based on set name
cf.sel='setname faceset'; % element selection based on a FaceId set

% Lower level set handling
% Generate a FaceSet from an EltSet
r1=cf.Stack{'eltset'};r1.type='FaceId';r1.data(:,2)=1;
cf.Stack{'set','faceset'}=r1;
% Generate a DOF set from a node set
r1=cf.Stack{'nodeset'};r1.type='DOF';r1.data=r1.data+0.02;
cf.Stack{'set','dofset'}=r1;
% Visualize set data in promodel stack
fecom(cf,'curtab Stack','eltset');
\end{verbatim}%enddoc


%  - - - - - - - - - - - - - - - - - - - - - - - - - - - - - - - - - - -
\ruic{feutil}{AddTest}{[,-EGID \tsi{i}][,\tsi{NodeShift},Merge,Combine]}\index{wire-frame plots}

{\tt model=feutil('AddTest',mo1,mo2);} {\sl Combine models}. 
When combining test and analysis models you typically want to overlay a detailed finite element mesh with a coarse wire-frame representation of the test configuration. These models coming from different origins you will want combine the two models in {\tt model}.

\begin{SDT}
Note that the earlier objective of combining test and FEM models is now more appropriately dealt with using \ts{SensDof} entries, see ~\ser{sensor} for sensor definitions and~\ser{dockCoTopo} for test/analysis correlation.  If you aim at combining several finite element models into an assembly, with proper handling of materials, element IDs, bases,\dots, you should rather use the more appropriate \lts{feutilb}{CombineModel} command.
\end{SDT}

\begin{itemize}
\item {\bf By default} the node sets are considered to be disjoint. New nodes are added starting from {\tt max(mo1.Node(:,1))+1} or from \tsi{NodeShift}+1 if the argument is specified. \texline Thus {\tt feutil('AddTest \tsi{NodeShift}',mo1,mo2)} adds {\tt mo2} nodes  to {\tt mo1.Node} while adding \tsi{NodeShift} to their initial identification number. The same \tsi{NodeShift} is added to node numbers in {\tt mo2.Elt} which is appended to {\tt mo1.Elt}. {\tt mo2} can be a wire frame matrix read with \ufread\ for example.
\item With command option \ts{Merge} it is assumed that some nodes are common but their numbering is not coherent. Non coincident nodes (as defined by the \ts{AddNode} command) are added to {\tt mo1.Node} and {\tt mo2.Elt} is renumbered according to resulting {\tt model.Node}. Command option \ts{Merge-Edge} is used to force mid-side nodes to be common if the end nodes are. Note that command \ts{Merge} will also merge all coincident nodes of {\tt mo2}.
\item With command option \ts{Combine} it is assumed that some nodes are common and their numbering is coherent. Nodes of {\tt mo2.Node} with new {\tt NodeId} values are added to {\tt mo1.Node} while common {\tt NodeId} values are assumed to be located at the same positions.
\item You can specify an {\tt EGID} value for the elements that are added using \ts{AddTest -EGID -1} for example. In particular negative {\tt EGID} values are display groups so that they will be ignored in model assembly operations. Command option \ts{keeptest} allows to retain existing test frames when adding a new one. If the same {\tt EGID} is declared, test frames are then combined in the same group.
\item Command option \ts{-NoOri} returns model without the {\tt Info,OrigNumbering} entry in the model stack.
\end{itemize}

%  - - - - - - - - - - - - - - - - - - - - - - - - - - - - - - - - - - -
\ruic{feutil}{Divide}{ \tsi{div1 div2 div3}}
{\tt model=feutil('Divide \tsi{div1 div2 div3}',model);}\\
\noindent {\sl Mesh refinement by division of elements.} 
\ts{Divide} applies to all groups in {\tt model.Elt}. To apply the division to a selection within the model use \lts{feutil}{ObjectDivide}. 

Division directions \tsi{div1 div2 div3} are here understood in the local element basis, thus depending on the declared node orders in the connectivity matrix that refer to the reference cell. Uneven divisions as function of the direction will thus require some care regarding the element declaration if the original mesh has been heterogeneously generated.

Currently supported divisions are

\begin{Eitem}

\item segments : elements with \beam\ parents are divided in \tsi{div1} segments of equal length.

\item quadrilaterals: elements with \quada\ or \quadb\ parents are divided in a regular mesh of \tsi{div1} by \tsi{div2} quadrilaterals.

\item hexahedrons: elements with \hexah\ or \hexav\ parents are divided in a regular grid of \tsi{div1} by \tsi{div2} by \tsi{div3} hexahedrons.

\item \triaa\ can be divided with an equal division of each segment specified by \tsi{div1}.

\item \mass\ and \celas\ elements are kept unchanged.

\end{Eitem}

\noindent The \ts{Divide} command applies element transformation schemes on the element parent topological structure. By default, the original element names are maintained. In case of trouble, element names can be controlled by declaring the proper parent name or use the \lts{feutil}{SetGroup}\ts{Name} command before and after \ts{divide}. 

The division preserves properties other than the node numbers, in addition final node numbering/ordering will depend on the MATLAB version. It is thus strongly recommended not to base meshing scripts on raw {\tt NodeId}.

You can obtain unequal divisions by declaring additional arguments whose lines give the relative positions of dividers. Note that this functionality has not been implemented for \quadb\ and \triaa\ elements.

For example, an unequal 2 by 3 division of a {\tt quad4} element would be obtained using\\
{\tt model=feutil('divide',[0 .1 1],[0 .5 .75 1],model)} (see also the {\tt gartfe} demo).

%begindoc
\begin{verbatim}
% Refining a mesh by dividing the elements
% Example 1 : beam1
femesh('Reset'); model=femesh('Testbeam1');  % build simple beam model
model=feutil('Divide 3',model); % divide by 3
cf=feplot(model); fecom('TextNode'); % plot model and display NodeId

% Example 2 : you may create a command string
femesh('Reset'); model=femesh('Testbeam1');  % build simple beam model
number=3;
st=sprintf('Divide %f',number);
model=feutil(st,model);
cf=feplot(model); fecom('TextNode')

% Example 3 : you may use uneven division
femesh('Reset'); model=femesh('Testquad4');  % one quad4 created
model=feutil('Divide',model,[0 .1 .2 1],[0 .3 1]); 
feplot(model);
\end{verbatim}%enddoc

An inconsistency in division for quad elements was fixed with version 1.105, you can obtain the consistent behavior (first division along element $x$) by adding the option \ts{-new} anywhere in the \ts{divide} command.

%  - - - - - - - - - - - - - - - - - - - - - - - - - - - - - - - - - - -
\ruic{feutil}{DivideInGroups}{}
{\tt elt=feutil('DivideInGroups',model);}\\
Finds groups that are not connected (no common node) and places each of these groups in a single element group.

%  - - - - - - - - - - - - - - - - - - - - - - - - - - - - - - - - - - -
\ruic{feutil}{DivideGroup}{ \tsi{i ElementSelectors}}
{\tt elt=feutil('DivideGroup \tsi{i ElementSelector}',model);}\\

Divides a single group \tsi{i} in two element groups. The first new element group is defined based on the element selectors (see \ser{findelt}).

For example 
{\tt elt=feutil('divide group 1 withnode\{x>10\}',model);}

\ruic{feutil}{EltId}{} % - - - - - - - - - - - - - - - - - - - 

{\tt [EltId]=feutil('EltId',elt)} returns the element identifier for each element in {\tt elt}. It currently does not fill {\tt EltId} for elements which do not support it. \\ 
{\tt [EltId,elt]=feutil('EltIdFix',elt)} returns an {\tt elt} where the element identifiers have been made unique.\\
Command option \ts{-elt} can be used to set new {\tt EltId}.\\
\begin{SDT}
Command option \ts{-model} can be used to set new {\tt EltId} and renumber model Stack data, a model structure must be input, and the output is then the model.
\end{SDT}

%begindoc
\begin{verbatim}
% Handling elements IDs, renumbering elements
model=femesh('TestHexa8')
[EltId,model.Elt]=feutil('EltIdFix',model.Elt); % Fix and get EltId
[model.Elt,EltIdPos]=feutil('eltid-elt',model,EltId*18); % Set new EltId
model.Elt(EltIdPos>0,EltIdPos(EltIdPos>0)) % New EltId
\end{verbatim}%continuedoc

\begin{SDT}
%continuedoc
\begin{verbatim}
% Renumber EltId with stack data
model=feutil('AddSetEltId',model,'all','groupall');
model=feutil('EltId-Model',model,EltId+1);
\end{verbatim}%enddoc
\end{SDT}

%  - - - - - - - - - - - - - - - - - - - - - - - - - - - - - - - - - - -
\ruic{feutil}{EltSetReplace}{}

Replace {\tt EltId} in {\tt EltId} sets with convertion table. This can be usefull when elements are modified, or refined and one would like to keep initial sets coherent with replaced parts. to ease up the procedure it is assumed that the original element sets are provided in meta-set format (see~\swref{Stack} for reference).

%begindoc
\begin{verbatim}
% Define a model with clean EltId
model=femesh('testhexa8');
[eltid,model.Elt]=feutil('EltIdFix;',model);
% Generate an EltId set
model=feutil('AddSetEltId',model,'comp','groupall');
% Localize elements in model belonging to set
i1=feutil('FindElt setname comp',model);
% Get global meta-set data from model
r1=feutil('AddSetEltId-Append-get',model,'_gsel');
% Now renumber EltId
eltid(:,2)=eltid(:,1)+1e3;
model.Elt=feutil('EltId-Elt',model,eltid(:,2));
% Call EltSetReplace to update sets with new eltid
model=feutil('EltSetReplace',model,r1,eltid);
% Check that the set in model is now coherent
i2=feutil('FindElt setname comp',model);
isequal(i1,i2)
\end{verbatim}%enddoc


%  - - - - - - - - - - - - - - - - - - - - - - - - - - - - - - - - - - -
\ruic{feutil}{Extrude}{ \tsi{nRep tx ty tz}}

\noindent {\sl Extrusion}.  Nodes, lines or surfaces of model are extruded \tsi{nRep} times with global translations \tsi{tx ty tz}.  Elements with a \mass\ parent are extruded into beams, element with a \beam\ parent are extruded into \quada\ elements, \quada\ are extruded into \hexah, and \quadb\ are extruded into \hexav.

You can create irregular extrusion. For example, \texline {\tt model=feutil('Extrude 0  0 0 1',model,[0 logspace(-1,1,5)])} will create an exponentially spaced mesh in the $z$ direction. The second argument gives the positions of the sections for an axis such that {\tt tx ty tz} is the unit vector.

%begindoc
\begin{verbatim}
% Extruding mesh parts to build a model
% Example 1 : beam
femesh('Reset'); model=femesh('Testbeam1'); % one beam1 created
model=feutil('Extrude 2 1 0 0',model); % 2 extrusions in x direction
cf=feplot(model);

% Example 2 : you may create the command string
number=2;step=[1 0 0];
st=sprintf('Extrude %f %f %f %f',[number step]);
femesh('Reset'); model=femesh('Testbeam1'); % one beam1 created
model=feutil(st,model);
cf=feplot(model);

% Example 3 : you may uneven extrusions in z direction
femesh('Reset'); model=femesh('Testquad4');
model=feutil('Extrude 0 0 0 1',model,[0 .1 .2 .5 1]);
     % 0 0 0 1        :  1 extrusion in z direction
     % [0 .1 .2 .5 1] :  where extrusions are made
feplot(model)
\end{verbatim}%enddoc


%  - - - - - - - - - - - - - - - - - - - - - - - - - - - - - - - - - - -
\ruic{feutil}{GetDof}{ \tsi{ElementSelectors}}

Command to obtain DOF from a model, or from a list of {\tt NodeId} and DOF.

\vs

Use {\tt mdof=feutil('GetDof',dof,NodeId);} to generate a DOF vector from a list of DOF indices {\tt dof}, a column vector ({\it e.g.} {\tt dof=[.01;.02;.03]}), and a list of {\tt NodeId}, a column vector. The result will be sorted by DOF, equivalent to {\tt mdof = [NodeId+dof(1);NodeId+dof(2);...]}.

Call {\tt mdof=feutil('GetDof',NodeId,dof);} will output a DOF vector sorted by {\tt NodeId}, equivalent to {\tt mdof = [NodeId(1)+dof;NodeId(2)+dof;...]}.

\vs

The nominal call to get DOFs used by a model is {\tt mdof=feutil('GetDOF',model)}. These calls are performed during assembly phases (\femk, \feload, ...). This supports elements with variable DOF numbers defined through the element rows or the element property rows. 
To find DOFs of a part of the model, you should add a \ts{ElementSelector} string to the \ts{GetDof} command string.

Note that node numbers set to zero are ignored by \feutil\  to allow elements with variable number of nodes.


%  - - - - - - - - - - - - - - - - - - - - - - - - - - - - - - - - - - -
\ruic{feutil}{FindElt}{ \tsi{ElementSelectors}}
\index{element!selection}

\noindent {\sl Find elements} based on a number of selectors described in \ser{findelt}. The calling format is 

{\tt [ind,elt] = feutil('FindElt \tsi{ElementSelector}',model);} 

where {\tt ind} gives the row numbers of the elements in {\tt model.Elt} (but not the header rows except for unique superelements which are only associated to a header row) and {\tt elt} (optional) the associated element description matrix. 

When operators are accepted, equality and inequality operators can be used. Thus {\tt group\verb+~=+[3 7]} or {\tt pro < 5} are acceptable commands. See also \lts{feutil}{SelElt}, \lts{feutil}{RemoveElt} and \lts{feutil}{DivideGroup}, the {\tt gartfe} demo, \fecom\ selections.


%  - - - - - - - - - - - - - - - - - - - - - - - - - - - - - - - - - - -
\ruic{feutil}{FindNode}{ \tsi{Selectors}}
\index{node!selection}

\noindent {\sl Find node numbers} based on a number of node selectors listed in \ser{findnode}. 

Different selectors can be chained using the logical operations \ts{\&} (finds nodes that verify both conditions), \ts{|} (finds nodes that verify one or both conditions). Condition combinations are always evaluated from left to right (parentheses are not accepted).

The calling format is\\
{\tt [NodeId,Node] = feutil('FindNode \tsi{NodeSelector}',model);} 

\noindent Output arguments are the {\tt NodeId} of the selected nodes and the selected nodes {\tt Node} as a second optional output argument. 

\noindent As an example you can show node numbers on the right half of the {\tt z==0} plane using the commands

\noindent{\tt fecom('TextNode',feutil('FindNode z==0 \& x>0',model))}

Following example puts markers on selected nodes
%begindoc
\begin{verbatim}
% Finding nodes and marking/displaying them in feplot
demosdt('demo ubeam'); cf=feplot; % load U-Beam model
fecom('ShowNodeMark',feutil('FindNode z>1.25',cf.mdl),'color','r')
fecom('ShowNodeMark-noclear',feutil('FindNode x>0.2*z|x<-0.2*z',cf.mdl),...
      'color','g','marker','o')
\end{verbatim}%enddoc

Note that you can give numeric arguments to the command as additional {\tt feutil} arguments. Thus the command above could also have been written 
{\tt feutil('FindNode z== \& x>=',0,0))}

 See also the {\tt gartfe} demo.


%  - - - - - - - - - - - - - - - - - - - - - - - - - - - - - - - - - - -
\ruic{feutil}{FixMPCMaster}{}

Resolution of MPC to define independent constraints, and redefine slave DOF.
The calling format is\\
{\tt [c1,islave]=feutil('fixMpcMaster',c)}

\noindent Input {\tt c} is either a constraint structure with fields {\tt .c} a constraint matrix, {\tt .DOF} a DOF vector coherent with the number of columns of {\tt .c} and optional {\tt .slave} field providing an initial list of slave indices in DOF. The input can directly be a constrain matrix, and an additional input is accepted in this case to provide the initial slave indices.

\noindent Output arguments are the recombined constraint matrix, and the column indices associated to slave DOF.

Trivial recombinations are first tested, along with wrong master definition checks. A complete resolution can be performed otherwise.

%begindoc
\begin{verbatim}
% Sample constraint resolutions
c=[1 0 -1 0;0 1 0 -1]
[c1,islave]=feutil('fixmpcmaster',c)

c=[1 0 -1 0 ;-1 1 -1 0]
[c1,islave]=feutil('fixmpcmaster',c)
\end{verbatim}%enddoc


\ruic{feutil}{GetEdge}{[Line,Patch]} % - - - - - - - - - - - - - - - - - - - - - - - -

These {\tt feutil} commands are used to create a model containing the 1D edges or 2D faces of a model. A typical call is

%begindoc
\begin{verbatim}
 % Generate a contour (nD-1) model from a nD model
 femesh('reset'); model=femesh('Testubeam');
 elt=feutil('GetEdgeLine',model); feutil('infoelt',elt)
\end{verbatim}%enddoc

\ts{GetEdgeLine} supports the following variants \ts{MatId} retains inter material edges, \ts{ProId} retains inter property edges, \ts{Group} retains inter group edges, \ts{all} does not eliminate internal edges, \ts{InNode} only retains edges whose node numbers are in a list given as an additional {\tt feutil} argument.

These commands are used for \ts{SelEdge} and \ts{SelFace} element selection commands. \ts{Selface} preserves the \ltt{EltId} and adds the \ltt{FaceId} after it to allow face set recovery.

\ruic{feutil}{GetElemF}{} % - - - - - - - - - - - - - - - - - - - - - - - - - - - - - -

\noindent {\sl Header row parsing.} In an element description matrix, element groups are separated by header rows (see \ser{elt}) which for the current group {\tt jGroup} is given by {\tt elt(EGroup(jGroup),:)} (one can obtain {\tt EGroup} - the positions of the headers in the element matrix - using \texline {\tt[EGroup,nGroup]=getegroup(model.Elt)}. The \ts{GetElemF} command, whose proper calling format is

\noindent 
{\tt [ElemF,opt,ElemP] = feutil('GetElemF',elt(EGroup(jGroup),:),[jGroup])}

\noindent returns the element/superelement name {\tt ElemF}, element options {\tt opt} and the parent element name {\tt ElemP}. It is expected that {\tt opt(1)} is the {\tt EGID} (element group identifier) when defined.

\ruic{feutil}{Get}{[Line,Patch]} % - - - - - - - - - - - - - - - - - - - - - - - - - - - - - -

{\tt Line=feutil('GetLine',node,elt)} returns a matrix of lines where each row has the form \texline {\tt [length(ind)+1 ind] }plus trailing zeros, and {\tt ind}  gives node indices (if the argument {\tt node} is not empty) or {\tt node} numbers (if {\tt node} is empty). {\tt elt} can be an element description matrix or a connectivity line matrix (see \feplot).  Each row of the {\tt Line} matrix corresponds to an element group or a line of a connectivity line matrix. For element description matrices, redundant lines are eliminated.

{\tt Patch=feutil('GetPatch',Node,Elt)} returns a patch matrix where each row (except the first which serves as a header) has the form {\tt [n1 n2 n3 n4 EltN GroupN]}.  The {\tt n}{\ti i} give node indices (if the argument {\tt Node} is not empty) or node numbers (if {\tt Node} is empty).  {\tt Elt} must be an element description matrix.  Internal patches (it is assumed that a patch declared more than once is internal) are eliminated.

 The \ts{all} option skips the internal edge/face elimination step. 
These commands are used in wire-frame and surface rendering.

%  - - - - - - - - - - - - - - - - - - - - - - - - - - - - - - - - - - -
\ruic{feutil}{GetNode}{ \tsi{Selectors}}

{\tt Node=feutil('GetNode \tsi{Selectors}',model)} returns a matrix containing nodes rather than NodeIds obtained with the \lts{feutil}{FindNode} command. The indices of the nodes in {\tt model.Node} can be returned as a 2nd optional output argument.
This command is equivalent to the {\tt feutil} call 

{\tt [NodeId,Node]=feutil('FindNode \tsi{Selectors}',model)}.

\ruic{feutil}{GetNormal}{[Elt,Node][,Map],GetCG} % - - - - - - - - - - - - - - - - - - - -

{\tt [normal,cg]=feutil('GetNormal[elt,node]',model)} returns normals to elements/nodes in {\tt model}.\\
{\tt CG=feutil('GetCG',model)} returns the CG locations. Command option \ts{-dir i} can be used to specify a local orientation direction other than the normal (this is typically used for composites).\\
{\tt MAP=feutil('getNormal Map',model)} returns a data structure with the following fields\index{normal}\index{Map}

\begin{tabular}{@{}p{.20\textwidth}@{}p{.80\textwidth}@{}}
\rz{\tt ID}     & column of identifier (as many as rows in the {\tt .normal} field). For {\tt .opt=2} contains the  {\tt NodeId}. For {\tt .opt=1} contains the  {\tt EltId}. \\
\rz{\tt normal} & $N\times 3 $ where each row specifies a vector at {\tt ID} or {\tt vertex}.\\
\rz{\tt opt}    & 1 for MAP at element center, 2 for map at nodes. \\
\rz{\tt color} & $N\times 1 $ optional real value used for color selection associated with the axes color limits.\\
\rz{\tt DefLen} & optional scalar giving arrow length in plot units.\\
\end{tabular}

\begin{SDT}
The {\tt MAP} data structure may be viewed using

{\tt fecom('ShowMap',MAP);fecom('ScaleOne'); }
\end{SDT}


%  - - - - - - - - - - - - - - - - - - - - - - - - - - - - - - - - - - -
\ruic{feutil}{Info}{[ ,Elt, Node\tsi{i}]}

{\tt feutil('Info',model);}
\noindent {\sl Information on model}.  
\ts{Info} by itself gives general information about {\tt model}. 
\ts{InfoNode}\tsi{i} gives information about all elements that are connected to node of NodeId \tsi{i}. 
 
%  - - - - - - - - - - - - - - - - - - - - - - - - - - - - - - - - - - -
\ruic{feutil}{Join}{[group \tsi{i}, \tsi{EltName}]}

\noindent {\sl Join the} {\sl groups} \tsi{i} or all the groups of type \tsi{EltName}. \ts{JoinAll} joins all the groups that have the same element name. Note that with the selection by group number, you can only join groups of the same type (with the same element name). \ts{JoinAll} joins all groups with identical element names.

You may join groups using there ID

%begindoc
\begin{verbatim}
% Joining groups of similar element types
femesh('Reset'); model=femesh('Test2bay');
% Join using group ID
feutil('Info',model);   % 2 groups at this step
model=feutil('JoinGroup1:2',model)  % 1 group now
feutil('Info',model);
% Join using element types
% Note you can give model (above) or element matrix (below)
femesh('Reset'); model=femesh('Test2bay'); 
model.Elt=feutil('Joinbeam1',model.Elt);  % 1 group now
\end{verbatim}%enddoc

% - - - - - - - - - - - - - - - - - - - - - - - -
\ruic{feutil}{Matid}{,\htr{feutil}{ProId},\htr{feutil}{MPID}}

{\tt MatId=feutil('MatId',model)} returns the element material identifier for each element in {\tt model.Elt}.\\
Command \ts{MatIdNew} provides a new model-wise unused material identifier. {\tt newId=feutil('MatIdNew',model);} \\
\begin{SDT}
One can also modify {\tt MatId} of the model giving a third argument.
{\tt model=feutil('MatId',model,r1)} {\tt r1} can be a global shift on all non zero {\tt MatId} or a matrix whose first column gives old {\tt MatId} and second new {\tt MatId} (this is not a vector for each element).

{\tt MatId} renumbering is applyed to elements, {\tt model.pl} and {\tt model.Stack 'mat'} entries. 
\end{SDT}
The \ts{ProId} command works similarly.\\

\ts{MPId} returns a matrix with three columns {\tt MatId}, {\tt ProId} and group numbers.\\
{\tt model.Elt=feutil('mpid',model,mpid)} can be used to set properties of elements in {\tt model.Elt} matrix.

% - - - - - - - - - - - - - - - - - - - - - - - -
\ruic{feutil}{Node}{[\htr{feutil}{trans},\htr{feutil}{rot},\htr{feutil}{mir},\htr{feutil}{DefShift}]} 

The command {\tt feutil('node [trans,rot,mir]',model,RO)} allows to move model nodes (or part of a model with a provided selection) with standard transformations :

\begin{itemize}
\item translation : \ts{trans} \tsi{x y z}
\item rotation : \ts{rot} \tsi{x1 x2 x3 n1 n2 n3 theta} with \tsi{xi} the coordinate of the node and \tsi{ni} the direction of the axe and \tsi{theta} the angle in degree 
\item plane symmetry : 
\begin{itemize}
\item plane x y or z : \ts{mir} \tsi{x}, \ts{mir} \tsi{y} or \ts{mir} \tsi{z}
\item point + normal : \ts{mir o} \tsi{x1 x2 x3 n1 n2 n3} with \tsi{xi} the coordinate of the node and \tsi{ni} the direction of the normal to the plane
\item plane equation : \ts{mir eq} \tsi{a b c d} defining the plane $aX+bY+cZ+d=0$
\item best plane defined by list of node coordinates : \texline {\tt feutil('node mir',model,struct('node',[\tsi{x1 y1 z1};\tsi{x2 y2 z2};...]))}
\item best plane defined by list of nodeids : \ts{mir "nodeid} \tsi{id1 id2 id3}\ts{"}
\end{itemize}
\item rigid body matrix : {\tt feutil('node',model,struct('rb',[4x4 RB matrix]))}
\end{itemize}

For each call, it is possible to either provide inputs as text string or as structure given on third argument with the field name corresponding to the wanted transformation. 

An node selection can be provided in the text command (\ts{sel"NodeElt"}) or as a text in a {.sel} field of the {\tt RO} stucture to apply the transformation on only a part of the model. See \lttts{FindNode}.

Here is an exhaustive list of examples

%begindoc
\begin{verbatim}
model=femesh('test tetra4'); % Load model wontaining a tetrahedron
model.Node=feutil('addnode',model.Node,[0-1 0 0]); % Add a node
model.Elt=feutil('addelt',model.Elt,'mass1',5); % Set this node as a mass1 element
feplot(model); % Display
% Displacement transformations
% translation in the direction [1 0 0] sepcified in the text command
model=feutil('node trans 1 0 0',model); feplot(model); 
% rotation of 180deg arround the axis defined by node [1 0 0] and vector [0 0 1]
RO=struct('rot',[1 0 0  0 0 1 180]); % rotation is the last number
% Only nodes in "group1" are moved
model=feutil('node -sel"group1"',model,RO); feplot(model); 
% Rigid body transformation (matrix in field rb) on nodes in group1
RO=struct('rb',[1 0 0 -1;0 1 0 0;0 0 1 0;0 0 0 1],'sel','group1');
model=feutil('node',model,RO); feplot(model); 
% mirror transformation
% Plane y=0
model=feutil('node mir y',model); feplot(model); 
% Same plane definined with node [0 0 0] and normal [0 1 0]
model=feutil('node mir o 0 0 0 0 1 0',model); feplot(model); 
% Same plane definined with nodeid 1 2 4
model=feutil('node mir nodeid 1 2 4',model); feplot(model);
% Same plane definined with equation 0*x+1*y+0*z+0=0, given as last
% argument in a structure
RO=struct('eq',[0 1 0 0]);
model=feutil('node mir',model,RO); feplot(model);
% Mirror with respect to the "best" plane passing through the node list 
RO=struct('node',[0 -0.1 0;1 0 0;0 0 1;1 0.3 1],'sel','group1');
model=feutil('node mir',model,RO); feplot(model); 
fecom('shownodemark',[0 -0.1 0;1 0 0;0 0 1;1 0.3 1]); % Show nodes defining the plane
\end{verbatim}
%enddoc

% xxx NodeDefShift

%  - - - - - - - - - - - - - - - - - - - - - - - - - - - - - - - - - - -
\ruic{feutil}{ObjectBeamLine}{ \tsi{i}, ObjectMass \tsi{i}}
{\tt elt=feutil('ObjectBeamLine \tsi{i}');}
\noindent {\sl Create a group of }\beam\ {\sl elements}.  The node numbers \tsi{i} define a series of nodes that form a continuous beam (for discontinuities use {\tt 0}), that is placed in {\tt elt} as a single group of \beam\ elements.

For example {\tt elt=feutil('ObjectBeamLine 1:3 0 4 5')} creates a group of three \beam\ elements between nodes {\tt 1 2}, {\tt 2 3}, and {\tt 4 5}.

An alternate call is {\tt elt=feutil('ObjectBeamLine',ind)} where {\tt ind} is a vector containing the node numbers. You can also specify a element name other than {\tt beam1} and properties to be placed in columns 3 and more using {\tt elt=feutil('ObjectBeamLine -\tsi{EltName}',ind,prop)}.

{\tt elt=feutil('ObjectMass 1:3')} creates a group of concentrated \mass\ elements at the declared nodes.

%begindoc
\begin{verbatim}
% Build a mesh by addition of defined beam lines and masses
model=struct('Node',[1 0 0 0  0  0 0;   2 0 0 0  0  0 .15; ... 
                     3 0 0 0 .4  1 .176;4 0 0 0 .4 .9 .176], 'Elt',[]);
prop=[100 100 1.1 0 0]; % MatId ProId nx ny nz
model.Elt=feutil('ObjectBeamLine 1 2 0 2 3 0 3 4',prop);
% or model.Elt=feutil('ObjectBeamLine',1:4);
model.Elt=feutil('ObjectMass',model,3,[1.1 1.1 1.1]);
%model.Elt(end+1:end+size(elt,1),1:size(elt,2))=elt;
feplot(model);fecom textnode
\end{verbatim}%enddoc


%  - - - - - - - - - - - - - - - - - - - - - - - - - - - - - - - - - - -
\ruic{feutil}{ObjectHoleInPlate}{}
{\tt model=feutil('ObjectHoleInPlate ...',model);}\\
\begin{tabular}{@{}L{.45\textwidth}@{}@{}L{.54\textwidth}@{}}%
\noindent \ingraph{40}{tr_hole} & \noindent {\sl Create a} \quada\ {\sl mesh of a hole in a plate.} The format is {\tt 'ObjectHoleInPlate \tsi{N0 N1 N2 r1 r2 ND1 ND2 NQ}'} giving the center node, two nodes to define the edge direction and distance, two radiuses in the direction of the two edge nodes (for elliptical holes), the number of divisions along a half quadrant of edge 1 and edge 2, the number of quadrants to fill (the figure shows 2.5 quadrants filled).\\
\end{tabular}

%begindoc
\begin{verbatim}
% Build a model of a plate with a hole
model=struct('Node',[1 0 0 0  0 0 0; 2 0 0 0  1 0 0; 3 0 0 0  0 2 0],'Elt',[]);
model=feutil('ObjectHoleInPlate 1 2 3 .5 .5 3 4 4',model);
model=feutil('Divide 3 4',model); % 3 divisions around, 4 divisions along radii
feplot(model)
% You could also use the call
model=struct('Node',[1 0 0 0  0 0 0; 2 0 0 0  1 0 0; 3 0 0 0  0 2 0],'Elt',[]);
%   n1 n2 n3 r1 r2 nd1 nd2 nq
r1=[ 1  2  3 .5 .5  3   4   4];
st=sprintf('ObjectHoleInPlate %f %f %f %f %f %f %f %f',r1);
model=feutil(st,model);
\end{verbatim}%enddoc


%  - - - - - - - - - - - - - - - - - - - - - - - - - - - - - - - - - - -
\ruic{feutil}{ObjectHoleInBlock}{}
{\tt model=feutil('ObjectHoleInBlock ...');}
\noindent {\sl Create a} \hexah\ {\sl mesh of a hole in a rectangular block.} The format is {\tt 'ObjectHoleInBlock \tsi{x0 y0 z0  nx1 ny1 nz1  nx3 ny3 nz3 dim1 dim2 dim3 r nd1 nd2 nd3 ndr}'} giving the center of the block (\tsi{x0 y0 z0}), the directions along the first and third dimensions of the block (\tsi{nx1 ny1 nz1  nx3 ny3 nz3}, third dimension is along the hole), the 3 dimensions (\tsi{dim1 dim2 dim3}), the radius of the cylinder hole (\tsi{r}), the number of divisions of each dimension of the cube (\tsi{nd1 nd2 nd3}, the 2 first should be even) and the number of divisions along the radius (\tsi{ndr}).

%begindoc
\begin{verbatim}
% Build a model of a cube with a cylindrical hole
model=feutil('ObjectHoleInBlock 0 0 0  1 0 0  0 1 1  2 3 3 .7  8 8 3 2') 
\end{verbatim}%enddoc


%  - - - - - - - - - - - - - - - - - - - - - - - - - - - - - - - - - - -
\ruic{feutil}{Object}{[Quad,Beam,Hexa] \tsi{MatId ProId}}

{\tt model=feutil('ObjectQuad \tsi{MatId ProId}',model,nodes,div1,div2)}
\noindent {\sl Create or add a model} containing {\tt quad4} {\sl elements}. The user must define a rectangular domain delimited by four nodes and the division in each direction ({\tt div1} and {\tt div2}). The result is a regular mesh. 

For example {\tt model=feutil('ObjectQuad 10 11',nodes,4,2)} returns model with 4 and 2 divisions in each direction with a {\tt MatId} 10 and a {\tt ProId} 11.

An alternate call is {\tt model=feutil('ObjectQuad 1 1',model,nodes,4,2)}: the quadrangular mesh is added to the model.

%begindoc
\begin{verbatim}
% Build a mesh based on the refinement of a single quad element
node = [0  0  0; 2  0  0; 2  3  0; 0  3  0];
model=feutil('Objectquad 1 1',node,4,3); % creates model 

node = [3  0  0; 5  0  0; 5  2  0; 3  2  0];
model=feutil('Objectquad 2 3',model,node,3,2); % matid=2, proid=3
feplot(model);
\end{verbatim}%enddoc


Divisions may be specified using a vector between {\tt [0,1]} :
%begindoc
\begin{verbatim}
% Build a mesh based on the custom refinement of a single quad element
node = [0  0  0; 2  0  0; 2  3  0; 0  3  0];
model=feutil('Objectquad 1 1',node,[0 .2 .6 1],linspace(0,1,10)); 
feplot(model);
\end{verbatim}%enddoc


Other supported object topologies are beams and hexahedrons with syntaxes

{\tt model=feutil('objectbeam',model,nodes,dvx,dvy);}

{\tt model=feutil('objecthexa',[Oxyz;OAxyz;OBxyz;OCxyz],divOA,divOB,divOC);}

For example
%begindoc
\begin{verbatim}
% Build a mesh based on the custom refinement of a single element
node = [0  0  0; 2  0  0;1  3  0; 1  3  1];
model=feutil('Objectbeam 3 10',node(1:2,:),4); % creates model 
model=feutil('Objecthexa 4 11',model,node,3,2,5); % creates model 
feutil('infoelt',model)
\end{verbatim}%enddoc


%  - - - - - - - - - - - - - - - - - - - - - - - - - - - - - - - - - - -
\ruic{feutil}{Object[Arc}{, Annulus, Circle, Cylinder, Disk]}

These object constructors follow the format

{\tt model=feutil('ObjectAnnulus x y z r1 r2 nx ny nz Nseg NsegR',model)}
with {\tt x y z} the coordinates of the center, {\tt nx ny nz} the coordinates of the normal to the plane containing the annulus, {\tt Nseg} the number of angular subdivisions, and {\tt NsegR} the number of segments along the radius. The resulting model is in \quada\ elements.

{\tt model=feutil('ObjectArc x y z x1 y1 z1 x2 y2 z2 Nseg obt',model)}
with {\tt x y z} the coordinates of the center, {\tt xi yi zi} the coordinates of the first and second points defining the arc boundaries, {\tt Nseg} the number of angular subdivisions, and {\tt obt} for obtuse, set to {\tt 1} to get the shortest arc between the two points or {\tt -1} to get the complementary arc. The resulting model is in \beam\ elements.

{\tt model=feutil('ObjectCircle xc yc zc r nx ny nz Nseg',model)}
with {\tt xc yc zc} the coordinates of the center, {\tt r} the radius, {\tt nx ny nz} the coordinates of the normal to the plane containing the circle, and {\tt Nseg} the number of angular subdivisions. The resulting model is in \beam\ elements.

{\tt model=feutil('ObjectCylinder x1 y1 z1 x2 y2 z2 r divT divZ',model)}
with {\tt xi yi zi} the coordinates of the centers of the cylinder base and top circles, {\tt r} the cylinder radius, {\tt divT} the number of angular subdivisions, and {\tt divZ} the number of subdivisions in the cylinder height. The resulting model is in \quada\ elements.

{\tt model=feutil('ObjectDisk x y z r nx ny nz Nseg NsegR',model)}
with {\tt x y z}, the coordinates of the center, {\tt r} the disk radius, {\tt nx ny nz} the coordinates of the normal to the plane containing the disk, {\tt Nseg} the number of angular subdivisions, and {\tt NsegR} the number of segments along the radius. The resulting model is in \quada\ elements. Command option \ts{-nodeg} avoids degenerate \quad\ elements by transforming them into \triaa\ elements.

For example:
%begindoc
\begin{verbatim}
% Build a mesh based on simple circular topologies
model=feutil('object arc 0 0 0 1 0 0 0 1 0 30 1');
model=feutil('object arc 0 0 0 1 0 0 0 1 0 30 1',model);
model=feutil('object circle 1 1 1 2 0 0 1 30',model);
model=feutil('object circle 1 1 3 2 0 0 1 30',model);
model=feutil('object cylinder 0 0 0  0 0 4 2 10 20',model);
model=feutil('object disk 0 0 0 3 0 0 1 10 3',model);
model=feutil('object disk -nodeg 1 0 0 3 0 0 1 10 3',model);
model=feutil('object annulus 0 0 0 2 3 0 0 1 10 3',model);
feplot(model)
\end{verbatim}%enddoc


%  - - - - - - - - - - - - - - - - - - - - - - - - - - - - - - - - - - -
\ruic{feutil}{ObjectDivide}{}

Applies a \lts{feutil}{Divide} command to a selection within the model. This is a packaged call to \lts{feutil}{RefineCell}, one thus has access to the following command options:
\begin{itemize}
\item \ts{-MPC} to generate {\tt MPC} constraints to enforce displacement continuity at non conforming interfaces
\item \ts{KnownNew} to add new nodes without check
\item \ts{-noSData} asks no to add model stack entry {\tt info,newcEGI} that provides the indices of new elements in model.
\end{itemize}

%begindoc
\begin{verbatim}
% Perform local mesh refinement
node = [0  0  0; 2  0  0; 2  3  0; 0  3  0];
model=feutil('Objectquad 1 1',node,4,3); % creates model 
model=feutil('ObjectDivide 3 2',model,'WithNode 1');
feplot(model);

% Perform a non uniform local mesh refinement with MPC
node = [0  0  0; 2  0  0; 2  3  0; 0  3  0];
model=feutil('Objectquad 1 1',node,4,3); % creates model 
model=feutil('ObjectDivide 3 2 -MPC',model,...
 'WithNode 1',[0 .2 1],[0 .25 .8 1]);
% display model and MPC constraint
feplot(model);
fecom(';promodelinit;proviewon;')
fecom('curtabCases','MPCedge');
\end{verbatim}%enddoc


%  - - - - - - - - - - - - - - - - - - - - - - - - - - - - - - - - - - -
\ruic{feutil}{Optim}{[Model, NodeNum, EltCheck]}
{\tt model.Node=feutil('Optim...',model);}\\
\noindent {\tt model.Node=feutil('OptimModel',model)} removes nodes unused in {\tt model.Elt} from {\tt model.Node}.\\
\begin{SDT}
This command is very partial, a thorough model optimization is obtained using \ltr{feutilb}{SubModel} with {\tt groupall} selection. {\tt model=feutilb('SubModel',model,'groupall');}. 
To recover used nodes the most complete command is \ltr{feutilb}{GetUsedNodes}.
\end{SDT}

{\tt model.Node=feutil('OptimNodeNum',model)} does a permutation of nodes in {\tt model.Node} such that the expected matrix bandwidth is smaller. This is only useful to export models, since here DOF renumbering is performed by \femk.\\
{\tt model=feutil('OptimEltCheck',model)} attempts to fix geometry pathologies (warped elements) in {\tt quad4}, {\tt hexa8} and {\tt penta6} elements.

{\tt model=feutil('OptimDegen',model)} detects degenerate elements and replaces them by the proper lower node number case {\tt hexa -> penta}. 


%  - - - - - - - - - - - - - - - - - - - - - - - - - - - - - - - - - - -
\ruic{feutil}{Orient}{, Orient \tsi{i} [ , n \tsi{nx ny nz}]}

{\sl Orient elements}.  For volumes and 2-D elements which have a defined orientation \texline {\tt model.Elt=feutil('Orient',model)} calls element functions with standard material properties to determine negative volume orientation and permute nodes if needed. This is in particular needed when generating models via \ts{Extrude} or \ts{Divide} operations which do not necessarily result in appropriate orientation (see \integrules). When elements are too distorted, you may have a locally negative volume. A warning about {\tt warped} volumes is then passed. You should then correct your mesh. 

Note that for 2D meshes you need to use 2D element names (\qfourp, {\tt t3p, ...}) rather than {\tt quad4, tria3, ...}. Typically  {\tt model.Elt=feutil('setgroup1 name q4p',model)}.

{\sl Orient normal of shell elements.} For plate/shell elements (elements with parents of type {\tt quad4}, {\tt quadb} or {\tt tria3}) in groups \tsi{i} of {\tt model.Elt},  {\tt model.Elt=feutil('Orient \tsi{i} n \tsi{nx ny nz}',model)} command computes the local normal and checks whether it is directed towards the node located at \tsi{nx ny nz}. If not, the element nodes are permuted to that a proper orientation is achieved. A \ts{-neg} option can be added at the end of the command to force orientation away rather than towards the nearest node.

{\tt model.Elt=feutil('Orient \tsi{i}',model,node)} can also be used to specify a list of orientation nodes. For each element, the closest node in {\tt node}  is then used for the orientation. {\tt node} can be a standard 7 column node matrix or just have 3 columns with global positions.

For example

%begindoc
\begin{verbatim}
% Specify element orientation
% Load example
femesh('Reset'); model=femesh('Testquad4'); 
model=feutil('Divide 2 3',model);
model.Elt=feutil('Dividegroup1 WithNode1',model); 
% Orient elements in group 2 away from [0 0 -1]
model.Elt=feutil('Orient 2 n 0 0 -1 -neg',model);
MAP=feutil('GetNormal MAP',model);MAP.normal
\end{verbatim}%enddoc

%  - - - - - - - - - - - - - - - - - - - - - - - - - - - - - - - - - - -
\ruic{feutil}{Quad2Lin}{, \htr{feutil}{Lin2Quad}, Quad2Tria, etc.}

\noindent {\sl Basic element type transformations.}


{\tt model=feutil('Lin2Quad epsl .01',model)} is the generic command to generate second order meshes.\\
\ts{Lin2QuadCyl} places the mid-nodes on cylindrical arcs.\\
\ts{Lin2QuadKnownNew} can be used to get much faster results if it is known that none of the new mid-edge nodes is coincident with an existing node.
\ts{Quad2Lin} performs the inverse operation.\\
\begin{SDT}
For this specific command many nodes become unecessary, command option \ts{-optim} performs a cleanup by removing these nodes from the \ltt{model}, and its {\tt Stack} and {\tt Case} entries.
\end{SDT}
%The inverse operation can be performed using \feutilb\ \ts{FirstOrder}.
\ts{Quad2Tria} searches elements for \quada\ element groups and replaces them with equivalent \triaa\ element groups.\\
\ts{Hexa2Tetra} replaces each \hexah\ element by 24 \tetra\ elements (this is really not a smart thing to do).\\
\ts{Hexa2Penta} replaces each \hexah\ element by 6 \tetra\ elements (warning : this transformation may lead to incompatibilities on the triangular faces).\\
\ts{Penta2Tetra} replaces each \penta\ element by 11 \tetra\ elements. 

Command option \ts{KnownNew} can be used for \ts{Hexa2Tetra}, \ts{Hexa2Penta}, and  \ts{Penta2Tetra}. Since these commands add nodes to the structure, quicker results can be obtained if it is known that none of the new nodes are coincident with existing ones. In a more general manner, this command option is useful if the initial model features coincident but free surfaces ({\it e.g.} two solids non connected by topology, when using coupling matrices). The default behavior will add only one node for both surfaces thus coupling them, while the \ts{KnownNew} alternative will add one for each.

%begindoc
\begin{verbatim}
% Transforming elements in a mesh, element type and order
% create 2x3 quad4 
femesh('Reset'); model=femesh('Testquad4'); 
model=feutil('Divide 2 3',model); 
model=feutil('Quad2Tria',model); % conversion
feplot(model)
% create a quad, transform to triangles, divide each triangle in 4
femesh('Reset'); model=femesh('Testquad4');
model=feutil('Quad2Tria',model);
model=feutil('Divide2',model);
cf=feplot(model); cf.model
% create a hexa8 and transform to hexa20
femesh('Reset'); model=femesh('Testhexa8');
model=feutil('Lin2Quad epsl .01',model);
feutil('InfoElt',model)
\end{verbatim}%enddoc


%  - - - - - - - - - - - - - - - - - - - - - - - - - - - - - - - - - - -
\ruic{feutil}{RefineCell}{, Beam \tsi{l}, ToQuad}

\begin{itemize}

\item The \ts{RefineCell} command is a generic element-wise mesh refinement command. Each element can be replaced by another mesh fitted in the initial topology. This is in particular used by \ts{RefineToQuad}.

For each element type, it is possible to define an interior mesh defined in the element reference configuration. \ts{RefineCell} then applies node and element additions in an optimized way to produce a final mesh in which all elements have been transformed.

A typical syntax is {\tt model=feutil('}\ts{RefineCell}{\tt ',model,R1)}, with {\tt model} a standard SDT model and {\tt R1} a running option structure providing in particular the cell refinement topologies.

In practice, cell refinement is defined for each element type in the reference configuration, giving additional nodes by edge, then face, then volume in increasing index. New nodes are computed using an operator performing weighted sums of initial cell coordinates. If no weights are given, arithmetic average is used.

Option structure {\tt R1} contains fields named as element types. These fields provide structures with fields
\begin{itemize}
\item {\tt edge} a cell array in the format {\tt \{[newId [oldId\_Av]], [weights]\}} providing the nodes to be added on the edges of the initial element. It is a 1 by 2 cell array. The first part is a matrix with as many lines as new nodes to be added, the first column {\tt newId} providing the new {\tt NodeId} of the reference configuration and the following ones {\tt oldId\_Av} the nodes of the initial cell used to generate the new coordinates. The second part is a weight matrix, with as many lines as new nodes and as many columns as {\tt oldId\_Av} providing the weights for each node. The {\tt weights} matrix can be left empty in which case equal weights will be used for each nodes. It can also be set a a scalar, and in this case the scalar coefficient will be used for each weight. {\tt newId} have to be given in increasing order.  This can be left blank if no node has to be added in edges.
\item {\tt face} a cell array in the same format than for field {\tt edge}, providing the nodes to be added on the edges of the initial element. {\tt newId} have to be given in increasing order and greater than the {\tt edge} new IDs. This can be left blank if no node has to be added in faces.
\item {\tt volume} a cell array in the same format than for field {\tt edge}, providing the nodes to be added in the volume of the initial element. {\tt newId} have to be given in increasing order and greater than the {\tt edge} new IDs and greater than the {\tt face} new IDs. This can be left blank if no node has to be added in the volume.
\item {\tt Elt} a cell array providing the elements defined in the reference configuration topology. This is a cell array in format {\tt \{ElemP, Elt\}}, {\tt ElemP} providing the new element types and {\tt Elt} an element matrix with no header providing the connectivies associated to {\tt ElemP}.
\item {\tt faces} For non symmetric transformations, it is possible to define a reference node ordering of the reference configuration that allows identifying a reference face of the reference configuration.
\item {\tt shift} For non symmetric transformations, {\tt shift} will identify the reference face in the {\tt faces} field to allow transformation for selected faces of elements.
\end{itemize}

A sample call to refine {\tt quad4} elements using \ts{RefineCell} is then

%begindoc
\begin{verbatim}
% refine cell sample call for iso quad refinement
model=femesh('testquad4'); % base quad element
% definition of the quad transformation
R1=struct('quad4',...
 struct('edge',{{[5 1 2;6 2 3;7 3 4;8 4 1],.5}},...
 'face',{{[9 1 2 3 4],.25}},...
 'Elt',{{'quad4',[1 5 9 8;5 2 6 9;9 6 3 7;9 7 4 8]}}));
mo1=feutil('refinecell',model,R1)
[eltid,mo1.Elt]=feutil('EltIdFix;',mo1);
% Visualization
cf=feplot(mo1); fecom('textnode')
\end{verbatim}%continuedoc

It is possible to restrain refinement to an element selection. This is realized by adding field {\tt set} to {\tt R1} containing a list of {\tt EltId} on which the refinement will be performed.

By default, the output model only contains the refined elements.

The following command options are available
\begin{itemize}
\item \ts{-Replace}\tsi{val} outputs the complete model on which selected elements have been refined. In this latter case, apparition of non conforming interfaces is possible. Set \tsi{val} to {\tt 2} to preserve properties stored in the model Stack.
\item \ts{-MPC} allows generating MPC constraints (on DOF 1,2,3) at non-conforming interfaces to enforce displacement continuity. Generated MPC are named {\tt MPCedge} and {\tt MPCface} respectively concerning nodes added on edges and faces.
\item \ts{-mpcALL} generates MPC entries relative to all new nodes (DOF 1,2,3). This allows field projection from the original mesh to the refined one. Generated MPC are named {\tt MPCedge}, {\tt MPCface} and {\tt MPCvolume} respectively concerning nodes added on edges, faces and volumes.
\item \ts{keepEP} preserves elements properties assignements based on the initial elements.
\item \ts{keepSets} preserves element sets by expanding all replaced elements by their refined versions in all sets.
\item \ts{AllElts} to force working on all element types whatever the input topologies. Missing ones will use the \ts{RefineToQuad} strategiy.
\item \ts{given} in combination with \ts{AllElts} not to use the \ts{RefineToQuad} strategiy on missing topologies.
\item \ts{KnownNew} new nodes are not merged.
\end{itemize}

%continuedoc
\begin{verbatim}
% local refine cell call with MPC generation
R1.set=[1]; % define an EltId set to refine
% call for MPC for new interface edges
mo1=feutil('refinecell-replace-mpc',mo1,R1);
% display refined model and MPC
cf=feplot(mo1);
fecom(cf,';promodelinit;proviewon;curtabCase;','MPCedge');
\end{verbatim}%enddoc

Command option \ts{KnownNew} adds new nodes without merging overlaying ones.

Command option \ts{-keepEP} asks to keep the element type (instead of the parent one) possible only if the refined cell features element sharing the same parent type than the initial element.

Command option \ts{-keepSets} asks to keep {\tt EltId} sets coherence by replacing original element IDs in the sets with the refined cell ones.

\vs

Non symmetric cell refinement requires the ability to detect the element orientation regarding the reference cell orientation. The strategy implemented is based on element face (for volume) or edge (for shells) identification, through the definition in the input structure of a field {\tt faces} providing the face indices of the reference model and a {\tt shift} index providing a reference face used in the reference cell. One can provide as many reference faces as necessary to uniquely define the reference cell orientation.

In this case, each element to be refined must be assigned a face (or edge) list selection for orientation purpose, with as many faces as specified in the {\tt .shift} field. The field {\tt set} in input structure {\tt R1} is then mandatory with as many additional columns as the number of reference cell, the first one providing the selected element IDs and the following ones the face (or edge) identifier corresponding (including order) to the reference cell orientation face.  See \ltr{feutil}{AddSetFaceId}, and \ltt{FindElt} commands to generate such element selection.

The following example provides a non-symmetric cell refinement of a side of a structure allowing an increase of node one side while keeping a continuous mesh.

%begindoc
\begin{verbatim}
% unsymmetric refine cell call
model=femesh('testquad4'); % base model
model=feutil('refineToQuad',model); % refine into 4 quad4
% fix eltid for clean element selection
[eltid,model.Elt]=feutil('EltIdFix;',model);
% define a non symmetric cell refinement
% here refinement is based on edge 1 2 using reference faces
R1=struct('quad4',...
 struct('edge',{{[5 1 2;6 1 2],...
 [2/3 1/3;1/3 2/3]}},...
 'face',{{[7 1:4;8 1:4],...
[1/6 1/3 1/3 1/6;1/3 1/6 1/6 1/3]}},...
'Elt',{{'quad4',[1 5 8 4;5 6 7 8;6 2 3 7;8 7 3 4]}},...
'faces',quad4('edge'),'shift',1));

% define a selection of edges to refine
elt=feutil('selelt seledge & innode{x==0}',model); 
% here easy recovery on elements for edge selection
% based on shell element
R1.set=elt(2:end,5:6);
% call refinement
mo1=feutil('refinecell-replace',model,R1)
cf=feplot(mo1); fecom('textnode')
\end{verbatim}%enddoc


\item The \ts{RefineBeam} command searches {\tt model.Elt} for beam elements and divides elements so that no element is longer than \tsi{l}. For \beam\ elements, transfer of pin flags properties are forwarded by keeping non null flags on the new beam elements for which a pre-existing node was flagged.

%begindoc
\begin{verbatim}
% Specific mesh refinement for beam
femesh('Reset'); model=femesh('Testbeam1'); % create a beam
model=feutil('RefineBeam 0.1',model);
\end{verbatim}%enddoc
One can give a model sub-selection ({\tt FindElt} command string) as 2nd argument, to refine only a part of the model beams.


\item The \ts{RefineBeamUni}\tsi{val} command uniformly refines all \beam elements into \tsi{val} elements. This command packages a \ltr{feutil}{ObjectDivide} call with command options \ts{KnownNew} and \ts{-noSData}. 
\begin{itemize}
\item Command option \ts{-pin} allows proper pin flag forwarding for \beam\ elements. transfer of pin flags properties are forwarded by keeping non null flags on the new beam elements for which a pre-existing node was flagged. This constitutes the main interest of the command.
\item Command option \ts{-MergeNew} asks to merge new nodes instead of simply adding them.
\end{itemize}

\item The  \ts{RefineToQuad} command transforms first order triangles, quadrangles, penta, tetra, and hexa to quad and hexa only while dividing each element each in two. The result is a conform mesh, be aware however that nodes can be added to your model boundaries. Using such command on model sub-parts will thus generate non conforming interfaces between the refined and non-refined parts.

By default, new nodes are added with an \ts{AddNode} command so matched new nodes are merged. Command option \ts{KnownNew} allows a direct addition of new nodes without checking.

%begindoc
\begin{verbatim}
% Refining mesh and transforming to quadrangle elements
model=femesh('testtetra4');model=feutil('RefineToQuad',model);
feplot(model);
\end{verbatim}%enddoc

\end{itemize}

% - - - - - - - - - - - - - - - - - - - - - - - - - - - - - - - - - - -
\ruic{feutil}{RefineLine}{\tsi{lc}}

The \ts{RefineLine} generates line uniform refinements in the provided line segments, so that gaps between two points along the provided line are not higher than a given characteristic length. This is useful for mesh seeding.
Command option \ts{-tolMerge}\tsi{val} allows merging points with gaps under the given tolerance, this can occur when providing merged series of points.

\begin{verbatim}
% Generate a line in 0-100 with fixed intermediate position at 32, with a maximum setp of 12.5
r1=feutil('RefineLine 12.5',[0 32 100])
% now with two given positions
r1=feutil('RefineLine 12.5',[0 32 33 100])
% merge points with gaps under2
r1=feutil('RefineLine 12.5 -tolMerge2',[0 32 33 100])
\end{verbatim}

%  - - - - - - - - - - - - - - - - - - - - - - - - - - - - - - - - - - -
\ruic{feutil}{RemoveElt}{ \tsi{ElementSelectors}}

{\tt [model.Elt,RemovedElt]=feutil('}\ts{RemoveElt} \tsi{ElementSelectors}{\tt',model);}

\noindent {\sl Element removal.} This function searches {\tt model.Elt} for elements which verify certain properties selected by \tsi{ElementSelectors} as a \lttts{FindElt} string, and removes these elements from the model description matrix. 2nd output argument {\tt RemovedElt} is optional and contains removed elements.
A sample call would be

%begindoc
\begin{verbatim}
% Removing elements in a model
% create 3x2 quad4 
femesh('Reset'); model=femesh('Testquad4');model=feutil('Divide 2 3',model); 
[model.Elt,RemovedElt]=feutil('RemoveElt WithNode 1',model);
feplot(model)
\end{verbatim}%enddoc

%  - - - - - - - - - - - - - - - - - - - - - - - - - - - - - - - - - - -
\ruic{feutil}{Remove [Pro, Mat]}{ \tsi{MatId, ProId}}

\noindent {\sl Mat, Pro removal} This function takes in argument the ID of a material or integration property and removes the corresponding entries in the model {\tt pl/il} fields and in the stack {\tt mat/pro} entries.

\begin{itemize}
\item Command option \ts{-all} removes all {\tt pl/il} entries found in the model and its stack.
\item Command option \ts{-unused} removes all {\tt pl/il} entries not used by any element.
\end{itemize}

\begin{SDT}
This call supports the {\tt info, Rayleigh} stack entry (see \swref{damp}), so that the data entries referring to removed IDs will also be removed.
By default, the non-linear properties are treated like normal properties. Care must thus be taken if a non-linear property that is not linked to specific elements is used. Command option~\ts{-unused} will alter this behavior and keep non-linear properties.
\end{SDT}

Sample calls are provided in the following to illustrate the use.
%begindoc
\begin{verbatim}
% Removing material and integration properties in a model
model=femesh('testhexa8');
model=stack_set(model,'pro','integ',p_solid('default'));
model=stack_set(model,'mat','steel',m_elastic('default steel'));
model=feutil('remove pro 110',model);
model=feutil('remove pro',model,111);
model=feutil('remove mat 100',model);
model=feutil('remove mat 100 pro 1',model);
model=feutil('remove pro -all',model); % Command option -all
model=feutil('remove mat pro -all',model);
model=femesh('testhexa8'); % Command option -unused
model=feutil('remove mat pro -unused',model);
\end{verbatim}%enddoc

%  - - - - - - - - - - - - - - - - - - - - - - - - - - - - - - - - - - -
\ruic{feutil}{Renumber}{}

{\tt model=feutil('}\ts{Renumber}{\tt',model,}\tsi{NewNodeNumbers}{\tt )} can be used to change the node numbers in the model. Currently nodes, elements, DOFs and deformations, nodeset, par, cyclic and other Case entries are renumbered.

\tsi{NewNodeNumbers} is the total new NodeIds vector. \tsi{NewNodeNumbers} can also be a scalar and then defines a global NodeId shifting. If \tsi{NewNodeNumbers} has two columns, first giving old NodeIds and second new NodeIds, a selective node renumbering is performed.

If \tsi{NewNodeNumbers} is not provided values {\tt 1:size(model.Node,1)} are used.  This command can be used to meet the OpenFEM requirement that node numbers be less than {\tt 2\verb+^+31/100}. Another application is to joint disjoint models with coincident nodes using

Command option \ts{-NoOri} asks not to add the {\tt info,OrigNumbering} data in the model stack. {\tt info,OrigNumbering} is only useful when the user needs to convert something specific linked to the new node numerotation that is outside model.

\begin{verbatim}
% Finding duplicate nodes and merging them
[r1,i2]=feutil('AddNode',model.Node,model.Node);
model=feutil('Renumber',model,r1(i2,1));
\end{verbatim}

Renumbering can also be applied to deformation curves, using the same syntax. Be aware however that to keep coherence between a deformation curve and a renumbered model, one should input \tsi{NewNodeNumbers} as the renumbered model stack entry {\tt info,OrigNumbering}.

%begindoc
\begin{verbatim}
% Renumering the nodes of a model, and its data
% simple model
model=femesh('testhexa8b');
% simple curve
def=fe_eig(model,[5 5 1e3]);
% first renumber model
model=feutil('renumber',model,1e4);
% then renumber def with renumbering info
r1=stack_get(model,'info','OrigNumbering','get');
def=feutil('renumber',def,r1);
\end{verbatim}%enddoc

%  - - - - - - - - - - - - - - - - - - - - - - - - - - - - - - - - - - -
\ruic{feutil}{RepeatSel}{ \tsi{nITE tx ty tz}}

\noindent {\sl Element group translation/duplication.} \ts{RepeatSel} repeats the elements of input {\tt model} \tsi{nITE} times with global axis translations \tsi{tx ty tz} between each repetition of the group. If needed, new nodes are added to {\tt model.Node}. An example is treated in the {\tt d\_truss} demo. 

%begindoc
\begin{verbatim}
% Build a mesh by replicating and moving sub-parts
femesh('Reset'); model=femesh('Testquad4');
model=feutil('Divide 2 3',model); 
model=feutil('RepeatSel 3 2 0 0',model); % 3 repetitions, tx=2
feplot(model)
% an alternate call would be
%                                            number, direction
% model=feutil(sprintf('Repeatsel %f %f %f %f', 3,  [2 0 0]))
\end{verbatim}%enddoc

%  - - - - - - - - - - - - - - - - - - - - - - - - - - - - - - - - - - -
\ruic{feutil}{Rev}{ \tsi{nDiv OrigID Ang nx ny nz}}

\noindent {\sl Revolution.} The elements of {\tt model} are taken to be the first meridian. Other meridians are created by rotating around an axis passing trough the node of number \tsi{OrigID} (or the origin of the global coordinate system) and of direction {\tt [}\tsi{nx ny nz}{\tt ]} (the default is the {\tt z} axis {\tt [0 0 1]}). \tsi{nDiv}+1 (for closed circle cases {\tt ang=360}, the first and last are the same) meridians are distributed on a sector of angular width \tsi{Ang} (in degrees). Meridians are linked by elements in a fashion similar to extrusion. Elements with a \mass\ parent are extruded into beams, element with a \beam\ parent are extruded into \quada\ elements, \quada\ are extruded into \hexah, and \quadb\ are extruded into \hexav.

The origin can also be specified by the {\sl x y z }values preceded by an \ts{o} using a command like {\tt model=feutil('Rev 10 o 1.0 0.0 0.0 \ \ \ 360 1 0 0')}.

You can obtain an uneven distribution of angles using a second argument. For example \texline {\tt model=feutil('Rev 0 101 40 0 0 1',model,[0 .25 .5 1])} will rotate around an axis passing by node {\tt 101} in direction {\ti z} and place meridians at angles 0 10 20 and 40 degrees. 

%begindoc
\begin{verbatim}
% Build a mesh by revolving a sub-part
model=struct('Node',[1 0 0 0  .2 0   0; 2 0 0 0  .5 1 0; ...  
                     3 0 0 0  .5 1.5 0; 4 0 0 0  .3 2 0],'Elt',[]);
model.Elt=feutil('ObjectBeamLine',1:4);
model=feutil('Divide 3',model);
model=feutil('Rev 40 o 0 0 0 360 0 1 0',model);
feplot(model)
fecom(';triax;view 3;showpatch')
% An alternate calling format would be
%      divi origin angle direct
% r1 = [40  0 0 0  360   0 1 0];
% model=feutil(sprintf('Rev %f o %f %f %f %f %f %f %f',r1))
\end{verbatim}%enddoc



%  - - - - - - - - - - - - - - - - - - - - - - - - - - - - - - - - - - -
\ruic{feutil}{RotateNode}{ \tsi{OrigID Ang nx ny nz}}

\noindent {\sl Rotation.} The nodes of {\tt model} are rotated by the angle \tsi{Ang} (degrees) around an axis passing trough the node of number \tsi{OrigID} (or the origin of the global coordinate system) and of direction {\tt [}\tsi{nx ny nz}{\tt ]} (the default is the {\tt z} axis {\tt [0 0 1]}). The origin can also be specified by the {\sl x y z} values preceded by an \ts{o}
{\tt model=feutil('RotateNode o 2.0 2.0 2.0 \ \ \  90  1 0 0',model)}
One can define as a second argument a list of NodeId or a FindNode string command to apply rotation on a selected set of nodes.
{\tt model=feutil('RotateNode o 2.0 2.0 2.0 \ \ \  90  1 0 0',model,'x==1')}

For example:

%begindoc
\begin{verbatim}
% Rotating somes nodes in a model
femesh('reset'); model=femesh('Testquad4'); model=feutil('Divide 2 3',model); 
% center is node 1, angle 30, aound axis z
%                                     Center angle  dir
st=sprintf('RotateNode %f %f %f %f %f',[1      30   0 0 1]);
model=feutil(st,model);  
feplot(model); fecom(';triax;textnode'); axis on
\end{verbatim}%enddoc

Similar operations can be realized using command \basis \ts{gnode}.


%  - - - - - - - - - - - - - - - - - - - - - - - - - - - - - - - - - - -
\ruic{feutil}{SelElt}{ \tsi{ElementSelectors}}

{\tt elt=feutil('SelElt  \tsi{ElementSelectors}',model)}

\noindent {\sl Element selection}. \ts{SelElt} extract selected element from {\tt model} that verify certain conditions. Available element selection commands are described under the \ts{FindElt} command and~\ser{findelt}. 

%  - - - - - - - - - - - - - - - - - - - - - - - - - - - - - - - - - - -
\ruic{feutil}{SetSel}{[Mat {\tsi j}, Pro {\tsi k}]}

\noindent {\sl Set properties of an element selection.} For a set of elements selected using a \lttts{FindElt} string command, you can modify the material property identifier \tsi{j} and/or the element property identifier \tsi{k}. 
For example

\begin{verbatim}
 % Assigning element properties to an element selection
 model=femesh('Testubeam')
 % Set MatId 10 and ProId 10 to all elements with z>1
 model.Elt=feutil('SetSel Mat10 Pro10',model,'withnode{z>1}');
 cf=feplot(model); 
 fecom(cf,'colordatamat'); % show matid with different colors
\end{verbatim}

%  - - - - - - - - - - - - - - - - - - - - - - - - - - - - - - - - - - -
\ruic{feutil}{SetGroup}{[{\tsi i},{\tsi name}] [Mat {\tsi j}, Pro {\tsi k}, EGID {\tsi e}, Name {\tsi s}]}

\noindent {\sl Set properties of a group.} For group(s) selected by number \tsi{i}, name \tsi{name}, or \ts{all} you can modify the material property identifier \tsi{j}, the element property identifier \tsi{k} of all elements and/or the element group identifier \tsi{e} or name \tsi{s}. For example

\begin{verbatim}
 % Assigning element properties by groups
 model.Elt=feutil('SetGroup1:3 Pro 4',model);
 model.Elt=feutil('SetGroup rigid Name celas',model) 
\end{verbatim}

If you know the column of a set of element rows that you want to modify, calls of the form {\tt model.Elt(feutil('FindElt \tsi{Selectors}',model),\tsi{ Column})= \tsi{ Value}} can also be used. 

%begindoc
\begin{verbatim}
 % Low level assignment of element properties
 femesh('Reset'); model=femesh('Testubeamplot');
 model.Elt(feutil('FindElt WithNode{x==-.5}',model),9)=2;
 cf=feplot(model); 
 cf.sel={'groupall','colordatamat'};
\end{verbatim}%enddoc

See \lts{feutil}{MPID} for higher level custom element properties assignments.

%  - - - - - - - - - - - - - - - - - - - - - - - - - - - - - - - - - - -
\ruic{feutil}{SetPro}{,\htr{feutil}{SetMat},\htr{feutil}{GetPro},\htr{feutil}{GetMat}}

\noindent {\sl Set an integration property data ({\tt ProId}) or material property ({\tt MatId}) to the model (enrich the list of matid and proid).} You can modify an {\tt il} or {\tt pl} property of ID \tsi{i} by giving its name and its value using an integrated call of the type

%begindoc
\begin{verbatim}
% Specifying material/integration rule parameters in a model
 model=femesh('testhexa8');model.il
 model=feutil('SetPro 111 IN=2',model,'MAP',struct('dir',1,'DOF',.01),'NLdata',struct('type','nl_inout'));
 feutilb('_writeil',model)
 % Now edit specific NLdata fields
 model=feutil('SetPro 111',model,'NLdataEdit',struct('Fu','Edited'));model.Stack{end}.NLdata
 mat=feutil('GetPl 100 -struct1',model) % Get Mat 100 as struct
\end{verbatim}%enddoc

The names related to the integration properties a documented in the {\tt p\_functions}, \psolid, \pshell, \pbeam, ... To get a type use calls of the form {\tt p\_pbeam('PropertyUnitTypeCell',1)}. 

The command can also be used to define additional property information : \ltt{pro.MAP} for field at nodes (\ltt{InfoAtNode}), \ltt{gstate} for field at integration points and \ltt{NLdata} for non linear behavior data (\nlspring). 

The \ts{GetPro} and \ts{GetMat} commands are the pending commands. For example:
%begindoc
\begin{verbatim}
 model=femesh('testhexa8');model.il
 rho=feutil('GetMat 100 rho',model) % get volumic mass
 integ=feutil('GetPro 111 IN',model) % get the integ rule
\end{verbatim}%enddoc
  
{\bf To assign {\tt proid} and {\tt matid} defined in the model to specific elements, see \ltr{feutil}{SetSel} and \lts{feutil}{SetGroup}}.
  
%  - - - - - - - - - - - - - - - - - - - - - - - - - - - - - - - - - - -
\ruic{feutil}{GetIl}{,\htr{feutil}{GetPl}}
The commands \lts{feutil}{GetIl} and \lts{feutil}{GetPl} respectively output the {\tt il} and {\tt pl} matrices of the model for the IDs used by elements. This command provides the values used during assembling procedures and aggregates the values stores in the {\tt model.il}, {\tt model.pl} fields and {\tt pro}, {\tt mat} entries in the model stack.


\ruic{feutil}{StringDOF}{} % - - - - - - - - - - - - - - - - - - - - - - - - - - - 

{\tt feutil('stringdof',sdof)} returns a cell array with cells containing string descriptions of the DOFs in {\tt sdof}.

%  - - - - - - - - - - - - - - - - - - - - - - - - - - - - - - - - - - -
\ruic{feutil}{SymSel}{ \tsi{ OrigID nx ny nz}}

\noindent {\sl Plane symmetry.  }\ts{SymSel} replaces elements in {\tt FEel0} by elements symmetric with respect to a plane going through the node of number \tsi{OrigID} (node {\tt 0} is taken to be the origin of the global coordinate system) and normal to the vector {\tt [}\tsi{nx ny nz}{\tt ]}. If needed, new nodes are added to {\tt FEnode}.  
Related commands are \lts{feutil}{TransSel}, \lts{feutil}{RotateSel} and \lts{feutil}{RepeatSel}.

%  - - - - - - - - - - - - - - - - - - - - - - - - - - - - - - - - - - -
\ruic{feutil}{Trace2Elt}{}
{\tt elt=feutil('Trace2Elt',ldraw);}\\
Convert the {\tt ldraw} trace line matrix (see \ltr{ufread}{82} for format details) to element matrix with {\tt beam1} elements. For example:

%begindoc
\begin{verbatim}
% Build a beam model from a trace line matrix
TEST.Node=[1001 0 0 0 0 0 0    ; 1003 0 0 0 0.2 0 0 ;
           1007 0 0 0 0.6 0 0  ; 1009 0 0 0 0.8 0 0 ;
           1015 0 0 0 0 0.2 0  ; 1016 0 0 0 0.2 0.2 0;
           1018 0 0 0 0.6 0.2 0; 1019 0 0 0 0.8 0.2 0]; 
L=[1001 1003 1007 1009];  
ldraw(1,[1 82+[1:length(L)]])=[length(L) L]; 
L=[1015 1016 1018 1019];
ldraw(2,[1 82+[1:length(L)]])=[length(L) L]; 
L=[1015 1001 0 1016 1003 0 1018 1007 0 1019 1009 0];
ldraw(3,[1 82+[1:length(L)]])=[length(L) L];
TEST.Elt=feutil('Trace2Elt',ldraw);
cf=feplot(TEST)
\end{verbatim}%enddoc

%  - - - - - - - - - - - - - - - - - - - - - - - - - - - - - - - - - - -
\ruic{feutil}{TransSel}{ \tsi{ tx ty tz}}

\noindent {\sl Translation of the selected element groups}.  \ts{TransSel} replaces elements by their translation of a vector {\tt [}\tsi{tx ty tz}{\tt ]} (in global coordinates).  If needed, new nodes are added.  Related commands are \lts{feutil}{SymSel}, \lts{feutil}{RotateSel} and \lts{feutil}{RepeatSel}.

%begindoc
\begin{verbatim}
% Translate and transform a mesh part
femesh('Reset'); model=femesh('Testquad4'); model=feutil('Divide 2 3',model); 
model=feutil('TransSel 3 1 0',model); % Translation of [3 1 0]
feplot(model); fecom(';triax;textnode')
\end{verbatim}%enddoc

Please, note that this command is usefull to translate only part of a model. If the full model must be translated, use \basis command \ts{gnode}. An example is given below.

%begindoc
\begin{verbatim}
% Translate all nodes of a model
femesh('Reset'); model=femesh('Testquad4'); model=feutil('Divide 2 3',model); 
model.Node=basis('gnode','tx=3;ty=1;tz=0;',model.Node);
feplot(model); fecom(';triax;textnode')
\end{verbatim}%enddoc


%  - - - - - - - - - - - - - - - - - - - - - - - - - - - - - - - - - - -
\ruic{feutil}{UnJoin}{ \tsi{Gp1 Gp2}}

{\sl Duplicate nodes which are common to two element ensembles.} To allow the creation of interfaces with partial coupling of nodal degrees of freedom, \ts{UnJoin} determines which nodes are common to the specified element ensembles. 

The command duplicates the common nodes between the specified element ensembles, and changes the node numbers of the second element ensemble to correspond to the duplicate set of nodes. The optional second output argument provides a two column matrix that gives the correspondence between the initial nodes and the duplicate ones. This matrix is coherent with the {\tt OrigNumbering} matrix format.

The following syntaxes are accepted
\begin{itemize}
\item {\tt [model,interNodes]=feutil('unjoin} \tsi{Gp1 Gp2}{\tt ',model);} Implicit group separation, \tsi{Gp1} (resp. \tsi{Gp2}) is the group identifier (as integer) of the first (resp. second) element groups to unjoin.
\item {\tt [model,interNodes]=feutil('unjoin',model,}\tsi{EltSel1},\tsi{EltSel2}{\tt );} Separation of two element selections. \tsi{EltSel1} (resp. {\tt EltSel2}) are either \ltt{FindElt} strings or {\tt EltId} vectors providing the element selections corresponding to each ensemble.
\item {\tt [model,interNodes=feutil('unjoin',model,RA);} general input with {\tt RA} as a structure. {\tt RA} has fields
\begin{itemize}
\item {\tt .type}, either \tsi{group}, \tsi{eltid} or \tsi{eltind} that provides the type of data for the selections, set to {\tt eltid} if omitted.
\item {\tt .sel1}, definition of the first element ensemble, the {\tt GroupId} for type group, either a \ltt{FindElt} string or a vector of {\tt EltId} or {\tt EltInd} depending on field {\tt .type}.
\item {\tt .sel2},  definition of the second element ensemble, same format as field {\tt .sel1}.
\item {\tt .NodeSel}, provides a \ltt{FindNode} selection command to restrict the second element ensemble. Optional, set to {\tt groupall} by default
\end{itemize}
\end{itemize}


%begindoc
\begin{verbatim}
 % Generate a disjointed interface between to parts in a model
femesh('Reset'); model=femesh('Test2bay');
feutil('FindNode group1 & group2',model) % nodes 3 4 are common

% Implicit call for group
mo1=feutil('UnJoin 1 2',model);
feutil('FindNode group1 & group2',mo1) % no common nodes in unjoined model
 
% Variant by specifying selections
mo1=feutil('UnJoin',model,'group 1','group 2');
feutil('FindNode group1 & group2',mo1) % no common nodes in unjoined model

% Variant with structure input, type "group"
RA=struct('type','group','sel1',1,'sel2',2);
mo1=feutil('UnJoin',model,RA);
feutil('FindNode group1 & group2',mo1) % no common nodes in unjoined model

% Variant with structure input, type "eltid" and string selections
RA=struct('type','eltid','sel1','group1','sel2','group 2');
mo1=feutil('UnJoin',model,RA);
feutil('FindNode group1 & group2',mo1) % no common nodes in unjoined model

% Advanced variants with structure and with selections as vectors
% Clean model EltId
[eltid,model.Elt]=feutil('eltidfix;',model);
i1=feutil('findelt group1',model);
i2=feutil('findelt group2',model);

% type "eltid"
RA=struct('type','eltid','sel1',eltid(i1),'sel2',eltid(i2));
mo1=feutil('UnJoin',model,RA);
feutil('FindNode group1 & group2',mo1) % no common nodes in unjoined model

% type "eltind"
RA=struct('type','eltind','sel1',i1,'sel2',i2);
mo1=feutil('UnJoin',model,RA);
feutil('FindNode group1 & group2',mo1) % no common nodes in unjoined model
\end{verbatim}%enddoc


%  - - - - - - - - - - - - - - - - - - - - - - - - - - - - - - - - - - -
\rmain{See also}

\noindent \khref{feutila}{feutila}, \femk, \fecom, \feplot, \ser{fem}, demos {\tt gartfe},  {\tt d\_ubeam}, {\tt beambar} ... 
%-----------------------------------------------------------------------------
\rtop{feutila}{feutila}

Advanced \feutil\ commands.

%  - - - - - - - - - - - - - - - - - - - - - - - - - - - - - - - - - - -
\ruic{feutil}{RotateSel}{ \tsi{ OrigID Ang nx ny nz}}

\noindent {\sl Rotation.} The elements of {\tt model} are rotated by the angle \tsi{Ang} (degrees) around an axis passing trough the node of number \tsi{OrigID} (or the origin of the global coordinate system) and of direction {\tt [}\tsi{nx ny nz}{\tt ]} (the default is the {\tt z} axis {\tt [0 0 1]}). The origin can also be specified by the {\sl x y z} values preceded by an \ts{o}

{\tt model=feutil('RotateSel o 2.0 2.0 2.0 \ \ \  90  1 0 0',model)}

Note that old nodes are kept during this process. If one simply want to rotate model nodes, see \lts{feutil}{RotateNode}. 

For example:

%begindoc
\begin{verbatim}
% Rotate and transform part of a mesh
femesh('reset'); model=femesh('Testquad4'); 
model=feutil('Divide 2 3',model); 
% center is node 1, angle 30, aound axis z
%                                     Center angle  dir
st=sprintf('RotateSel %f %f %f %f %f',[1      30   0 0 1]);
model=feutil(st,model);  
feplot(model); fecom(';triax;textnode'); axis on
\end{verbatim}%enddoc














%       Copyright (c) 2001-2024 by INRIA and SDTools, All Rights Reserved.
%       Use under OpenFEM trademark.html license and LGPL.txt library license
%       $Revision: 1.132 $  $Date: 2024/09/13 06:28:10 $

%----------------------------------------------------------------------------
\rtop{fe\_c}{fe_c}

\noindent DOF selection and input/output shape matrix construction.\index{degree of freedom (DOF)!definition vector}\index{degree of freedom (DOF)!active}\index{degree of freedom (DOF)!selection}\index{input shape matrix b}\index{output shape matrix c}\index{b}\index{c}

\rsyntax\begin{verbatim}
c            = fe_c(mdof,adof)
c            = fe_c(mdof,adof,cr,ty)
b            = fe_c(mdof,adof,cr)'
[adof,ind,c] = fe_c(mdof,adof,cr,ty)
ind          = fe_c(mdof,adof,'ind',ty)
adof         = fe_c(mdof,adof,'dof',ty)
labels       = fe_c(mdof,adof,'dofs',ty)
\end{verbatim}\nlvs

\rmain{Description}

\noindent This function is quite central to the flexibility of DOF numbering in the \toolbox. FE model matrices are associated to {\sl DOF definition vectors} which allow arbitrary DOF numbering (see \ser{mdof}).  \fec\ provides simplified ways to extract the indices of particular DOFs (see also \ser{adof}) and to construct input/output matrices. The input arguments for \fec\ are

\vs\noindent\begin{tabular}{@{}p{.15\textwidth}@{}p{.85\textwidth}@{}}
%
 \rz{\tt mdof} & {\sl DOF definition vector} for the matrices of interest (be
              careful not to mix DOF definition vectors of different models) \\
         \rz{\tt adof} & {\sl active DOF definition vector}. \\
 \rz{\tt cr} & {\sl output matrix associated to the active DOFs}. The default for
           this argument is the identity matrix.
           {\tt cr} can be replaced by a string {\tt 'ind'} or {\tt 'dof'}
           specifying the unique output argument desired then. \\

\rz{\tt ty} & {\sl active/fixed option} tells \fec\   whether the DOFs in
           {\tt adof} should be kept ({\tt ty=1} which is the default) or
           on the contrary deleted ({\tt ty=2}). \\
\end{tabular}

\vs The input {\tt adof} can be a standard DOF definition vector but can also contain wild cards as follows

\lvs\noindent
{\tt NodeID.0\ \ \ \ \ \ }  means all the DOFs associated to node {\tt NodeID} \\
{\tt \hbox{\ \ \ \ \ }0.DofID\ \ } means {\tt DofID} for all nodes having such a DOF\\
{\tt -EltID.0\ \ \ \ \ \ } means all the DOFs associated to element {\tt  EltID}

\vs The convention that DOFs {\tt .07} to {\tt .12} are the opposite of DOFs {\tt .01} to {\tt .06} is supported by \fec, but this should really only be used for combining experimental and analytical results where some sensors have been positioned in the negative directions.

The output argument {\tt adof} is the actual list of DOFs selected with the input argument.  \fec\ seeks to preserve the order of DOFs specified in the input {\tt adof}.  In particular for models with nodal DOFs only and

\begin{Eitem}
   \item {\tt adof} contains no wild cards: no reordering is performed.
   \item {\tt adof} contains node numbers: the expanded {\tt adof} shows all DOFs of the different nodes in the order given by the wild cards.
\end{Eitem}

The first use of \fec\ is the {\bf extraction} of particular DOFs from a DOF definition vector (see \hyperlink{adof}{{\tt b,c}} \forlatex{ page \pageref{s*adof}}).
One may for example want to restrict a model to 2-D motion in the $xy$ plane (impose a fixed boundary condition\index{boundary condition}). This is achieved as follows

\begin{verbatim}
 % finding DOF indices by extension in a DOF vector
 [adof,ind] = fe_c(mdof,[0.01;0.02;0.06]);
 mr = m(ind,ind); kr = k(ind,ind);
\end{verbatim}


Note {\tt adof=mdof(ind)}.  The vector {\tt adof} is the DOF definition vector linked to the new matrices {\tt kr} and {\tt mr}.

Another usual example is to fix the DOFs associated to particular nodes (to achieve a clamped boundary condition). One can for example fix nodes 1 and 2 as follows

\begin{verbatim}
% finding DOF indices by NodeId in a DOF vector
 ind = fe_c(mdof,[1 2],'ind',2);
 mr = m(ind,ind); kr = k(ind,ind);
\end{verbatim}


\noindent Displacements that do not correspond to DOFs can be fixed using \fecoor.


The second use of \fec\ is the creation of {\bf input/output shape matrices} \begin{SDT}(see \hyperlink{bc}{{\tt b,c}} \forlatex{ page \pageref{s*bc}})\end{SDT}.  These matrices contain the position, direction, and scaling information that describe the linear relation between particular applied forces (displacements) and model coordinates. \fec\ allows their construction without knowledge of the particular order of DOFs used in any model (this information is contained in the DOF definition vector {\tt mdof}). For example the output shape matrix linked to the relative $x$ translation of nodes 2 and 3 is simply constructed using

\begin{verbatim}
% Generation of observation matrices
 c=fe_c(mdof,[2.01;3.01],[1 -1])
\end{verbatim}

For reciprocal systems, input shape matrices are just the transpose 
of the collocated output shape matrices so that the same function can 
be used to build point load patterns.\index{reciprocity}

\ruic{fe_c}{Example}{}

Others examples may be found in \adof\ section.

\rmain{See also}

\noindent \femk, \feplot, \fecoor, \feload, \adof, 
\begin{SDT}
\norss 
\end{SDT}
\newline

%------------------------------------------------------------------------------
\rtop{fe\_case}{fe_case}\index{cases}

UI function to handle FEM computation {\sl cases}

\rsyntax\begin{verbatim}
  Case = fe_case(Case,'EntryType','Entry Name',Data)
  fe_case(model,'command' ...)
\end{verbatim}\nlvs


\rmain{Description}

{\sl FEM computation cases} contains information other than nodes and elements used to describe a FEM computation. Currently supported entries in the \hyperlink{stackref}{case stack} are


\vs\noindent\begin{tabular}{@{}p{.15\textwidth}@{}p{.85\textwidth}@{}}
\rz\ltt{cyclic} & (SDT) used to support cyclic symmetry conditions \\
\rz\ltt{DofLoad} & loads defined on DOFs (handled by  \feload) \\
\rz\ltt{DofSet} & (SDT) imposed displacements on DOFs \\
\rz\ltt{FixDof} & used to eliminated DOFs specified by the stack data \\
\rz\ltt{FSurf} & surface load defined on element faces (handled by \feload). This will be phased out since surface load elements associated with volume loads entries are more general. \\
\rz\ltt{FVol} & volume loads defined on elements (handled by  \feload) \\
\rz\ltt{info} & used to stored non standard entries \\
\rz\tt{KeepDof} & (obsolete) used to eliminated DOFs not specified by the stack data. These entries are less general than {\tt FixDof} and should be avoided. \\
\rz\ltt{map}  & field of normals at nodes \\
\rz\ltt{mpc}  & multiple point constraints \\
\rz\ltt{rbe3}  & a flavor of MPC that enforce motion of a node a weighted average \\
\rz\ltt{par}  & are used to define physical parameters (see \ltr{upcom}{Par} commands) \\
\rz\ltt{rigid} & linear constraints associated with rigid links \\
\rz\ltt{SensDof} & (SDT) Sensor definitions \\
\end{tabular}

\fecase\ is called by the user to initialize (when {\tt Case} is not provided as first argument) or modify cases ({\tt Case} is provided).

Accepted commands are
%\ts{AddToCase (i)} allows specification of the active case (by number in the model stack) for multiple case models. See the example below.

\ruic{fe_case}{Get}{, \htr{fe\_case}{T}, \htr{fe\_case}{Set}, \htr{fe\_case}{Remove}, \htr{fe\_case}{Reset} ...} % - - 
\begin{itemize}

\item {\tt [Case,CaseName]=fe\_case(model,'GetCase')} returns the current case. \\
\ts{GetCase{\ti i}} returns case number \tsi{i} (order in the model stack). \ts{GetCase{\ti Name}} returns a case with name \tsi{Name} and creates it if it does not exist. Note that the Case name cannot start with \ts{Case}.

\item {\tt data=fe\_case(model,'GetData {\ti EntryName}')} returns data associated with the case entry {\ti \ts{EntryName}}.

\item {\tt model=fe\_case(model,'SetData {\ti EntryName}',data)} sets data associated with the case entry {\ti \ts{EntryName}}.

\item {\tt [Case,NNode,ModelDOF]=fe\_case(model,'GetT');} returns a congruent transformation matrix which verifies constraints. Details are given in~\ser{mpc}. \texline {\tt CaseDof=fe\_case(model,'GetTDOF')} returns the case DOF (for model DOF use \texline {\tt feutil('getdof',model)}). If fields {\tt Case.T} and {\tt Case.DOF} are already defined, they will be reused. Use command option \ts{new} to force a reset of these fields.

\item {\tt model=fe\_case(model,'Remove','{\ti EntryName}')} removes the entry with name \tsi{EntryName}.

\item \ts{Reset} empties all information in the case stored in a model structure \texline {\tt  model = fe\_case(model,'reset')}

\item \ltr{fe\_case}{SetCurve} has a load reference a curve in model Stack. For example \texline {\tt model=fe\_case(model,'SetCurve','Point load 1','input');} associates \ts{Point load 1} to curve \ts{input}. See \ser{curve} for more details on curves format and \ltr{fe\_case}{SetCurve} for details on the input syntax.

\item \ts{stack\_get} applies the command to the case rather than the model. For example \texline {\tt  des = fe\_case(model,'stack\_get','par')}
\item \ts{stack\_set} applies the command to the case rather than the model. For example \texline {\tt  model = fe\_case(model,'stack\_set','info','Value',1)}
\item \ts{stack\_rm} applies the command to the case rather than the model. For example \texline {\tt  model = fe\_case(model,'stack\_rm','par')}

\end{itemize}

\begin{SDT}
\rmain{Commands for advanced constraint generation} % - - - - - - - - - - - - - 

\ruic{fe\_case}{AutoSPC}{} % - - - - - - - - - - - - - - - - - - - -
Analyses the rank of the stiffness matrix at each node and generates a \ts{fixdof} case entry for DOFs found to be singular: 
\begin{verbatim}
 model = fe_case(model,'autospc')
\end{verbatim}

\ruic{fe\_case}{Assemble}{} % - - - - - - - - - - - - - - - - - - - -
Calls used to assemble the matrices of a model. See \ltr{fe\_mknl}{Assemble} and section \ref{s*feass} for optimized assembly strategies.\\

\ruic{fe\_case}{Build}{ {\ti Sec} epsl {\ti d}} % - - - - - - - - - - - - - - - - - - - -

{\tt model = fe\_cyclic('build (N) epsl (d)',model,LeftNodeSelect)} is used to append a cyclic constraint entry in the current case.


\ruic{fe\_caseg}{ConnectionEqualDOF}{} % - - - - - - - - - - - - - - - - - - -
{\tt fe\_caseg('Connection EqualDOF',model,'name',{\ti DOF1},{\ti DOF2})} generates a set of \lts{fe\_case}{MPC} connecting each DOF of the vector {\ti DOF1} (slaves) to corresponding DOF in {\ti DOF2} (masters). {\ti DOF1} and {\ti DOF2} can be a list of {\tt NodeId}, in that case all corresponding DOF are connected, or only DOF given as a \ts{ -dof }{\ti \ts{DOFs}} command option.

Following example defines 2 disjointed cubes and connects them with a set of \lts{fe\_case}{MPC} between DOFs along x and y of the given nodes,
%begindoc
\begin{verbatim}
% Build a Multiple Point Constraint (MPC) with DOF equalization
% Generate a cube model
cf=feplot; cf.model=femesh('testhexa8');
% duplicate the cube and translate
cf.mdl=feutil('repeatsel 2 0.0 0.0 1.5',cf.mdl);
% build the connection
cf.mdl=fe_caseg('Connection EqualDOF -id7 -dof 1 2',cf.mdl, ...
    'link1',[5:8]',[9:12]');
% display the result in feplot
cf.sel='reset'; % reset feplot display
% open feplot pro and view the built connection
fecom(cf,'promodelviewon');fecom(cf,'curtab Cases','link1');
\end{verbatim}%enddoc

The option \ts{-id }{\ti \ts{i}} can be added to the command to specify a MPC ID {\ti \ts{i}} for export to other software. Silent mode is obtained by adding \ts{;} at the end of the command.

By default a DOF input mismatch will generate an error. Command option \ts{-safe} allows DOF mismatch in the input by applying the constraint only to DOF existing in both lists. If no such DOF exists the constraint is not created.

\ruic{fe\_case}{ConnectionPivot}{} % - - - - - - - - - - - - - - - - - - - - - - - -

This command generates a set of \lts{fe\_case}{MPC} defining a pivot connection between two sets of nodes. It is meant for use with volume or shell models with no common nodes. For beams the pin flags (columns 9:10 of the element row) are typically more appropriate, see \beam for more details.

The command specifies the DOFs constraint at the pivot (in the example DOF 6 is free), the local $z$ direction and the location of the pivot node. One then gives the model, the connection name, and node selections for the two sets of nodes. 

%begindoc
\begin{verbatim}
% Build a pivot connection between plates
 model=demosdt('demoTwoPlate');
 model=fe_caseg('Connection Pivot 12345 0 0 1 .5 .5 -3 -id 1111', ...
  model,'pivot','group1','group2');
 def=fe_eig(model);feplot(model,def)
\end{verbatim}%enddoc

The option \ts{-id }{\ti \ts{i}} can be added to the command to specify a MPC ID {\ti \ts{i}} for export to other software. Silent mode is obtained by adding \ts{;} at the end of the command.

\ruic{fe\_case}{ConnectionSurface}{} % - - - - - - - - - - - - - - - - - - - -

\htt{ConnectionSurface} implements node to surface connections trough constraints or elasticity. \texline {\tt fe\_caseg('ConnectionSurface {\ti DOFs}',model,'name',NodeSel1,Eltsel2)} generates a set of \lts{fe\_case}{MPC} connecting of \tsi{DOFs} of a set of nodes selected by {\tt NodeSel1} (this is a \hyperlink{findnode}{node selection} string) to a surface selected by {\tt EltSel2} (this is an \hyperlink{findelt}{element selection} string). \ts{ConnectionSurface} performs a match between two selections using \ltr{feutilb}{Match} and exploits the result with \ltr{feutilb}{MpcFromMatch}.

The following example links $x$ and $z$ translations of two plates

%begindoc
\begin{verbatim}
% Build a surface connection between two plates
 model=demosdt('demoTwoPlate');
 model=fe_caseg('Connection surface 13 -MaxDist0.1',model,'surface', ...
   'z==0', ...                          % Selection of nodes to connect
   'withnode {z==.1 & y<0.5 & x<0.5}'); % Selection of elements for matching
 def=fe_eig(model);feplot(model,def)
\end{verbatim}%enddoc


Accepted command options are 
\begin{itemize}
\item \ts{Auto} will run an automated refinement of then provided element selections \hyperlink{findelt}{element selection} to locate areas of possible interactions.
\item \ts{-aTol} provides a custom tolerance in \ts{Auto} mode to detect intersecting volume extensions where the match will be performed. By default one will consider 10 times the mesh characteristic length.
\item \ts{-id }\tsi{i} can be added to the command to specify a MPC ID {\ti \ts{i}} for export to other software. 
\item \ts{-Radius }\tsi{val} can be used to increase the search radius for the \ltr{feutilb}{Match} operation.
\item \ts{-radEst}\tsi{val} can be used to exploit a radius based on the average mesh edge length of the elements selected for matching multiplied by \tsi{val} (0.1 to get 10\% of the average mesh edge length). This command is exclusive with \ts{-Radius}, the priority is on {\ts -Radius}
\item \ts{-MaxDist }\tsi{val} eliminates matched node with distance to the matched point within the element higher than \ts{val}. This is typically useful for matches on surfaces where the node can often be external. Using a \ts{-MaxDist} is required for \ts{-Dof}.
\item \ts{-kp }\tsi{val} is used to give the stiffness (force/length) for a penalty based implementation of the constraint. The stiffness matrix of the penalized bilateral connection is stored in a superelement with the constraint name.
\item \ts{-KpAuto}\tsi{val} is used is \ts{-kp} is not present to ask for an automated estimation of the penalization stiffness based on mesh size and flange materials. The objective is to get a saturated stiffness not altering numerical conditionning. \tsi{val} is optionnal. It wiil be used as a correction factor to the default computed stiffness. To get 10\% of the automated stiffness use 0.1.
\item \ts{-dens} uses  a slave surface. In conjunction with \ts{-kp} the coefficient provided is used as a surface stiffness density. With this option, the first selection must rethrow a face selection.
\item \ts{-Dof }\tsi{val} can be used to build surface connections of non structural DOFs (thermal fields, ...).
\item \ts{-MatchS} uses a surface based matching strategy that may be significantly faster.
\item \ts{-disjCut} will attempt at splitting the generated connection by disjointed connected areas of the surface (second selection), the result is either a series of {\tt mpc} or a model with multiple SE depending on the mode.
\item Silent mode is obtained by adding \ts{;} at the end of the command.
\end{itemize}

It is also possible to define the \ts{ConnectionSurface} implicitly, to let the constraint resolution be performed after full model assembly. The \ts{ConnectionSurface} is then defined as an \ts{MPC}, which {\tt data} structure features fields {\tt .type} equal to {\tt ConnectionSurface} with possible command options, and field {\tt .sel} giving in a cell array a sequence {\tt \{NodeSel1, EltSel2\}}, as defined in the explicit definition. The following example presents the implicit \ts{ConnectionSurface} definition equivalent to the above explicit one.

%begindoc
\begin{verbatim}
% Build a surface connection between two plates
% using implicit selections
model=demosdt('demoTwoPlate');
 model=fe_case(model,'mpc','surface',...
struct('type','Connection surface 13 -MaxDist0.1',...
'sel',{{'z==0','withnode {z==.1 & y<0.5 & x<0.5}'}}));
def=fe_eig(model);feplot(model,def)
\end{verbatim}%enddoc

%begindoc
\begin{verbatim}
% Build a penalized surface connection 
% with a given sitffness density between two plates
model=demosdt('demoTwoPlate');
model=fe_caseg('Connection surface 123 -MaxDist 0.1 -kp1e8 -dens',model,...
 'surface',...
 'withnode{z==0}&selface',...
 'withnode {z==.1 & y<0.5 & x<0.5}')
def=fe_eig(model);cf=feplot(model,def);
fecom(cf,'promodelinit');
fecom(cf,'curtabStack','surface');
fecom(cf,'proviewon');
\end{verbatim}%enddoc


{\bf Warning} volume matching requires that nodes are within the element. To allow exterior nodes, you should add a \ts{\& selface} at the end of the element selection string for matching.


\ruic{fe\_case}{ConnectionScrew}{} % - - - - - - - - - - - - - - - - - - - -

{\tt fe\_caseg('Connection Screw',model,'name',data)}

This command generates a set of RBE3 defining a screw connection. Nodes to be connected are defined in planes from their distance to the axis of the screw. The connected nodes define a master set enforcing the motion of a node taken on the axis of the screw with a set of RBE3 (plane type 1) or rigid links (plane type 0) ring for each plane. 

In the case where rigid links are defined, the command appends a group of \rigid\ elements to the model case.

Real screws can be represented by beams connecting all the axis slave nodes, this option is activated by adding the field {\tt MatProId} in the {\tt data} structure.  

{\tt data} defining the screw is a data structure with following fields:

\vs\noindent\begin{tabular}{@{}p{.2\textwidth}@{}p{.8\textwidth}@{}}

\rz{\tt Origin} & a vector {\tt [x0 y0 z0]} defining the origin of the screw.\\
\rz{\tt axis}   & a vector {\tt [nx ny nz]} defining the direction of the screw axis.\\
\rz{\tt radius} & defines the radius of the screw.\\
\rz{\tt planes} & a matrix with as many lines as link rings. Each row is of the form {\tt [z0 type ProId zTol rad stype zTol2]} where \\
& {\tt z0} is the plane distance to the origin along the axis of the screw \\
& {\tt type} is the type of link: 0 for {\tt rigid} and 1 for {\tt rbe3} \\
& {\tt ProId} is the ProId of the elements containing nodes to connect. This limits the plane search to the elements of given {\tt ProId}. By default, a zero value can be used, in which case all elements will be considered for the search \\
& {\tt zTol} is the plane position tolerance, nodes within {\tt z0-zTol} to {\tt z0+zTol} will be detected\\
& {\tt rad} is the radius considered for this plane detection, if a zero value is given the base radius is used \\
& {\tt stype} defines the node search type. A value of 0 (default) will use a spherical search of radius {\tt rad} aorund the origin (only practical for perfectly planar definitions). A value of 1 will use a cylindrical node search along the screw axis from the origin, with symmetric distance from the origin defined by {\tt zTol}. A value of 2 implements a cylindrical node search with non-symmetric height tolerances from origin, using from {\tt zTol} to {\tt zTol2}\\
& {\tt zTol2} second side height tolerance for {\tt stype=2} (non-symmetric height cylinder based node search) \\
%\rz{\tt rigid} & Optional. This field (in lower case) must be added if a rigid link is created ({\tt type 0}), containing the value {\tt [ Inf abs('rigid')]}.\\
\rz{\tt MatProId} & Optional. If present beams are added to connect slave nodes at the center of each link ring. It is a vector {\tt [MatId ProId]} defining the {\tt MatId} and the {\tt ProId} of the beams. For new MatId, default material is steel and for new ProId, default beam section is a circle with provided radius.\\
\rz{\tt MasterCelas} & Optional. It defines the \celas\ element which is added if this field is present. It is of the form {\tt [0 0 -DofID1 DofID2 ProID EltID Kv Mv Cv Bv]}. The first node of the celas is the slave node of the rbe3 ring and the second is added at the same location. This can be useful to reduce a superelement keeping the center of the rings in the interface.\\
\rz{\tt NewNode} & Optional. If it is omitted or equal to 1 then a new slave node is added to the model at the centers of the link rings. If it equals to 0, existent model node can be kept.\\
\rz{\tt Nnode} & Optional. Gives the number of points to retain in each plane. 
\end{tabular}

For each plane, nodes are searched following the {\tt stype} strategy. The found nodes are then connected to the center node which is strictly defined at height {\tt z0} on the axis provided. The heights provided as {\tt z0}, {\tt zTol} and {\tt zTol2} must be understood along the axis provided and not as function of the main frame coordinates.

In the case of a {\tt rigid} connection, nodes detection should be non intersecting to avoid multiple slaves. Overlapping slave node selection is avoided by sequentially eliminating used nodes in the following detections. Selection priority is thus performed following the plane order sequence.

One can also define more generally planes as a cell array whose each row defines a plane and is of the form {\tt \{z0 type st\}} where {\tt z0} and {\tt type} are defined above and {\tt st} is a \hyperlink{findnode}{FindNode string}. {\tt st} can contain \ts{\$FieldName} tokens that will be replaced by corresponding {\tt data.FieldName} value (for example {\tt 'cyl<= \$radius o \$Origin \$axis \& inElt\{ProId \$ProId\}'} will select nodes in cylinder of radius {\tt data.radius}, origin {\tt data.Origin} and axis {\tt data.axis}, and in elements of ProId {\tt data.ProId}).

Silent mode is obtained by adding \ts{;} at the end of the command.

Following example creates a test model, and adds 2 {\tt rbe3} rings in 2 planes.

%begindoc
\begin{verbatim}
% Sample connection builds commands for screws using rigid or RBE3
model=demosdt('demoscrew layer 0 40 20 3 3 layer 0 40 20 4'); % create model
r1=struct('Origin',[20 10 0],'axis',[0 0 1],'radius',3, ...
          'planes',[1.5 1 111 1 3.1;
                    5.0 1 112 1 4;], ...
          'MasterCelas',[0 0 -123456 123456 10 0 1e14], ...
          'NewNode',0);
model=fe_caseg('ConnectionScrew',model,'screw1',r1);
cf=feplot(model); % show model 
fecom('promodelviewon');fecom('curtab Cases','screw1');

% alternative definintion using a beam
model=demosdt('demoscrew layer 0 40 20 3 3 layer 0 40 20 4'); % create model
r1=struct('Origin',[20 10 0],'axis',[0 0 1],'radius',3, ...
          'planes',[1.5 1 111 1 3.1;
                    5.0 1 112 1 4;], ...
          'MasterCelas',[0 0 -123456 123456 10 0 1e14], ...
          'MatProId',[110 1001],...
          'NewNode',0);
model=fe_caseg('ConnectionScrew',model,'screw1',r1);
cf=feplot(model); % show model 
fecom('promodelviewon');fecom('curtab Cases','screw1');

% alternative definition with a load, two beam elements are created
model=demosdt('demoscrew layer 0 40 20 3 3 layer 0 40 20 4'); % create model
model=fe_caseg('ConnectionScrew -load1e5;',model,'screw1',r1);
def=fe_eig(model,[5 15 1e3]);

% alternative definition with a load, two beam elements are created
% and a pin flag is added to release the beam compression
model=demosdt('demoscrew layer 0 40 20 3 3 layer 0 40 20 4'); % create model
model=fe_caseg('ConnectionScrew -load1e5 -pin1;',model,'screw1',r1);
def1=fe_eig(model,[5 15 1e3]);

% a new rigid body mode has been added due to the pin flag addition
[def.data(7) def1.data(7)]
\end{verbatim}%enddoc

\vs

Command option \ts{-load}\tsi{val} allows defining a loading force of amplitude \tsi{val} to the screw in the case where a beam is added to model the screw (through the {\tt MatId} optional field). To this mean the last beam element (in the order defined by the {\tt planes} entry) is split in two at a tenth of its length and a compression force is added to the larger element that is exclusively inside the beam. In complement, command option \ts{-pin}\tsi{pdof} allows defining pin flags with identifiers \tsi{pdof} to the compressed \beam element.


\end{SDT}

\rmain{Entries} % - - - - - - - - - - - - - - - - - - - - - - - - - - - - - - 

The following paragraphs list available entries not handled by \feload\ or \upcom.

\ruic{fe\_case}{cyclic}{ (SDT)} % - - - - - - - - - - - - - - - - - - - - - - - - - - - -

\rz\htt{cyclic} entries are used to define sector edges for cyclic symmetry computations. They are generated using the \ltr{fe\_cyclic}{Build} command.

\ruic{fe\_case}{FixDof}{} % - - - - - - - - - - - - - - - - - - - - - - - - - - -

\htt{FixDof} entries correspond to rows of the {\tt Case.Stack} cell array giving {\tt \{'FixDof', Name, Data\}}. {\tt Name} is a string identifying the entry. {\tt data} is a column DOF definition vector (see \ser{adof}) or a string defining a \hyperlink{findnode}{node selection} command. You can also use \\{\tt data=struct('data',DataStringOrDof,'ID',ID)} to specify a identifier.

You can now add DOF and ID specifications to the \ts{findnode} command. For example {\tt 'x==0 -dof 1 2 -ID 101'} fixes DOFs x and y on the {\tt x==0} plane and generates an {\tt data.ID} field equal to 101 (for use in other software).

The following command gives syntax examples. An example is given at the end of the \fecase\ documentation.

\begin{verbatim}
% Declare a clamping constraint with fixdof
 model = fe_case(model,'FixDof','clamped dofs','z==0', ...
    'FixDof','SimpleSupport','x==1 & y==1 -DOF 3', ...
    'FixDof','DofList',[1.01;2.01;2.02], ...
    'FixDof','AllDofAtNode',[5;6], ...
    'FixDof','DofAtAllNode',[.05]);
\end{verbatim}


\ruic{fe\_case}{Grav}{} % - - - - - - - - - - - - - - - - - - - - - - - - - - - - - -

Integraton of gravity loading. This command generates a type {\tt DofLoad} entry on all system DOF coherent with a gravity loading.
Input is a structure with field {\tt .dir} as a 1x3 line vector giving the acceleration vector in the global frame.

Note that gravity loading is different from a volumic load, as the mass is required. Its practical implementation requires a mass matrix assembly. This command will thus provide relevant results once the mesh and material assignments are complete.

%begindoc
\begin{verbatim}
% Implementation of gravity loading
model=femesh('testhexa8');
model=fe_case(model,'grav','gravity',struct('dir',[0 0 -9.81]));
% generated type is DofLoad
r1=fe_case(model,'stack_get','DofLoad','gravity','get')
\end{verbatim} %enddoc

\ruic{fe\_case}{map}{} % - - - - - - - - - - - - - - - - - - - - - - - - - - - - - - -

\htt{map} entries are used to define maps for normals at nodes. These entries are typically used by shell elements or by meshing tools. {\tt Data} is a structure with fields

\begin{itemize}
 \item {\tt .normal} a N by 3 matrix giving the normal at each node or element
 \item {\tt .ID} a N by 1 vector giving identifiers. For normals at integration points, element coordinates can be given as two or three additional columns.
 \item {\tt .opt} an option vector. {\tt opt(1)} gives the type of map (1 for normals at element centers, 2 for normals at nodes, 3 normals at integration points specified as additional columns of {\tt Data.ID}).
 \item {\tt .vertex} an optional N by 3 matrix giving the location of each vector specified in {\tt .normal}. This can be used for plotting. 
\end{itemize}

\ruic{fe\_case}{MPC}{} % - - - - - - - - - - - - - - - - - - - - - - - -

\htt{MPC} (multiple point constraint) entries are rows of the {\tt Case.Stack} cell array giving {\tt \{'MPC', Name, Data\}}. {\tt Name} is a string identifying the entry. {\tt Data} is a structure with fields {\tt Data.ID} positive integer for identification. {\tt Data.c} is a sparse matrix whose columns correspond to DOFs in {\tt Data.DOF}. {\tt c} is the constraint matrix such that $\ma{c} \ve{q} = \ve{0}$ for $q$ defined on {\tt DOF}.

{\tt Data.slave} is an optional vector of slave DOFs in {\tt Data.DOF}. If the vector does not exist, it is filled by \feutil\ {\tt FixMpcMaster}.

Note that the current implementation has no provision for using local coordinates in the definition of MPC (they are assumed to be defined using global coordinates).

\ruic{fe\_case}{par}{ (SDT)} % - - - - - - - - - - - - - - - - - - - - - - - - - - -

\htt{par} entries are used to define variable coefficients in element selections. It is nominally used through \ltr{upcom}{Par} commands but other routines may also use it~\cite{bal43}.

High level calls to define model parameters are packaged in \ltr{fe\_caseg}{Par}.

\begin{SDT}
At a lower level, command \ts{ParAdd} allows quickly defining a parameter:

{\tt model=fe\_case(model,'}\ts{ParAdd}\tsi{type nom min max scale}{\tt ',pname,par)}

Inputs defines the parameter as
\begin{itemize}
\item \tsi{type} the types are associted to the \lts{fe\_mknl}{MatType} with tokens that refer to an internal numeric value
\begin{itemize}
\item \tsi{k} for stiffness (1).
\item \tsi{m} for mass (2).
\item \tsi{c} for viscous damping (3.1).
\item \tsi{t} for shell thickness (3).
\item \tsi{ki} for hyteretic damping (4).
\item \tsi{kg} for non-linear geometry stiffness (5).
\item \tsi{0} for no matrix association (0)
\item \tsi{-2} internal only (-2), to force a superelement matrix not to undergo low-level assembly checks for parametric assemblies (\lts{fe\_mknl}{MatType} option {\tt -1}
\end{itemize}
\item \tsi{nom} the parameter nominal value.
\item \tsi{min} the parameter minimal value.
\item \tsi{max} the parameter maximal value.
\item \tsi{scale} the parameter varying scale,
\begin{itemize}
\item \tsi{1} linear variation
\item \tsi{2} or \tsi{lo} for logarithmic variation
\item more types are available in generic parameters, see \ferange.
\end{itemize}
\item {\tt pname} defines the parameter name.
\item {\tt par} provides the parameter defintion as structure, string inputs will override the original values. At low level, the following fields are admissible
\begin{itemize}
\item {\tt .coef} a 5 value row vector providing the parameter values as {\tt [type nom min max scale]}, defined above. {\tt type} is used for assembly. The following values are overriden by \ferange if a more advanced definition is used.
\item {\tt .sel} provides the element selection on which the parameter is applied. During assembly, a submodel based on the selection is assembled with the required type to provide the associated matrix.
\item {\tt .zCoef} (optionnal) defines a non standard weighting coefficient rule for the matrix.
\item More entries can be added conforming to \ferange parameter definition.
\end{itemize}
\end{itemize}

\end{SDT}

\ruic{fe\_case}{RBE3}{ (SDT)} % - - - - - - - - - - - - - - - - - - - - - - - - - - 

\htt{rbe3} constraints enforce the motion of a slave node as a weighted average of master nodes. Two definition strategies are supported in SDT, either direct or implicit. There are known robustness problems with the current implementation of this constraint.

The direct definition explicitly declares each node with coupled DOFs and weighting in a {\tt data} field. Several {\tt rbe3} constrains can be declared in {\tt data.data}. Each row of {\tt data.data} codes a set of constraints following the format

{\tt Rbe3ID NodeIdSlave DofSlave Weight1 DofMaster1 NodeId1 Weight2 ...}

{\tt DofMaster} and {\tt DofSlave} code which DOFs are used (123 for translations, 123456 for both translations and rotations). You can obtain the expression of the RBE3 as a MPC constraint using {\tt data=fe\_mpc('rbe3c',model,'CaseEntryName')}. 

When reading NASTRAN models an alternate definition 

{\tt Rbe3ID NodeIdSlave DofSlave Weight DofMaster NodeId1 NodeId2 ...}
%
may exist. If the automated attempt to detect this format fails you can fix the entry using {\tt model=fe\_mpc('FixRbe3 Alt',model)}. 

Other manipulations include
\begin{itemize}
\item When using mixtures of shell and volume elements, use {\tt model=fe\_mpc('RBE3MasterDofClean',model)} to check master DOF selection.
\item Separating RBE3 input lines from a global one can be performed with {\tt model=fe\_mpc('RBE3Split',model)}.
\item Verifying unique RBE3 IDs can be performed with {\tt model=fe\_mpc('RBE3Id',model)};
\item Convertion to RBE2 can be obtained with {\tt model=fe\_mpc('Rbe3ToRbe2',model,list)}. {\tt list} is an optionnal input to restrict the transformation to selected {\tt RBE3} entries.
\end{itemize}


The implicit definition handles {\it Node Selectors} described in \ser{findnode} to define the {\tt rbe3}. The input is then a structure:

\begin{verbatim}
% Define a RBE3 constraint
data=struct('SlaveSel','NodeSel',...
            'MasterSel','NodeSel',...
            'DOF', DofSlave,...
            'MasterDOF', DofMaster);
\end{verbatim}

\ts{SlaveSel} is the slave node selection (typically a single node), \ts{MasterSel} is the master node selection, \ts{DOF} is the declaration of the slave node coupling, \ts{MasterDOF} is the declaration of the master nodes coupling (same for all master nodes).

Grounding or coupling the slave node movement is possible through the use of a {\tt celas}, as shown in the example below featuring an implicit \ltt{rbe3} definition. In a practical approach, the slave node is duplicated and a \celas\ element is generated between the two, which allows the definition of global movement stiffness.  Constraining the rotation of a drilled block around its bore axis is considered using a global rotation stiffness.

%begindoc
\begin{verbatim}
% Integrated generation of an RBE3 constraint in a model
% Definition of a drilled block around y
model=feutil('ObjectHoleInBlock 0 0 0   1 0 0   0 1 0  2 2 2 .5 4 4 4'); 
model=fe_mat('DefaultIl',model); % default material properties
model=fe_mat('defaultPl',model); % default element integration properties
% Generation of the bore surface node set
[i1,r1]=feutil('Findnode cyl ==0.5 o 0 0 0 0 1 0',model);
model=feutil('AddsetNodeId',model,'bolt',r1(:,1));
% Generation of the slave node driving the global bore movement
model.Node(end+[1:2],1:7)=[242 0 0 0 0 0 0;244 0 0 0  0 0 0];
% Addition of the celas element between the slave node and its duplicate
model.Elt(end+[1:2],1:7)=[inf abs('celas') 0;242 244 123456 0 0 0 1e11];
model=feutil('AddSetNodeId',model,'ref_rot',244);
% Definition of the RBE3 constraint
data=struct('SlaveSel','setname ref_rot',...
            'MasterSel','setname bolt',...
            'DOF',123456,... % Slave node constrained on 6 DOF
            'MasterDOF',123); % Master only use translation
model=fe_case(model,'rbe3','block_mov',data);
% Grounding the global y rotation (leaving the celas stiffness work)
model=fe_case(model,'fixdof','ClampBlockRot',242.05);
% 5 rigid body modes model obtained
def=fe_eig(model,[5 20 1e3]);
cf=feplot(model,def);fecom('curtabCases','rbe3');fecom('ProViewOn');
\end{verbatim}%enddoc


\ruic{fe\_case}{rigid}{} % - - - - - - - - - - - - - - - - - - - - - - - - - - - - -

See details under \rigid\ which also illustrates the \ts{RigidAppend} command.

\ruic{fe\_case}{Sens}{ ... (SDT)} % - - - - - - - - - - - - - - - - - - - - - - - - - - -

\ltt{SensDof} entries are detailed in ~\ser{sensor}. Command options \ts{vel} and {\tt acc} can be used to specify that certain sensors should measure velocity or acceleration. They are stored as rows of the {\tt Case.Stack} cell array giving {\tt \{'SensDof', Name, data\}}.\index{data structure!sens}. 

To properly retrieve a unique {\tt SensDof} from the model, command {\tt [wire,name]=fe\_case('GetSensDof',model)} looks in the model {\tt Case} with this strategy :
\begin{itemize}
\item If only one {\tt SensDof} is defined, return this {\tt SensDof} and its name
\item If several {\tt SensDof}s are defined return {\tt SensDof} {\tt Test} if there, else return first {\tt SensDof} in the list
\item Empty return if no {\tt SensDof} found
\end{itemize}

To get back the observation matrix, use the command {\tt Sens=fe\_case(model,'sens','SensName')} as detailed in \lts{sensor}{Sens} for both full and reduced models.

{\tt R1=fe\_case('sensobserve',model,'SensEntryName',def); iiplot(R1)} can be used to extract observations at sensors associated with a given response. The \ts{SensEntryName} can be omitted if a single sensor set exist. \texline {\tt Sens=fe\_case(model,'sens','SensName');R1=fe\_case('sensobserve',Sens,def);} is also acceptable 

\ruic{fe\_case}{un=0}{} % - - - - - - - - - - - - - - - - - - - - - - - - - - - - - - - -

{\tt model=fe\_case(model,'un=0','Normal motion',map);} where {\tt map} gives normals at nodes generates an {\tt mpc} case entry that enforces the condition $\ve{u}^T\ve{n}=0$ at each node of the map.


\ruic{fe\_case}{SetCurve}{} % - - - - - - - - - - - - - - - - - - - - - - - - - - - - -

To associate a time variation to a compatible case entry, one adds a field {\tt curve} to the case entry structure. This field is a cell array that is of the same length as the number of solicitation contained in the case entry. 

Each curve definition in the cell array can be defined as either
\begin{itemize}
\item a string referring to the name of a curve stacked in the model (recommended)
\item a curve structure
\item a string that will be interpreted on the fly by \fecurve when the load is assembled, see {\tt fe\_curve('TestList')} to get the corresponding strings
\end{itemize}

The assignation is performed using 

{\tt model = fe\_case(model,'SetCurve',EntryName,CurveName,Curve,ind);}

with
\begin{itemize}
\item {\tt EntryName} the case entry to which the curve will be assigned. Use \ts{?} to find name automatically if only one exists. 
\item {\tt CurveName} a string or a cell array of string with the name of the curves to assign
\item {\tt Curve} (optional)  a curve or a cell array of curves that will be assigned (if not in model stack), they will be set in the model stack and only their names will be mentioned in the case entry
\item {\tt ind} (optional) the index of the curves to assign in the {\tt curve} field, if several solicitation are present in the case entry considered. If {\tt ind} is omitted the whole field {\tt curve} of the case entry will be replaced by {\tt CurveName}.
\end{itemize}

In practice, a variant call is supported for retro-compatibility but is not recommended for use,

{\tt model = fe\_case(model,'SetCurve',EntryName,Curve,ind);}

allows a direct assignation of non stacked curves to the case entry with the same behavior than for the classical way.

\vs

Multiple curve assignation at once to a specific {\tt EntryName} is supported with the following rules
\begin{itemize}
\item {\tt CurveName}, {\tt Curve} (optional) and {\tt ind} (mandatory) have the same sizes. In this case, all given curves will be assigned to the case entry with their provided index
\item A single{\tt CurveName} and {\tt Curve} is provided with a vector of indices. In this case, all indexed curves will be assigned to the new provided one
\end{itemize}

\vs

To remove a curve assignation to a case entry. Command

{\tt model = fe\_case(model,'SetCurve',EntryName,'remove');}

will remove the field {\tt curve} from case entry {\tt EntryName}.


\vs The flexibility of the command imposes some restriction to the curve names. Name {\tt remove} and {\tt TestVal} with {\tt Val} begin a keyword used by \ltr{fe\_curve}{Test} cannot be used.

\vs

The following example illustrate the use of \ts{SetCurve} to assign curves to case entries

%begindoc
\begin{verbatim}
% Sample calls to assign curves to load cases
% generate a sample cube model
 model=femesh('testhexa8'); 
 % clamp the cube bottom
 model=fe_case(model,'FixDof','clamped dofs','z==0');
 % load a DOF of the cube base
 model=fe_case(model,'DofLoad','in',struct('def',1,'DOF',5.02));
 % generate a curve loading transient pattern
 R1=fe_curve('testramp t1.005 yf1');
 % assign the curve to the load case
 model=fe_case(model,'SetCurve','in','tramp',R1);
 
 % add a new load case with two sollicitations
 model=fe_case(model,'DofLoad','in2',...
  struct('def',[1 0;0 1],'DOF',[6.02;6.03]));
 % assign a new transient variation to both directions
 model=fe_case(model,'SetCurve','in2','tramp1',...
  fe_curve('testramp t0.5 yf1'),1:2);
 % modify the first direction only to tramp instead of tramp1
 model=fe_case(model,'SetCurve','in2','tramp',1);
 
 % remove the curve assigned to input in
 model=fe_case(model,'SetCurve','in','remove')
\end{verbatim}%enddoc

\ruic{fe\_case}{Examples}{} % - - - - - - - - - - - - - - - - - - - - - - - - - - - - - - -


Here is an example combining various \fecase\ commands

%begindoc
\begin{verbatim}
% Sample fe_case commands for boundary conditions, connections, and loads
 femesh('reset');
 model = femesh('test ubeam plot');
 % specifying clamped dofs (FixDof) 
 model = fe_case(model,'FixDof','clamped dofs','z==0');
 % creating a volume load
 data  = struct('sel','GroupAll','dir',[1 0 0]);
 model = fe_case(model,'FVol','Volumic load',data);
 % assemble active DOFs and matrices
 model=fe_mknl(model);
 % assemble RHS (volumic load)
 Load  = fe_load(model,'Case1');
 % compute static response
 kd=ofact(model.K{2});def.def= kd\Load.def; ofact('clear',kd)
 Case=fe_case(model,'gett'); def.DOF=Case.DOF;
 % plot displacements
 feplot('initdef',def);
 fecom(';undef;triax;showpatch;promodelinit');
\end{verbatim}%enddoc


\rmain{See also}
\noindent \femk, \fecase








%       Copyright (c) 2001-2017 by INRIA and SDTools, All Rights Reserved.
%       Use under OpenFEM trademark.html license and LGPL.txt library license
%       $Revision: 1.69 $  $Date: 2025/06/25 17:34:38 $

%------------------------------------------------------------------------------
\rtop {fe\_curve}{fe_curve}

\noindent Generic handling of curves and signal processing utilities

\rsyntax\begin{verbatim}
  out=fe_curve('command',MODEL,'Name',...);
\end{verbatim}\nlvs
 

\ruic{fe\_curve}{Commands}{}

\fecurve\ is used to handle curves and do some basic signal processing. The format for curves is described in \ser{curve}. The \iiplot\ interface may be used to plot curves and a basic call would be {\tt iiplot(Curve)} to plot curve data structure {\tt Curve}.

Accepted commands are


\ruic{fe\_curve}{bandpass}{ {\ti Unit f\_min f\_max}} %  - - - - - - - - - - - - - - - - - - - - - - - - - -
{\tt out=fe\_curve('BandPass {\ti Unit f\_min f\_max}',signals);} \\
realizes a true bandpass filtering (i.e. using {\tt fft()} and {\tt ifft()}) of time signals contained in curves {\tt signals}. {\ti \ts{f\_min}} and {\ti \ts{f\_max}} are given in units {\ti \ts{Unit}}, whether Hertz(\ts{Hz}) or Radian(\ts{Rd}). With no {\ti \ts{Unit}}, {\tt f\_min} and {\tt f\_max} are assumed to be in Hertz. 

%begindoc
\begin{verbatim}
% apply a true bandpasss filter to a signal
out=fe_curve('TestFrame');% 3 DOF oscillator response to noisy input
fe_curve('Plot',out{2});  % "unfiltered" response
filt_disp=fe_curve('BandPass Hz 70 90',out{2}); % filtering 
fe_curve('Plot',filt_disp); title('filtered displacement');
\end{verbatim}%enddoc


\ruic{fe\_curve}{datatype}{ [,cell]} %  - - - - - - - - - - - - - - - - - - -

{\tt out=fe\_curve('DataType',DesiredType);} \\
returns a data structure describing the data type, useful to fill {\tt .xunit} and {\tt .yunit} fields for curves definition. {\tt DesiredType} could be a string or a number corresponding to the desired type.  With no {\tt DesiredType}, the current list of available types is displayed. One can specify the unit with {\tt out=fe\_curve('DataType',DesiredType,'unit');}.

{\tt DataTypeCell} returns a cell array rather than data structure to follow the specification for \hyperlink{curve}{curve data structures}. Command option \ts{-label}\tsi{"lab"} allows directly providing a custom label named \tsi{lab} in the data type.

\ruic{fe\_curve}{get}{Curve,Id,IdNew} %  - - - - - - - - - - - - - - - - - - - - - - - - - -
\begin{itemize}
\item \ts{GetCurve}: recover curve by name
{\tt curve=fe\_curve('getcurve',model,'curve\_name');} \\
extracts curve \ts{curve\_name} from {\tt model.Stack} or the possible curves attached to a load case. If the user does not specify any name, all the curves are returned in a cell array.
\item \ts{GetId}\tsi{val}: recover curve by {\tt .ID} field, equal to \tsi{val}. {\tt curve=fe\_curve('GetId val',model);}
\item \ts{GetIdNew}: generate a new identifier interger unused in any curve of the model.
{\tt i1=fe\_curve('GetIdNew',model);}
\end{itemize}

\ruic{fe\_curve}{h1h2}{ {\ti input\_channels}} %  -  - - - - - - - - - - - - - - - - - -
{\tt FRF=fe\_curve('H1H2 input\_channels',frames,'window');} \\
computes H1 and H2 FRF estimators along with the coherence from time signals contained in cell array {\tt frames} using window \ts{window}. The time vector is given in {\tt frames\{1\}.X} while {\ti input\_channels} tells which columns of in {\tt frames\{1\}.Y} are inputs. If more than one input channel is specified, true MIMO FRF estimation is done, and H$\nu$ is used instead of H2. When multiple frames are given, a mean estimation of FRF is computed. 

Note: To ensure the proper assembly of H1 and H$\nu$ in MIMO FRF estimation case, a weighing based on maximum time signals amplitude is used. To use your own, use \\
{\tt FRF=fe\_curve('H1H2 input\_channels',frames,window,weighing);} \\
where {\tt weighing} is a vector containing weighing factors for each channel. To avoid weighing, use \texline {\tt FRF=fe\_curve('H1H2 input\_channels',frames,window,0);} . For an example see \texline {\tt sdtweb('start\_time2frf','h1h2')}


\ruic{fe\_curve}{noise}{} %  - - - - - - - - - - - - - - - - - - - - - - - - - -

OBSOLETE : use \ltr{fe\_curve}{Test}\ts{Noise} instead

{\tt noise=fe\_curve('Noise',Nw\_pt,fs,f\_max);} \\
computes a {\tt Nw\_pt} points long time signal corresponding to a ``white noise'', with sample frequency {\tt fs} and a unitary power spectrum density until {\tt f\_max}. {\tt fs/2} is taken as {\tt f\_max} when not specified. The general shape of noise power spectrum density, extending from {\tt 0 } to {\tt fs/2}, can be specified instead of {\tt f\_max}.

%begindoc
\begin{verbatim}
% computes a 2 seconds long white noise, 1024 Hz of sampling freq.
% with "rounded" shape PSD    
fs=1024; sample_length=2;
Shape=exp(fe_curve('window 1024 hanning'))-1; 
noise_h=fe_curve('noise',fs*sample_length,fs,Shape);
noise_f=fe_curve('fft',noise_h);
figure(1);
subplot(211);fe_curve('plot -gca',noise_h);axis tight;
subplot(212);fe_curve('plot -gca',noise_f);axis tight;
\end{verbatim}%enddoc


\ruic{fe\_curve}{plot}{} %  - - - - - - - - - - - - - - - - - - - - - - - - - -

{\tt fe\_curve('plot',curve);} plots the curve {\tt curve}. \\
{\tt fe\_curve('plot',fig\_handle,curve);} plots {\tt curve} in the figure with handle {\tt fig\_handle}.\\
{\tt fe\_curve('plot',model,'curve\_name');} plots the curve of {\tt model.Stack} named \ts{curve\_name}.\\
{\tt fe\_curve('plot',fig\_handle,model,curve\_name);} plots curve named {\tt curve\_name} stacked in {\tt .Stack} field of model {\tt model}.

%begindoc
\begin{verbatim}
% Plot a fe_curve signal
% computes a 2 seconds long white noise, 1024 Hz of sampling freq.
fs=1024; sample_length=2;
noise=fe_curve('noise',fs*sample_length,fs);noise.Xlab{1}={'Time','s',[]}
noise.Xlab{2}={'Force','N',fe_curve('DataType','Excit. force')};
noise.name='Input force';cdm.plot(noise)
\end{verbatim}%enddoc
 

\ruic{fe\_curve}{resspectrum}{ [{\ti True, Pseudo}] [{\ti Abs., Rel.}] [{\ti Disp., Vel., Acc.}]} %  - - - - - - - - - - - - - - - - - - - - - - - - - -

{\tt out=fe\_curve('ResSpectrum',signal,freq,damp);} \\
computes the response spectrum associated to the time signal given in {\tt signal}. Time derivatives can be obtained with option \ts{-v} or \ts{-a}. Time integration with option \ts{+v} or \ts{+a}. Pseudo derivatives with option \ts{PseudoA} or \ts{PseudoV}. {\tt freq} and {\tt damp} are frequencies (in Hz) and damping ratios vectors of interest for the response spectra. For example

%begindoc
\begin{verbatim}
wd=fileparts(which('d_ubeam'));
% read the acceleration time signal
bagnol_ns=fe_curve(['read' fullfile(wd,'bagnol_ns.cyt')]);

% read reference spectrum
bagnol_ns_rspec_pa= fe_curve(['read' fullfile(wd,'bagnol_ns_rspec_pa.cyt')]);
% compute response spectrum with reference spectrum frequencies
% vector and 5% damping
RespSpec=fe_curve('ResSpectrum PseudoA',...
                  bagnol_ns,bagnol_ns_rspec_pa.X/2/pi,.05);

fe_curve('plot',RespSpec); hold on;
plot(RespSpec.X,bagnol_ns_rspec_pa.Y,'r');
legend('fe\_curve','cyberquake');
\end{verbatim}%enddoc


\ruic{fe\_curve}{returny}{} %  - - - - - - - - - - - - - - - - - - - - - - - - - -

If curve has a {\tt .Interp} field, this interpolation is taken in account. If {\tt .Interp} field is not present or empty, it uses a degree 2 interpolation by default. 

To force a specific interpolation (over passing {\tt .interp field}, one may insert the \ts{-linear}, \ts{-log} or \ts{-stair} option string in the command. 

To extract a curve {\tt curve\_name} and return the values {\tt Y} corresponding to the input {\tt X}, the syntax is

{\tt y = fe\_curve('returny',model,curve\_name,X);} \\

Given a {\tt curve} data structure, to return the values {\tt Y} corresponding to the input {\tt X}, the syntax is

{\tt y = fe\_curve('returny',curve,X);} \\


\ruic{fe\_curve}{set}{} %  - - - - - - - - - - - - - - - - - - - - - - - - - -

This command sets a curve in the model. 3 types of input are allowed:

\begin{itemize}
\item A data structure, {\tt model=fe\_curve(model,'set',curve\_name,data\_structure)}

\item A string to interprete, {\tt model=fe\_curve(model,'set',curve\_name,string)}

\item A name referring to an existing curve (for load case only), {\tt model=fe\_curve( model, 'set LoadCurve',load\_case,chanel,curve\_name)}. {\bf This last behavior is obsolete} and should be replaced in your code by a more general call to \ltr{fe\_case}{SetCurve}.
\end{itemize}

When you want to associate a curve to a load for time integration it is preferable to use a formal definition of the time dependence (if not curve can be interpolated or extrapolated).

The following example illustrates the different calls.

%begindoc
\begin{verbatim}
% Sample curve assignment to modal loads in a model
model=fe_time('demo bar'); q0=[];

% curve defined by a by-hand data structure:
c1=struct('ID',1,'X',linspace(0,1e-3,100), ...
     'Y',linspace(0,1e-3,100),'data',[],...
     'xunit',[],'yunit',[],'unit',[],'name','curve 1');
model=fe_curve(model,'set','curve 1',c1);
% curve defined by a string to evaluate (generally test fcn):
model=fe_curve(model,'set','step 1','TestStep t1=1e-3');
% curve defined by a reference curve:
c2=fe_curve('test -ID 100 ricker dt=1e-3 A=1');
model=fe_curve(model,'set','ricker 1',c2);
c3=fe_curve('test eval sin(2*pi*1000*t)'); % 1000 Hz sinus
model=fe_curve(model,'set','sin 1',c3);

% define Load with curve definition
LoadCase=struct('DOF',[1.01;2.01],'def',1e6*eye(2),...
            'curve',{{fe_curve('test ricker dt=2e-3 A=1'),...
                      'ricker 1'}});
model = fe_case(model,'DOFLoad','Point load 1',LoadCase);

% modify a curve in the load case
model=fe_case(model,'SetCurve','Point load 1','TestStep t1=1e-3',2);

% the obsolete but supported call was
model=fe_curve(model,'set LoadCurve','Point load 1',2,'TestStep t1=1e-3');

% one would prefer providing a name to the curve, 
% that will be stacked in the model
model=fe_case(model,'SetCurve','Point load 1',...
 'my\_load','TestStep t1=1e-3',2);
\end{verbatim}%enddoc


\ruic{fe\_curve}{Test}{ ...} %  - - - - - - - - - - - - - - - - - - - - - - -

The {\tt test} command handles a large array of analytic and tabular curves. 
In OpenFEM all parameters of each curve must be given in the proper order.  In SDT you can specify only the ones that are not the default using their name.

When the abscissa vector (time, frequency, ...) is given as shown in the example, a tabular result is returned.  

Without output argument the curve is simply plotted.

\begin{SDT}
%begindoc
\begin{verbatim}
% Standard generation of parametered curves
fe_curve('test')  % lists curently implemented curves

t=linspace(0,3,1024); % Define abscissa vector
% OpenFEM format with all parameters (should be avoid):
C1=fe_curve('test ramp 0.6 2.5 2.3',t);
C2=fe_curve('TestRicker 2 2',t);

% SDT format non default parameters given with their name
%  definition is implicit and will be applied to time vector
%  during the time integration: 
C3=fe_curve('Test CosHan f0=5 n0=3 A=3'); 
C4=fe_curve('testEval 3*cos(2*pi*5*t)');

% Now display result on time vector t:
C3=fe_curve(C3,t);C4=fe_curve(C4,t)
figure(1);plot(t,[C1.Y C2.Y C4.Y C3.Y]);
legend(C1.name,C2.name,C4.name,C3.name)
\end{verbatim}%enddoc

\end{SDT}


A partial list of accepted test curves follows

\begin{itemize}
\item \ts{Testsin}, \ts{Testcos},  \ts{TestTan},  \ts{TestExp}, accept parameters \ts{T} period and \ts{A} amplitude. \ts{-stoptime Tf} will truncate the signal. 

\item \ts{TestRamp t0=}\tsi{t0}\ts{\ t1=}\tsi{t1}\ts{\ Yf=}\tsi{Yf} has a ramp starting at zero until \tsi{t0} and going up to \tsi{Yf} at \tsi{t1}. The number of intermediate value can be controlled with the abscissa vector.\\
To define a gradual load, for non linear static for example, a specific call with a \ts{Nstep} parameter can be performed : \ts{TestRamp NStep=}\tsi{NStep}\ts{\ Yf=}\tsi{Yf}. For example, to define a 20 gradual steps to 1e-6 :{\tt R1=fe\_curve('TestRamp NStep=20 Yf=1e-6');}

\item \ts{TestRicker dt=}\tsi{dt}\ts{\ A=}\tsi{A}\ts{\ t0=}\tsi{t0} generates a Ricker function typically used to represent impacts of duration \tsi{dt} and amplitude \tsi{A}, starting from time \tsi{t0}.

\item \ts{TestSweep fmin=}\tsi{fmin}\ts{\ fmax=}\tsi{fmax}\ts{\ t0=}\tsi{t0}\ts{\ t1=}\tsi{t1} generates a sweep cosine from \tsi{t0} to \tsi{t1}, with linear frequency sweeping from \tsi{f0} to \tsi{f1}.

$Y=cos(2*pi*\left(fmin+\left(fmax-fmin\right)*\frac{t-t0}{t1-t0})*\left(t-t0\right)\right)$ for $t0<t<t1$, $Y=0$ elsewhere.

%Note that \tsi{f1} is not the final instant frequency of the signal (which will be $2*f1-f0$ for ascending sweep).
% used to be $f0+2*\frac{f1-f0}{NStep}$ for max final freq : but wrong

\item \ts{TestStep t1=}\tsi{t1} generates a step which value is one from time 0 to time \tsi{t1}.

\item \ts{TestNoise -window"}\tsi{window}\ts{"} computes a time signal corresponding to a white noise, with the power spectrum density specified as the \tsi{window} parameter. For example \ts{TestNoise "Box A=1 min=0 max=200"} defines a unitary power spectrum density from 0 Hz to 200 Hz.

\item \ts{TestBox A=}\tsi{A}\ts{\ min=}\tsi{min}\ts{\ max=}\tsi{max} generates a sample box signal from \tsi{min} to \tsi{max} abscissa, with an amplitude \ts{A}.

\item \ts{TestEval {\ti str}} generates the signal obtained by evaluating the string \tsi{str} function of \ts{t}. For example {\tt R1=fe\_curve('Test eval sin(2*pi*1000*t)',linspace(0,0.005,501)); iiplot(R1)}


\end{itemize}

One can use {\tt fe\_curve('TestList')} to obtain a cell array of the test keywords recognized.

\ruic{fe\_curve}{testframe}{} %  - - - - - - - - - - - - - - - - - - - - - - - - - -

{\tt out=fe\_curve('TestFrame');} computes the time response of a 3 DOF oscillator to a white noise and fills the cell array {\tt out} with noise signal in cell 1 and time response in cell 2. See \swref{fe\_curve('TestFrame')} to open the function at this example.


\ruic{fe\_curve}{timefreq}{} %  - - - - - - - - - - - - - - - - - - - - - - - - - -

{\tt out=fe\_curve('TimeFreq',Input,xf);} \\
computes response of a system with given transfer functions {\tt FRF} to time input {\tt Input}. Sampling frequency and length of time signal {\tt Input} must be coherent with frequency step and length of given transfer {\tt FRF}.

%begindoc
\begin{verbatim}
% Plot time frequency diagrams of signals
fs=1024; sample_length=2;                   % 2 sec. long white noise
noise=fe_curve('noise',fs*sample_length,fs);% 1024 Hz of sampling freq.
[t,f,N]=fe_curve('getXTime',noise);

% FRF with resonant freq. 50 100 200 Hz, unit amplitude, 2% damping 
xf=nor2xf(2*pi*[50 100 200].',.02,[1 ; 1 ; 1],[1 1 1],2*pi*f);

Resp=fe_curve('TimeFreq',noise,xf); % Response to noisy input
fe_curve('Plot',Resp); title('Time response');
\end{verbatim}%enddoc


\ruic{fe\_curve}{Window}{ ... } %  - - - - - - - - - - - - - - - - - - - - - - - - - -

Use {\tt fe\_curve} \ts{window} to list implemented windows. The general calling format is \texline {\tt win=fe\_curve('Window Nb\_pts Type Arg');} which computes a {\tt {\ti Nb\_pts}} points window. The default is a symmetric window (last point at zero), the command option \ts{-per} clips the last point of a $N+1$ long symmetric window. 

For the exponential window the arguments are three doubles. {\tt win = fe\_curve('Window 1024 Exponential 10 20 10');} returns an exponential window with 10 zero points, a 20 point flat top, and a decaying exponential over the 1004 remaining points with a last point at {\tt exp(-10)}.


{\tt win = fe\_curve('Window 1024 Hanning');} returns a 1024 point long hanning window.


\rmain{See also}

\noindent \feload, \fecase










%------------------------------------------------------------------------------
\rtop{fe\_eig}{fe_eig}

\noindent Computation of normal modes associated to a second order undamped model.\index{normal mode!computation and normalization}%
\index{orthogonality conditions}\index{eigenvalue}%
\index{degree of freedom (DOF)!active}%
\index{mass!normalization} \index{sparse eigensolution}\index{mode!normal}

\rsyntax\begin{verbatim}
[phi, wj] = fe_eig(m,k)
[phi, wj] = fe_eig(m,k,opt,imode)
\end{verbatim}\nlvs

\rmain{Description}

The non current {\sl SDT 3} version of this function is included in OpenFEM. 


\noindent The normal modeshapes {\tt phi}=$\phi$ and frequencies {\tt wj= sqrt(diag(}$\Omega^2${\tt))} are solution of the undamped eigenvalue problem 

\begin{displaymath}
  - \ma M \ve{\phi_j}\omega_j^2+\ma K \ve{\phi_j}=\ve 0
\end{displaymath}

\noindent and verify the two orthogonality conditions

\begin{displaymath}
  \ma{\phi} ^T \ma M _{N\times N} \ma{\phi} _{N\times N}= \ma I 
  _{N\times N} \ \hbox{and}\  \ma{\phi} ^T \ma K \ma{\phi}= \diag{\Omega_j^2}
\end{displaymath}


Outputs are modeshapes (columns of {\tt phi}) and frequencies {\tt wj} in {\bf rad/s}.

\noindent \feeig\ implements various algorithms to solve this problem for modes and frequencies. Many options are available and it is important that you read the notes below to understand how to properly use them. The format of the option vector {\tt opt} is \vs\\
{\tt [method nm Shift Print Thres]} (default values are {\tt [2 0 0 0 1e-5]}) \vs

\begin{tabular}{@{}p{.15\textwidth}@{}p{.85\textwidth}@{}}
%
{\tt method} & {\bf 0} alternate full matrix solution (old method of {\sl SDT} 1.0)

{\bf 1} subspace iteration which allows computing the lowest modes of a large problem where sparse mass and stiffness matrices are used. For cases with more than 1000 DOF, the \ofact\ object is used to factor the stiffness.

{\bf 2} {\bf default} full matrix solution

{\bf 3} Lanczos algorithm which allows to compute the lowest modes of a large problem using an unpivoted {\tt lu} decomposition of the stiffness matrix

{\bf 4} same as method 3 but uses a Cholesky decomposition of the stiffness matrix (less general).

...

\end{tabular}\par\begin{tabular}{@{}p{.15\textwidth}@{}p{.85\textwidth}@{}}
%
{\tt method} & {\bf 1} subspace iteration is the only partial solver included in this version of the {\tt fe\_eig} developed by SDTools.
{\bf 2} is a FULL solver cleaning up results of the \matlab\ {\tt eig} function.

\\

{\tt nm} &  number of modes to be returned. A non-integer or negative {\tt nm}, is used as the desired {\tt fmax} in {\bf Hz} for iterative solvers. \\

{\tt shift} & value of mass shift (should be non-zero for systems with {\bf rigid body modes}, see notes below).  The subspace iteration method supports iterations without mass shift for structures with rigid body modes.  This method is used by setting the shift value to {\tt Inf}. \\

{\tt print} & level of printout ({\tt 0} none, {\tt 11} maximum) \\

{\tt thres} & threshold for convergence of modes (default {\tt 1e-5} for the subspace iteration method) \\

\end{tabular}


\rmain{Notes}

\begin{Eitem}

\item For systems with rigid body modes, you must specify a mass-shift. A good value is about one tenth of the first flexible frequency squared, but the Lanczos algorithm tends to be sensitive to this value (you may occasionally need to play around a little). If you do not find the expected number of rigid body modes, this is often the reason.

\item Memory usage may be affected by the choice of a {\tt skyline method}.

\end{Eitem}

\rmain{See also}

\noindent \feceig, \femk

%       Copyright (c) 2001-2015 by INRIA and SDTools, All Rights Reserved.
%       Use under OpenFEM trademark.html license and LGPL.txt library license
%       $Revision: 1.35 $  $Date: 2019/05/06 15:45:51 $

%-----------------------------------------------------------------------------
\rtop{fe\_gmsh}{fe_gmsh}

GMSH interface. You can download GMSH at \url{http://www.geuz.org/gmsh/} and tell where to find GMSH using

\rsyntax\begin{verbatim}
setpref('OpenFEM','gmsh','/path_to_binary/gmsh.exe') % Config
model=fe_gmsh(command,model,...)
model=fe_gmsh('write -run','FileName.stl')
\end{verbatim}

\rmain{Description}

The main operation is the automatic meshing of surfaces.


%  - - - - - - - - - - - - - - - - - - - - - - - - - - - - - - - - - - -
\ruic{fe\_gmsh}{Example}{}

This example illustrates the automatic meshing of a plate
%begindoc
\begin{verbatim}
 FEnode = [1 0 0 0  0 0 0; 2 0 0 0  1 0 0; 3 0 0 0  0 2 0];
 femesh('objectholeinplate 1 2 3 .5 .5 3 4 4');
 model=femesh('model0');
 model.Elt=feutil('selelt seledge ',model);
 model.Node=feutil('getnode groupall',model);
 model=fe_gmsh('addline',model,'groupall');
 model.Node(:,4)=0; % reset default length
 mo1=fe_gmsh('write del.geo -lc .3 -run -2 -v 0',model);
 feplot(mo1)
\end{verbatim}%enddoc



This other example makes a circular hole in a plate 
%beginddoc
\begin{verbatim}
% Hole in plate :
model=feutil('Objectquad 1 1',[0 0 0; 1 0 0;1 1 0;0 1 0],1,1); %
model=fe_gmsh('addline -loop1',model,[1 2; 2 4]);
model=fe_gmsh('addline -loop1',model,[4 3; 3 1]);
model=fe_gmsh('AddFullCircle -loop2',model,[.5 .5 0; .4 .5 0; 0 0 1]);
 
model.Stack{end}.PlaneSurface=[1 2]; 
mo1=fe_gmsh('write del.geo -lc .02 -run -2 -v 0',model)
feplot(mo1)
\end{verbatim}%enddoc


To allow automated running of GMSH from MATLAB, this function uses a \ts{info,GMSH} stack entry with the following fields

\lvs\noindent\begin{tabular}{@{}p{.2\textwidth}@{}p{.8\textwidth}@{}}
\rz{\tt .Line} & one line per row referencing {\tt NodeId}. Can be defined using \ts{addline} commands.\\
\rz{\tt .Circle} & define properties of circles.\\
\rz{\tt .LineLoop} & rows define a closed line as combination of elementary lines. Values are row indices in the {\tt .Line} field.
One can also define {\tt LineLoop} from circle arcs (or mixed arcs and lines) using a cell array whose each row describes a lineloop as {\tt \{'{\ti LineType}',LineInd,...\}} where {\ti LineType} can be {\tt Circle} or {\tt Line} and {\tt LineInd} row indices in corresponding {\tt .Line} or {\tt .Circle} field.\\
\rz{\tt .TransfiniteLines} & Defines lines which seeding is controlled. \\
\rz{\tt .PlaneSurface} & rows define surfaces as a combination of line loops, values are row indices in the {\tt .LineLoop} field. Negative values are used to reverse the line orientation. 1st column describes the exterior contour, and followings the interiors to be removed. As {\tt .PlaneSurface} is a matrix, extra columns can be filled by zeros.\\
\rz{\tt .EmbeddedLines} & define line indices which do not define mesh contours but add additional constrains to the final mesh (see Line In Surface in the {\tt gmsh} documentation. \\
\rz{\tt .SurfaceLoop} & rows define a closed surface as combination of elementary surfaces. Values are row indices in the {\tt .PlaneSurface} field.
\end{tabular}

The local mesh size is defined at nodes by GMSH. This is stored in column 4 of the {\tt model.Node}. Command option \ts{-lc}\tsi{val} in the command resets the value \tsi{val} for all nodes that do not have a prior value. 

%  - - - - - - - - - - - - - - - - - - - - - - - - - - - - - - - - - - -
\ruic{fe\_gmsh}{Add}{...}

Typical calls are of the form  {\tt [mdl,RO]=fe\_gmsh('Add Cmd',mdl,data)}. The optional second output argument can be used to obtain additional information like the {\tt LoopInfo}. Accepted command options are

\begin{itemize}

\item \ts{-loop i} is used to add the given geometries and append the associated indices into the {\tt LineLoop(i)}.
\item \ts{FullCircle} defines a circle defined using {\tt data} with rows giving  center coordinates, an edge node coordinates and the normal in the last row. 4 arcs of circle are added. In the {\tt LineLoop} field the entry has the form {\tt \{'Circle',[ind1 ind2 ind3 ind4]\}} where {\tt indi} are the row indices of the 4 arcs of circle created in {\tt .Circle} field. \\

\item \ts{CircleArc} defines a circle arc using {\tt data} 
\begin{itemize}
\item 3x3 matrix, with 1rst row giving center coordinates, second and third rows are respectively the first and second edges defined by node coordinates.
\item 3x1 vector, giving the 3 NodeId (center, 1st and 2nd edge) as a column instead of x y z. 
\item with a \ts{-tangent1} option, 3x3 matrix whose 1st row defines a tangent vector of the circle arc at the 1st edge node (resp. at the second edge node with the option \ts{-tangent2}). 2nd row defines the 1st edge node coordinates and third row the 2nd edge node coordinate.
\end{itemize}

\item \ts{Disk} ...\\

\item \ts{Line} accepts multiple formats. {\tt data} can be a 2 column matrix which each row defines a couple of points from their {\tt NodeId}. 

{\tt data} can also be a 2 by 3 matrix defining the coordinates of the 2 extremities. 

{\tt data} can also be a string defining a line selection. 

\begin{itemize}
\item It is possible to specify a seeding on the line for further meshing operation using additional arguments \ts{seed} and the number of nodes to seed on the line. \texline {\it E.g.}:
{\tt mdl=fe\_gmsh('AddLine',mdl,data,'seed',5);} will ask {\tt gmsh} to place 5 nodes on each line declared in \ts{data}.
\item It is possible to define line constrains in mesh interiors using embedded lines (depending on the {\tt gmsh} version). 
{\tt mdl=fe\_gmsh('AddLine',mdl,data,'embed',1);} will thus declare the edges found in data not as line loops defining surfaces, but as interior mesh constrains. This feature is only supported for lines specified as selections.
\end{itemize}

\item \ts{AddLine3} can be used to declare splines instead of lines in the geometry. For this command to work, \ts{beam3}~elements must be used, so that a middle node exists to be declared as the spline control point.
For this command, data can only be an element string selection.
\end{itemize}

%  - - - - - - - - - - - - - - - - - - - - - - - - - - - - - - - - - - -
\ruic{fe\_gmsh}{config}{}

The {\tt fe\_gmsh} function uses the {\tt OpenFEM} preference to launch the GMSH mesher.

\begin{verbatim}
 setpref('OpenFEM','gmsh','$HOME_GMSH/gmsh.exe')
\end{verbatim}
% $ % not to disturb emacs display

%  - - - - - - - - - - - - - - - - - - - - - - - - - - - - - - - - - - -
\ruic{fe\_gmsh}{Ver}{}

Command \ts{Ver} returns the version of {\tt gmsh}, the version is transformed into a double to simplify hierarchy handling ({\it e.g.} version 2.5.1 is transformed into 251).
This command also provides a good test to check your {\tt gmsh} setup as the output will be empty if {\tt gmsh} could not be found.

%  - - - - - - - - - - - - - - - - - - - - - - - - - - - - - - - - - - -
\ruic{fe\_gmsh}{Read}{}

{\tt fe\_gmsh('read FileName.msh')} reads a mesh from the GMSH output format. Starting with GMSH 4 {\tt .msh} is an hybrid between mesh and CAD so that direct reading is not possible. You should then use an extention {\tt .ext} field to force GMSH to export to a format that SDT supports ()

% - - - - - - - - - - - - - - - - - - - - - - - - - - - - - - - - - - - - 
\ruic{fe\_gmsh}{Write}{}

{\tt fe\_gmsh('write FileName.geo',model);} writes a model ({\tt .Node}, {\tt .Elt}) and geometry data in {\tt model.Stack{'info','GMSH'}} into a {\tt .geo} file which root name is specified as {\tt FileName} (if you use {\tt del.geo} the file is deleted on exit). 

\begin{itemize}
\item Command option \ts{-lc} allows specifying a characteristic length. You can also define a nodewise characteristic length by setting non zero values in {\tt model.Node(:,4)}.
\item Command option \ts{-multiple} can be used for automated meshing of several closed contours. The default behavior will define a single Plane Surface combining all contours, while \ts{-multiple} variant will declare each contour as a single Plane Surface.
\item Command option \ts{-keepContour} can be used to force {\tt gmsh} not to add nodes in declared line objects ({\tt Transfinite Line} feature).
\item Command option \ts{-spline} can be used (when lines have been declared using command \ts{AddLine3} from \ts{beam3}~elements) to write spline objects instead of line objects in the {\tt .geo} file
\item {\tt .stl} writing format is also supported, by using extension {\tt .stl} instead of {\tt .geo} in the command line.
\item Command option \ts{-run} allows to run {\tt gmsh} on the written file for meshing. All characters in the command after {\tt -run} will be passed to the {\tt gmsh} batch call performed. {\tt fe\_gmsh} then outputs the model processed by {\tt gmsh}, which is usually written in {\tt .msh} file format.

All text after the \ts{-run} is propagated to GMSH, see sample options below. \\
It also possible to add a different output file name \tsi{NewFile.msh}, using  {\tt model=fe\_gmsh('write NewFile.msh -run','FileName.stl')}.

\item Conversion of files through {\tt fe\_gmsh} into {\tt .msh}, or SDT/OpenFEM format is possible, for all input files readable by {\tt gmsh}. Use command option \ts{-run} and specify in second argument the file name.\\ 
For example: {\tt model=fe\_gmsh('write -run','FileName.stl')} convert {\tt .stl} to {\tt .mesh} then open into SDT/OpenFem. Some warning can occur if no {\tt FileName.mesh} is given, but without effect on the result. \\
\end{itemize}

Known options for the {\tt run} are
\begin{itemize}
\item \ts{-1} or \ts{-2} or \ts{-3}) specifies the meshing dimension.
\item \ts{-order 2} uses quadratic elements.
\item \ts{-v 0} makes a silent run. 
\item \ts{-clmax float} sets maximum mesh size, \ts{-clmin float} for minimum.
\end{itemize}

From a geometry file the simplest meshing call is illustrated below

%begindoc
\begin{verbatim}
 filename=demosdt('download-back http://www.sdtools.com/contrib/component8.step')
 RO=struct( ... % Predefine materials
   'pl',m_elastic('dbval -unit TM 1 steel'), ...
   'ext','.msh', ... % Select output format by extension (use .m MATLAB for GMSH>4)
   'sel','selelt eltname tetra10', ... % Elements to retain at end
   'Run','-3 -order 2 -clmax 3 -clmin 2 -v 0');  %RunCommand
  model=fe_gmsh('write',filename,RO);
\end{verbatim}%enddoc

It is also possible to write GMSH post-processing command lines, written at the end of the file (see the GMSH documentation) by providing a cell array (one cell by command line) in the field {\tt .Post} of the {\tt RO} structure.

%  - - - - - - - - - - - - - - - - - - - - - - - - - - - - - - - - - - -
\rmain{See also}

\noindent {\tt missread}




%-----------------------------------------------------------------------------

%------------------------------------------------------------------------------
\rtop{fe\_load}{fe_load}

Interface for the assembly of distributed and multiple load patterns\index{load}\index{b}

\rsyntax
\begin{verbatim}
 Load = fe_load(model)
 Load = fe_load(model,Case)
 Load = fe_load(model,'NoT')
 Load = fe_load(model,Case,'NoT')
\end{verbatim}

\rmain{Description}


\feload\ is used to assemble loads (left hand side vectors to FEM problems). Loads are associated with \ltt{case} structures with at least a {\tt Case.Stack} field giving all the case entries. Addition of entries to the cases, it typically done using \fecase.

To compute the load, the \hyperlink{model}{model} (a structure with fields {\tt .Node}, {\tt .Elt}, {\tt .pl}, {\tt .il}) must generally be provided with the syntax {\tt Load=fe\_load(model)}. In general simultaneous assembly of matrices and loads  detailed in \ser{feass} is preferable. 

The option \lttts{NoT} argument is used to require loads defined on the full list of DOFs rather than after constraint eliminations computed using {\tt Case.T'*Load.def}.  

The rest of this manual section describes supported load types and the associated type specific data.

\ruic{fe\_load}{curve}{} % - - - - - - - - - - - - - - - - - - - - - - - - - - -

The frequency or time dependence of a load can be specified as a {\tt data.curve} field in the load case entry. This field is a cell array specifying the dependence for each column of the applied loads. 

Each entry can be a \hyperlink{curve}{curve data structure}, or a string referring to an existing curve (stored in the {\tt model.Stack}), to describe frequency or time dependence of loads. 

Units for the load are defined through the {\tt .lab} field (in $\ve{F}=\ma{B}\ve{u}$ one assumes $u$ to be unitless thus $F$ and $B$ have the same unit systems).

\ruic{fe\_load}{DofLoad}{, \htr{fe\_load}{DofSet}} % - - - - - - - - - - - - - - - - - - - - - - - - - - -

{\sl Loads at DOFs \htt{DofLoad} and prescribed displacements \htt{DofSet}} entries are described by the following data structure

\lvs\noindent\begin{tabular}{@{}p{.15\textwidth}@{}p{.85\textwidth}@{}}
\rz{\tt bset.DOF }  &  column vector containing  a \hyperlink{adof}{DOF selection} \\
\rz{\tt bset.def }  &  matrix of load/set for each DOF (each column is a load/set case and the rows are indexed by {\tt Case.DOF }). With two DOFs, {\tt def=[1;1]} is a single input at two DOFs, while {\tt def=eye(2)} corresponds to two inputs.\\
\rz{\tt bset.name } &  optional name of the case\\
\rz{\tt bset.lab }  &  optional cell array giving label, unit label , and unit info (see \fecurve\ \ts{DataType}) for each load (column of {\tt bset.def})\\
\rz{\tt bset.curve}  &  see \ltr{fe\_load}{curve}\\
\rz{\tt bset.KeepDof}  &  when {\tt ==1} choose to keep DOF being set in the working DOF vector (not all solvers support this option)\\
\end{tabular}

Typical initialization is illustrated below

%begindoc
\begin{verbatim}
% Applying a load case in a model
 model = femesh('testubeam plot');
 % Simplified format to declare unit inputs
 model=fe_case(model,'DofLoad','ShortTwoInputs',[362.01;258.02]); 

 % General format with amplitudes at multiple DOF
 % At node 365, 1 N in x and 1.1 N in z 
 data=struct('DOF',[365.01;365.03],'def',[1;1.1]); 
 data.lab=fe_curve('datatype',13);
 model=fe_case(model,'DofLoad','PointLoad',data);

 Load = fe_load(model);
 feplot(model,Load); fecom(';scaleone;undefline;ch1 2') % display
\end{verbatim}%enddoc

When sensors are defined in SDT, loads collocated with sensors can be defined using \ltr{sensor}{DofLoadSensDof}.


\ruic{fe\_load}{FVol}{} % - - - - - - - - - - - - - - - - - - - - - - - - - - - -

\htt{FVol} entries use {\tt data} is a structure with fields

\lvs\noindent\begin{tabular}{@{}p{.15\textwidth}@{}p{.85\textwidth}@{}}
\rz{\tt data.sel}  &  \rz an \hyperlink{findelt}{element selection} (or a\hyperlink{elt}{model description matrix} but this is not acceptable for non-linear applications). \\
\rz{\tt data.dir}  &  a 3 by 1 cell array specifying the value in each global direction x, y, z. Alternatives for this specification are detailed below . The field can also be specified using {\tt .def} and {\tt .DOF} fields. \\
\rz{\tt data.lab}  &  cell array giving label, unit label , and unit info (see \fecurve\ \ts{DataType}) for each load (column of {\tt data.def})\\
\rz{\tt data.curve}  &  see \ltr{fe\_load}{curve}\\
%
\end{tabular}

Each cell of {\tt Case.dir} can give a constant value, a position dependent value defined by a string \ts{FcnName} that is evaluated using \\{\tt fv(:,jDir)=eval(FcnName)} or {\tt fv(:,jDir)=feval(FcnName,node)} if the first fails. Note that {\tt node} corresponds to nodes of the model in the global coordinate system and you can use the coordinates \ts{x,y,z} for your evaluation. The transformation to a vector defined at {\tt model.DOF} is done using {\tt vect=elem0('VectFromDir',model,r1,model.DOF)}, you can look the source code for more details.

For example 

%begindoc
\begin{verbatim}
% Applying a volumic load in a model
 model = femesh('testubeam');
 data=struct('sel','groupall','dir',[0 32 0]);
 data2=struct('sel','groupall','dir',{{0,0,'(z-1).^3.*x'}});
 model=fe_case(model,'FVol','Constant',data, ...
                     'FVol','Variable',data2);
 Load = fe_load(model); 
 feplot(model,Load);fecom(';colordataz;ch2'); % display
\end{verbatim}%enddoc


Volume loads are implemented for all elements, you can always get an example using the elements self tests, for example {\tt [model,Load]=beam1('testload')}.

Gravity loads are not explicitly implemented (care must be taken considering masses in this case and not volume). You should use the product of the mass matrix with the rigid body mode corresponding to a uniform acceleration.

\newpage
\ruic{fe\_load}{FSurf}{} % - - - - - - - - - - - - - - - - - - - - - - - - - - - -


\htt{FSurf} entries use {\tt data} a structure with fields

\lvs\noindent\begin{tabular}{@{}p{.2\textwidth}@{}p{.8\textwidth}@{}}
\rz{\tt data.sel} &  a vector of {\tt NodeId} in which the faces are contained (all the nodes in a loaded face/edge must be contained in the list). {\tt data.sel} can also contain any valid \hyperlink{findnode}{node selection} (using string or cell array format). \\
 & the optional {\tt data.eltsel} field can be used for an optional element selection to be performed before selection of faces with {\tt feutil('selelt innode',model,data.sel)}. The surface is obtained using

\begin{verbatim}
% Surface selection mechanism performed for a FSurf input
 if isfield(data,'eltsel'); 
  mo1.Elt=feutil('selelt',mo1,data.eltsel);
 end
 elt=feutil('seleltinnode',mo1, ...
     feutil('findnode',mo1,r1.sel));
\end{verbatim}

\\

\rz{\tt data.set} & Alternative specification of the loaded face by specifying a face \ts{set} name to be found in {\tt model.Stack} \\
\rz{\tt data.def}  &  a vector with as many rows as {\tt data.DOF} specifying a value for each DOF.\\
\rz{\tt data.DOF}  &  DOF definition vector specifying what DOFs are loaded. Note that pressure is DOF {\tt .19} and generates a load opposite to the outgoing surface normal. Uniform pressure can be defined using wild cards as show in the example below. \\
\rz{\tt data.lab}  &  cell array giving label, unit label ,and unit info (see \fecurve\ \ts{DataType}) for each load (column of {\tt data.def})\\
\rz{\tt data.curve}  &  see \ltr{fe\_load}{curve}\\
\rz{\tt data.type}  &  string giving \ts{'surface'} (default) or \ts{'edge'} (used in the case of 2D models where external surfaces are edges) \\
%
\end{tabular}

Surface loads are defined by surface selection and a field defined at nodes. The surface can be defined by a set of nodes ({\tt data.sel} and possibly {\tt data.eltsel} fields. One then retains faces or edges that are fully contained in the specified set of nodes. For example

%begindoc
\begin{verbatim}
% Applying a surfacing load case in a model using selectors
 model = femesh('testubeam plot');
 data=struct('sel','x==-.5', ... 
             'eltsel','withnode {z>1.25}','def',1,'DOF',.19);
 model=fe_case(model,'Fsurf','Surface load',data);
 Load = fe_load(model); feplot(model,Load);
\end{verbatim}%continuedoc


Or an alternative call with the cell array format for {\tt data.sel}
%
%continuedoc
\begin{verbatim}
% Applying a surfacing load case in a model using node lists
 data=struct('eltsel','withnode {z>1.25}','def',1,'DOF',.19);
 NodeList=feutil('findnode x==-.5',model);
 data.sel={'','NodeId','==',NodeList};
 model=fe_case(model,'Fsurf','Surface load',data);
 Load = fe_load(model); feplot(model,Load);
\end{verbatim}%enddoc


Alternatively, one can specify the surface by referring to a \ts{set} entry in {\tt model.Stack}, as shown in the following example


%begindoc
\begin{verbatim}
% Applying a surfacing load case in a model using sets
 model = femesh('testubeam plot');

 % Define a face set
 [eltid,model.Elt]=feutil('eltidfix',model);
 i1=feutil('findelt withnode {x==-.5 & y<0}',model);i1=eltid(i1);
 i1(:,2)=2; % fourth face is loaded
 data=struct('ID',1,'data',i1);
 model=stack_set(model,'set','Face 1',data);

 % define a load on face 1
 data=struct('set','Face 1','def',1,'DOF',.19);
 model=fe_case(model,'Fsurf','Surface load',data);
 Load = fe_load(model); feplot(model,Load)
\end{verbatim}%enddoc

The current trend of development is to consider surface loads as surface elements and transform the case entry to a volume load on a surface.

\rmain{See also} % - - - - - - - - - - - - - - - - - - - - - - - - - - -

\noindent \fec, \fecase, \femk


%------------------------------------------------------------------------------
\rtop{fe\_mat}{fe_mat}


\noindent Material / element property handling utilities.\index{element!property row}\index{material properties}

\rsyntax\begin{verbatim}
  out = fe_mat('convert si ba',pl);
  typ=fe_mat('m_function',UnitCode,SubType)
  [m_function',UnitCode,SubType]=fe_mat('type',typ)
  out = fe_mat('unit')
  out = fe_mat('unitlabel',UnitSystemCode)
  [o1,o2,o3]=fe_mat(ElemP,ID,pl,il)
\end{verbatim}

\rmain{Description} % - - - - - - - - - - - - - - - - - - - - - - - - - -

Material definitions can be handled graphically using the \ts{Material} tab in the model editor (see \ser{femp}). For general information about material properties, you should refer to \ser{pl}. For information about element properties, you should refer to \ser{il}. For assignment of material properties to model elements, see \ltr{feutil}{SetGroup} \ts{Mat} or \ser{feut}.

The main user accessible commands in \femat\ are listed below

\ruic{fe\_mat}{Convert}{,Unit}

The \ts{convert} command supports conversions from \ts{unit1} to \ts{unit2} with the general syntax \texline {\tt pl\_converted = fe\_mat('convert unit1 unit2',pl);}. 

For example convert from SI to BA and back
%begindoc
\begin{verbatim}
% Sample unit convertion calls
 mat = m_elastic('default'); % Default is in SI
 % convert mat.pl from SI unit to BA unit
 pl=fe_mat('convert SIBA',mat.pl)
 % for section properties IL, you need to specify -il
 fe_mat('convert -il MM',p_beam('dbval 1 circle .01'))
 % For every system but US you don't need to specify the from
 pl=fe_mat('convert BA',mat.pl)
 % check that conversion is OK
 pl2=fe_mat('convert BASI',pl);
 fprintf('Conversion roundoff error : %g\n',norm(mat.pl-pl2(1:6))/norm(pl))
 fe_mat('convertSIMM') % Lists defined units and coefficients
 coef=fe_mat('convertSIMM',2.012) % conversion coefficient for force/m^2
\end{verbatim}%enddoc

Convertion coefficients can be recovered by calling the convertion token without further arguments as \tsi{convert unit1 unit2}. For a more exploitable version, one can recover a structure providing each convertion coefficients per labelled units.

%begindoc
\begin{verbatim}
% recover convertion coefficients per unit label
r1=fe_mat('convertSIMM','struct')
\end{verbatim}%enddoc

Supported units are either those listed with {\tt  fe\_mat('convertSIMM')} which shows the index of each unit in the first column or ratios of any of these units. Thus, 2.012 means the unit 2 (force) divided by unit 12 (surface), which in this case is equivalent to unit 1 pressure.

{\tt out=fe\_mat('unitsystem')} returns a {\tt struct} containing the information characterizing standardized unit systems supported in the universal file format.

\vs\noindent\begin{tabular}{@{}p{.05\textwidth}@{}p{.1\textwidth}@{}p{.35\textwidth}@{}p{.05\textwidth}@{}p{.1\textwidth}@{}p{.35\textwidth}@{}}
%
ID&  & Length and Force & ID & & \\
1&\rz\ts{SI} & Meter, Newton  & 7&\rz\ts{IN} & Inch, Pound force \\
2&\rz\ts{BG} & Foot, Pound f  & 8&\rz\ts{GM} & Millimeter, kilogram force\\
3&\rz\ts{MG} & Meter, kilogram f & 9&\rz\ts{TM} & Millimeter, Newton\\
4&\rz\ts{BA} & Foot, poundal   & 9&\rz\ts{MU} & micro-meter, kiloNewton\\
5&\rz\ts{MM} & Millimeter, milli-newton & 9&\rz\ts{US} & User defined\\
6&\rz\ts{CM} & Centimeter, centi-newton \\
%
\end{tabular}
%
Unit codes 1-8 are defined in the universal file format specification and thus coded in the material/element property type (column 2). Other unit systems are considered user types and are associated with unit code 9. With a unit code 9, {\tt fe\_mat} \ts{convert} commands must give both the initial and final unit systems.

{\tt out=fe\_mat('unitlabel',UnitSystemCode)} returns a standardized list of unit labels corresponding in the unit system selected by the {\tt UnitSystemCode} shown in the table above. To recover a descriptive label list (like {\tt density}), use {\tt US} as {\tt UnitSystemCode}.

When defining your own properties material properties, automated unit conversion is implemented automatically through tables defined in the \ltr{p\_fun}{PropertyUnitType} command.

\ruic{fe\_mat}{GetPl}{\htr{fe\_mat}{GetIl}} % - - - - - - - - - - - - - - - - - - - - - - - - - - - - - -
 
{\tt pl = fe\_mat('getpl',model)} is used to robustly return the material property matrix {\tt pl} (see \ser{pl}) independently of the material input format.

Similarly {\tt il = fe\_mat('getil',model)} returns the element property matrix \hyperlink{il}{{\tt il}}.

\ruic{fe\_mat}{Get}{[Mat,Pro]} % - - - - - - - - - - - - - - - - - - - - - - - - - - - - - -
 
{\tt r1 = fe\_mat('GetMat {\ti Param}',model)}
This command can be used to extract given parameter \tsi{Param} value in the model properties.
For example one can retrieve density of matid 111 as following\\
{\tt rho=fe\_mat('GetMat 111 rho',model);}\\

\ruic{fe\_mat}{Set}{[Mat,Pro]} % - - - - - - - - - - - - - - - - - - - - - - - - - - - - - -
 
{\tt r1 = fe\_mat('SetMat {\ti MatId}  {\ti Param}={\ti value}',model)}\\
{\tt r1 = fe\_mat('SetPro {\ti ProId}  {\ti Param}={\ti value}',model)}\\

This command can be used to set given parameter \tsi{Param} at the value \tsi{value} in the model properties. 
For example one can set density of matid 111 at 5000 as following\\
{\tt rho=fe\_mat('SetMat 111 rho=5000',model);}\\

\ruic{fe\_mat}{Type}{} % - - - - - - - - - - - - - - - - - - - - - - - - - - - - - -

 The type of a material or element declaration defines the function used to handle it.  

{\tt typ=fe\_mat('m\_function',UnitCode,SubType)} returns a real number which codes the material function, unit and sub-type. 
\begin{itemize}
\item \ts{m\_function}s are \ts{.m} or \ts{.mex} files whose name starts with \ts{m\_}. 

They are used to interpret the material properties. 

See as an example the \melastic\ reference.

\item The {\tt UnitCode} is a number between 1 and 9 (or the associated two letters labels, see table in \ltr{fe\_mat}{Convert}). 

It gives the unit of the material data to ensure coherent if different units are used between material properties.

\item The {\tt SubType} is a also a number between 1 and 9.

It allows selection of material subtypes within the same material function (for example, \melastic\ supports subtypes : 1 isotropic solid, 2 fluid, 3 anisotropic solid, ...).

\end{itemize}

{\bf Note} : the code type {\tt typ} should be stored in column 2 of material property rows (see \ser{pl}).

To decode a {\tt typ} number, us command

{\tt [m\_function,UnitCode,SubType]=fe\_mat('typem',typ)}

Similarly, element properties are handled by {\tt p\_} functions which also use \femat\ to code the type (see \pbeam, \pshell\ and \psolid).

\ruic{fe\_mat}{ElemP}{} % - - - - - - - - - - - - - - - - - - - - - - - - - - - - - - -

Calls of the form {\tt [o1,o2,o3]=fe\_mat(ElemP,ID,pl,il)} are used by element functions to request constitutive matrices. This call is really for developers only and you should look at the source code of each element.

% a few constitutive law formulas would be useful

\rmain{See also}

\noindent \melastic, \pshell, element functions in chapter~\ref{s*eltfun}, \ltr{feutil}{SetMat}

%------------------------------------------------------------------------------
\rtop{fe\_mknl, fe\_mk}{fe_mknl}

\noindent Assembly of finite element model matrices.\index{assembly}

\rsyntax\begin{verbatim}
 [m,k,mdof] = fe_mknl(model);
 [Case,model.DOF]=fe_mknl('init',model); 
 mat=fe_mknl('assemble',model,Case,def,MatType);
\end{verbatim}

\rmain{Description}

\begin{SDT}
 {\bf The exact procedure used for assembly often needs to be optimized in detail to avoid repetition of unnecessary steps. SDT typically calls an internal procedure} implemented in {\tt fe\_caseg} \ts{Assemble} and detailed in~\ser{feass}. This documentation is meant for low level calls.
\end{SDT}


{\tt fe\_mknl} (and the obsolete \femk) take models and return assembled matrices and/or right hand side vectors. 

Input arguments are

\begin{itemize}

\item \hyperlink{model}{{\tt model}} a model data structure describing \hyperlink{node}{nodes}, \hyperlink{elt}{elements}, \hyperlink{pl}{material properties}, \hyperlink{il}{element properties}, and possibly a \ltt{case}.

\item \ltt{case} data structure describing loads, boundary conditions, etc. This may be stored in the model and be retrieved automatically using {\tt fe\_case(model,'GetCase')}. 

\item \hyperlink{def}{{\tt def}} a data structure describing the current state of the model for model/residual assembly using {\tt fe\_mknl}. {\tt def} is expected to use model DOFs. If {\tt Case} DOFs are used, they are reexpanded to model DOFs using {\tt def=struct('def',Case.T*def.def,'DOF',model.DOF)}. This is currently used for geometrically non-linear matrices.

\item {\tt MatType} or {\tt Opt} describing the desired output, appropriate handling of linear constraints, etc. 

\end{itemize}

Output formats are

\begin{itemize}

\item {\tt model} with the additional field {\tt model.K} containing the matrices. The corresponding types are stored in {\tt model.Opt(2,:)}. The {\tt model.DOF} field is properly filled.
\item {\tt [m,k,mdof]} returning both mass and stiffness when {\tt Opt(1)==0}
\item {\tt [Mat,mdof]} returning a matrix with type specified in {\tt Opt(1)}, see {\tt MatType} below.

\end{itemize}

{\tt mdof} is the \hyperlink{mdof}{DOF definition vector} describing the DOFs of output matrices. 

When fixed boundary conditions or linear constraints are considered, {\tt mdof} is equal to the set of master or independent degrees of freedom {\tt Case.DOF} which can also be obtained with \texline {\tt fe\_case(model,'gettdof')}. Additional unused DOFs can then be eliminated unless {\tt Opt(2)} is set to 1 to prevent that elimination. To prevent constraint elimination in {\tt fe\_mknl} use \ts{Assemble NoT}.

In some cases, you may want to assemble the matrices but not go through the constraint elimination phase. This is done by setting {\tt Opt(2)} to 2. {\tt mdof} is then equal to {\tt model.DOF}.

This is illustrated in the example below

%begindoc
\begin{verbatim}
% Low level assembly call with or without constraint resolution
 model =femesh('testubeam');
 model.DOF=[];% an non empty model.DOF would eliminate all other DOFs
 model =fe_case(model,'fixdof','Base','z==0');
 model = fe_mk(model,'Options',[0 2]); 
 [k,mdof] = fe_mk(model,'options',[0 0]); 
 fprintf('With constraints %i DOFs\n',size(k,1)); 
 fprintf('Without          %i DOFs',size(model.K{1},1));
 Case=fe_case(model,'gett');
 isequal(Case.DOF,mdof) % mdof is the same as Case.DOF
\end{verbatim}%enddoc


For other information on constraint handling see~\ser{mpc}.


Assembly is decomposed in two phases. The initialization prepares everything that will stay constant during a non-linear run. The assembly call performs other operations.

\ruic{fe\_mknl}{Init}{}

The {\tt fe\_mknl} \ts{Init} phase initializes the {\tt Case.T} (basis of vectors verifying linear constraints see \ser{mpc}, resolution calls \ltr{fe\_case}{Get}\ts{T}, {\tt Case.GroupInfo} fields (detailed below) and {\tt Case.MatGraph} (preallocated sparse matrix associated with the model topology for optimized (re)assembly). \texline {\tt Case.GroupInfo} is a cell array with rows giving information about each element group in the model (see \ser{GroupInfo} for details). 

Command options are the following
\begin{itemize}
\item \ts{NoCon} {\tt Case = fe\_mknl('initNoCon', model)} can be used to initialize the case structure without building the matrix connectivity (sparse matrix with preallocation of all possible non zero values).
\item  \ts{Keep} can be used to prevent changing the {\tt model.DOF} DOF list. This is typically used for submodel assembly.
\item  \ts{-NodePos} saves the {\tt NodePos} node position index matrix for a given group in its {\tt EltConst} entry. 
\item  \ts{-gstate} is used force initialization of group stress entries.
\item  \ts{new} will force a reset of {\tt Case.T}.

\end{itemize}

The initialization phase is decomposed into the following steps

\begin{enumerate}
 \item Generation of a complete list of DOFs using the {\tt feutil('getdof',model)} call.
 \item get the material and element property tables in a robust manner (since some data can be replicated between the {\tt pl,il} fields and the \ts{mat,pro} stack entries. Generate node positions in a global reference frame.

 \item For each element group, build the {\tt GroupInfo} data (DOF positions).
 \item For each element group, determine the unique pairs of {\tt [MatId ProId]} values in the current group of elements and build a separate {\tt integ} and {\tt constit} for each pair. One then has the constitutive parameters for each type of element in the current group. {\tt pointers} rows 6 and 7 give for each element the location of relevant information in the \ltt{integ} and \ltt{constit} tables.

This is typically done using an {\tt [integ,constit,ElMap]=ElemF('integinfo')} command, which in most cases is really being passed directly to a {\tt p\_fun('BuildConstit')} command. 

\ltt{ElMap} can be a structure with fields beginning by {\tt RunOpt\_},  {\tt Case\_} and {\tt eval} which allows execution of specific callbacks at this stage.

\item For each element group, perform other initializations as defined by evaluating the callback string obtained using {\tt elem('GroupInit')}. For example, initialize integration rule data structures \ltt{EltConst}, define local bases or normal maps in \ltt{InfoAtNode}, allocate memory for internal state variables in \ltt{gstate}, ...

\item If requested (call without \ts{NoCon}), preallocate a sparse matrix to store the assembled model. This topology assumes non zero values at all components of element matrices so that it is identical for all possible matrices and constant during non-linear iterations.
 
\end{enumerate}

% - - - - - - - - - - - - - - - - - - - - - - - - - - - - - - - - - -
\ruic{fe\_mknl}{Assemble}{ [ , NoT]}

The second phase, assembly, is optimized for speed and multiple runs (in non-linear sequences it is repeated as long as the element connectivity information does not change). In \femk\ the second phase is optimized for robustness.  The following example illustrates the interest of multiple phase assembly

%begindoc
\begin{verbatim}
% Low level assembly calls
 model =femesh('test hexa8 divide 100 10 10');
 % traditional FE_MK assembly
 tic;[m1,k1,mdof] = fe_mk(model);toc

 % Multi-step approach for NL operation
 tic;[Case,model.DOF]=fe_mknl('init',model);toc
 tic;
 m=fe_mknl('assemble',model,Case,2);
 k=fe_mknl('assemble',model,Case,1);
 toc
\end{verbatim}%enddoc


% - - - - - - - - - - - - - - - - - - - - - - - - - - - - - - - - - -
\ruic{fe\_mknl}{MatType}{: matrix identifiers}

Matrix types (sometimes also noted \httts{mattyp} or \httts{MatType} in the documentation) are numeric indications of what needs to be computed during assembly. Currently defined types for OpenFEM are

\begin{itemize}
\item {\tt 0} mass and stiffness assembly.  {\tt 1} stiffness, {\tt 2} mass, {\tt 3} viscous damping,  {\tt 4} hysteretic damping 
\item {\tt 5} tangent stiffness in geometric non-linear mechanics (assumes a static state given in the call. In SDT calls (see~\ser{feass}), the case entry {\tt 'curve','StaticState'} is used to store the static state.
\begin{SDT}
\item {\tt 3} viscous damping. Uses \ts{info,Rayleigh} case entries if defined, see example in~\ser{hyst}. 
\item {\tt 4} hysteretic damping. Weighs the stiffness matrices associated with each material with the associated loss factors. These are identified by the key word {\tt Eta} in \lts{p\_fun}{PropertyUnitType} commands.
\end{SDT}
\item {\tt 7} gyroscopic coupling in the body fixed frame, {\tt 70} gyroscopic coupling in the global frame. {\tt 8} centrifugal softening. 
\item {\tt 9} is reserved for non-symmetric stiffness coupling (fluid structure, contact/friction, ...);
\item {\tt 20} to assemble a lumped mass instead of a consistent mass although using common integration rules at Gauss points.
\item {\tt 100} volume load, {\tt 101} pressure load, {\tt 102} inertia load, {\tt 103} initial stress load. Note that some load types are only supported with the {\tt mat\_og} element family; 
\item {\tt 200} stress at node, {\tt 201} stress at element center, {\tt 202} stress at Gauss point
\item {\tt 251} energy associated with matrix type 1 (stiffness), {\tt 252} energy associated with matrix type 2 (mass), ...  
\item {\tt 300} compute initial stress field associated with an initial deformation. This value is set in {\tt Case.GroupInfo\{jGroup,5\}} directly (be careful with the fact that such direct modification INPUTS is not a MATLAB standard feature). {\tt 301} compute the stresses induced by a thermal field. For pre-stressed beams, {\tt 300} modifies  {\tt InfoAtNode=Case.GroupInfo\{jGroup,7\}}.
\begin{SDT}
\item {\tt -1, -1.1} submodel selected by parameter, see~\ser{feass}.
\item {\tt -2, -2.1} specific assembly of superelements with label split, see~\ser{feass}.
\end{SDT}

\end{itemize}

% - - - - - - - - - - - - - - - - - - - - - - - - - - - - - - - - - -
\ruic{fe\_mknl}{NodePos}{}

{\tt NodePos=fe\_mknl('NodePos',NNode,elt,cEGI,ElemF)} is used to build the node position index matrix for a given group. {\tt ElemF} can be omitted. {\tt NNode} can be replaced by {\tt node}. 

% - - - - - - - - - - - - - - - - - - - - - - - - - - - - - - - - - -
\ruic{fe\_mknl}{nd}{}

{\tt nd=fe\_mknl('nd',DOF);} is used to build and optimized object to get indices of DOF in a DOF list.

% - - - - - - - - - - - - - - - - - - - - - - - - - - - - - - - - - -
\ruic{fe\_mknl}{OrientMap}{}

This command is used to build the \ltt{InfoAtNode} entry. The {\tt 'Info','EltOrient'} field is a possible stack entry containing appropriate information before step 5 of the {\tt init} command. The preferred mechanism is to define an material map associated to an element property as illustrated in \ser{VectFromDir}. 


% - - - - - - - - - - - - - - - - - - - - - - - - - - - - - - - - - -
\ruic{fe\_mknl}{of\_mk}{}

{\tt of\_mk} is the mex file supporting assembly operations. You can set the number of threads used with {\tt  of\_mk('setomppro',8)}.

% - - - - - - - - - - - - - - - - - - - - - - - - - - - - - - - - - -
\ruic{fe\_mk}{obsolete}{}


\rsyntax\begin{verbatim}
 model      = fe_mk(model,'Options',Opt)
 [m,k,mdof] = fe_mk( ... ,[0       OtherOptions])
 [mat,mdof] = fe_mk( ... ,[MatType OtherOptions])
\end{verbatim}


\femk\ options are given by calls of the form {\tt fe\_mk(model,'Options',Opt)} or the obsolete \texline {\tt fe\_mk(node,elt,pl,il,[],adof,opt)}.

\lvs\noindent\begin{tabular}{@{}p{.15\textwidth}@{}p{.85\textwidth}@{}}
%
\rz{\tt opt(1)} & \rz{\lts{fe\_mknl}{MatType}} see above \\
\rz{\tt opt(2)} &  if active DOFs are specified using {\tt model.DOF} (or the obsolete call with {\tt adof}), DOFs in
 {\tt model.DOF} but not used by the model (either linked to no element or with a zero on the matrix or both the mass and stiffness diagonals) are eliminated unless {\tt opt(2)} is set to {\tt 1} (but case constraints are then still considered) or {\tt 2} (all constraints are ignored). \\
\rz{\tt opt(3)} &  Assembly method (0 default, 1 symmetric mass and stiffness (OBSOLETE), 2 disk (to be preferred for large problems)). The disk assembly method creates temporary files using the \sdtdef\ \ts{tempname} command.  This minimizes memory usage so that it should be preferred for very large models.
 \\
\rz{\tt opt(4)} & \rz{\tt 0} (default) nothing done for less than 1000 DOF method 1 otherwise. {\tt 1} DOF numbering optimized using current \ofact\ {\tt SymRenumber} method. Since new solvers renumber at factorization time this option is no longer interesting. 
\end{tabular}



{\tt [m,k,mdof]=fe\_mk(node,elt,pl,il)} returns mass and stiffness matrices when given  \hyperlink{node}{nodes}, \hyperlink{elt}{elements}, \hyperlink{pl}{material properties}, \hyperlink{il}{element properties} rather than the corresponding model data structure. 

{\tt [mat,mdof]=fe\_mk(node,elt,pl,il,[],adof,opt)} lets you specify
DOFs to be retained with {\tt adof} (same as defining a \ltt{case} entry with {\tt \{'KeepDof', 'Retained', adof\}}). 

 These formats are kept for backward compatibility but they do not allow support of local coordinate systems, handling of boundary conditions through cases, ...


\rmain{Notes}

\femk\ no longer supports complex matrix assembly in order to allow a number of speed optimization steps. You are thus expected to assemble the real and imaginary parts successively.


\rmain{See also}

\noindent Element functions in chapter~\ref{s*eltfun}, 
\fec, \feplot, \feeig, \upcom, \femat, \femesh, etc.








% --------------------------------------------------------------------
%       Copyright (c) 2001-2014 by INRIA and SDTools, All Rights Reserved.
%       Use under OpenFEM trademark.html license and LGPL.txt library license
%       $Revision: 1.32 $  $Date: 2016/09/02 16:38:50 $

%---------------------------------------------------------------------------
\rtop{fe\_stress}{fe_stress}

\noindent Computation of stresses and energies for given deformations.

\rsyntax\begin{verbatim}
Result = fe_stress('Command',MODEL,DEF)
  ...  = fe_stress('Command',node,elt,pl,il, ...)
  ...  = fe_stress( ... ,mode,mdof)
\end{verbatim}

\rmain{Description}

You can display stresses and energies directly using \ltr{fecom}{ColorData}\ts{Ener} commands and use \festress\ to analyze results numerically. {\tt MODEL} can be specified by four input arguments {\tt node}, {\tt elt}, {\tt pl} and {\tt il} (those used by \femk, see also \ser{node} and following), a data structure with fields {\tt .Node}, {\tt .Elt}, {\tt .pl}, {\tt .il}, or a database wrapper with those fields.

The deformations {\tt DEF} can be specified using two arguments: {\tt mode} and associated DOF definition vector {\tt mdof} or a structure array with fields {\tt .def} and {\tt .DOF}.

% - - - - - - - - - - - - - - - - - - - -
\ruic{fe\_stress}{Ener}{ [m,k]{\ti ElementSelection}} 

{\sl Element energy computation}.  For a given shape, the levels of strain and kinetic energy in different elements give an indication of how much influence the modification of the element properties may have on the global system response. This knowledge is a useful analysis tool to determine regions that may need to be updated in a FE model. Accepted command options are 

\begin{itemize}
\item \ts{-MatDes}\tsi{val} is used to specify the matrix type (see \lts{fe\_mknl}{MatType}). \ts{-MatDes 5} now correctly computes energies in pre-stressed configurations. 
\item \ts{-curve} should be used to obtain energies in the newer \hyperlink{curve}{curve} format. {\tt Ek.X\{1\}} gives as columns \ts{EltId,vol,MatId,ProId,GroupId} so that passage between energy and energy density can be done dynamically. 

\item \ts{ElementSelection} (see the \hyperlink{findelt}{element selection} commands) used to compute energies in part of the model only. The default is to compute energies in all elements. A typical call to get the strain energy in a material of ID 1 would then be
{\tt R1=fe\_stress('Ener -MatDes1 -curve matid1',model,def);}

\end{itemize}

Obsolete options are

\begin{itemize}
\item \ts{m}, \ts{k} specify computation of kinetic or strain energies. For backward compatibility, \festress\ returns {\tt  [StrainE,KinE]} as two arguments if no element selection is given.
\item \ts{dens} changes from the default where the element energy and {\bf not} energy density is computed. This may be more appropriate when displaying energy levels for structures with uneven meshes.

\item Element energies are computed for deformations in {\tt DEF} and the result is returned in the data structure {\tt RESULT} with fields {\tt .data} and {\tt .EltId} which specifies which elements were selected. A {\tt .vol} field gives the volume or mass of each element to allow switching between energy and energy density.

\end{itemize}



The strain and kinetic energies of an element are defined by

\begin{displaymath}
  E^e_{strain}=\frac{1}{2}\phi^TK_{element}\phi \hbox{\ \ and \ \ } 
  E^e_{kinetic}=\frac{1}{2}\phi^TM_{element}\phi
\end{displaymath}


For complex frequency responses, one integrates the response over one cycle, which corresponds to summing the energies of the real and imaginary parts and using a factor 1/4 rather than 1/2. 
 
\begin{SDT}
\ruic{fe\_stress}{feplot}{} % - - - - - - - - - - - - - - - - - - - -

\feplot\ allows the visualization of these energies using a color coding. You should compute energies once, then select how it is displayed. Energy computation clearly require material and element properties to be defined in \lts{fecom}{InitModel}.

The earlier high level commands \ltr{fecom}{ColorData}\ts{K} or \ts{ColorDataM} don't store the result and thus tend to lead to the need to recompute energies multiple times. The preferred strategy is illustrated below.

\begin{verbatim}
% Computing, storing and displaying energy data
 demosdt('LoadGartFe'); % load model,def 
 cf=feplot(model,def);cf.sel='eltname quad4';fecom ch7
 % Compute energy and store in Stack
 Ek=fe_stress('ener -MatDes 1 -curve',model,def)
 cf.Stack{'info','Ek'}=Ek;
 % Color is energy density by element
 feplot('ColorDataElt  -dens -ColorBarTitle "Ener Dens"',Ek);
 % Color by group of elements
 cf.sel={'eltname quad4', ... % Just the plates
   'ColorDataElt -ColorBarTitle "ener" -bygroup -edgealpha .1', ...
   Ek}; % Data with no need to recompute
 fecom(cf,'ColorScale One Off Tight') % Default color scaling for energies
\end{verbatim}

Accepted \ts{ColorDataElt} options are

\begin{itemize}
\item \ts{-dens} divides by element volume. Note that this can be problematic for mixed element types (in the example above, the volume of  {\tt celas} springs is defined as its length, which is inappropriate here).
\item \ts{-frac} divides the result by the total energy (equal to the square of the modal frequency for normal modes).
\item \ts{-byGroup} sums energies within the same element group. Similarly \ts{-byProId} and \ts{-byMatId} group by property identifier. When results are grouped, the {\tt fecom('InfoMass')} command gives a summary of results.

\end{itemize}

The color animation mode is set to {\tt ScaleColorOne}.  

\end{SDT}

\ruic{fe\_stress}{Stress}{} % - - - - - - - - - - - - - - - - - - - - - - - - - - - - - - -

{\tt out=fe\_stress('stress {\ti CritFcn Options}',MODEL,DEF,{\ti EltSel})} returns the stresses evaluated at elements of {\tt Model} selected by {\ti EltSel}. 

The \tsi{CritFcn} part of the command string is used to select a criterion. Currently supported criteria are

\lvs\begin{tabular}{@{}p{.15\textwidth}@{}p{.85\textwidth}@{}}
%
\rz\ts{sI, sII, sIII} &  principal stresses from max to min. \ts{sI} is the default.\\
\rz\ts{mises} &  Returns the von Mises stress (note that the plane strain case is not currently handled consistently).\\
\rz\ts{-comp }\tsi{i} &  Returns the stress components of index \tsi{i}. This component index is giving in the engineering rather than tensor notation (before applying the {\tt TensorTopology} transformation).
%
\end{tabular}

Supported command \ts{Options} (to select a restitution method, ...) are
\begin{itemize}

\item \ts{AtNode}  average stress at each node (default). Note this is not currently weighted by element volume and thus quite approximate. Result is a structure with fields {\tt .DOF} and {\tt .data}. \\
\item \ts{AtCenter}  stress at center or mean stress at element stress restitution points. Result is a structure with fields {\tt .EltId} and {\tt .data}.\\
\item\ts{AtInteg}  stress at integration points ({\tt *b} family of elements).\\
\item\ts{Gstate}  returns a case with {\tt Case.GroupInfo\{jGroup,5\}} containing the group \ltt{gstate}. This will be typically used to initialize stress states for non-linear computations. For multiple deformations, {\tt gstate} the first {\tt nElt} columns correspond to the first deformation.\\
\item \ts{-curve}  returns the output using the \ltt{curve} format.
%
\end{itemize}

\begin{SDT}
The \ltr{fecom}{ColorData}\ts{Stress} directly calls \festress\ and displays the result. For example, run the basic element test {\tt q4p} \ts{testsurstress}, then display various stresses using
%begindoc
\begin{verbatim}
% Using stress display commands
 q4p('testsurstress')
 fecom('ColorDataStress atcenter')
 fecom('ColorDataStress mises')
 fecom('ColorDataStress sII atcenter')
\end{verbatim}%enddoc
\end{SDT}

To obtain strain computations, use the strain material as shown below.

%begindoc
\begin{verbatim}
% Accessing stress computation data (older calls)
 [model,def]=hexa8('testload stress');
 model.pl=m_elastic('dbval 100 strain','dbval 112 strain');
 model.il=p_solid('dbval 111 d3 -3');
 data=fe_stress('stress atcenter',model,def)
\end{verbatim}%enddoc

% - - - - - - - - - - - - - - - - - - - - - - - - - - - - - - - - - - - - - -
\ruic{fe\_stress}{CritFcn}{} 

For stress processing, one must often distinguish the raw stress components associated with the element formulation and the desired output. {\tt CritFcn} are callback functions that take a local variable {\tt r1} of dimensions (stress components $\times$ nodes $\times$ deformations) and to replace this variable with the desired stress quantity(ies). For example 

\begin{verbatim}
% Sample declaration of a user defined stress criterium computation
 function out=first_comp(r1)
  out=squeeze(r1(1,:,:,:));
\end{verbatim}

would be a function taking the first component of a computed stress. \swref{fe\_stress(''Principal'')} provides stress evaluations classical for mechanics.

For example, a list of predefined {\tt CritFcn} callback :
\begin{itemize}
\item Von Mises : {\tt CritFcn='r1=of\_mk(''StressCrit'',r1,''VonMises'');lab=''Mises'';';}
\item YY component : {\tt CritFcn='r1=r1(2,:,:,:);lab=''Syy'';'}
\end{itemize}


Redefining the {\tt CritFcn} callback is in particular used in the \ts{StressCut} functionality, see \ser{corstress}.

\rmain{See also}

\noindent \femk, \feplot, \fecom
 


%------------------------------------------------------------------------------
\rtop {fe\_super}{fe_super}

\noindent Generic element function for superelement support.
\index{element!function}

\rmain{Description}

The non current {\sl SDT 3} version of this function is included in OpenFEM. Use the {\tt help} \ts{fecom} command to get help.


\rmain{See also}

\noindent \fesuper, the {\tt d\_cms2} demonstration

%%% Local Variables: 
%%% mode: latex
%%% TeX-master: "tr"
%%% TeX-master: t
%%% End: 

