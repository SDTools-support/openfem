%       Copyright (c) 2001-2017 by INRIA and SDTools, All Rights Reserved.
%       Use under OpenFEM trademark.html license and LGPL.txt library license
%       $Revision: 1.69 $  $Date: 2025/06/25 17:34:38 $

%------------------------------------------------------------------------------
\rtop {fe\_curve}{fe_curve}

\noindent Generic handling of curves and signal processing utilities

\rsyntax\begin{verbatim}
  out=fe_curve('command',MODEL,'Name',...);
\end{verbatim}\nlvs
 

\ruic{fe\_curve}{Commands}{}

\fecurve\ is used to handle curves and do some basic signal processing. The format for curves is described in \ser{curve}. The \iiplot\ interface may be used to plot curves and a basic call would be {\tt iiplot(Curve)} to plot curve data structure {\tt Curve}.

Accepted commands are


\ruic{fe\_curve}{bandpass}{ {\ti Unit f\_min f\_max}} %  - - - - - - - - - - - - - - - - - - - - - - - - - -
{\tt out=fe\_curve('BandPass {\ti Unit f\_min f\_max}',signals);} \\
realizes a true bandpass filtering (i.e. using {\tt fft()} and {\tt ifft()}) of time signals contained in curves {\tt signals}. {\ti \ts{f\_min}} and {\ti \ts{f\_max}} are given in units {\ti \ts{Unit}}, whether Hertz(\ts{Hz}) or Radian(\ts{Rd}). With no {\ti \ts{Unit}}, {\tt f\_min} and {\tt f\_max} are assumed to be in Hertz. 

%begindoc
\begin{verbatim}
% apply a true bandpasss filter to a signal
out=fe_curve('TestFrame');% 3 DOF oscillator response to noisy input
fe_curve('Plot',out{2});  % "unfiltered" response
filt_disp=fe_curve('BandPass Hz 70 90',out{2}); % filtering 
fe_curve('Plot',filt_disp); title('filtered displacement');
\end{verbatim}%enddoc


\ruic{fe\_curve}{datatype}{ [,cell]} %  - - - - - - - - - - - - - - - - - - -

{\tt out=fe\_curve('DataType',DesiredType);} \\
returns a data structure describing the data type, useful to fill {\tt .xunit} and {\tt .yunit} fields for curves definition. {\tt DesiredType} could be a string or a number corresponding to the desired type.  With no {\tt DesiredType}, the current list of available types is displayed. One can specify the unit with {\tt out=fe\_curve('DataType',DesiredType,'unit');}.

{\tt DataTypeCell} returns a cell array rather than data structure to follow the specification for \hyperlink{curve}{curve data structures}. Command option \ts{-label}\tsi{"lab"} allows directly providing a custom label named \tsi{lab} in the data type.

\ruic{fe\_curve}{get}{Curve,Id,IdNew} %  - - - - - - - - - - - - - - - - - - - - - - - - - -
\begin{itemize}
\item \ts{GetCurve}: recover curve by name
{\tt curve=fe\_curve('getcurve',model,'curve\_name');} \\
extracts curve \ts{curve\_name} from {\tt model.Stack} or the possible curves attached to a load case. If the user does not specify any name, all the curves are returned in a cell array.
\item \ts{GetId}\tsi{val}: recover curve by {\tt .ID} field, equal to \tsi{val}. {\tt curve=fe\_curve('GetId val',model);}
\item \ts{GetIdNew}: generate a new identifier interger unused in any curve of the model.
{\tt i1=fe\_curve('GetIdNew',model);}
\end{itemize}

\ruic{fe\_curve}{h1h2}{ {\ti input\_channels}} %  -  - - - - - - - - - - - - - - - - - -
{\tt FRF=fe\_curve('H1H2 input\_channels',frames,'window');} \\
computes H1 and H2 FRF estimators along with the coherence from time signals contained in cell array {\tt frames} using window \ts{window}. The time vector is given in {\tt frames\{1\}.X} while {\ti input\_channels} tells which columns of in {\tt frames\{1\}.Y} are inputs. If more than one input channel is specified, true MIMO FRF estimation is done, and H$\nu$ is used instead of H2. When multiple frames are given, a mean estimation of FRF is computed. 

Note: To ensure the proper assembly of H1 and H$\nu$ in MIMO FRF estimation case, a weighing based on maximum time signals amplitude is used. To use your own, use \\
{\tt FRF=fe\_curve('H1H2 input\_channels',frames,window,weighing);} \\
where {\tt weighing} is a vector containing weighing factors for each channel. To avoid weighing, use \texline {\tt FRF=fe\_curve('H1H2 input\_channels',frames,window,0);} . For an example see \texline {\tt sdtweb('start\_time2frf','h1h2')}


\ruic{fe\_curve}{noise}{} %  - - - - - - - - - - - - - - - - - - - - - - - - - -

OBSOLETE : use \ltr{fe\_curve}{Test}\ts{Noise} instead

{\tt noise=fe\_curve('Noise',Nw\_pt,fs,f\_max);} \\
computes a {\tt Nw\_pt} points long time signal corresponding to a ``white noise'', with sample frequency {\tt fs} and a unitary power spectrum density until {\tt f\_max}. {\tt fs/2} is taken as {\tt f\_max} when not specified. The general shape of noise power spectrum density, extending from {\tt 0 } to {\tt fs/2}, can be specified instead of {\tt f\_max}.

%begindoc
\begin{verbatim}
% computes a 2 seconds long white noise, 1024 Hz of sampling freq.
% with "rounded" shape PSD    
fs=1024; sample_length=2;
Shape=exp(fe_curve('window 1024 hanning'))-1; 
noise_h=fe_curve('noise',fs*sample_length,fs,Shape);
noise_f=fe_curve('fft',noise_h);
figure(1);
subplot(211);fe_curve('plot -gca',noise_h);axis tight;
subplot(212);fe_curve('plot -gca',noise_f);axis tight;
\end{verbatim}%enddoc


\ruic{fe\_curve}{plot}{} %  - - - - - - - - - - - - - - - - - - - - - - - - - -

{\tt fe\_curve('plot',curve);} plots the curve {\tt curve}. \\
{\tt fe\_curve('plot',fig\_handle,curve);} plots {\tt curve} in the figure with handle {\tt fig\_handle}.\\
{\tt fe\_curve('plot',model,'curve\_name');} plots the curve of {\tt model.Stack} named \ts{curve\_name}.\\
{\tt fe\_curve('plot',fig\_handle,model,curve\_name);} plots curve named {\tt curve\_name} stacked in {\tt .Stack} field of model {\tt model}.

%begindoc
\begin{verbatim}
% Plot a fe_curve signal
% computes a 2 seconds long white noise, 1024 Hz of sampling freq.
fs=1024; sample_length=2;
noise=fe_curve('noise',fs*sample_length,fs);noise.Xlab{1}={'Time','s',[]}
noise.Xlab{2}={'Force','N',fe_curve('DataType','Excit. force')};
noise.name='Input force';cdm.plot(noise)
\end{verbatim}%enddoc
 

\ruic{fe\_curve}{resspectrum}{ [{\ti True, Pseudo}] [{\ti Abs., Rel.}] [{\ti Disp., Vel., Acc.}]} %  - - - - - - - - - - - - - - - - - - - - - - - - - -

{\tt out=fe\_curve('ResSpectrum',signal,freq,damp);} \\
computes the response spectrum associated to the time signal given in {\tt signal}. Time derivatives can be obtained with option \ts{-v} or \ts{-a}. Time integration with option \ts{+v} or \ts{+a}. Pseudo derivatives with option \ts{PseudoA} or \ts{PseudoV}. {\tt freq} and {\tt damp} are frequencies (in Hz) and damping ratios vectors of interest for the response spectra. For example

%begindoc
\begin{verbatim}
wd=fileparts(which('d_ubeam'));
% read the acceleration time signal
bagnol_ns=fe_curve(['read' fullfile(wd,'bagnol_ns.cyt')]);

% read reference spectrum
bagnol_ns_rspec_pa= fe_curve(['read' fullfile(wd,'bagnol_ns_rspec_pa.cyt')]);
% compute response spectrum with reference spectrum frequencies
% vector and 5% damping
RespSpec=fe_curve('ResSpectrum PseudoA',...
                  bagnol_ns,bagnol_ns_rspec_pa.X/2/pi,.05);

fe_curve('plot',RespSpec); hold on;
plot(RespSpec.X,bagnol_ns_rspec_pa.Y,'r');
legend('fe\_curve','cyberquake');
\end{verbatim}%enddoc


\ruic{fe\_curve}{returny}{} %  - - - - - - - - - - - - - - - - - - - - - - - - - -

If curve has a {\tt .Interp} field, this interpolation is taken in account. If {\tt .Interp} field is not present or empty, it uses a degree 2 interpolation by default. 

To force a specific interpolation (over passing {\tt .interp field}, one may insert the \ts{-linear}, \ts{-log} or \ts{-stair} option string in the command. 

To extract a curve {\tt curve\_name} and return the values {\tt Y} corresponding to the input {\tt X}, the syntax is

{\tt y = fe\_curve('returny',model,curve\_name,X);} \\

Given a {\tt curve} data structure, to return the values {\tt Y} corresponding to the input {\tt X}, the syntax is

{\tt y = fe\_curve('returny',curve,X);} \\


\ruic{fe\_curve}{set}{} %  - - - - - - - - - - - - - - - - - - - - - - - - - -

This command sets a curve in the model. 3 types of input are allowed:

\begin{itemize}
\item A data structure, {\tt model=fe\_curve(model,'set',curve\_name,data\_structure)}

\item A string to interprete, {\tt model=fe\_curve(model,'set',curve\_name,string)}

\item A name referring to an existing curve (for load case only), {\tt model=fe\_curve( model, 'set LoadCurve',load\_case,chanel,curve\_name)}. {\bf This last behavior is obsolete} and should be replaced in your code by a more general call to \ltr{fe\_case}{SetCurve}.
\end{itemize}

When you want to associate a curve to a load for time integration it is preferable to use a formal definition of the time dependence (if not curve can be interpolated or extrapolated).

The following example illustrates the different calls.

%begindoc
\begin{verbatim}
% Sample curve assignment to modal loads in a model
model=fe_time('demo bar'); q0=[];

% curve defined by a by-hand data structure:
c1=struct('ID',1,'X',linspace(0,1e-3,100), ...
     'Y',linspace(0,1e-3,100),'data',[],...
     'xunit',[],'yunit',[],'unit',[],'name','curve 1');
model=fe_curve(model,'set','curve 1',c1);
% curve defined by a string to evaluate (generally test fcn):
model=fe_curve(model,'set','step 1','TestStep t1=1e-3');
% curve defined by a reference curve:
c2=fe_curve('test -ID 100 ricker dt=1e-3 A=1');
model=fe_curve(model,'set','ricker 1',c2);
c3=fe_curve('test eval sin(2*pi*1000*t)'); % 1000 Hz sinus
model=fe_curve(model,'set','sin 1',c3);

% define Load with curve definition
LoadCase=struct('DOF',[1.01;2.01],'def',1e6*eye(2),...
            'curve',{{fe_curve('test ricker dt=2e-3 A=1'),...
                      'ricker 1'}});
model = fe_case(model,'DOFLoad','Point load 1',LoadCase);

% modify a curve in the load case
model=fe_case(model,'SetCurve','Point load 1','TestStep t1=1e-3',2);

% the obsolete but supported call was
model=fe_curve(model,'set LoadCurve','Point load 1',2,'TestStep t1=1e-3');

% one would prefer providing a name to the curve, 
% that will be stacked in the model
model=fe_case(model,'SetCurve','Point load 1',...
 'my\_load','TestStep t1=1e-3',2);
\end{verbatim}%enddoc


\ruic{fe\_curve}{Test}{ ...} %  - - - - - - - - - - - - - - - - - - - - - - -

The {\tt test} command handles a large array of analytic and tabular curves. 
In OpenFEM all parameters of each curve must be given in the proper order.  In SDT you can specify only the ones that are not the default using their name.

When the abscissa vector (time, frequency, ...) is given as shown in the example, a tabular result is returned.  

Without output argument the curve is simply plotted.

\begin{SDT}
%begindoc
\begin{verbatim}
% Standard generation of parametered curves
fe_curve('test')  % lists curently implemented curves

t=linspace(0,3,1024); % Define abscissa vector
% OpenFEM format with all parameters (should be avoid):
C1=fe_curve('test ramp 0.6 2.5 2.3',t);
C2=fe_curve('TestRicker 2 2',t);

% SDT format non default parameters given with their name
%  definition is implicit and will be applied to time vector
%  during the time integration: 
C3=fe_curve('Test CosHan f0=5 n0=3 A=3'); 
C4=fe_curve('testEval 3*cos(2*pi*5*t)');

% Now display result on time vector t:
C3=fe_curve(C3,t);C4=fe_curve(C4,t)
figure(1);plot(t,[C1.Y C2.Y C4.Y C3.Y]);
legend(C1.name,C2.name,C4.name,C3.name)
\end{verbatim}%enddoc

\end{SDT}


A partial list of accepted test curves follows

\begin{itemize}
\item \ts{Testsin}, \ts{Testcos},  \ts{TestTan},  \ts{TestExp}, accept parameters \ts{T} period and \ts{A} amplitude. \ts{-stoptime Tf} will truncate the signal. 

\item \ts{TestRamp t0=}\tsi{t0}\ts{\ t1=}\tsi{t1}\ts{\ Yf=}\tsi{Yf} has a ramp starting at zero until \tsi{t0} and going up to \tsi{Yf} at \tsi{t1}. The number of intermediate value can be controlled with the abscissa vector.\\
To define a gradual load, for non linear static for example, a specific call with a \ts{Nstep} parameter can be performed : \ts{TestRamp NStep=}\tsi{NStep}\ts{\ Yf=}\tsi{Yf}. For example, to define a 20 gradual steps to 1e-6 :{\tt R1=fe\_curve('TestRamp NStep=20 Yf=1e-6');}

\item \ts{TestRicker dt=}\tsi{dt}\ts{\ A=}\tsi{A}\ts{\ t0=}\tsi{t0} generates a Ricker function typically used to represent impacts of duration \tsi{dt} and amplitude \tsi{A}, starting from time \tsi{t0}.

\item \ts{TestSweep fmin=}\tsi{fmin}\ts{\ fmax=}\tsi{fmax}\ts{\ t0=}\tsi{t0}\ts{\ t1=}\tsi{t1} generates a sweep cosine from \tsi{t0} to \tsi{t1}, with linear frequency sweeping from \tsi{f0} to \tsi{f1}.

$Y=cos(2*pi*\left(fmin+\left(fmax-fmin\right)*\frac{t-t0}{t1-t0})*\left(t-t0\right)\right)$ for $t0<t<t1$, $Y=0$ elsewhere.

%Note that \tsi{f1} is not the final instant frequency of the signal (which will be $2*f1-f0$ for ascending sweep).
% used to be $f0+2*\frac{f1-f0}{NStep}$ for max final freq : but wrong

\item \ts{TestStep t1=}\tsi{t1} generates a step which value is one from time 0 to time \tsi{t1}.

\item \ts{TestNoise -window"}\tsi{window}\ts{"} computes a time signal corresponding to a white noise, with the power spectrum density specified as the \tsi{window} parameter. For example \ts{TestNoise "Box A=1 min=0 max=200"} defines a unitary power spectrum density from 0 Hz to 200 Hz.

\item \ts{TestBox A=}\tsi{A}\ts{\ min=}\tsi{min}\ts{\ max=}\tsi{max} generates a sample box signal from \tsi{min} to \tsi{max} abscissa, with an amplitude \ts{A}.

\item \ts{TestEval {\ti str}} generates the signal obtained by evaluating the string \tsi{str} function of \ts{t}. For example {\tt R1=fe\_curve('Test eval sin(2*pi*1000*t)',linspace(0,0.005,501)); iiplot(R1)}


\end{itemize}

One can use {\tt fe\_curve('TestList')} to obtain a cell array of the test keywords recognized.

\ruic{fe\_curve}{testframe}{} %  - - - - - - - - - - - - - - - - - - - - - - - - - -

{\tt out=fe\_curve('TestFrame');} computes the time response of a 3 DOF oscillator to a white noise and fills the cell array {\tt out} with noise signal in cell 1 and time response in cell 2. See \swref{fe\_curve('TestFrame')} to open the function at this example.


\ruic{fe\_curve}{timefreq}{} %  - - - - - - - - - - - - - - - - - - - - - - - - - -

{\tt out=fe\_curve('TimeFreq',Input,xf);} \\
computes response of a system with given transfer functions {\tt FRF} to time input {\tt Input}. Sampling frequency and length of time signal {\tt Input} must be coherent with frequency step and length of given transfer {\tt FRF}.

%begindoc
\begin{verbatim}
% Plot time frequency diagrams of signals
fs=1024; sample_length=2;                   % 2 sec. long white noise
noise=fe_curve('noise',fs*sample_length,fs);% 1024 Hz of sampling freq.
[t,f,N]=fe_curve('getXTime',noise);

% FRF with resonant freq. 50 100 200 Hz, unit amplitude, 2% damping 
xf=nor2xf(2*pi*[50 100 200].',.02,[1 ; 1 ; 1],[1 1 1],2*pi*f);

Resp=fe_curve('TimeFreq',noise,xf); % Response to noisy input
fe_curve('Plot',Resp); title('Time response');
\end{verbatim}%enddoc


\ruic{fe\_curve}{Window}{ ... } %  - - - - - - - - - - - - - - - - - - - - - - - - - -

Use {\tt fe\_curve} \ts{window} to list implemented windows. The general calling format is \texline {\tt win=fe\_curve('Window Nb\_pts Type Arg');} which computes a {\tt {\ti Nb\_pts}} points window. The default is a symmetric window (last point at zero), the command option \ts{-per} clips the last point of a $N+1$ long symmetric window. 

For the exponential window the arguments are three doubles. {\tt win = fe\_curve('Window 1024 Exponential 10 20 10');} returns an exponential window with 10 zero points, a 20 point flat top, and a decaying exponential over the 1004 remaining points with a last point at {\tt exp(-10)}.


{\tt win = fe\_curve('Window 1024 Hanning');} returns a 1024 point long hanning window.


\rmain{See also}

\noindent \feload, \fecase








