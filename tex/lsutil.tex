
%------------------------------------------------------------------------------
\rtop{lsutil}{lsutil}

Level set utilities.


\rsyntax
\begin{verbatim}
model=lsutil('cut',model,li,RO)
def=lsutil('gen',model,li)
lsutil('ViewLs',model,li) 
\end{verbatim}


\rmain{Description}

{\tt lsutil} provides a number of tools for level-set creation and manipulation. 

Some commands return the model structure while others return the value of the level-set. Plot outputs are also available.

\noindent Available {\tt lsutil} commands are


\ruic{lsutil}{edge}{[cut, sellevellines, self2, gensel]} % - - - - - - - - - - - - - - - - - - - - - - - - - - - -

\ruic{lsutil}{eltset}{} % - - - - - - - - - - - - - - - - - - - - - - - - - - - -

\ruic{lsutil}{gen}{[-max]} % - - - - - - - - - - - - - - - - - - - - - - - - - - - -

{\sl Level-set computation}. This call takes 2 arguments: {\tt model} a standard model and {\tt li} data to build LS functions. {\tt li} can be  a structure or a cellarray containing structures. Required field in each structure is {\tt .shape}, a string defining the form of the LS. Accepted shapes are
\begin{itemize}
\item    {\tt  "rect"}: additionnal required fields are {\tt  .lx}, {\tt  .ly}, {\tt  .xc}, {\tt  .yc} and {\tt  .alpha};
\item    {\tt  "box"}: additionnal required fields are {\tt  .lx}, {\tt  .ly}, {\tt  .lz}, {\tt  .xc}, {\tt  .yc}, {\tt  .zc}, {\tt  .nx}, {\tt  .ny} and {\tt  .nz}; 
\item	{\tt  "circ"}: additionnal required fields are {\tt  .xc}, {\tt  .yc} and {\tt  .rc};
\item	{\tt  "sphere"}: additionnal required fields are {\tt  .xc}, {\tt  .yc}, {\tt  .zc} and {\tt  .rc}; 
\item	{\tt  "cyl"}: additionnal required fields are {\tt  .xc}, {\tt  .yc}, {\tt  .zc}, {\tt  .rc}, {\tt  .nx}, {\tt  .ny}, {\tt  .nz}, {\tt  .z0} and {\tt  .z1}.  \\Optional field is {\tt  .toAxis}.
\item	{\tt  "cyla"}: additionnal required fields are {\tt  .xc}, {\tt  .yc}, {\tt  .zc}, {\tt  .rc}, {\tt  .nx}, {\tt  .ny} and {\tt  .nz}.
\item	{\tt  "toseg"}: additionnal required fields are {\tt  .orig}, {\tt  .normal}, {\tt  .z0} and {\tt  .z1}. Optional field is {\tt  .rc}.
\item	{\tt  "toplane"}: additionnal required fields are {\tt  .xc}, {\tt  .yc}, {\tt  .zc}, {\tt  .nx}, {\tt  .ny} and {\tt  .nz}. \\Optional field is {\tt  .lc}.
\item	{\tt  "tes"}: additionnal required field is {\tt  .distInt}.
\item	{\tt  "cnem"}: additionnal required fields are {\tt  .xyz} and {\tt  .val}. Optional field is {\tt  .box}.
\item	{\tt  "interp"}: additionnal required field is {\tt  .distInt}. Optional field is {\tt  .box}.
\item	{\tt  "distFcn"}: additionnal required field is {\tt  .distInt}.
\end{itemize}
Instead of using coordinates ({\tt  .xc}, {\tt  .yc}, {\tt  .zc}) to define center of those shapes, user can provide a nodeId in the field {\tt  .idc}.
Other optional fields are accepted, namely {\tt  .rsc} to scale LS values, {\tt  .LevelList} to xxx


\ruic{lsutil}{cut}{[,face2tria]} % - - - - - - - - - - - - - - - - - - - - - - - - - - - -

Accepted options are

\begin{itemize}
%\item {\tt .newTol} tolerance on fractional distance to edge end considered as identical to end node. Default .01.
\item {\tt .tolE}  fractional distance to edge end considered used to
enforce node motion. 
\item {\tt .Fixed} nodes that should not be moved.
\item {\tt .keepOrigMPID} not to alter elements MPID.  By default added elements inherits the original element property.
\item {\tt .keepSets} to update {\tt EltId} sets present in model so that added elements are also added in {\tt EltId} sets to which original elements belonged.

\end{itemize}

Here a first example with placement of circular piezo elements 
%begindoc
\begin{verbatim}
  RO=struct('dim',[400 300 8],'tolE',.3);
  [mdl,li]=ofdemos('LS2d',RO);lsutil('ViewLs',mdl,li);
  li{1} % Specification of a circular level set
  mo3=lsutil('cut',mdl,li,RO);
  lsutil('ViewLs',mo3,li); % display the level set
  fecom('ShowFiPro') % Show element properties
\end{verbatim}%enddoc

Now a volume example

%begindoc
\begin{verbatim}
  RO=struct('dim',[10 10 40],'tolE',.1);
  [model,li]=ofdemos('LS3d',RO);li{1} % Spherical cut
  mo3=lsutil('cut',model,li,RO);
  cf=feplot(mo3);feplot('ShowFiMat')
  
  % Now do a cylinder cut
  li={struct('shape','cyl','xc',.5,'yc',.5,'zc',1,'nx',0,'ny',0,'nz',-1, ...
      'rc',.2,'z0',-.4,'z1',.4,'mpid',[200 300])};
  mo3=lsutil('cut',model,li,RO);feplot(mo3);
  cf.sel={'innode {x>=.5}','colordatamat -edgealpha.1'}
  fecom('ShowFiPro') % Show element properties
\end{verbatim}%enddoc


\vs

Command \ts{CutFace2Tria} transforms faces of selected elements into a triangular mesh. Combination with the \ts{cut} command, it ensures that the cut interface only features triangular elements. This can be useful to perform tet remeshing of one of the cut volumes while ensuring mesh compatibility at the interface.

Syntax is {\tt model=lsutil('CutFace2Tria',model,sel);} with {\tt model} a standard model, and {\tt sel} either
\begin{itemize}
\item A cell-array of level sets that was used to cut the model. Element selection is peformed using~\lts{lsutil}{mpid} command.
\item An {\tt EltId} or {\tt FaceId} set structure.
\item An element matrix.
\item A~\lttts{FindElt} string
\item A list of {\tt EltId} or {\tt FaceId}
\end{itemize}

%begindoc
\begin{verbatim}
% Generate a cube model
RO=struct('dim',[10 10 40],'tolE',.1);
[model]=ofdemos('LS3d',RO); 
model=stack_rm(model,'info','EltOrient');
% Transform one face to use triangles
model=lsutil('CutFace2Tria',model,'selface & innode{x==0}');
\end{verbatim}%enddoc


\ruic{lsutil}{mpid}{} % - - - - - - - - - - - - - - - - - - - - - - - - - - - -

Command \ts{MPID} assigns {\tt MatId} and {\tt ProId} provided in the level set data structure, or by default as indices of level sets to which they belong.

%begindoc
\begin{verbatim}
RO=struct('dim',[10 10 40],'tolE',.1);
[model,li]=ofdemos('LS3d',RO);li{1} % Spherical cut
% li{1} features MatId 200 and ProId 300
% assign these properties to elements in level set
model=lsutil('mpid',model,li);
feplot(model)
fecom('ShowFiPro');
\end{verbatim}%enddoc

\ruic{lsutil}{surf}{[,stream,frompoly,remesh,fromrectmesh]} % - - - - - - - - - - - - - - - - - - - - - - - - - - - -


%\ruic{lsutil}{split}{} % - - - - - - - - - - - - - - - - - - - - - - - - - - - -

%\ruic{lsutil}{view}{} % - - - - - - - - - - - - - - - - - - - - - - - - - - - -

%\ruic{lsutil}{3dintersect}{} % - - - - - - - - - - - - - - - - - - - - - - - - - - - -

\rmain{See also}

\noindent \feplot 
