\documentclass[10pt]{article}

\title{Aide m\'emoire pour CVS}
\author{Frank G\'ENOT}
\date{}

\begin{document}
\maketitle
Ce document constitue un petit r\'esum\'e des commandes usuelles indispensables
pour d\'ebuter avec {\tt CVS} ({\tt C}oncurrent {\tt V}ersion {\tt S}ystem).
Il s'appuie sur la version en place sous AFS Linux \`a l'INRIA
Rocquencourt \`a la date de cr\'eation de ce document, \`a savoir, la 
1.11.1p1 ({\tt"cvs --version"}).

\section{Introduction}
{\tt CVS} est un outil de gestion de sources qui garde/trace automatiquement les
diff\'erentes versions d'un m\^eme fichier. Parmi ces nombreuses fonctionnalit\'es,
on peut lui demander d'envoyer un mel aux diff\'erents d\'eveloppeurs 
intervenants sur le projet en cours \`a chaque fois qu'une modification est
enregistr\'ee.

\section{Variables d'environnement}
\subsection{Machine AFS Linux du projet MACS}
L'utilisation de {\tt CVS} n\'ecessite que les deux variables suivantes soient d\'eclar\'ees
dans votre environnement :
\begin{itemize}
\item {\tt EDITOR} qui contient le chemin d'acc\`es complet \`a votre \'editeur 
favori (emacs, nedit, vi, etc.). Cette variable sera utilis\'ee lorsque {\tt CVS}
vous demandera de taper un texte pour commenter un changement de version.
\item {\tt CVSROOT} qui contient le chemin d'acc\`es au r\'epertoire o\`u {\tt CVS}
va/a stock\'e ses fichiers. Ce r\'epertoire constitue la r\'eference (le d\'ep\^ot). Normalement,
personne ne fait d'intervention directe dessus \`a la main, \`a part au moment
de la cr\'eation.
\end{itemize}
Par exemple, pour OpenFEM, si vous utilisez {\tt tcsh}, rajoutez
dans votre {\tt $\sim$/.(t)cshrc} de votre compte AFS les commandes :
\begin{verbatim}
setenv EDITOR /usr/local/bin/emacs
setenv CVSROOT /net/rouget/local/cvs/OpenFEM
\end{verbatim}
Cela vaut la peine de faire un {\tt "tree \$CVSROOT"} pour comprendre
la structure du repertoire de d\'ep\^ot. 

\subsection{Machine ne voyant pas le r\'epertoire de d\'ep\^ot}
Dans ce cas, on peut toujours utiliser {\tt CVS} mais il faut d\'eclarer une
variable suppl\'ementaire :
\begin{itemize}
\item {\tt CVS\_RSH} qui d\'ecrit le type de protocole que {\tt CVS} va utiliser pour
se connecter \`a la machine serveur (qui g\`ere le r\'epertoire de d\'ep\^ot). 
\end{itemize}
A priori, le protocole le plus utilis\'e maintenant est {\tt SSH}. 
Par exemple, pour {\tt OpenFEM}, \`a partir d'un portable, si vous utilisez {\tt tcsh}, les variables
\`a d\'eclarer sont les suivantes :
\begin{verbatim}
setenv EDITOR /usr/local/bin/emacs
setenv CVSROOT :ext:<username>@rouget.inria.fr:/local/cvs/OpenFEM
setenv CVS_RSH ssh
\end{verbatim}
o\`u {\tt <username>} est votre nom d'utilisateur sur {\tt rouget}. Il vous faut donc avoir
un compte sur la machine serveur, et avoir configurer {\tt SSH}
correctement aussi bien c\^ot\'e client que c\^ot\'e serveur (fichiers du r\'epertoire
{\tt $\sim$/.ssh}). \\
Pour info, la version de protocole actuellement en place sur les machines
{\tt AFS} est la 1.5 :
\begin{verbatim}
rouget:~> ssh -V
SSH Version 1.2.27 [i686-unknown-linux], protocol version 1.5.
Standard version.  Does not use RSAREF.
\end{verbatim}

\section{Un exemple simple d'utilisation compl\`ete}
\subsection{Initialisation d'un nouveau projet}
On commence par cr\'eer un r\'epertoire o\`u CVS va stocker ses fichiers (d\'ep\^ot):
\begin{verbatim}
rouget:> cd /var/tmp
rouget:/var/tmp> mkdir cvsdep
\end{verbatim}
On met \`a jour la variable {\tt CVSROOT} et on initialise le r\'epertoire :
\begin{verbatim}
rouget:/var/tmp> setenv CVSROOT /var/tmp/cvsdep
rouget:/var/tmp> cvs init
\end{verbatim}
Un petit {\tt "tree \$CVSROOT"} vous permettra de constater qu'un sous-repertoire
de nom {\tt CVSROOT} \`a \'et\'e cr\'e\'e : il contient divers fichiers de configuration,
d'historique, de log.\\\\ 
On cr\'ee un r\'epertoire de travail sous {\tt \$CVSROOT} avec les droits d'acc\`es 
pour les futurs d\'eveloppeurs :
\begin{verbatim}
rouget:/var/tmp> mkdir $CVSROOT/The_Project
rouget:/var/tmp> chgrp -R macs $CVSROOT/The_Project
rouget:/var/tmp> chmod g+w -R $CVSROOT/The_Project
\end{verbatim}
Ca y est, on peut maintenant travailler, et on n'aura plus
\`a faire de manip directement sur le r\'epertoire de d\'ep\^ot.
Pour remplir tout de suite le r\'epertoire, consulter {\tt cvs import}.

\subsection{Ajout/modification de fichiers/r\'epertoires}
Je (mais ca pourrait \^etre n'importe qui de chez MACS qui
a les droits d'acc\`es au r\'epertoire de d\'ep\^ot)
me place dans mon r\'epertoire de travail perso :
\begin{verbatim}
rouget:/var/tmp> mkdir ~/MA_COPIE_DU_PROJET
rouget:/var/tmp> cd ~/MA_COPIE_DU_PROJET
\end{verbatim}
Je commence par faire un "check-out" de la version courante :
\begin{verbatim}
rouget:~/MA_COPIE_DU_PROJET> cvs checkout The_Project
cvs checkout: Updating The_Project
rouget:~/MA_COPIE_DU_PROJET> tree .
.
`-- The_Project
    `-- CVS
        |-- Entries
        |-- Repository
        `-- Root
 
2 directories, 3 files
\end{verbatim}
J'ai ainsi obtenu une copie locale du code source du d\'ep\^ot. 
Je me place dans le r\'epertoire de ma copie locale :
\begin{verbatim}
rouget:~/MA_COPIE_DU_PROJET> cd The_Project/ 
\end{verbatim}
J'\'edite un nouveau fichier, je l'ajoute et je valide.
\begin{verbatim}
rouget:~/MA_COPIE_DU_PROJET/The_Project> emacs prog.f77
rouget:~/MA_COPIE_DU_PROJET/The_Project> cvs add prog.f77
cvs add: scheduling file `prog.f77' for addition
cvs add: use 'cvs commit' to add this file permanently
rouget:~/MA_COPIE_DU_PROJET/The_Project> cvs commit
cvs commit: Examining .
RCS file: /var/tmp/cvsdep/The_Project/prog.f77,v
done
Checking in prog.f77;
/var/tmp/cvsdep/The_Project/prog.f77,v  <--  prog.f77
initial revision: 1.1
done
\end{verbatim}
Au passage (au moment du "commit"), {\tt CVS} m'a ouvert un \'editeur (variable {\tt \$EDITOR})
pour que je tape un commentaire du style "Premi\`ere version".
On peut rajouter autant de fichiers que l'on veut, on n'utilise la commande add 
que pour rajouter un fichier :
\begin{verbatim}
rouget:~/MA_COPIE_DU_PROJET/The_Project> cvs add prog.f77
cvs add: prog.f77 already exists, with version number 1.1
Exit 1
\end{verbatim}
On peut valider ("commit") un fichier pr\'ecis en le rajoutant en argument \`a {\tt cvs commit}. 
Il existe des options de commande permettant de valider les sous-r\'epertoires, ou bien uniquement
le r\'epertoire courant.

\subsection{Intervention d'un second d\'eveloppeur}
Maintenant, imaginons qu'un second d\'eveloppeur (Toto) 
entre en jeu, fait son "check out", rajoute un autre fichier {\tt prog1.f77} et le valide par "commit". 
Pour se synchroniser, je fais un "update" de mon r\'epertoire :
\begin{verbatim}
rouget:~/MA_COPIE_DU_PROJET/The_Project> cvs update
cvs update: Updating .
U prog1.f77
\end{verbatim}
Je vois que prog1.f77 a \'et\'e modifi\'e (effectivement, il vient d'\^etre cr\'ee par Toto).
Je consulte son historique pour savoir qui a fait quoi dessus pour l'instant :
\begin{verbatim}
rouget:~/MA_COPIE_DU_PROJET/The_Project> cvs log prog1.f77
 
RCS file: /var/tmp/cvsdep/The_Project/prog1.f77,v
Working file: prog1.f77
head: 1.1
branch:
locks: stricts sur un m�me fichier, voire une m�me r�vision.
access list:
symbolic names:
keyword substitution: kv
total revisions: 1;     selected revisions: 1
description:
----------------------------
revision 1.1
date: 2002/11/06 15:42:28;  author: Toto;  state: Exp;
 
 
Prog1 est une version amelioree de Prog
=============================================================================
\end{verbatim}
\subsection{Un exemple de conflit}
Imaginons maintenant un cas o\`u plusieurs utilisateurs travaillent sur un m\^eme fichier
en m\^eme temps. Toto finit ses modifications avant moi et fait un "commit". Il 
a une course \`a faire d'urgence et oublie de m'avertir. Lorsque je vais essayer de faire
un "commit" de ma nouvelle version, j'ai un conflit :
\begin{verbatim}
rouget:~/MA_COPIE_DU_PROJET/The_Project> cvs commit prog.f77
cvs commit: Up-to-date check failed for `prog.f77'
cvs [commit aborted]: correct above errors first!
Exit 1
\end{verbatim}
Je tente de r\'eparer :
\begin{verbatim}
rouget:~/MA_COPIE_DU_PROJET/The_Project> cvs update
cvs update: Updating .
RCS file: /var/tmp/cvsdep/The_Project/prog.f77,v
retrieving revision 1.1
retrieving revision 1.2
Merging differences between 1.1 and 1.2 into prog.f77
rcsmerge: warning: conflicts during merge
cvs update: conflicts found in prog.f77
C prog.f77
\end{verbatim}
Parfois {\tt CVS} arrive \`a r\'esoudre lui-m\^eme le conflit, mais pas toujours.
Par exemple, ici, ca n'a pas march\'e. Par contre, le r\'esultat du "update" est un 
nouveau fichier {\tt prog.f77} contenant le diff entre ma version et celle que Toto
a d\'ej\`a valid\'ee. En fait, durant la phase de merge, {\tt CVS} s'appuye sur diff et sur
de "puissants" outils de merge de code source. M\^eme si l'op\'eration r\'eussit, il est 
vivement conseill\'e de v\'erifier ce qu'il a fait r\'eellement.\\
Il faut donc \^etre prudent. Le plus simple me semble d'utiliser reguli\`erement la commande
{\tt cvs status} pour v\'erifier qu'on travaille sur une version up to date.
Int\'eressante est aussi la commande {\tt cvs diff <nom-de-fichier>} qui compare le fichier
actuelle et celle du d\'ep\^ot.

\subsection{Une fois termin\'e}
Une fois le travail termin\'e, vous n'\^etes pas oblig\'e de garder une version chez vous.
Par exemple, vous pouvez simplement d\'etruire votre copie locale par :
\begin{verbatim}
rouget:~/MA_COPIE_DU_PROJET/The_Project> cd ../..  
rouget:~> \rm -rf MA_COPIE_DU_PROJET
\end{verbatim}
Mais, si vous voulez \^etre s\^ur que vous n'avez pas oubli\'e de valider certaines modifs, 
mieux vaut faire :
\begin{verbatim}
rouget:~/MA_COPIE_DU_PROJET/The_Project> cd ..
rouget:~/MA_COPIE_DU_PROJET> cvs release -d The_Project/ 
\end{verbatim}
{\tt CVS} vous avertira alors des fichiers derni\`erement modifi\'es et non valid\'es ("commit").

\section{Obtenir de l'aide sur une commande}
Il y a une aide en ligne pour chaque commande de {\tt CVS}. Par exemple :
\begin{verbatim}
rouget:~> cvs --help commit
Usage: cvs commit [-nRlf] [-m msg | -F logfile] [-r rev] files...
    -n          Do not run the module program (if any).
    -R          Process directories recursively.
    -l          Local directory only (not recursive).
    -f          Force the file to be committed; disables recursion.
    -F logfile  Read the log message from file.
    -m msg      Log message.
    -r rev      Commit to this branch or trunk revision.
(Specify the --help global option for a list of other help options)
Exit 1
\end{verbatim}

\section{Gestion des versions}
Lorsqu'on est nombreux \`a travailler sur un code, ou si le code change souvent, 
on est amen\'e \`a sortir de temps \`a marquer certaines versions afin de
pouvoir se rep\'erer. Cette marque n'a pas de rapport avec le num\'ero
de r\'evision (obtenu par exemple par {\tt cvs log}). 
Par exemple, pour {\tt OpenFEM}, 
il s'agit de marquer la version actuellement distribu\'ee sur le web. 
On attribue alors un "label symbolique", qui servira de rep\`ere par la suite.
Certaines commandes de {\tt CVS} comme {\tt cvs export} (pour sortir une
version "clean" sans les fichiers suppl\'ementaires produits par {\tt CVS}) 
demande un tel label comme param\`etre.
On commence donc par affecter au projet un label de version :
\begin{verbatim}
rouget:~> cvs rtag OpenFEM_stable matlab/openfem
\end{verbatim}
Le label est affect\'e \`a tous les fichiers (\`a part si on utilise des options
de la commande {\tt cvs rtag}). Un fichier peut \'evidemment avoir plusieurs
label.\\
On peut affect\'e un label a un fichier directement par :
\begin{verbatim}
rouget:~>cvs tag <label> <nom-de-fichier>
\end{verbatim}
Comme un label correspond \`a une r\'evision du fichier, on peut l'exploiter pour
voir les diff\'erences avec entre la version actuelle (copie dans le 
repertoire local) et la version identifi\'ee par le label dans le r\'epertoire
de d\'ep\^ot :
\begin{verbatim}
rouget:~/MA_COPIE_DU_PROJET/matlab/openfem> cvs diff -r OpenFEM_stable fe_load.m
\end{verbatim}
De m\^eme, pour un "checkout", je peux demander de r\'ecuperer une
copie locale d'une version ant\'erieure par :
\begin{verbatim}
rouget:~> mkdir vieille_version
rouget:~> cd vielle_version
rouget:~/vielle_version> cvs checkout -r OpenFEM_old matlab/openfem
\end{verbatim}
Finalement, imaginons que la version actuelle du projet me satisfasse. 
Pour produire une version clean, il me suffit de faire la chose suivante :
\begin{verbatim}
rouget:~> cvs rtag OpenFEM_web_7_nov_2002 matlab/openfem
rouget:~> mkdir tmp
rouget:~> cd tmp
rouget:~/tmp> cvs export -r OpenFEM_web_7_nov_2002 matlab/openfem
\end{verbatim}
et sous {\tt $\sim$/tmp/matlab}, j'obtiens un r\'epertoire {\tt openfem} contenant la 
distribution.

\section{Modification de l'arborescence du projet}
Imaginons que je d\'esire \`a un moment donn\'e d\'eplacer/d\'etruire (cf. {\tt cvs --help delete} 
ou {\tt cvs --help remove}) un r\'epertoire du projet.\\
Cette manip n'est malheureusement pas facile \`a faire : comme le grand int\'eret 
de {\tt CVS} est d'offrir la possibilit\'e de r\'ecup\'erer des versions 
ant\'erieures du projet, un r\'epertoire ayant \'et\'e un jour ajout\'e 
dans le d\'ep\^ot ne sera jamais supprim\'e (sinon {\tt CVS} ne serait pas comment faire
pour revenir en arri\`ere). Il est donc possible de r\'ecup\'erer des r\'epertoires
vides \`a la suite d'un {\tt checkout}. Le mieux est alors d'utiliser l'option {\tt -P} 
qui ignore ces r\'epertoires parasites.\\
Revenons au probl\`eme initial et pr\'ecisons le cas difficile : je d\'esire d\'eplacer toute
une arborescence. Un simple {\tt mv} du r\'epertoire dans ma copie locale suivi
d'un {\tt cvs commit} ne marche pas, car les commandes {\tt cvs add}, {\tt cvs delete}, ne prennent
comme argument que des fichiers !!! La m\'ethode standard est, dans un premier temps, de cr\'eer l'arborescence
cible de r\'epertoires, de copier tous les fichiers dedans et de les valider par {\tt cvs add}, pour finalement
d\'etruire les fichiers dans le r\'epertoire initial par {\tt cvs delete}, et faire un {\tt cvs commit}.\\
Cette manip peut \^etre v\'eritablement fastidieuse. Voici la technique la plus efficace \`a laquelle je suis arriv\'e.
\begin{itemize}
\item Je commence par obtenir une copie chez moi de la derniere {\it distribution} (cf. section pr\'ec\'edente).
\item Je fais les manips de r\'epertoires sur cette distribution.
\item Je cr\'ee un nouveau r\'epertoire de d\'ep\^ot.
\item J'importe la nouvelle distribution dans le d\'ep\^ot.
\end{itemize}
Voici un exemple pr\'ecis : je d\'esire d\'eplacer le r\'epertoire {\tt matlab/opemfem} d'un cran
vers le haut. 
\begin{verbatim}
rouget:~> cvs rtag OpenFEM_actu matlab/openfem
rouget:~> mkdir tmp
rouget:~> cd tmp
rouget:~/tmp> cvs export -r OpenFEM_actu matlab/openfem

rouget:~/tmp> mv matlab/openfem openfem
rouget:~/tmp> \rm -rf matlab

rouget:~/tmp> mkdir /net/rouget/local/newcvs
rouget:~/tmp> setenv CVSROOT /net/rouget/local/newcvs
rouget:~/tmp> cvs init

rouget:~/tmp> cd openfem
rouget:~/tmp/openfem> cvs import -m "Nouvelle arborescence" openfem V0 start 
\end{verbatim}
Evidemment, c'est un nouveau d\'ep\^ot, je suis incapable de revenir en arri\`ere sur les versions !
Mais cette m\'ethode demeure n\'eanmoins la plus simple, l'essentiel \'etant de la faire 
le moins souvent possible, et surtout de garder les anciens r\'epertoires de d\'epot au cas o\`u \ldots.\\
Au passage, toutes les commandes de {\tt CVS} acceptent l'option {\tt -d $<$r\'epertoire\_de\_dep\^ot$>$} qui
permet de faire les manips sur le r\'epertoire de d\'ep\^ot point\'e par 
{\tt $<$r\'epertoire\_de\_d\'ep\^ot$>$} plut\^ot que sur celui d\'esign\'e par d\'efaut par 
la variable d'environnement {\tt \$CVSROOT}.


\section{Remarques importantes concernant {\tt OpenFEM}}
Il n'y a aucun int\'er\^et \`a ce que le r\'epertoire de d\'ep\^ot 
contienne les librairies compil\'ees obtenues par {\tt ofutil('mexall')} 
sous {\tt Matlab}. Je pense
donc qu'une bonne politique consiste \`a {\em \'eviter absolument} des
commandes du style {\tt cvs commit} sans argument, qui r\'epercute toutes 
modifications sur le d\'ep\^ot, et donc \'evidemment
les librairies et les .o si vous les avez cr\'e\'es dans votre
r\'epertoire local !!!\\
Pensez toujours \`a l'argument (typiquement le nom
du fichier que vous venez de modifier/tester/valider) dans 
{\tt cvs commit <nom-du-fichier>}.


\section{Autres documents}
Une doc en fran\c{c}ais tr\`es simple :
\begin{verbatim}
http://www.idealx.org/fr/doc/cvs/cvs.html
\end{verbatim}
La documentation en ligne officielle accessible \`a partir de :
\begin{verbatim}
http://www.cvshome.org
\end{verbatim}

\end{document}
