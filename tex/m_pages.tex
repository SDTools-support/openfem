%       Copyright (c) 2001-2015 by INRIA and SDTools, All Rights Reserved.
%       Use under OpenFEM trademark.html license and LGPL.txt library license
%       $Revision: 1.68 $  $Date: 2021/03/09 07:04:46 $

%------------------------------------------------------------------------------
\rtop{m\_elastic}{m_elastic}

 Material function for elastic solids and fluids.

\rsyntax\begin{verbatim}
 mat= m_elastic('default') 
 mat= m_elastic('database name') 
 mat= m_elastic('database -therm name') 
 pl = m_elastic('dbval MatId name');
 pl = m_elastic('dbval -unit TM MatId name');
 pl = m_elastic('dbval -punit TM MatId name');
 pl = m_elastic('dbval -therm MatId name');
\end{verbatim}

\rmain{Description}

This help starts by describing the main commands of {\tt m\_elastic} : \ts{Database} and \ts{Dbval}. Materials formats supported by {\tt m\_elastic} are then described.

\ruic{m\_elastic}{Database}{,\htr{m\_elastic}{Dbval}] [-unit TY] [,MatiD]] Name} % - - - - - - - - - - - - - - 

A material property function is expected to store a number of standard materials. See \ser{pl} for material property interface.

{\tt m\_elastic('database Steel')} returns a the data structure describing steel.\\
{\tt m\_elastic('dbval 100 Steel')} only returns the property row. 

%begindoc
\begin{verbatim}
  % List of materials in data base
  m_elastic info
  % examples of row building and conversion
  pl=m_elastic([100 fe_mat('m_elastic','SI',1) 210e9 .3 7800], ...
    'dbval 101 aluminum', ...
    'dbval 200 lamina .27 3e9 .4 1200 0  790e9 .3 1780 0');
  pl=fe_mat('convert SITM',pl);
  pl=m_elastic(pl,'dbval -unit TM 102 steel')
\end{verbatim}%enddoc

Command option \ts{-unit} asks the output to be converted in the desired unit system.
Command option \ts{-punit} tells the function that the provided data is in a desired unit system (and generates the corresponding type).
Command option \ts{-therm} asks to keep thermal data (linear expansion coefficients and reference temperature) if existing.

You can generate orthotropic shell properties using the \ts{Dbval 100 lamina VolFrac Ef nu\_f rho\_f G\_f E\_m nu\_m Rho\_m G\_m} command which gives fiber and matrix characteristics as illustrated above (the volume fraction is that of fiber). 

The default material is steel.


To orient fully anisotropic materials, you can use the following command

%begindoc
\begin{verbatim}
 % Behavior of a material grain assumed orthotropic
 C11=168.4e9; C12=121.4e9; C44=75.4e9; % GPa
 C=[C11 C12 C12 0 0 0;C12 C11 C12 0 0 0;C12 C12 C11 0 0 0;
   0 0 0 C44 0 0;    0 0 0 0 C44 0;    0 0 0 0 0 C44]; 

 pl=[m_elastic('formulaPlAniso 1',C,basis('bunge',[5.175 1.3071 4.2012]));
     m_elastic('formulaPlAniso 2',C,basis('bunge',[2.9208 1.7377 1.3921]))];
\end{verbatim}%enddoc


\rmain{Subtypes}
{\tt m\_elastic} supports the following material subtypes\vs\vs

\ruic{m\_elastic}{1}{ : standard isotropic}

\noindent {\sl Standard isotropic materials}, see~\ser{feelas3d} and ~\ser{feelas2d}, are described by a row of the form

\begin{verbatim}
 [MatID   typ   E nu rho G Eta Alpha T0]
\end{verbatim}


\noindent with {\tt typ} an identifier generated with the {\tt fe\_mat('m\_elastic','SI',1)} command, $E$ (Young's modulus), $\nu$ (Poisson's ratio), 
$\rho$ (density), $G$ (shear modulus, set to $G=E/2(1+\nu)$ if equal to zero). $\eta$ loss factor for hysteretic damping modeling. $\alpha$ thermal expansion coefficient. $T_0$ reference temperature.
$G=E/2(1+\nu)$

By default $E$ and $G$ are interdependent through $G=E/2(1+\nu)$. One can thus define either $E$ and $G$ to use this property. If $E$ or $G$ are set to zero they are replaced on the fly by their theoretical expression. Beware that modifying only E or G, either using \feutil \ts{SetMat} or by hand, will not apply modification to the other coefficient. In case where both coefficients are defined, in thus has to modify both values accordingly.


\ruic{m\_elastic}{2}{ : acoustic fluid} % - - - - - - - - - - - - - - - - - - - - - -

\noindent {\sl Acoustic fluid} , see~\ser{feacoustics},are described by a row of the form

\begin{verbatim}
 [MatId typ rho C eta R]
\end{verbatim}


\noindent with {\tt typ} an identifier generated with the {\tt fe\_mat('m\_elastic','SI',2)} command, $\rho$ (density), $C$ (velocity) and $\eta$ (loss factor). The bulk modulus is then given by $K=\rho C^2$. 

For walls with an impedance (see~\ltr{p\_solid}{3} form 8), the real part of the impedance, which corresponds to a viscous damping on the wall is given by {\tt $Z=\rho C R$}. If an imaginary part is to be present, one will use $Z=\rho C R(1+i \eta)$. In an acoustic tube the absorbtion factor is given by $\alpha=\frac{4R}{((R+1)^2+(R\eta)^2)}$. 

\ruic{m\_elastic}{3}{ : 3-D anisotropic solid} % - - - - - - - - - - - - - - - - - - - - - 

{\sl 3-D Anisotropic solid}, see~\ser{feelas3d}, are described by a row of the form

\begin{verbatim}
 [MatId typ Gij rho eta A1 A2 A3 A4 A5 A6 T0]
\end{verbatim}


with {\tt typ} an identifier generated with the {\tt fe\_mat('m\_elastic','SI',3)} command, $rho$ (density), $eta$ (loss factor) and $Gij$ a row containing 

\begin{verbatim}
 [G11 G12 G22 G13 G23 G33 G14 G24 G34 G44 ...
  G15 G25 G35 G45 G55 G16 G26 G36 G46 G56 G66]
\end{verbatim}

Note that shear is ordered $g_{yz}, g_{zx}, g_{xy}$ which may not be the convention of other software.

SDT supports material handling through 

\begin{itemize}
\item material bases defined for each element xx
\item orientation maps used for material handling are described in~\ser{VectFromDir}. It is then expected that the six components {\tt v1x,v1y,v1z,v2x,v2y,v2z} are stored sequentially in the interpolation table.It is then usual to store the MAP in the stack entry \ts{info,EltOrient}.
\end{itemize}


\ruic{m\_elastic}{4}{ : 2-D anisotropic solid} % - - - - - - - - - - - - - - - - - - - - - 

{\sl 2-D Anisotropic solid}, see~\ser{feelas2d}, are described by a row of the form

\begin{verbatim}
 [MatId typ E11 E12 E22 E13 E23 E33 rho eta a1 a2 a3 T0]
\end{verbatim}


with {\tt typ} an identifier generated with the {\tt fe\_mat('m\_elastic','SI',4)} command, $rho$ (density), $eta$ (loss factor) and $Eij$ elastic constants and $ai$ anisotropic thermal expansion coefficients.

\ruic{m\_elastic}{5}{ : shell orthotropic material} % - - - - - - - - - - - - - - - - - - - - - 

{\sl shell orthotropic material}, see~\ser{feshell} corresponding to NASTRAN MAT8,  are described by a row of the form

\begin{verbatim}
 [MatId typ E1 E2 nu12 G12 G1z G2z Rho A1 A2 T0 Xt Xc Yt Yc S Eta ...
   F12 STRN]
\end{verbatim}


with {\tt typ} an identifier generated with the {\tt fe\_mat('m\_elastic','SI',5)} command, $rho$ (density), ... See \ltr{m\_elastic}{Dbval}\ts{lamina} for building. 

\ruic{m\_elastic}{6}{ : Orthotropic material} % - - - - - - - - - - - - - - - - - - - - - - - - - -

{\sl 3-D orthotropic material}, see~\ser{feelas3d}, are described by a set of engineering constants, in a row of the form

\begin{verbatim}
 [MatId typ E1 E2 E3 Nu23 Nu31 Nu12 G23 G31 G12 rho a1 a2 a3 T0 eta]
\end{verbatim}


with {\tt typ} an identifier generated with the {\tt fe\_mat('m\_elastic','SI',6)} command, $Ei$ (Young modulus in each direction), $\nu ij$ (Poisson ratio), $Gij$ (shear modulus), $rho$ (density), $ai$ (anisotropic thermal expansion coefficient), $T_0$ (reference temperature), and $eta$ (loss factor).
Care must be taken when using these conventions, in particular, it must be noticed that

\begin{eqsvg}{m_elastic_ortho_mat}
 \nu_{ji} = \frac{E_j}{E_i} \nu_{ij}
\end{eqsvg}

\rmain{See also}

  \Ser{femp}, \ser{pl}, \femat, \pshell, \ltr{feutil}{SetMat}


%------------------------------------------------------------------------------
\rtop{m\_heat}{m_heat}

 Material function for heat problem elements.

\rsyntax\begin{verbatim}
 mat= m_heat('default') 
 mat= m_heat('database name') 
 pl = m_heat('dbval MatId name');
 pl = m_heat('dbval -unit TM MatId name');
 pl = m_heat('dbval -punit TM MatId name');
\end{verbatim}

\rmain{Description}

This help starts by describing the main commands of {\tt m\_heat} : \ts{Database} and \ts{Dbval}. Materials formats supported by {\tt m\_heat} are then described.

\ruic{m\_heat}{Database}{,Dbval] [-unit TY] [,MatiD]] Name} % - - - - - - - - - - - - - - 

A material property function is expected to store a number of standard materials. See \ser{pl} for material property interface.

{\tt m\_heat('DataBase Steel')} returns a the data structure describing steel.\\
{\tt m\_heat('DBVal 100 Steel')} only returns the property row. 

%begindoc
\begin{verbatim}
  % List of materials in data base
  m_heat info
  % examples of row building and conversion
  pl=m_heat('DBVal 5 steel');
  pl=m_heat(pl,...
    'dbval 101 aluminum', ...
    'dbval 200 steel');
  pl=fe_mat('convert SITM',pl);
  pl=m_heat(pl,'dbval -unit TM 102 steel')
\end{verbatim}%enddoc

\rmain{Subtypes}
{\tt m\_heat} supports the following material subtype\vs\vs

\ruic{m\_heat}{1}{ : Heat equation material} % - - - - - - - - - - - - - - - - - - - -

\begin{verbatim}
   [MatId fe_mat('m_heat','SI',2) k rho C Hf]
\end{verbatim}

\begin{itemize}
\item {\tt k} conductivity
\item {\tt rho} mass density
\item {\tt C}  heat capacity
\item {\tt Hf} heat exchange coefficient
\end{itemize}

\rmain{See also}

  \Ser{femp}, \ser{pl}, \femat, \pheat

%------------------------------------------------------------------------------
\rtop{m\_hyper}{m_hyper}

 Material function for hyperelastic solids.

\rsyntax\begin{verbatim}
 mat= m_hyper('default') 
 mat= m_hyper('database name') 
 pl = m_hyper('dbval MatId name');
 pl = m_hyper('dbval -unit TM MatId name');
 pl = m_hyper('dbval -punit TM MatId name');
\end{verbatim}

\rmain{Description}

Function based on {\tt m\_elastic} function adapted for hyperelastic material. Only subtype 1 is currently used:

\ruic{m\_hyper}{1}{ : Nominal hyperelastic material}

\noindent {\sl Nominal hyperelastic materials} are described by a row of the form

\begin{verbatim}
 [MatID   typ  rho Wtype C_1 C_2 K]
\end{verbatim}


\noindent with {\tt typ} an identifier generated with the {\tt fe\_mat('m\_hyper','SI',1)} command, $rho$ (density), $Wtype$ (value for Energy choice), $C_1$, $C_2$, $K$ (energy coefficients).\\
\noindent Possible values for $Wtype$ are:

$$
\begin{array}{ll}
0: & W = C_1(J_1-3) + C_2(J_2-3) + K(J_3-1)^2\\
1: & W = C_1(J_1-3) + C_2(J_2-3) + K(J_3-1) - (C_1 + 2C_2 + K)\ln(J_3)
\end{array}
$$

Other energy functions can be added by editing the {\tt hyper.c Enpassiv} function.

In RivlinCube test, m\_hyper is called in this form:
\begin{verbatim}
model.pl=m_hyper('dbval 100 Ref'); % this is where the material is defined
\end{verbatim}


the hyperelastic material called ``Ref'' is described in the database of {\tt m\_hyper.m} file:
\begin{verbatim}
  out.pl=[MatId fe_mat('type','m_hyper','SI',1) 1e-06 0 .3 .2 .3];
  out.name='Ref';
  out.type='m_hyper';
  out.unit='SI';
\end{verbatim}


Here is an example to set your material property for a given structure model:
\begin{verbatim}
model.pl = [MatID fe_mat('m_hyper','SI',1) typ rho Wtype C_1 C_2 K];
model.Elt(2:end,length(feval(ElemF,'node')+1)) = MatID;
\end{verbatim}


