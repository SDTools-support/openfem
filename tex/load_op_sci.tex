%       Copyright (c) 2001-2002 by INRIA and SDTools, All Rights Reserved.
%       Use under OpenFEM trademark.html license and LGPL.txt library license
%       $Revision: 1.5 $  $Date: 2014/08/28 12:52:58 $

\documentclass[a4paper]{article}
\usepackage{graphicx}
\author{Claire DELFORGE}
\title{Compilation et chargement d'OpenFEM Scilab}
\date{ }
\begin{document}
\maketitle
Le lecteur trouvera ci-dessous les informations utiles \`a la compilation et au chargement d'OpenFEM Scilab.\\
Lors de l'installation d'OpenFEM Scilab, l'utilisateur devra compiler les routines C et Fortran, puis charger OpenFEM dans Scilab.
\section{Structure d'OpenFEM Scilab}
Le sch\'ema ci-dessous repr\'esente la structure du r\'epertoire \verb+openfem_scilab+.\\
Les fichiers Fortran et C \`a compiler sont dans les sous-r\'epertoires \verb+fmex+, \verb+medit+, \verb+nopo+, \verb+temps_reel+ et \verb+util+.
\begin{center}
\begin{figure}[!h]
\ingraph{80}{structure} %\includegraphics[angle=270,width=12cm]{structure}
\caption{Structure d'OpenFEM Scilab}
\end{figure}
\end{center}

\section{Compilation des routines C et Fortran}
\subsection{Compilation de toutes les routines}
On peut compiler l'ensemble des routines C et Fortran par une m\^eme action \`a l'aide de la fonction \verb+ofutil.sci+. Cette op\'eration est utile lors de l'installation d'OpenFEM Scilab. La marche \`a suivre est la suivante :
\begin{enumerate}
\item Se placer dans le r\'epertoire \verb+openfem_scilab+ et lancer Scilab.
\item Charger en m\'emoire la fonction \verb+ofutil.sci+ en tapant : \verb+getf ofutil.sci+
\item Taper \verb+ofutil('mexall')+ : la compilation est lanc\'ee.
\end{enumerate}
\subsection{Compilation routine par routine}
L'utilisateur peut \^etre amen\'e \`a compiler s\'epar\'ement les routines C et Fortran d'OpenFEM (s'il veut recompiler un code modif\'e par exemple\ldots). La fonction \verb+ofutil.sci+ permet de compiler s\'epar\'ement les routines d'OpenFEM.
\begin{itemize}
\item \verb+ofutil('of_mk')+ : compilation de la routine \verb+of_mk.c+ (situ\'ee dans \verb+mex/fmex+)
\item \verb+ofutil('medit')+ : compilation des routines \verb+write_mesh.c+, \verb+write_bb.c+ (situ\'ees dans le r\'epertoire \verb+mex/cmex/medit+) et des routines \verb+realtime.c+ et \verb+tempsreel.c+ (situ\'ees dans le r\'epertoire \verb+mex/cmex/temps_reel+) 
\item \verb+ofutil('nopo2sd')+ : compilation de la routine \verb+nopo2sd.c+ (situ\'ee dans \verb+mex/cmex/nopo+)
\item \verb+ofutil('sp_util')+ : compilation de la routine \verb+sp_util.c+ (situ\'ee dans \verb+mex/cmex/util+)
\end{itemize}

\section{Chargement dans Scilab des fonctions OpenFEM}
\subsection{Chargement manuel}
\begin{enumerate}
\item Se placer dans le r\'epertoire \verb+openfem_scilab+ et lancer Scilab.
\item Dans la fen\^etre Scilab, taper : \verb+exec ofutil.sce+. Les fonctions OpenFEM se chargent et sont maintenant utilisables \`a partir de cette fen\^etre Scilab.
\end{enumerate}
Il faut noter que cette op\'eration doit \^etre effectu\'ee \`a chaque fois qu'on ouvre une fen\^etre Scilab. Pour \'eviter de r\'ep\'eter cette op\'eration \`a chaque utilisation, on peut rendre automatique le chargement d'OpenFEM.
\subsection{Chargement automatique}
Pour rendre automatique le chargement d'OpenFEM Scilab, il faut cr\'eer un fichier \verb+.scilab+ dans son home, contenant les instructions suivantes:\\
\begin{verbatim}
repwd = pwd();
chdir('.../openfem_scilab'); // specifier le chemin d'acces a OpenFEM Scilab  
exec ofutil.sce
chdir(repwd);
\end{verbatim}
Ce fichier sera automatiquement ex\'ecut\'e \`a chaque fois que l'utilisateur ouvrira une fen\^etre Scilab.

\end{document}
