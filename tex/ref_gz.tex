%       Copyright (c) 2001-2014 by INRIA and SDTools, All Rights Reserved.
%       Use under OpenFEM trademark.html license and LGPL.txt library license
%       $Revision: 1.21 $  $Date: 2018/10/12 17:07:45 $


%-----------------------------------------------------------------------------
\rtop{iimouse}{iimouse}


\noindent Mouse related callbacks for GUI figures.

\rsyntax\begin{verbatim}
iimouse
iimouse('ModeName')
iimouse('ModeName',Handle)
\end{verbatim}\nlvs

\rmain{Description}


The non current {\sl SDT 3} version of this function is included in OpenFEM. Use the {\tt help} \ts{fecom} command to get help.


% - - - - - - - - - - - - - - - - - - - - - - - - - - - - - - - -
\rmain{See also}

\noindent  \iicom, \fecom, \iiplot, \iiplot


%------------------------------------------------------------------------------
\rtop{lsutil}{lsutil}

Level set utilities.


\rsyntax
\begin{verbatim}
model=lsutil('cut',model,li,RO)
def=lsutil('gen',model,li)
lsutil('ViewLs',model,li) 
\end{verbatim}


\rmain{Description}

{\tt lsutil} provides a number of tools for level-set creation and manipulation. 

Some commands return the model structure while others return the value of the level-set. Plot outputs are also available.

\noindent Available {\tt lsutil} commands are


\ruic{lsutil}{edge}{[cut, sellevellines, self2, gensel]} % - - - - - - - - - - - - - - - - - - - - - - - - - - - -

\ruic{lsutil}{eltset}{} % - - - - - - - - - - - - - - - - - - - - - - - - - - - -

\ruic{lsutil}{gen}{[-max]} % - - - - - - - - - - - - - - - - - - - - - - - - - - - -

{\sl Level-set computation}. This call takes 2 arguments: {\tt model} a standard model and {\tt li} data to build LS functions. {\tt li} can be  a structure or a cellarray containing structures. Required field in each structure is {\tt .shape}, a string defining the form of the LS. Accepted shapes are
\begin{itemize}
\item    {\tt  "rect"}: additional required fields are {\tt  .lx}, {\tt  .ly}, {\tt  .xc}, {\tt  .yc} and {\tt  .alpha};
\item    {\tt  "box"}: additional required fields are {\tt  .lx}, {\tt  .ly}, {\tt  .lz}, {\tt  .xc}, {\tt  .yc}, {\tt  .zc}, {\tt  .nx}, {\tt  .ny} and {\tt  .nz}; 
\item	{\tt  "circ"}: additional required fields are {\tt  .xc}, {\tt  .yc} and {\tt  .rc};
\item	{\tt  "sphere"}: additional required fields are {\tt  .xc}, {\tt  .yc}, {\tt  .zc} and {\tt  .rc}; 
\item	{\tt  "cyl"}: additional required fields are {\tt  .xc}, {\tt  .yc}, {\tt  .zc}, {\tt  .rc}, {\tt  .nx}, {\tt  .ny}, {\tt  .nz}, {\tt  .z0} and {\tt  .z1}.  \\Optional field is {\tt  .toAxis}.
\item	{\tt  "cyla"}: additional required fields are {\tt  .xc}, {\tt  .yc}, {\tt  .zc}, {\tt  .rc}, {\tt  .nx}, {\tt  .ny} and {\tt  .nz}.
\item	{\tt  "toseg"}: additional required fields are {\tt  .orig}, {\tt  .normal}, {\tt  .z0} and {\tt  .z1}. Optional field is {\tt  .rc}.
\item	{\tt  "toplane"}: additional required fields are {\tt  .xc}, {\tt  .yc}, {\tt  .zc}, {\tt  .nx}, {\tt  .ny} and {\tt  .nz}. \\Optional field is {\tt  .lc}.
\item	{\tt  "tes"}: additional required field is {\tt  .distInt}.
\item	{\tt  "cnem"}: additional required fields are {\tt  .xyz} and {\tt  .val}. Optional field is {\tt  .box}.
\item	{\tt  "interp"}: additional required field is {\tt  .distInt}. Optional field is {\tt  .box}.
\item	{\tt  "distFcn"}: additional required field is {\tt  .distInt}.
\end{itemize}
Instead of using coordinates ({\tt  .xc}, {\tt  .yc}, {\tt  .zc}) to define center of those shapes, user can provide a nodeId in the field {\tt  .idc}.
Other optional fields are accepted, namely {\tt  .rsc} to scale LS values, {\tt  .LevelList} to fixed target levels.


\ruic{lsutil}{cut}{[,face2tria]} % - - - - - - - - - - - - - - - - - - - - - - - - - - - -

Accepted options are

\begin{itemize}
%\item {\tt .newTol} tolerance on fractional distance to edge end considered as identical to end node. Default .01.
\item {\tt .tolE}  fractional distance to edge end considered used to
enforce node motion. 
\item {\tt .Fixed} nodes that should not be moved.
\item {\tt .keepOrigMPID} not to alter elements MPID.  By default added elements inherits the original element property.
\item {\tt .keepSets} to update {\tt EltId} sets present in model so that added elements are also added in {\tt EltId} sets to which original elements belonged.

\end{itemize}

Here a first example with placement of circular piezo elements 

%begindoc
\begin{verbatim}
  RO=struct('dim',[400 300 8],'tolE',.3);
  [mdl,li]=ofdemos('LS2d',RO);lsutil('ViewLs',mdl,li);
  li{1} % Specification of a circular level set
  mo3=lsutil('cut',mdl,li,RO);
  % xxxGV dtopoly / lc / damage area 
  lsutil('ViewLs',mo3,li); % display the level set
  fecom('ShowFiPro') % Show element properties
\end{verbatim}%enddoc

xxx show figure xxx 

Now a volume example

%begindoc
\begin{verbatim}
  RO=struct('dim',[10 10 40],'tolE',.1);
  [model,li]=ofdemos('LS3d',RO);li{1} % Spherical cut
  mo3=lsutil('cut',model,li,RO);
  cf=feplot(mo3);feplot('ShowFiMat')
  
  % Now do a cylinder cut
  li={struct('shape','cyl','xc',.5,'yc',.5,'zc',1,'nx',0,'ny',0,'nz',-1, ...
      'rc',.2,'z0',-.4,'z1',.4,'mpid',[200 300])};
  mo3=lsutil('cut',model,li,RO);feplot(mo3);
  cf.sel={'innode {x>=.5}','colordatamat -edgealpha.1'}
  fecom('ShowFiPro') % Show element properties
\end{verbatim}%enddoc


\vs

Command \ts{CutFace2Tria} transforms faces of selected elements into a triangular mesh. Combination with the \ts{cut} command, it ensures that the cut interface only features triangular elements. This can be useful to perform tetrahedra remeshing of one of the cut volumes while ensuring mesh compatibility at the interface.

Syntax is {\tt model=lsutil('CutFace2Tria',model,sel);} with {\tt model} a standard model, and {\tt sel} either
\begin{itemize}
\item A cell-array of level sets that was used to cut the model. Element selection is peformed using~\lts{lsutil}{mpid} command.
\item An {\tt EltId} or {\tt FaceId} set structure.
\item An element matrix.
\item A~\lttts{FindElt} string
\item A list of {\tt EltId} or {\tt FaceId}
\end{itemize}

%begindoc
\begin{verbatim}
% Generate a cube model
RO=struct('dim',[10 10 40],'tolE',.1);
[model]=ofdemos('LS3d',RO); 
model=stack_rm(model,'info','EltOrient');
% Transform one face to use triangles
model=lsutil('CutFace2Tria',model,'selface & innode{x==0}');
\end{verbatim}%enddoc


\ruic{lsutil}{mpid}{} % - - - - - - - - - - - - - - - - - - - - - - - - - - - -

Command \ts{MPID} assigns {\tt MatId} and {\tt ProId} provided in the level set data structure, or by default as indices of level sets to which they belong.

%begindoc
\begin{verbatim}
RO=struct('dim',[10 10 40],'tolE',.1);
[model,li]=ofdemos('LS3d',RO);li{1} % Spherical cut
% li{1} features MatId 200 and ProId 300
% assign these properties to elements in level set
model=lsutil('mpid',model,li);
feplot(model)
fecom('ShowFiPro');
\end{verbatim}%enddoc

\ruic{lsutil}{surf}{[,stream,frompoly,remesh,fromrectmesh]} % - - - - - - - - - - - - - - - - - - - - - - - - - - - -


%\ruic{lsutil}{split}{} % - - - - - - - - - - - - - - - - - - - - - - - - - - - -

%\ruic{lsutil}{view}{} % - - - - - - - - - - - - - - - - - - - - - - - - - - - -

%\ruic{lsutil}{3dintersect}{} % - - - - - - - - - - - - - - - - - - - - - - - - - - - -

\ruic{lsutil}{dToSurf}{, dToPoly, ...} % - - - - - - - - - - 

Low level implementation of distance computations are implemented in subfunctions such as {\tt lsutil('@dToSurf'), dToCirc, dToSphere, tToPlane, dToTri, dToPoly, ...} which can be used for level set building, but also other applications such as contact.  

\rmain{See also}

\noindent \feplot 


%------------------------------------------------------------------------------
\rtop{nopo}{nopo}

Imports nopo files (cf. Modulef)

\rsyntax\begin{verbatim}
 model      = nopo('read -v -p type FileName')
 [Node,Elt] = nopo('read -v -p type FileName')
\end{verbatim}\nlvs

\rmain{Description}
 
\ruic{nopo}{read}{}
 The \ts{-v} option is used for verbose output. The optional \ts{-p} \tsi{type} gives the type of problem described in the nopo file, this allows proper translation to OpenFEM element names.  Supported types are

 {\tt '2D','3D','AXI','FOURIER','INCOMPRESSIBLE','PLAQUE','COQUE'}.

\rmain{See also}  \medit


%------------------------------------------------------------------------------
\rtop{medit}{medit}

Export to {\tt Medit} format

\rsyntax\begin{verbatim}
[indnum,scale] = medit('write FileName',model)
[indnum,scale] = medit('write FileName',model,def,[opt])
[indnum,scale] = medit('write FileName',model,def,'a',[opt])
[indnum,scale] = medit('write FileName',model,[],strain)
[indnum,scale] = medit('write FileName',model,def,strain,[opt])
[indnum,scale] = medit('write FileName',model,def,strain,'a',[opt])
\end{verbatim}\nlvs


\rmain{Description}

{\tt Medit} is an interactive mesh visualization software, developed by the Gamma project at INRIA-Rocquencourt. 

{\tt Medit} executable is freely available at \href{http://www-rocq.inria.fr/gamma/medit}{\tt http://www-rocq.inria.fr/gamma/medit}.
Documentation can also be obtained.

{\tt medit} is an interface to {\tt Medit} software. This function creates files needed by {\tt Medit} and runs the execution of these files in {\tt Medit}. Users must download and install {\tt Medit} themselves. It is not provided in {\tt OpenFEM}.

Input arguments are the following :

\ruic{medit}{FileName:}{}
file name where information for Medit will be written, no extension must be given in FileName.
\ruic{medit}{model}{ :}
a structure defining the model. It must contain at least fields .Node and .Elt.
\ruic{medit}{def}{ :}
a structure defining deformations that users want to visualize. It must contain at least fields .def and .DOF.
\ruic{medit}{strain}{ :}
structure defining coloring, must at least contain :

* fields .data and .EltId  if coloring depends on elements

* fields .data and .DOF    if coloring depends on nodes

Strain can be obtained by a call to {\tt fe\_stres}.\\
For example, {\tt strain=fe\_stres('ener',FEnode,FEel0,pl,il,md1,mdof)} generates a structure with fields .data and .EltId (depending on elements), and {\tt strain = fe\_stres('stress mises',FEnode,FEel0,pl,[],def,mdof)}  generates a structure with fields .data and .DOF (depending on nodes). See demo {\tt d\_ubeam} or test {\tt test\_medit}.
\ruic{medit}{opt}{ :}
option vector, {\tt opt = [numdef nb\_imag scale\_user]} with 

* {\tt numdef} : mode to display number

* {\tt nb\_imag} : number of files to create the animation of deformations

* {\tt scale\_user}  : display scale (a parameter for increasing the deformations) 

\ruic{medit}{indnum}{ :}
returns the nodes numbering used by Medit 

\ruic{medit}{scale}{ :}
returns the scale that was used to display the deformations


\rmain{Use}

\begin{verbatim}
medit('write FileName',model)
%displays the mesh defined by model
medit('write FileName',model,def,[opt])
%displays the mesh defined by model in a window and
% the deformation defined by def in an other window
medit('write FileName',model,def,'a',[opt])
%animates deformations defined by def on model
medit('write FileName',model,[],strain)
%displays the mesh defined by model and colors it with the help of strain
medit('write FileName',model,def,strain,[opt])
%displays the mesh defined by model in a window and 
%the deformation defined by def with colors due to strain in an other window
medit('write FileName',model,def,strain,'a',[opt])
%animates the deformations defined by def on model and
% colors them with the help of strain
\end{verbatim}

Note that nodes and faces references are given to Medit. You can visualize faces references by pressing the right mouse button, selecting ``Shading'' in the ``Render mode'' menu and then selecting ``toggle matcolors'' in the ``Colors, Materials'' menu.


%------------------------------------------------------------------------------
\rtop{of2vtk}{of2vtk}

Export model and deformations to {\tt VTK} format for visualization purposes.

\rsyntax\begin{verbatim}
opfem2VTK(FileName,model)
opfem2VTK(FileName,model,val1,...,valn)
\end{verbatim}

\rmain{Description}

Simple function to write the mesh corresponding to the structure model and associated data currently in the ``Legacy VTK file format'' for visualization.

To visualize the mesh using VTK files you may use {\tt ParaView} which is freely available\\
at \href{http://www.paraview.org/HTML/Download.html}{\tt http://www.paraview.org} or any other visualization software supporting {\tt VTK} file formats.

%begindoc
\begin{verbatim}
try;tname=nas2up('tempname.vtk');catch;tname=[tempname '.vtk'];end
model=femesh('testquad4');

NodeData1.Name='NodeData1';NodeData1.Data=[1 ; 2 ; 3 ; 4];
NodeData2.Name='NodeData2';NodeData2.Data=[0 0 1;0 0 2;0 0 3;0 0 4];
of2vtk('fic1',model,NodeData1,NodeData2);

EltData1.Name ='EltData1' ;EltData1.Data =[ 1 ];
EltData2.Name ='EltData2' ;EltData2.Data =[ 1 2 3];
of2vtk('fic2',model,EltData1,EltData2);

def.def = [0 0 1 0 0 0 0 0 2 0 0 0 0 0 3 0 0 0 0 0 4 0 0 0 ]'*[1 2]; 
def.DOF=reshape(repmat((1:4),6,1)+repmat((1:6)'/100,1,4),[],1)
def.lab={'NodeData3','NodeData4'};
of2vtk('fic3',model,def);

EltData3.EltId=[1];EltData3.data=[1];EltData3.lab={'EltData3'};
EltData4.EltId=[2];EltData4.data=[2];EltData4.lab={'EltData4'};
of2vtk('fic4',model,EltData3,EltData4);
\end{verbatim}%enddoc

The default extention \ts{.vtk} is added if no extention is given.

Input arguments are the following:

\ruic{of2vtk}{FileName}{}
file name for the VTK output, no extension must be given in FileName, ``FileName.vtk'' is automatically created.

\ruic{of2vtk}{model}{}
a structure defining the model. It must contain at least fields {\tt .Node} and {\tt .Elt}.\\
{\bf FileName and model fields are mandatory}.

\ruic{of2vtk}{vali}{}
To create a VTK file defining the mesh and some data at nodes/elements (scalars, vectors) you want to visualize, you must specify as many inputs \emph{vali} as needed. \emph{vali} is a structure defining the data: {\tt $vali=struct('Name',ValueName,'Data',Values)$}. Values can be either a table of scalars ($Nnode \times 1$ or $Nelt \times 1$) or vectors ($Nnode \times 3$ or $Nelt \times 3$) at nodes/elements.
Note that a deformed model can be visualized by providing nodal displacements as data (e.g. in ParaView using the ``warp'' function).



%------------------------------------------------------------------------------
\rtop{ofutil}{ofutil}


OpenFEM utilities

\rsyntax\begin{verbatim}
 ofutil commands
\end{verbatim}\nlvs

\rmain{Description}

This function is used for compilations, path checking, documentation generation, ...

 Accepted commands are
 
\lvs\begin{tabular}{@{}p{.15\textwidth}@{}p{.85\textwidth}@{}}
{\tt   Path     } & checks path consistency with possible removal of SDT\\
{\tt   mexall   } & compiles all needed DLL\\
{\tt   of\_mk    } & compiles of\_mk.f (see openfem/mex directory)\\
{\tt   nopo2sd  } & compiles nopo2sd.c (located in openfem/mex directory)\\
{\tt   sp\_util  } & compiles sp\_ufil.c\\
{\tt   zip      } & creates a zip archive of the OpenFEM library \\
{\tt   hevea    } & generates documentation with HEVEA\\
{\tt   latex    } & generates documentation with LaTeX\\
%
\end{tabular}
 
%  HP-UX start MATAB R13 : setenv LD_PRELOAD /usr/lib/libF90.sl;matlab

%------------------------------------------------------------------------------
%       Copyright (c) 2001-2017 by INRIA and SDTools, All Rights Reserved.
%       Use under OpenFEM trademark.html license and LGPL.txt library license
%       $Revision: 1.27 $  $Date: 2018/12/10 17:00:00 $

%----------------------------------------------------------------------------
\rtop{ofact}{ofact}

 Factored matrix object.
\index{matrix!ofact}\index{object!ofact}\index{matrix!sparse/full}

\rmain{Syntax}

\begin{verbatim}
ofact
ofact('method MethodName');
kd=ofact(k); q = kd\b;  ofact('clear',kd);
kd=ofact(k,'MethodName')
\end{verbatim}

\rmain{Description}

The factored matrix object \ofact\ is designed to let users write code
that is independent of the library used to solve static problems of
the form $\ma{K}\ve{q}=\ve{F}$.  For FEM applications, choosing the
appropriate library for that purpose is crucial. Depending on the case
you may want to use full, skyline, or sparse solvers. Then within
each library you may want to specify options (direct, iterative,
in-core, out-of-core, parallel, ... ).

Using the \ofact\ object in your code, lets you specify method at run
time rather than when writing the code. Typical steps are

\begin{verbatim}
ofact('method spfmex'); % choose method
kd = ofact(k);          % create object and factor
static = kd\b           % solve
ofact('clear',kd)       % clear factor when done
\end{verbatim}

For single solves {\tt static=ofact(k,b)} performs the three steps (factor, solve clear) in a single pass.

The first step of method selection provides an open architecture that lets users introduce new solvers with no need to rewrite functions that use {\tt ofact} objects. Currently available methods are listed simply by typing 

\begin{verbatim}
>> ofact

Available factorization methods for OFACT object
->  spfmex : SDT sparse LDLt solver
   sp_util : SDT skyline solver
        lu : MATLAB sparse LU solver
    mtaucs : TAUCS sparse solver
   pardiso : PARDISO sparse solver
      chol : MATLAB sparse Cholesky solver
   *psldlt : SGI sparse solver (NOT AVAILABLE ON THIS MACHINE)
\end{verbatim}

%
and the method used can be selected with {\tt ofact('method MethodName')}. SDTools maintains pointers to pre-compiled solvers at \url{http://www.sdtools.com/faq/FE_ofact.html}.

The factorization {\tt kd = ofact(k);} and resolution steps {\tt static = kd\verb+\+b} can be separated to allow multiple solves with a single factor. Multiple solves are essential for eigenvalue and quasi-newton solvers. {\tt static = ofact(k)\verb+\+b} is of course also correct.

The clearing step is needed when the factors are not stored as \matlab\ variables. They can be stored in another memory pile, in an out-of-core file, or on another computer/processor. Since for large problems, factors require a lot of memory. Clearing them is an important step.

Historically the object was called {\tt skyline}. For backward compatibility reasons, a {\tt skyline} function is provided.

\ruic{ofact}{umfpack}{} % - - - - - - - - - - - - - - - - - - - - - - - - - - - -

To use UMFPACK as a \ofact\ solver you need to install it on your machine. This code is available at \href{http://www.cise.ufl.edu/research/sparse/umfpack}{www.cise.ufl.edu/research/sparse/umfpack}.

\ruic{ofact}{pardiso}{} % - - - - - - - - - - - - - - - - - - - - - - - - - - - -

\begin{OPENFEM}
For installation, see \ser{usepack}.\\
\end{OPENFEM}

Based on the Intel MKL (Math Kernel Library), you should use version 8 and after.

By default the pardiso call used in the ofact object is set for symmetric matrices. For non-symmetric matrices, you have to complement the ofact standard command for factorization with the character string {\tt 'nonsym'}. Moreover, when you pass a matrix from Matlab to PARDISO, you  {\bf must transpose} it in order to respect the PARDISO sparse matrix format.\\
Assuming that $k$ is a real non-symmetric matrix and $b$ a real vector, the solution $q$ of the system $k.q=b$ is computed by the following sequence of commands:

\begin{verbatim}
ofact pardiso                   % select PARDISO solver
kd = ofact('fact nonsym',k');   % factorization
q=kd\b;                         % solve
ofact('clear',kd);              % clear ofact object
\end{verbatim}

The factorization is composed of two steps: symbolic and numerical factorization.
For the first step the solver needs only the sparse matrix structure (i.e. non-zeros location), whereas the actual data stored in the matrix are required in the second step only. Consequently, for a problem with a unique factorization, you can group the steps. This is done with the standard command \emph {ofact('fact',...)}.\\
In case of multiple factorizations with a set of matrices having the same sparse structure, only the second step should be executed for each factorization, the first one is called just for the first factorization. This is possible using the commands {\tt 'symbfact'} and {\tt 'numfact'} instead of 'fact' as follows:

\begin{verbatim}
kd = ofact('symbfact',k);   % just one call at the beginning
...
kd = ofact('numfact',k,kd); % at each factorization
q=kd\b;                     %
...
ofact('clear',kd);          % just one call at the end
\end{verbatim} 

You can extend this to {\bf non-symmetric systems} as described above.\\

\begin{SDT}
\ruic{ofact}{spfmex}{} % - - - - - - - - - - - - - - - - - - - - - - - - - - - -

{\tt spfmex} is a sparse multi-frontal solver based on \href{http://www.netlib.org/linalg/spooles/spooles.2.2.html}{{\tt spooles}} a compiled version is provided with SDT distributions. 

\ruic{ofact}{sp\_util}{} % - - - - - - - - - - - - - - - - - - - - - - - - - - - -

 The skyline matrix storage is a traditional form to store the sparse symmetric matrices corresponding to FE models. For a full symmetric matrix {\tt kfull}

\begin{verbatim}
 kfull=[1  2
           10  5  8  14
               6  0  1
                  9  7
         sym.        11  19
                         20]
\end{verbatim}

The non-zero elements of each column that are above the diagonal are stored sequentially in the data field {\tt k.data} from the diagonal up (this is known as the reverse Jenning's representation) and the indices of the elements of {\tt k} corresponding to diagonal elements of the full matrix are stored in an index field {\tt k.ind}. Here

\begin{verbatim}
 k.data = [1; 10; 2; 6; 5; 9; 0; 8; 11; 7; 1; 14; 20; 19; 0]
 k.ind  = [1; 2; 4; 6; 9; 13; 15];
\end{verbatim}

For easier manipulations and as shown above, it is assumed that the index field {\tt k.ind} has one more element than the number of columns of {\tt kfull} whose value is the index of a zero which is added at the end of the data field {\tt k.data}.

If you have imported the {\tt ind} and {\tt data} fields from an external code, {\tt ks = ofact (data, ind)} will create the ofact object. You can then go back to the \matlab\ sparse format using {\tt sparse(ks)} (but this needs to be done before the matrix is factored when solving a static problem).

% - - - - - - - - - - - - - - - - - - - - - - - - -
\rmain{Generic commands}

\ruic{ofact}{verbose}{}

{\it Persistent solver verbosity handling.}
By default, solvers tend to provide several information for debugging purposes. For production such level of verbosity can be undesirable as it will tend to fill-up logs and slow down the process due to multiple display outputs. One can then toggle the {\tt silent} option of {\tt ofact} with this command.

{\tt ofact('silent','on');}, or {\tt ofact('silent')} will make the solver silent.
{\tt ofact('silent','off');} will switch back the solver to verbose.

It is possible to activate the verbosity level during the solver selection, using token \ts{-silent} to get a silent behavior or \ts{-v} to get a verbose behavior. {\bf Note that a space must exist between the solver name and other tokens}.

\begin{verbatim}
ofact('spfmex -silent') % selected the spfmex_utils solver with silent option
ofact('spfmex -v') % selects the spfmex_utils solver with verbose option
\end{verbatim}


\ruic{ofact}{\_sel}{}

{\it Advanced solver selection with parameter customization}.
Solvers use default parameters to work, but it is sometimes usefull to tweak these values for specific configurations. This command further allows generic solver selection from GUI inputs.

By default, one can call {\tt ofact('\_sel','solver')}, possibly with the \ts{-silent} token. Direct parameter tweaking is currently supported for {\tt spfmex} only, where the {\tt MaxDomainSize} (default to 32), and {\tt MaxZeros} (default to 0.01) can be provided. For larger models, it is suggested to use a {\tt MaxZeros} value set to 0.1.

\begin{verbatim}
ofact('_sel','spfmex 32 .1') % tweaks the MaxZeros spfmex solver value to 0.1
\end{verbatim}



\end{SDT}
% - - - - - - - - - - - - - - - - - - - - - - - - - - 
\rmain{Your solver}

To add your own solver, simply add a file called {\tt MySolver\_utils.m} in the {\tt \verb+@+ofact} directory. This function must accept the commands detailed below.

Your object can use the fields {\tt .ty} used to monitor what is stored in the object (0 unfactored ofact, 1 factored ofact, 2 LU, 3 Cholesky, 5 other), {\tt .ind}, {\tt .data} used to store the matrix or factor in true ofact format, {\tt .dinv} inverse of diagonal (currently unused), {\tt .l} L factor in {\tt lu} decomposition or transpose of Cholesky factor, {\tt .u} U factor in {\tt lu} decomposition or Cholesky factor, {\tt .method} other free format information used by the object method.

\ruic{ofact}{method}{}

Is used to define defaults for what the solver does. 

\ruic{ofact}{fact}{}

This is the callback that is evaluated when {\tt ofact} initializes a new matrix.

\ruic{ofact}{solve}{}

This is the callback that is evaluated when {\tt ofact} overloads the matrix left division ({\tt \verb+\+})

\ruic{ofact}{clear}{}

\ts{clear} is used to provide a clean up method when factor information is not stored within the \ofact\ object itself. For example, in persistent memory, in another process or on another computer on the network.

\ruic{ofact}{silent}{}

\ts{silent} handled the verbosity level of your solver.


\rmain{See also}
\begin{SDT}
 \feeig, \fereduc
\end{SDT}
\begin{OPENFEM}
 \feeig
\end{OPENFEM}
%------------------------------------------------------------------------------




%------------------------------------------------------------------------------
%------------------------------------------------------------------------------
\rtop{sp\_util}{sp_util}

  Sparse matrix utilities.\index{matrix!ofact}\index{matrix!sparse/full}

\rmain{Description}

This function should be used as a \ts{mex} file. The \ts{.m} file version does not support all functionalities, is significantly slower and requires more memory. 

The \ts{mex} code {\bf is not} \matlab\ {\bf clean}, in the sense that it often modifies input arguments. You are thus not encouraged to call \sputil\ yourself.

The following comments are only provided, so that you can understand the purpose of various calls to \sputil.

\begin{itemize}
\item {\tt sp\_util} with no argument returns its version number.

\item {\tt sp\_util('ismex')} true if \sputil\ is a \ts{mex} file on your platform/path.

\item {\tt ind=sp\_util('profile',k)} returns the profile of a sparse matrix (assumed to be symmetric). This is useful to have an idea of the memory required to store a Cholesky factor of this matrix.

\item {\tt ks=sp\_util('sp2sky',sparse(k))} returns the structure array used by the \ofact\ object.

\item {\tt ks = sp\_util('sky\_dec',ks)} computes the LDL' factor of a ofact object and replaces the object data by the factor. The \ts{sky\_inv} command is used for forward/backward substitution (take a look at the {\tt \verb+@+ofact\verb+\+mldivide.m} function).  \ts{sky\_mul} provides matrix multiplication for unfactored ofact matrices. 

\item {\tt k = sp\_util('nas2sp',K,RowStart,InColumn,opt)} is used by \nasread\ for fast transformation between NASTRAN binary format and \matlab\ sparse matrix storage.

\item {\tt k = sp\_util('spind',k,ind)} renumbering and/or block extraction of a matrix. The input and output arguments {\tt k} MUST be the same. This is not typically acceptable behavior for \matlab\ functions but the speed-up compared with {\tt k=k(ind,ind)} can be significant.

\item {\tt k = sp\_util('xkx',x,k)} coordinate change for {\tt x} a 3 by 3 matrix and DOFs of {\tt k} stacked by groups of 3 for which the coordinate change must be applied.

\item {\tt ener = sp\_util('ener',ki,ke,length(Up.DOF),mind,T)} is used by \upcom\ to compute energy distributions in a list of elements. Note that this function does not handle numerical round-off problems in the same way as previous calls.

\item {\tt k = sp\_util('mind',ki,ke,N,mind)} returns the square sparse matrix {\tt k} associated to the vector of full matrix indices {\tt ki} (column-wise position from {\tt 1} to {\tt \verb|N^2|}) and associated values {\tt ke}. This is used for finite element model assembly by \femk\ and \upcom. In the later case, the optional argument {\tt mind} is used to multiply the blocks of {\tt ke} by appropriate coefficients.  \ts{mindsym} has the same objective but assumes that {\tt ki,ke} only store the upper half of a symmetric matrix.

\item {\tt sparse = sp\_util('sp2st',k)} returns a structure array with fields corresponding to the \matlab\ sparse matrix object. This is a debugging tool.

\item {\tt sp\_util('setinput',mat,vect,start)} places vector {\tt vect} in matrix {\tt mat} starting at C position {\tt start}. Be careful to note that {\tt start} is modified to contain the end position. 

\end{itemize}

%------------------------------------------------------------------------------
\rtop{stack\_get,stack\_set,stack\_rm}{stack_get}

Stack handling functions.

\rsyntax\begin{verbatim}
[StackRows,index]=stack_get(model,typ);
[StackRows,index]=stack_get(model,typ,name);
[StackRows,index]=stack_get(model,typ,name,opt);
Up=stack_set(model,typ,name,val)
Up=stack_rm(model,typ,name);
Up=stack_rm(model,typ);
Up=stack_rm(model,'',name);
[model,r1]=stack_rm(model,typ,name,opt);
\end{verbatim}\nlvs

\rmain{Description}

The {\tt .Stack} field is used to store a variety of information, in a $N$ by $3$ cell array with each row of the form {\tt \{'type','name',val\}} (see \ser{model} or \ser{stackref} for example). The purpose of this cell array is to deal with an unordered set of data entries which can be classified by type and name.

Since sorting can be done by name only, names should all be distinct. If the types are different, this is not an obligation, just good practice. 

In get and remove calls, {\tt typ} and {\tt name} can start by \ts{\#} to use a regular expression based on matching (use {\tt doc regexp} to access detailed documentation on regular expressions). To avoid selection by {\tt typ} or {\tt name} one can set it to an empty string.

Command options can be given in {\tt opt} to recover stack lines or entries.
\begin{itemize}
\item {\tt stack\_get} outputs selected sub-stack lines by default. 
\begin{itemize}
\item Using {\tt opt} set to \ts{get} or to \ts{GetData} allows directly recovering the content of the stack entry instead of the stack line.
\item Using {\tt opt} set to {\tt multi} asks to return sub stack lines for multiple results, this is seldom used.
\end{itemize}

\item {\tt stack\_rm} outputs the model from which stack lines corresponding to typ and name have been removed.
\begin{itemize}
\item Using {\tt opt} set to \ts{get} will output in a second argument the removed lines.
\item Using {\tt opt} set to \ts{GetData} will output in a second argument the content of the removed lines. If several lines are removed, 
\end{itemize}
\end{itemize}

%begindoc
\rsyntax\begin{verbatim}
% Sample calls to stack_get and stack_rm
Case.Stack={'DofSet','Point accel',[4.03;55.03];
            'DofLoad','Force',[2.03];
            'SensDof','Sensors',[4 55 30]'+.03};

% Replace first entry
Case=stack_set(Case,'DofSet','Point accel',[4.03;55.03;2.03]);
Case.Stack

% Add new entry
Case=stack_set(Case,'DofSet','P2',[4.03]);
Case.Stack

% Remove entry
Case=stack_rm(Case,'','Sensors');Case.Stack

% Get DofSet entries and access
[Val,ind]=stack_get(Case,'DofSet')
Case.Stack{ind(1),3} % same as Val{1,3}
% Direct access to cell content
[Val,ind]=stack_get(Case,'DofSet','P2','get')

% Regular expression match of entries starting with a P
stack_get(Case,'','#P*')

% Remove Force entry and keep it
[Case,r1]=stack_rm(Case,'','Force','get')
\end{verbatim}\nlvs%enddoc


\begin{SDT}
SDT provides simplified access to stacks in \feplot\ (see \ser{FEPointers}) and \iiplot\ figures (see \ser{CurveStack}). {\tt cf.Stack\{'Name'\}} can be used for direct access to the stack, and {\tt cf.CStack\{'Name'\}} for access to FEM model case stacks.
\end{SDT}



%------------------------------------------------------------------------------






