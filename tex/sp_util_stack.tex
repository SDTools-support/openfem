%------------------------------------------------------------------------------
\rtop{sp\_util}{sp_util}

  Sparse matrix utilities.\index{matrix!ofact}\index{matrix!sparse/full}

\rmain{Description}

This function should be used as a \ts{mex} file. The \ts{.m} file version does not support all functionalities, is significantly slower and requires more memory. 

The \ts{mex} code {\bf is not} \matlab\ {\bf clean}, in the sense that it often modifies input arguments. You are thus not encouraged to call \sputil\ yourself.

The following comments are only provided, so that you can understand the purpose of various calls to \sputil.

\begin{itemize}
\item {\tt sp\_util} with no argument returns its version number.

\item {\tt sp\_util('ismex')} true if \sputil\ is a \ts{mex} file on your platform/path.

\item {\tt ind=sp\_util('profile',k)} returns the profile of a sparse matrix (assumed to be symmetric). This is useful to have an idea of the memory required to store a Cholesky factor of this matrix.

\item {\tt ks=sp\_util('sp2sky',sparse(k))} returns the structure array used by the \ofact\ object.

\item {\tt ks = sp\_util('sky\_dec',ks)} computes the LDL' factor of a ofact object and replaces the object data by the factor. The \ts{sky\_inv} command is used for forward/backward substitution (take a look at the {\tt \verb+@+ofact\verb+\+mldivide.m} function).  \ts{sky\_mul} provides matrix multiplication for unfactored ofact matrices. 

\item {\tt k = sp\_util('nas2sp',K,RowStart,InColumn,opt)} is used by \nasread\ for fast transformation between NASTRAN binary format and \matlab\ sparse matrix storage.

\item {\tt k = sp\_util('spind',k,ind)} renumbering and/or block extraction of a matrix. The input and output arguments {\tt k} MUST be the same. This is not typically acceptable behavior for \matlab\ functions but the speed-up compared with {\tt k=k(ind,ind)} can be significant.

\item {\tt k = sp\_util('xkx',x,k)} coordinate change for {\tt x} a 3 by 3 matrix and DOFs of {\tt k} stacked by groups of 3 for which the coordinate change must be applied.

\item {\tt ener = sp\_util('ener',ki,ke,length(Up.DOF),mind,T)} is used by \upcom\ to compute energy distributions in a list of elements. Note that this function does not handle numerical round-off problems in the same way as previous calls.

\item {\tt k = sp\_util('mind',ki,ke,N,mind)} returns the square sparse matrix {\tt k} associated to the vector of full matrix indices {\tt ki} (column-wise position from {\tt 1} to {\tt \verb|N^2|}) and associated values {\tt ke}. This is used for finite element model assembly by \femk\ and \upcom. In the later case, the optional argument {\tt mind} is used to multiply the blocks of {\tt ke} by appropriate coefficients.  \ts{mindsym} has the same objective but assumes that {\tt ki,ke} only store the upper half of a symmetric matrix.

\item {\tt sparse = sp\_util('sp2st',k)} returns a structure array with fields corresponding to the \matlab\ sparse matrix object. This is a debugging tool.

\item {\tt sp\_util('setinput',mat,vect,start)} places vector {\tt vect} in matrix {\tt mat} starting at C position {\tt start}. Be careful to note that {\tt start} is modified to contain the end position. 

\end{itemize}

%------------------------------------------------------------------------------
\rtop{stack\_get,stack\_set,stack\_rm}{stack_get}

Stack handling functions.

\rsyntax\begin{verbatim}
[StackRows,index]=stack_get(model,typ);
[StackRows,index]=stack_get(model,typ,name);
[StackRows,index]=stack_get(model,typ,name,opt);
Up=stack_set(model,typ,name,val)
Up=stack_rm(model,typ,name);
Up=stack_rm(model,typ);
Up=stack_rm(model,'',name);
[model,r1]=stack_rm(model,typ,name,opt);
\end{verbatim}\nlvs

\rmain{Description}

The {\tt .Stack} field is used to store a variety of information, in a $N$ by $3$ cell array with each row of the form {\tt \{'type','name',val\}} (see \ser{model} or \ser{stackref} for example). The purpose of this cell array is to deal with an unordered set of data entries which can be classified by type and name.

Since sorting can be done by name only, names should all be distinct. If the types are different, this is not an obligation, just good practice. 

In get and remove calls, {\tt typ} and {\tt name} can start by \ts{\#} to use a regular expression based on matching (use {\tt doc regexp} to access detailed documentation on regular expressions). To avoid selection by {\tt typ} or {\tt name} one can set it to an empty string.

Command options can be given in {\tt opt} to recover stack lines or entries.
\begin{itemize}
\item {\tt stack\_get} outputs selected sub-stack lines by default. 
\begin{itemize}
\item Using {\tt opt} set to \ts{get} or to \ts{GetData} allows directly recovering the content of the stack entry instead of the stack line.
\item Using {\tt opt} set to {\tt multi} asks to return sub stack lines for multiple results, this is seldom used.
\end{itemize}

\item {\tt stack\_rm} outputs the model from which stack lines corresponding to typ and name have been removed.
\begin{itemize}
\item Using {\tt opt} set to \ts{get} will output in a second argument the removed lines.
\item Using {\tt opt} set to \ts{GetData} will output in a second argument the content of the removed lines. If several lines are removed, 
\end{itemize}
\end{itemize}

%begindoc
\rsyntax\begin{verbatim}
% Sample calls to stack_get and stack_rm
Case.Stack={'DofSet','Point accel',[4.03;55.03];
            'DofLoad','Force',[2.03];
            'SensDof','Sensors',[4 55 30]'+.03};

% Replace first entry
Case=stack_set(Case,'DofSet','Point accel',[4.03;55.03;2.03]);
Case.Stack

% Add new entry
Case=stack_set(Case,'DofSet','P2',[4.03]);
Case.Stack

% Remove entry
Case=stack_rm(Case,'','Sensors');Case.Stack

% Get DofSet entries and access
[Val,ind]=stack_get(Case,'DofSet')
Case.Stack{ind(1),3} % same as Val{1,3}
% Direct access to cell content
[Val,ind]=stack_get(Case,'DofSet','P2','get')

% Regular expression match of entries starting with a P
stack_get(Case,'','#P*')

% Remove Force entry and keep it
[Case,r1]=stack_rm(Case,'','Force','get')
\end{verbatim}\nlvs%enddoc


\begin{SDT}
SDT provides simplified access to stacks in \feplot\ (see \ser{FEPointers}) and \iiplot\ figures (see \ser{CurveStack}). {\tt cf.Stack\{'Name'\}} can be used for direct access to the stack, and {\tt cf.CStack\{'Name'\}} for access to FEM model case stacks.
\end{SDT}


