%       Copyright (c) 2001-2015 by INRIA and SDTools, All Rights Reserved.
%       Use under OpenFEM trademark.html license and LGPL.txt library license
%       $Revision: 1.123 $  $Date: 2022/06/16 15:13:51 $

\Tchapter{Element reference}{eltfun}\thispagestyle{empty}


%------------------------------------------------------------------------------
%------------------------------------------------------------------------------
%------------------------------------------------------------------------------

Element functions supported by {\sl OpenFEM} are listed below. The rule is to have element families (2D and 3D) with families of formulations selected through element properties and implemented for all standard shapes


\lvs\noindent\begin{tabular}{|p{.15\textwidth}|p{.75\textwidth}|}\hline
\multicolumn{2}{|c|}{{\sc  3-D volume element shapes}}\\ \hline
\rz  \hexah\   &  8-node 24-DOF brick \\
\rz  \hexav\   &  20-node 60-DOF brick   \\
\rz  \hexac\   &  27-node 81-DOF brick \\
\rz  \penta\   &  6-node 18-DOF pentahedron \\
\rz  \pentb\   &  15-node 45-DOF pentahedron \\
\rz  \tetra\   &  4-node 12-DOF  tetrahedron \\
\rz  \tetrb\   &  10-node 30-DOF tetrahedron  \\ \hline
\end{tabular}%-------------------------------------------

\lvs\noindent\begin{tabular}{|p{.15\textwidth}|p{.75\textwidth}|}\hline
\multicolumn{2}{|c|}{{\sc 2-D volume element shapes}}\\ \hline
\rz  \qfourp\   &  4-node quadrangle \\
\rz  \qfivep\   &  5-node quadrangle \\
\rz  \qeightp\  &  8-node quadrangle \\
\rz  {\tt q9a}  &  9-node quadrangle \\
\rz  \tthreep\  &  3-node 6-DOF triangle \\
\rz  \tsixp\    &  6-node 12-DOF triangle \\ \hline
\end{tabular}%-------------------------------------------




Supported problem formulations are listed in \ser{feform}, in particular one considers 2D and 3D elasticity, acoustics, hyperelasticity, fluid/structure coupling, piezo-electric volumes, ...

Other elements, non generic elements, are listed below

\lvs\noindent\begin{tabular}{|p{.15\textwidth}|p{.75\textwidth}|}\hline
\multicolumn{2}{|c|}{{\sc  3-D plate/shell Elements}}\\ \hline
\rz  \dktp\   &  3-node 9-DOF discrete Kirchoff plate \\
\rz  \quadc\  &  4-node 20-DOF shell \\
\rz  \quadb\   &   quadrilateral 4-node 20/24-DOF plate/shell\\
\rz  quad9   &   (display only) \\
\rz  \quadb\   &   quadrilateral 8-node 40/48-DOF plate/shell \\
\rz  \triaa\   &   3-node 15/18-DOF thin plate/shell element \\
\rz  \triac\   &  6-node 36DOF thin plate/shell element  \\ \hline
\end{tabular}%-------------------------------------------


\lvs\noindent\begin{tabular}{|p{.15\textwidth}|p{.75\textwidth}|}\hline
\multicolumn{2}{|c|}{{\sc  Other elements}}\\ \hline
\rz \bare\    &   standard 2-node 6-DOF bar \\
\rz \beam\   &   standard 2-node 12-DOF Bernoulli-Euler beam \\
\rz \beamt\  &   pretensionned 2-node 12-DOF Bernoulli-Euler beam \\
\rz  beam3   &   (display only)\\
\rz \celas\    &  scalar springs and penalized rigid links \\
\rz \mass\    &  concentrated mass/inertia element\\
\rz \massb\    &  concentrated mass/inertia element with offset \\
\rz \rigid\    &  handling of linearized rigid links  \\ \hline
\end{tabular}%-------------------------------------------


\begin{SDT}
\lvs\noindent\begin{tabular}{|p{.15\textwidth}|p{.75\textwidth}|}\hline
\multicolumn{2}{|c|}{{\sc Utility elements}}\\ \hline
\rz\fesuperb\   & element function for general superelement support \\
\rz\integrules\   & FEM integration rule support \\ \hline
\rz\fsc\   & fluid/structure coupling capabilities \\\hline
\end{tabular}%-------------------------------------------
\end{SDT}

\begin{OPENFEM}
\lvs\noindent\begin{tabular}{|p{.15\textwidth}|p{.75\textwidth}|}\hline
\multicolumn{2}{|c|}{{\sc Utility elements}}\\ \hline
\rz\fesuperb\   & generic element support \\
\rz\integrules\   & FEM integration rule support \\ \hline
\end{tabular}%-------------------------------------------
\end{OPENFEM}


\renewcommand \thesection {}

%-------------------------------------------------------------------------
\rtop{bar1}{bar1}

\noindent Element function for a 6 DOF traction-compression bar element.\index{element!bar}\index{bar element}%


\rmain{Description}

\noindent    The \bare\   element corresponds to the standard linear interpolation for axial traction-compression. The element DOFs are the standard translations at the two end nodes (DOFs {\tt .01} to {\tt .03}).

\centre{\ingraph{40}{E_Bar1}}

In a model description matrix, {\sl 
element property rows} for \bare\ elements follow the standard format 
(see \ser{elem0}).

\begin{verbatim}
 [n1 n2 MatID ProID EltID]
\end{verbatim}


Isotropic elastic materials are the only supported (see \melastic).

For supported element properties see \pbeam. Currently, \bare\ only uses the element area {\tt A} with the format

\begin{verbatim}
 [ProID  Type   0  0  0 A] 
\end{verbatim}


\rmain{See also}

\noindent \melastic, \pbeam, \femk, \feplot\  

%------------------------------------------------------------------------------
\rtop{beam1, beam1t}{beam1}

\noindent Element functions for a 12 DOF beam element. {\tt beam1t} is a 2 node beam with pretension available for non-linear cable statics and dynamics.
\index{element!beam}\index{beam element}

\rmain{Description}

\rui{beam1} % - - - - - - - - - - - - - - - - - - - - - - - - - - - -

\ingraph{25}{E_Beam}

In a model description matrix, {\sl element property rows} for \beam\ 
  elements follow the format

\begin{verbatim}
 [n1 n2 MatID ProID nR 0 0 EltID p1 p2 x1 y1 z1 x2 y2 z2]
\end{verbatim}


\noindent where

\lvs\noindent\begin{tabular}{@{}p{.15\textwidth}@{}p{.85\textwidth}@{}}
\rz{\tt n1,n2}  & node numbers of the nodes connected \\
\rz{\tt MatID} & material property identification number\\
\rz{\tt ProID} & element section property identification number\\
\rz{\tt nr 0 0} & number of node not in the beam direction defining bending plane 1  in this case $\ve{v}$ is the vector going from {\tt n1} to {\tt nr}. If {\tt nr} is undefined it is assumed to be located at position [1.5 1.5 1.5]. \\
\rz{\tt vx vy vz} & alternate method for defining the bending plane 1 by giving the components of a vector in the plane but not collinear to the beam axis. If {\tt vy} and {\tt vz} are zero, {\tt vx} {\bf must be negative or not an integer}. {\tt MAP=beam1t('map',model)} returns a normal vector MAP giving the vector used for bending plane 1. This can be used to check your model.
\\
\rz{\tt p1,p2}  & pin flags. These give a list of DOFs to be released (condensed before assembly). For example, 456 will release all rotation degrees of freedom. Note that the DOFS are defined in the local element coordinate system.  \\
\rz{\tt x1,...}  & optional components in global coordinate system of offset vector at node 1 (default is no offset) \\
\rz{\tt x2,...}  & optional components of offset vector at node 2 \\
\end{tabular}

Isotropic elastic materials are the only supported (see \melastic). \pbeam\ describes the section property format and associated formulations.

\begin{SDT}

Failure to define orientations is a typical error with beam models. In the following example, the definition of bending plane 1 using a vector is illustrated. 
%begindoc
\begin{verbatim}
  cf=feplot(femesh('test2bay'));
  % Map is in very variable direction due to undefined nr
  % This is only ok for sections invariant by rotation
  beam1t('map',cf.mdl);fecom('view3'); 
  
  % Now define generator for bending plane 1
  i1=feutil('findelt eltname beam1',cf.mdl); % element row index
  cf.mdl.Elt(i1,5:7)=ones(size(i1))*[-.1 .9 0]; % vx vy vz
  beam1t('map',cf.mdl);fecom('view2'); 
\end{verbatim}%enddoc

\end{SDT}

{\tt beam1} adds secondary inertia effects which may be problematic for extremely short beams and {\tt beam1t} may then be more suitable.

\rui{beam1t} % - - - - - - - - - - - - - - - - - - - - - - - - - - - -

For the bending part, this element solves
%
\begin{eqsvg}{poutreq}
\rho A (\ddot w -\Omega^2 w) + \frac{\partial^2}{\partial x^2}\br{E I_y\frac{\partial^2 w}{\partial x^2}}
-  \frac{\partial }{\partial x}\br{ T\frac{\partial w}{\partial x}}-f = 0 
\end{eqsvg}
%
with boundary conditions in transverse displacement
%
\begin{eqsvg}{beam1_1}
w= \hbox{given or}\ \ F=T\frac{\partial w}{\partial x}-EI_y\frac{\partial^3 w}{\partial x^3}
\end{eqsvg}
%
and rotation
%
\begin{eqsvg}{beam1_2}
\frac{\partial w}{\partial x}=\ \hbox{given or}\ \ M=EI_y\frac{\partial^2 w}{\partial x^2}
\end{eqsvg}


This element has an internal state stored in a \ltt{InfoAtNode} structure where each column of \texline {\tt Case.GroupInfo\{7\}.data} gives the local basis, element length and tension {\tt [bas(:);L;ten]}. Initial tension can be defined using a {\tt .MAP} field in the element property. 

This is a simple example showing how to impose a pre-tension~:

%begindoc
\begin{verbatim}
 model=femesh('TestBeam1 divide 10');
 model=fe_case(model,'FixDof','clamp',[1;2;.04;.02;.01;.05]);
 model.Elt=feutil('SetGroup 1 name beam1t',model);
 d1=fe_eig(model,[5 10]); 
 model=feutil('setpro 112',model,'MAP', ...
   struct('dir',{{'1.5e6'}},'lab',{{'ten'}}));
 d2=fe_eig(model,[5 10]); 

 figure(1);plot([d2.data./d1.data-1]);
 xlabel('Mode index');ylabel('Frequency shift');
\end{verbatim}%enddoc


Strains in a non-linear Bernoulli Euler section are given by 

\begin{eqsvg}{beam1_3}
%\sigma_{33}=\frac{N}{A}-\frac{M_2+(L-x_3)T_1}{I_2}+\frac{M_1-(L-x_3)T_2}{I_1}
\epsilon_{11}=\br{\frac{\partial u}{\partial x}+\frac{1}{2}\br{\frac{\partial w_0}{\partial x}^2}}-z\frac{\partial^2 w_0}{\partial x^2}
\end{eqsvg}
%


\rmain{See also}

\noindent \pbeam, \melastic, \femk, \feplot 

%------------------------------------------------------------------------------
\rtop{celas,cbush}{celas}

\noindent element function for scalar springs and penalized rigid links
\index{element!rigid link}\index{rigid link}\index{scalar spring}

\rmain{Description}

{\tt celas} and {\tt cbush} implement similar spring elements with a somewhat more convenient handling of local bases in {\tt cbush}. Properties can typically either be given in the element or defined in a \pspring\ property. 

\rui{celas}

\noindent In an model description matrix a group of {\tt celas} elements starts with a header row {\tt [Inf  abs('celas') 0 ...]} followed by element property rows following the format

{\tt [n1 n2 DofID1 DofID2 ProID EltID Kv Mv Cv Bv]}

with

\lvs\begin{tabular}{@{}p{.15\textwidth}@{}p{.85\textwidth}@{}}
%
\rz{\tt n1,n2}  & node numbers of the nodes connected. Grounded springs are obtained by setting {\tt n1} or {\tt n2} to {\tt 0}. \\
\rz{\tt DofID} & Identification of selected DOFs. \\
& {\sl For rigid links}, the first node defines the rigid body motion of the other extremity slave node. Motion between the slave node and the second node is then penalized. {\tt DofID} (positive) defines which DOFs of the slave node are connected by the constraint. Thus {\tt [1 2 123 0 0 0 1e14]} will only impose the penalization of node translations {\tt 2} by motion of node {\tt 1}, while  {\tt [1 2 123456 0 0 0 1e14]} will also penalize the difference in rotations.
\begin{figure}[H]
\centering
\ingraph{40}{celasrigid}
\end{figure}
\\
& {\sl For scalar springs}, {\tt DofID1} (negative) defines which DOFs of node 1 are connected to which of node 2. {\tt DofID2} can be used to specify different DOFs on the 2 nodes. For example {\tt [1 2 -123 231 0 0 1e14]} connects DOFs 1.01 to 2.02, etc. Use of negative {\tt DofID1} will only activate additional DOF if explicitly given.
\begin{figure}[H]
\centering
\ingraph{20}{celasscalar}
\end{figure}\\
\rz{\tt ProID} &   Optional property identification number (see format below)\\
\rz{\tt Kv} & Optional stiffness value used as a weighting associated with the constraint. If {\tt Kv} is zero (or not given), the default value in  the element property declaration is used. If this is still zero, {\tt Kv} is set to {\tt 1e14}.\\
\rz{\tt Bv} & Optional stiffness hysteretic damping value : stiffness given by $K_v+iB_v$ (rather than $Kv(1+i\eta)$ when using \pspring). 
\end{tabular}

\pspring\ properties for {\tt celas} elements take the form {\tt [ProID type KvDefault m c eta S] }

By default a \celas\ element will activate all 6 mechanical DOF in the model. If the \celas\ element is not linked to other elements using these DOF ({\it e.g.} 3D elements do not use DOF 4-6), there will be a risk of null stiffness occurrence at assembly. To alleviate this problem use negative {\tt DofID1} that will only activate additional DOF in the specified list. One can also fix the spurious DOF as a boundary condition.

Below is the example of a 2D beam on elastic supports.
%begindoc
\begin{verbatim}
model=femesh('Testbeam1 divide 10');
model=fe_case(model,'FixDof','2D',[.01;.02;.04]);
model.Elt(end+1,1:6)=[Inf abs('celas')]; % spring supports
model.Elt(end+[1:2],1:7)=[1 0 -13 0   0 0 1e5;2 0 -13 0   0 0 1e5];
def=fe_eig(model,[5 10 0]); feplot(model,def);
\end{verbatim}%enddoc

When using local displacement bases (non zero DID\index{DID}), the stiffness is defined in the local basis and transformed to global coordinates.

\rui{cbush}

The element property row is defined by 

\begin{verbatim}
[n1 n2 MatId  ProId EltId x1  x2 x3 EDID  S OCID S1 S2 S3]
[n1 n2 MatId  ProId EltId NodeIdRef 0 0 EDID  S OCID S1 S2 S3]
\end{verbatim}

The orientation of the spring (basis $x_e,y_e,z_e$) can be specified by 
\begin{itemize}
\item {\tt EDID=0} implies the use of the global basis. Local directions require a non-null {\tt EDID}.
\item {\tt EDID>0} specifies a coordinate system for element orientation. This behaviour is pre-emptive.
\item {\tt EDID=-1} allows defining a coordinate system based on the element properties, see example in {\tt sdtweb('d\_fetime','CbushOrient')}. One can then use {\tt x1, x2, x3} that specifies an  orientation vector $v$ to refine local direction definitions. {\bf Note {\tt x1} should not be an integer if {\tt x2} and {\tt x3} are zero}. The orientation vector $v$ does not define the same direction depending on nodal colocality:
\begin{itemize}
\item  For {\bf coincident} {\tt n1,n2}, orientation vector given as {\tt x1,x2,x3} can be used to specify $x_e$ (this differs from figure and is not compatible with NASTRAN). To specify $y_e$ for coincident nodes, you must use classically defined {\tt EDID} with an externally defined basis.
\item  For distinct {\tt n1,n2}, element $x_e$ is along $n_2-n_1$, other directions are defined as follows
\begin{itemize}
\item giving orientation vector $v$ as {\tt x1,x2,x3} specifies $y_e$ in the plane given by $x_e$ and $v$. Note {\tt x1} should not be an integer if {\tt x2} and {\tt x3} are zero to distinguish from the {\tt NodeIdRef} case.
\item {\tt NodeIdRef,0,0} specifies the use of a node number to create $v=n_{ref}-n_1$.
\end{itemize}
\end{itemize}
\end{itemize}


The spring/damper is located at a position interpolated between {\tt n1} and {\tt n2} using {\tt S}, such that $x_i = S n_1 + (1-S) n_2$. The midpoint is used by default, that-is-to-say {\tt S} is taken at 0.5 if left to zero. To use other locations, specify a non-zero {\tt OCID} and an offset {\tt S1,S2,S3}.

It is possible to set {\tt n2} to {\tt 0} to define a grounded \cbush.

ProId properties for cbush elements are defined using \pspring\ and take the form [ProId Type k1:k6 c1:c6 Eta SA ST EA ET m v]. Matid is here for reference but currently unused.

\begin{figure}[H]
\centering
\ingraph{30}{cbush}
\end{figure}

\rmain{See also}

\pspring, \rigid

%------------------------------------------------------------------------
\rtop{dktp}{dktp}

\noindent  2-D 9-DOF Discrete Kirchhoff triangle\index{plate element}\index{element!plate}

\rmain{Description}

\centre{\ingraph{50}{e_dktp}}

In a model description matrix, {\bf element property rows} for {\tt dktp}  elements follow the standard format

\begin{verbatim}
 [n1 n2 n3 MatID ProID EltID Theta] 
\end{verbatim}


giving the node identification numbers {\tt ni}, material {\tt MatID}, property {\tt ProID}. Other {\bf optional} information is {\tt EltID} the element identifier, {\tt Theta} the angle between material $x$ axis and element $x$ axis (currently unused)

The elements support isotropic materials declared with a material entry described in \melastic. Element property declarations follow the format described in \pshell.


The \dktp\ element uses the {\tt et*dktp} routines.

 There are three vertices nodes for this triangular Kirchhoff plate element and the normal deflection \mbox{ $W(x,y)$} is cubic along each edge.

We start with a 6-node triangular element with a total
                $~D.O.F = 21~$~: 
 \begin{itemize}
   \item five degrees of freedom at corner nodes~:
    \begin{eqsvg}{dktp_1}
        \mbox{ $W(x,y)$} ~,~ {\partial\mbox{ $W$} \over \partial{x}}
        ~,~ {\partial\mbox{ $W$} \over \partial{y}} ~,~ {\theta}_{x}
        ~,~ {\theta}_{y} ~~(deflection \mbox{ $W$} ~~ and
        ~~ rotations ~~ {\theta})
    \end{eqsvg}
   \item two degrees of freedom ${\theta}_{x}$ ~and~ $ {\theta}_{y} $
         at mid side nodes.
 \end{itemize}

Then, we impose no transverse shear deformation 
   $ {\gamma}_{xz}~=~0 $  and $ {\gamma}_{yz}~=~0 $ at selected nodes to
   reduce the total $ DOF = 21 - 6*2 = 9 $~: 
 \begin{itemize}
   \item three degrees of freedom at each of the vertices of the triangle.
    \begin{eqsvg}{dktp_2}
        \mbox{ $W(x,y)$} ~,~ 
        {\theta}_{x}=({\partial\mbox{ $W$} \over \partial{x}})
       ~,~ {\theta}_{y}=({\partial\mbox{ $W$} \over \partial{y}})
    \end{eqsvg}
 \end{itemize}

The coordinates of the reference element's vertices are
\mathsvg{\hat{S_1} (0.,0.)}{dktp_l1}, 
\mathsvg{\hat{S_2} (1.,0.)}{dktp_l2} and
\mathsvg{\hat{S_3} (0.,1.)}{dktp_l3}.


Surfaces are integrated using a 3 point rule $\omega_k = {1 \over 3}$  and $b_k$  mid side node.



\rmain{See also}

\noindent  \femat, \melastic, \pshell, \femk, \feplot 

%------------------------------------------------------------------------------
\rtop{fsc}{fsc}

Fluid structure/coupling with non-linear follower pressure support. \index{element!fluid}

\rmain{Description}

Elasto-acoustic coupling is used to model structures containing a compressible, non-weighing fluid, with or without a free surface.

\begin{center}
\ingraph{50}{fsc1}
\end{center}

The FE formulation for this type of problem can be written as~\ecite{mor4}

\begin{eqsvg}{newformu}
s ^2 \ma{\ba{cc} M&0\\ C^{T}&K_p \ea} \ve{\ba{c} q\\p\ea}+
\ma{\ba{cc} K(s)&-C\\ 0&F \ea} \ve{\ba{c} q\\p\ea}
 = \ve{\ba{c}F^{ext}\\0\ea}
\end{eqsvg}

with $q$ the displacements of the structure, $p$ the pressure variations in the fluid and $F^{ext}$ the external load applied to the structure, where 

\begin{eqsvg}{fsc_2}\nonumber
\ba{c}
\int_{\Omega_S}{\sigma_{ij}{(u)}\epsilon_{ij}{(\delta u)} dx}\Rightarrow\delta q^{T} K q\\
\int_{\Omega_S}{\rho_S u.\delta u dx}\Rightarrow\delta q^{T} M q\\
\frac{1}{\rho_F}\int_{\Omega_F}{\nabla p \nabla \delta p dx}\Rightarrow\delta p^{T} F p\\
\frac{1}{\rho_F c^2}\int_{\Omega_F}{p \delta p dx}\Rightarrow\delta p^{T} K_p p\\
\int_{\Sigma}{p \delta u.n dx}\Rightarrow\delta q^{T} C p\\
\ea
\end{eqsvg}

To assemble fluid/structure coupling matrix you should declare a set of surface elements (any topology) with property {\tt p\_solid('dbval 1 fsc')}. The $C$ matrix (solid forces induced by pressure field) is assembled with the stiffness (matrix type 1), while the $C^T$ matrix (fluid pressure due to normal velocity of solid) is assembled with the mass (matrix type 2).

Some formulations, consider a surface impedance proportional to the pressure. This matrix can be computed by defining a group of surface elements with an acoustic material (see \ltr{m\_elastic}{2}) and a standard surface integration rule ({\tt p\_solid('dbval 1 d2 -3'))}. This results in a mass given by

\begin{eqsvg}{fsc_3}
 \delta p^{T} K_p p=\frac{1}{\rho_F c^2}\int_{\Omega_F}{\delta p p dx}
\end{eqsvg}


\rmain{Follower force}

One uses the identity
\begin{eqsvg}{fsc_4}
n \, dS = \frac{\partial \su{x}}{\partial r} \wedge \frac{\partial \su{x}}{\partial s} \, drds,
\end{eqsvg}
where $(r,s)$ designate local coordinates of the face (assumed such
that the normal is outgoing). Work of the pressure is thus:
\begin{eqsvg}{fsc_5}
\delta W_p = - \int_{r,s} \Pi \,  \bigl(\frac{\partial \su{x}}{\partial r} \wedge \frac{\partial \su{x}}{\partial s}\bigr) \cdot\delta\su{v}\,drds.
\end{eqsvg}
On thus must add the non-linear stiffness term:
\begin{eqsvg}{fsc_6}
-d\delta W_p = \int_ {r,s} \Pi \, \bigl(\frac{\partial d\su{u}}{\partial r} \wedge \frac{\partial \su{x}}{\partial s}
+ \frac{\partial \su{x}}{\partial r} \wedge \frac{\partial d\su{u}}{\partial s}\bigr) \cdot\delta\su{v}\,drds.
\end{eqsvg}

Using \mathsvg{\frac{\partial \su{x}} {\partial r}=\{x_{1,r} \; x_{2,r} \; x_{3,r}\}^T}{fsc_l1} (idem for $s$), and also
%
\begin{eqsvg}{fsc_7}
[Axr]= \left(
\begin{array}{ccc}
0 & -x_{,r3} & x_{,r2}\\
x_{,r3} & 0 & -x_{,r1}\\
-x_{,r2} & x_{,r1} & 0
\end{array}
\right),\quad
[Axs]= \left(
\begin{array}{ccc}
0 & -x_{,s3} & x_{,s2}\\
x_{,s3} & 0 & -x_{,s1}\\
-x_{,s2} & x_{,s1} & 0
\end{array}
\right),
\end{eqsvg}
%
this results in
%
\begin{eqsvg}{fsc_8}
\begin{array}{l}
\bigl(\frac{\partial d\su{x}}{\partial r} \wedge \frac{\partial \su{x}}{\partial s}
+ \frac{\partial \su{x}}{\partial r} \wedge \frac{\partial d\su{x}}{\partial s}\bigr) \cdot\delta\su{v} =\\
\quad \{\delta q_{ik}\}^T \ve{N_k} \bigl( Axr_{ij} \{N_{l,s}\}^T - Axs_{ij} \{N_{l,r}\}^T \bigr) \{d q_j\}.
\end{array}
\end{eqsvg}

Base tests : {\tt fsc3 \ts{testsimple}} and {\tt fsc3 \ts{test}}.

In the RivlinCube test \begin{OPENFEM}(see \ser{rivlin})\end{OPENFEM}, the pressure on each free face is given by

\begin{eqsvg}{fsc_9}
\begin{array}{cccc}
\Pi_1 = -\frac{1+\lambda_1}{(1+\lambda_2)(1+\lambda_3)} \Sigma_{11} & on & face & (x_1=l_1)\\
\Pi_2 = -\frac{1+\lambda_2}{(1+\lambda_1)(1+\lambda_3)} \Sigma_{22} & on & face & (x_2=l_2)\\
\Pi_3 = -\frac{1+\lambda_3}{(1+\lambda_1)(1+\lambda_2)} \Sigma_{33} & on & face & (x_3=l_3).
\end{array}
\end{eqsvg}

% NOT IN OPENFEM --------------------------------------------------------------
\begin{SDT}

\rmain{Implementation}

Fluid structure coupling has been renewed in SDT 7.2. In particular \fluid\ elements and \fsc\ elements are obsolete and are not supported with these commands. One must now use classical volume elements assigned with proper \melastic properties for fluid modelling and classical ND-1 elements for the interface, assigned with proper \psolid properties.

Implementation strategy relies on a solid model to which fluid and coupling superelements are added. This easens solid or fluid coupled models. Reduced solutions are generated by default through the generation of a pre-assembled reduced model using free solid modes and free fluid modes.

A coupled fluid structure model is thus composed of three sub-models
\begin{itemize}
\item a solid model. This model remains the base working model throughout the procedure
\item a fluid model. This model will be handled as a superelement in the base structure model. No mesh compatibility assumption is made between solid and fluid meshes. Fluid model will then be renumbered upon addition unless mesh compatibility is stated.
\item a fluid interface coupling model. This model will be handled as a superelement in the base structure model. These coupling elements rely on ND-1 interface \psolid\ formulations. As their name suggest they must be compatible with the fluid interface mesh. Solid mesh compatiblity is ensured through the generation of a \ltt{ConnectionSurface} MPC in case of need. Providing interface elements based on the solid structure mesh is possible only if fluid compatibility is ensured.
\end{itemize}

To help keeping track of performed modelling operations and main options associated to fluid structure models, a running option structure is stored in the base model stack as {\tt info,fscOpt}. This structure can be {\it a priori} defined by the user to store options once for all. It will be completed on the fly during the procedure execution. This structure will contain in particular
\begin{itemize}
\item {\tt .name} the fluid superelement name, see~\ltr{fsc}{AddFluid}.
\item {\tt .cname} the fluid interface coupling superelement name, see~\ltr{fsc}{AddCoupling}.
\item {\tt .MPCname} the fluid interface coupling elements MPC connection to the solid mesh (if needed), see~\ltr{fsc}{AddCoupling}.
\item several reduction options as documented in~\ltr{fsc}{SolveMVR}
\end{itemize}

% ----------------------------------------------------------------------------
\rmain{Commands}

% - - - - - - - - - - - - - - - - - - - - 
\ruic{fsc}{AddFluid}{}

\ts{AddFluid} generates one or several fluid superlements in a model. The base model either already contains the elements modelling the fluid in which case an element selection can be provided, or an external model to be added. By default the fluid model is considered as an independent mesh {\it i.e.} not compatible with the solid structure. If that is the case, command option \ts{-combine} does not renumber the fluid superlement to keep mesh compatibility.

Syntax is {\tt model=fsc('AddFluid',model,mof,RO);} with
\begin{itemize}
\item {\tt model} the base model.
\item {\tt mof} either a fluid model or a string \ltt{FindElt} selection providing fluid elements in {\tt model}.
\item {\tt RO} an optional structure input of running options.
\end{itemize}

Output {\tt model} contains a fluid superelement and stack entry {\tt info,fscOpt} keeping track of the fluid superlement name and other options.

Command options can be defined in three ways. It can either be specifed in the \ts{AddFluid} command string with a {\tt -} prefix, or provided as an additional argument {\tt RO} or as a structure stored in {\tt model.Stack\{'info','fscOpt'\}}. Potential multiple definitions are handed by the following priority rule. Options are taken in input argument {\tt RO} if provided, it is then completed by the string command option parsing.  Additional fields specified in {\tt info,fscOpt} will eventually be added.

Fluid parameters are directly propagated to the global model if {\tt mof} is provided as an assembled model, see \ltr{fe\_case}{par} and {\tt matdes -1.1} in \lts{mattyp}. In such case parameters declared in {\tt mof} are translated as superelement parameters associated to the fluid superelement matrices in the base model.

The following options are available
\begin{itemize}
\item \ts{-combine} tells that the fluid model is compatible with the solid model, so that no fluid renumbering will be performed.
\item \ts{-name''}\tsi{fluid}\ts{''} provides the fluid superelement name. By default this is set to {\tt fluid}. Refer to \ltr{fesuper}{s\_} for supelement naming conventions and restrictions.
\item \ts{-skipPar} to skip fluid parameter definitions. By default fluid parameters are propagated as superelement parameters to the base model if the fluid is provided as an assembled structure.
\end{itemize}

%begindoc
\begin{verbatim}
% Generate a demonstration model containing a structure and a fluid (no interfaces)
mo1=fsc('TestModel');
% Declare the fluid and generate the superlement, based on the material
mo1=fsc('AddFluid',mo1,'matid1');
\end{verbatim}%continuedoc

% - - - - - - - - - - - - - - - - - - - -- 
\ruic{fsc}{AddCoupling}{}

\ts{AddCoupling} generates fluid interface coupling elements between the solid structure and the fluid model. The fluid model must have already been declared with command \ltr{fsc}{AddFluid}. These elements must be defined as compatible interfaces to the fluid model.
The base model either already contains the fluid interface elements or an externally defined fluid coupling model is provided, or it can be eventually defined on the fly by specifying a selection with \ltt{FindElt} providing the interface topology. In the latter case, one must keep in mind that the fluid coupling interface must be compatible with the fluid model, two cases exist depending on the provided selection.
\begin{itemize}
\item The \ltt{FindElt} selection is based on the fluid model. Compatibility with the fluid is directly ensured. An {\tt MPC} will be generated to couple the solid structure to the fluid interface using \ltt{ConnectionSurface}.
\item The \ltt{FindElt} selection is based on the solid model. One can declare that the selection is based on the solid by using command option \ts{-InSol}. Otherwise the selection is first tested on the fluid model, the selection is then applied to the solid if no matching element has been found in the fluid. In this case, the fluid coupling elements topology is compatible with the solid mesh in addition to its assumed compatibility to the fluid mesh. Solid based selection can thus only be used if the fluid mesh is compatible with the solid mesh.
\end{itemize}

Syntax is {\tt model=fsc('AddCoupling',model,moc,RO);} with
\begin{itemize}
\item {\tt model} the base model.
\item {\tt moc}  either a fluid interface model or a string \ltt{FindElt} selection providing either fluid interface elements or at least their topology.
\item {\tt RO} is an optional structure input of running options.
\end{itemize}

Output {\tt model} contains a coupling superelement and stack entry {\tt info,fscOpt} keeping track of the fluid and fluild interface superlements names and other options. In particular stack entry {\tt info,fscOpt} contains the fields
\begin{itemize}
\item {\tt .cname} the name of the coupling superelement
\item {\tt .name} the name of the fluid superelement
\item {\tt .MPCname} the name of the MPC constraint coupling the fluid interface elements to the solid. It remains empty if compatible meshes have been used so that no such MPC was needed.
\end{itemize}

Command options can be defined in two ways. It can either be specifed in the \ts{AddCoupling} command string with a {\tt -} prefix, or provided as an additional argument {\tt RO}. Potential multiple definitions are handed by the following priority rule. Options are taken in input argument {\tt RO} if provided, it is then completed by the string command option parsing. 


The following options are available
\begin{itemize}
\item \ts{-name''}\tsi{fscoup}\ts{''} provides the fluid coupling interface superlement name. By default this is set to {\tt fscoup}. Refer to \ltr{fesuper}{s\_} for supelement naming conventions and restrictions.
\item \ts{-InSol} to tell that coupling element selection must be performed on the solid model. In this case the resulting topology is assumed to be compatible with the fluid mesh.
\item \ts{-MPCname''}\tsi{FSCoupling}\ts{''} provides the base name of the \ltt{ConnectionSurface} MPC to be generated if needed to couple the fluid interface coupling to the solid.
\item \ts{-get} is used to output the coupling model only.
\item \ts{-Integ}\tsi{val} provides the integration rule to be applied for the fluid interface coupling elements. By default \tsi{val} is set to {\tt -1} for an integration rule peformed at center points. A classical alternative is to use rule {\tt -3} that will use the default law of the corresponding topology for solid applications.
\end{itemize}

%continuedoc
\begin{verbatim}
% Now declare coupling
% by default fluid surface is taken
mo1=fsc('AddCoupling',mo1);
\end{verbatim}%continuedoc


% - - - - - - - - - - - - - - - - - - - - 
\ruic{fsc}{SolveMVR}{[,-direct]}

\ts{SolveMVR} generates and assembles the reduced fluid structure coupled model. The solid and fluid are respectively projected onto their reduction basis and the fluid interface model is then projected. Reduced matrices are then assembled maintaining solid, fluid and coupling contributions apart. The global assembly rule is provided in a {\tt zCoef} stack entry, see \ser{zcoef}. 

Syntax is {\tt model=fsc('SolveMVR',model,RO);} with
\begin{itemize}
\item {\tt model} the base model containing a fluid superelement and a fluid interface coupling superelement,
\item {\tt RO} is an optional structure input of running options.
\end{itemize}

Output model is the base model, with additional stack entry {\tt SE,MVR} containing the reduced assembled fluid coupled model. This model itself is assembled and contains in its stack entry {\tt info,zCoef} the assembly rule, see \ser{zcoef}. The MVR entry contains the following fields
\begin{itemize}
\item {\tt .K} the assembled matrices.
\item {\tt .Klab} the assembled matrices labels.
\item {\tt .Opt} the assembled matrices types, see \lts{mattyp} for reference.
\item {\tt .TR} the restitution structure, see \ltr{fesuper}{SEDef} for reference.
\item {\tt .Stack} the additionally stacked information, in particular {\tt info,zCoef}, see \ser{zcoef} for reference.
\end{itemize}
If command option \ts{-direct} was used the following fields are also present
\begin{itemize}
\item {\tt .br} the reduced command matrix
\item {\tt .cr} the reduced observation matrix
\item {\tt .lab\_in} the labels associated to the reduced command matrix columns.
\item {\tt .lab\_out} the labels associated to the reduced observation matrix lines.
\end{itemize}

Command options should be defined with the {\tt RO} input with supported fields
\begin{itemize}
\item \ts{.NoFirstK} not to perform first order corrections on the fluid reduction basis if not set to {\tt 0} -- can be set in the command string.
\item \ts{.direct} to perform load and sensor projections and thus prepare the output model for direct FRF computations if not set to {\tt 0} -- can be set in the command string.
\item \ts{keepT} to store the solid reduction basis in {\tt model.Stack\{'curve','TR'\}} and the fluid reduction basis in {\tt model.Stack\{'curve','TF'\}} -- can be set in the command string.
\item \ts{.name}  provides the fluid superelement name, by default \tsi{fluid} is looked for.
\item \ts{.cname} provides the fluid interface coupling superelement name, by default, \tsi{fscoup} is looked for.
\item \ts{.SolidZeta} to provide a global damping ratio associated to the solid part. By default the result of {\tt fe\_def('defZeta',model)} is used, linked either to model entry {\tt info,DefaultZeta} or to the preference {\tt sdtdef DefaultZeta} with a factory setting value of 0.01. The corresponding loss factor is two times the damping ratio.
\item \ts{FluiEta} to provide a global fluid loss factor. By default the preference {\tt sdtdef FluiEta} is used with a factory setting value of zero. A loss factor is twice the associated damping ratio.
\item \ts{SEigOpt} to provide an {\tt EigOpt} option for the optional computation of solid modes in the procedure, see \feeig\ . The default values are recovered using {\tt fe\_def('defEigOpt',model)}.
\item \ts{FEigOpt} to provide an {\tt EigOpt} option for the computation of fluid modes in the procedure see \feeig\ . The default values are recovered using {\tt fe\_def('defEigOpt',model)}.
\item \ts{matdes} to provide the types of matrices to be assembled for the solid. By default this field is left empty and a standard assembly for mass and stiffness (types {\tt 2 1}) is performed. Hysteretic damping matrices (type {\tt 4}) are added if {\tt RO.DefaultZeta} or {\tt RO.FluiEta} are not null. See \lts{mattyp} for matrix types description.
\end{itemize}

%continuedoc
\begin{verbatim}
% Generate the reduced coupled fluid structure model
mo1=fsc('SolveMVR-keepT',mo1);  
% Recover the assembled model
MVR=stack_get(mo1,'SE','MVR','get');
\end{verbatim}%continuedoc


% - - - - - - - - - - - - - - - - - - - - 
\ruic{fsc}{SolveEig}{}

\ts{SolveEig} computes complex eigenmodes of the reduced fluid structure coupled model generated by \ltr{fsc}{SolveMVR}.

Syntax is {\tt def=fsc('SolveEig',model);}.

The following command options are available
\begin{itemize}
\item \ts{scale} to mass scale the output modes.
\item \ts{fr}\tsi{val} to use an input frequency of \tsi{val} for matrix coefficient resolution in case of property dependency. The default is set to {\tt 1}.
\end{itemize}

%continuedoc
\begin{verbatim}
% Compute coupled modes
def=fsc('SolveEig',mo1);
% Restitue modes on full DOF
dfull=fesuper('sedef',MVR.TR,def);
\end{verbatim}%enddoc



% ----------------------------------------------------------------------------

\rmain{Obsolete Non-conform conforming match}

SDT supports non conforming element for fluid/structure coupling \index{element!fluid} terms corresponding to the structure are computed using the classical elements of the SDT, and terms corresponding to the fluid are computed using the fluid elements (see \fluid).

The coupling term $C$ is computed using fluid/structure coupling elements (\fsc\ elements).

Only one integration point on each element (the center of gravity) is used to evaluate $C$.

When structural and fluid meshes do not match at boundaries, pairing of elements needs to be done.
The pairing procedure can be described for each element.
For each fluid element $F_i$, one takes the center of gravity $G_{f,i}$ (see figure), and searches the solid element $S_i$ which is in front of the center of gravity, in the direction of the normal to the fluid element $F_i$. The projection of $G_{f,i}$ on the solid element, $P_i$, belongs to $S_i$, and one computes the reference coordinate $r$ and $s$ of $P_i$ in $S_i$ (if $S_i$ is a quad4, $-1<r<1$ and $-1<s<1$). Thus one knows the weights that have to be associated to each node of $S_i$. The coupling term will associate the DOFs of $F_i$ to the DOFs of $S_i$, with the corresponding weights.

\begin{center}
\ingraph{50}{fsc2}
\end{center}

\end{SDT}
\rmain{See also}

\noindent \psolid, \melastic
%------------------------------------------------------------------------------
\rtop{hexa8, penta6, tetra4, and other 3D volumes}{hexa8}

\noindent Topology holders for 3D volume elements.
\index{element!solid}\index{solid element}

\rmain{Description}

The \hexah\, \hexav\, {\tt hexa27}, \penta\, \pentb\, \tetra\ and \tetrb\  elements are standard topology reference for 3D volume FEM problems. 


In a model description matrix, {\bf element property rows} for \hexah\ and \hexav\   elements follow the standard format with no element property used.  The generic format for an element containing $i$ nodes is 
{\tt [n1 ... ni MatID ProId EltId]}.
For example, the \hexah\ format is {\tt [n1 n2 n3 n4 n5 n6 n7 n8 MatID ProId EltId]}. 

These elements only define topologies, the nature of the problem to be solved should be specified using a property entry, see~\ser{feform} for supported problems and  \psolid, \pheat, ... for formats. 


Integration rules for various topologies are described under \integrules. Vertex coordinates of the reference element can be found using an \integrules\ command containing the name of the element such as {\tt r1=integrules('q4p');r1.xi}.


{\bf Backward compatibility note} : if no element property entry is defined, or with a {\tt p\_solid} entry with the integration rule set to zero, the element defaults to the historical 3D mechanic elements described in \ser{of_mk_subs}. 

\rmain{See also}

\femat, \melastic, \femk, \feplot  

\begin{OPENFEM}See \ser{rivlin} \end{OPENFEM}.

%------------------------------------------------------------------------------
\rtop{integrules}{integrules}

Command function for FEM integration rule support.

\rmain{Description}

This function groups integration rule manipulation utilities used by various elements. In terms of notations, a field $u$ is interpolated within an element by shapes functions $N_i$ and values of the field at nodes $u_i$
\begin{eqsvg}{integrules_2}
 u(x,y,z)=\sum_i N_i(r,s,t)u_i
\end{eqsvg} 

The relation between physical coordinates $x,y,z$ and element coordinates $r,s,t$ is itself described by a mapping associated with shape functions. When computing an integral, one selects a number of Gauss points $r_g,s_g,t_g$ and associated weights $w_g$ leading to an approximation of the integral as

\begin{eqsvg}{integrulesa}
 \int_V f(x,y,z) dV \approx \sum_g f(r_g,s_g,t_g) J w_g 
\end{eqsvg} 
%
where $J$ is the determinant of the Jacobian of the transform from reference to physical coordinates. The field {\tt .wjdet} is used to denote the local value of the product $J w_g$. 
%
The following calls generate the reference {\tt EltConst} data structure, see \ser{eltconst}.

\rui{\htr{integrules}{Gauss}} % - - - - - - - - - - - - - - - - - - - - - - - -

This command supports the definition of Gauss points and associated weights. It is called with {\tt integrules('Gauss Topology',RuleNumber)}. Supported topologies are {\tt 1d} (line), {\tt q2d} (2D quadrangle), {\tt t2d} (2D triangle), t{\tt 3d} (3D tetrahedron), {\tt p3d} (3D prism), {\tt h3d} (3D hexahedron). \texline {\tt integrules('Gauss q2d')} will list available 2D quadrangle rules.
\begin{itemize}
\item \htt{Integ} {\tt -3} is always the default rule for the order of the element.
\item {\tt -2} a rule at nodes.
\item {\tt -1} the rule at center.
\end{itemize}

\begin{verbatim}
    [ -3]    [ 0x1 double]    'element dependent default'        
    [ -2]    [ 0x1 double]    'node'        
    [ -1]    [ 1x4 double]    'center'      
    [102]    [ 4x4 double]    'gefdyn 2x2'  
    [  2]    [ 4x4 double]    'standard 2x2'
    [109]    [ 9x4 double]    'Q4WT'        
    [103]    [ 9x4 double]    'gefdyn 3x3'  
    [104]    [16x4 double]    'gefdyn 4x4'  
    [  9]    [ 9x4 double]    '9 point'     
    [  3]    [ 9x4 double]    'standard 3x3'
    [  2]    [ 4x4 double]    'standard 2x2'
    [ 13]    [13x4 double]    '2x2 and 3x3' 
\end{verbatim}


\rui{bar1,beam1,beam3} % - - - - - - - - - - - - - - - - - - - - - - - -

For integration rule selection, these elements use the 1D rules which list you can find using \texline {\tt integrules('Gauss1d')}.

Geometric orientation convention for segment \index{segment} is $\bullet$ (1) $\rightarrow$ (2)

One can show the edge using {\tt {\ti elt\_name}} \ts{edge} (e.g. {\tt beaml \ts{edge}}).

\rui{t3p,t6p} % - - - - - - - - - - - - - - - - - - - - - - - -

Vertex coordinates of the reference element can be found using {\tt r1=integrules('tria3');r1.xi}.

\begin{figure}[H]
\centering
\ingraph{50}{e_t3p} 
 \caption{{\tt t3p} reference element.}
\end{figure}

Vertex coordinates of the reference element can be found using {\tt r1=integrules('tria6');r1.xi}.

\begin{figure}[H]
\centering
\ingraph{50}{e_t6p} 
 \caption{{\tt t6p} reference element.}
\end{figure}

For integration rule selection, these elements use the 2D triangle rules which list you can find using {\tt integrules('Gausst2d')}.

Geometric orientation convention for triangle \index{triangle}is to number anti-clockwise in the
two-dimensional case (in the three-dimensional case, there is no orientation).\\
   $\bullet$ edge [1]: (1) $\rightarrow$ (2)  (nodes 4, 5,... if there are supplementary nodes) 
   $\bullet$ edge [2]: (2) $\rightarrow$ (3)  (...) 
   $\bullet$ edge [3]: (3) $\rightarrow$ (1)  (...)

One can show the edges or faces using {\tt {\ti elt\_name}} \ts{edge} or {\ti elt\_name} \ts{face} (e.g. {\tt t3p \ts{edge}}).

\rui{q4p, q5p, q8p} % - - - - - - - - - - - - - - - - - - - - - - - -

Vertex coordinates of the reference element can be found using {\tt r1=integrules('quad4');r1.xi}.

\begin{figure}[H]
\centering
\ingraph{50}{e_q4p} 
 \caption{{\tt q4p} reference element.}
\end{figure}

\begin{figure}[H]
\centering
\ingraph{50}{E_q5p} 
 \caption{{\tt q5p} reference element.}
\end{figure}

Vertex coordinates of the reference element can be found using the {\tt r1=integrules('quadb');r1.xi}.

\begin{figure}[H]
\centering
\ingraph{50}{e_q8p} 
 \caption{{\tt q8p} reference element.}
\end{figure}

For integration rule selection, these elements use the 2D quadrangle rules which list you can find using {\tt integrules('Gaussq2d')}.

Geometric orientation convention for quadrilateral\index{quadrilateral} is to number anti-clockwise (same remark as for the triangle)\\
   $\bullet$ edge [1]: (1) $\rightarrow$ (2) (nodes 5, 6, ...) 
   $\bullet$ edge [2]: (2) $\rightarrow$ (3) (...) 
   $\bullet$ edge [3]: (3) $\rightarrow$ (4) 
   $\bullet$ edge [4]: (4) $\rightarrow$ (1)

One can show the edges or faces using {\tt {\ti elt\_name}} \ts{edge} or {\ti elt\_name} \ts{face} (e.g. {\tt q4p \ts{edge}}).

\rui{tetra4,tetra10} % - - - - - - - - - - - - - - - - - - - - - - - -

3D tetrahedron geometries with linear and quadratic shape functions. Vertex coordinates of the reference element can be found using {\tt r1=integrules('tetra4');r1.xi} (or command {\tt 'tetra10'}).

\begin{figure}[H]
\centering
\ingraph{50}{E_Tetra4} % E fig3p1d
 \caption{{\tt tetra4} reference element.}
\end{figure}

\begin{figure}[H]
\centering
\ingraph{50}{E_Tetra10} % e_fig3p2c
 \caption{{\tt tetra10} reference element.}
\end{figure}

For integration rule selection, these elements use the 3D pentahedron rules which list you can find using {\tt integrules('Gausst3d')}.

Geometric orientation convention for tetrahedron\index{tetrahedron} is to have trihedral $(\vec{12},\vec{13},\vec{14})$ direct ($\vec{ij}$ designates the vector from point i to point j).\\
   $\bullet$ edge [1]: (1) $\rightarrow$ (2)  (nodes 5, ...) 
   $\bullet$ edge [2]: (2) $\rightarrow$ (3)  (...) 
   $\bullet$ edge [3]: (3) $\rightarrow$ (1)  \\
   $\bullet$ edge [4]: (1) $\rightarrow$ (4) 
   $\bullet$ edge [5]: (2) $\rightarrow$ (4) 
   $\bullet$ edge [6]: (3) $\rightarrow$ (4) (nodes ...,
   $p$)

   All faces, seen from the exterior, are described anti-clockwise:   \\
   $\bullet$ face [1]: (1) (3) (2) (nodes p+1, ...) 
   $\bullet$ face [2]: (1) (4) (3) (...)  \\
   $\bullet$ face [3]: (1) (2) (4) 
   $\bullet$ face [4]: (2) (3) (4)

One can show the edges or faces using {\tt {\ti elt\_name}} \ts{edge} or {\ti elt\_name} \ts{face} (e.g. {\tt tetra10 \ts{face}}).

\rui{penta6, penta15} % - - - - - - - - - - - - - - - - - - - - - - - - - - - - - -

3D prism geometries with linear and quadratic shape functions. Vertex coordinates of the reference element can be found using  {\tt r1=integrules('penta6');r1.xi} (or command {\tt 'penta15'}).

\begin{figure}[H]
\centering
\ingraph{50}{E_Penta6} % e_fig3r1d
 \caption{{\tt penta6} reference element.}
\end{figure}


\begin{figure}[H]
\centering
\ingraph{50}{E_Penta15} % e_fig3r2c
 \caption{{\tt penta15} reference element.}
\end{figure}

For integration rule selection, these elements use the 3D pentahedron rules which list you can find using {\tt integrules('Gaussp3d')}.

Geometric orientation convention for pentahedron\index{pentahedron} is to have trihedral $(\vec{12},\vec{13},\vec{14})$ direct \\
   $\bullet$ edge [1]: (1) $\rightarrow$ (2)  (nodes 7, ...) 
   $\bullet$ edge [2]: (2) $\rightarrow$ (3)  (...) 
   $\bullet$ edge [3]: (3) $\rightarrow$ (1)  \\
   $\bullet$ edge [4]: (1) $\rightarrow$ (4) 
   $\bullet$ edge [5]: (2) $\rightarrow$ (5) 
   $\bullet$ edge [6]: (3) $\rightarrow$ (6) \\
   $\bullet$ edge [7]: (4) $\rightarrow$ (5) 
   $\bullet$ edge [8]: (5) $\rightarrow$ (6) 
   $\bullet$ edge [9]: (6) $\rightarrow$ (4) (nodes ..., p)

   All faces, seen from the exterior, are described anti-clockwise. \\
   $\bullet$ face [1]~: (1) (3) (2)        (nodes p+1, ...) 
   $\bullet$ face [2]~: (1) (4) (6) (3)     
   $\bullet$ face [3]~: (1) (2) (5) (4) \\
   $\bullet$ face [4]~: (4) (5) (6) 
   $\bullet$ face [5]~: (2) (3) (6) (5)

One can show the edges or faces using {\tt {\ti elt\_name}} \ts{edge} or {\ti elt\_name} \ts{face} (e.g. {\tt penta15 \ts{face}}).

\rui{hexa8, hexa20, hexa21, hexa27} % - - - - - - - - - - - - - - - - - - - - 

3D brick geometries, using linear {\tt hexa8}, and quadratic shape functions.  
Vertex coordinates of the reference element can be found using {\tt r1=integrules('hexa8');r1.xi} (or command {\tt 'hexa20'}, {\tt 'hexa27'}).

\begin{figure}[H]
\centering
\ingraph{50}{E_Hexa8} % e_fig3q1d
 \nlvs\nlvs\nlvs\caption{{\tt hexa8} reference topology.}
\end{figure}

\begin{figure}[H]
\centering
\ingraph{50}{E_Hexa20} % e_fig3q2c
 \nlvs\nlvs\nlvs\caption{{\tt hexa20} reference topology.}
\end{figure}

For integration rule selection, these elements use the 3D hexahedron rules which list you can find using {\tt integrules('Gaussh3d')}.

Geometric orientation convention for hexahedron\index{hexahedron} is to have trihedral $(\vec{12},\vec{14},\vec{15}) $ direct \\
   $\bullet$ edge [1]: (1) $\rightarrow$ (2)  (nodes 9, ...) 
   $\bullet$ edge [2]: (2) $\rightarrow$ (3)  (...) 
   $\bullet$ edge [3]: (3) $\rightarrow$ (4)  \\
   $\bullet$ edge [4]: (4) $\rightarrow$ (1) 
   $\bullet$ edge [5]: (1) $\rightarrow$ (5) 
   $\bullet$ edge [6]: (2) $\rightarrow$ (6) \\
   $\bullet$ edge [7]: (3) $\rightarrow$ (7) 
   $\bullet$ edge [8]: (4) $\rightarrow$ (8) 
   $\bullet$ edge [9]: (5) $\rightarrow$ (6) \\
   $\bullet$ edge [10]: (6) $\rightarrow$ (7) 
   $\bullet$ edge [11]: (7) $\rightarrow$ (8) 
   $\bullet$ edge [12]: (8) $\rightarrow$ (5) (nodes ..., p)

   All faces, seen from the exterior, are described anti-clockwise. \\
   $\bullet$ face [1]~: (1) (4) (3) (2)  (nodes p+1, ...) 
   $\bullet$ face [2]~: (1) (5) (8) (4) \\  
   $\bullet$ face [3]~: (1) (2) (6) (5) 
   $\bullet$ face [4]~: (5) (6) (7) (8) \\
   $\bullet$ face [5]~: (2) (3) (7) (6) 
   $\bullet$ face [6]~: (3) (4) (8) (7)

One can show the edges or faces using {\tt {\ti elt\_name}} \ts{edge} or {\ti elt\_name} \ts{face} (e.g. {\tt hexa8 \ts{face}}).

\rui{\httts{BuildNDN}} % - - - - - - - - - - - - - - - - - - - - - - - - - - -

The commands are extremely low level utilities to fill the {\tt .NDN} field for a given set of nodes. The calling format is {\tt of\_mk('BuildNDN',type,rule,nodeE)} where {\tt type} is an {\tt int32} that specifies the rule to be used : 2 for 2D, 3 for 3D, 31 for 3D with xyz sorting of NDN columns, 23 for surface in a 3D model, 13 for a 3D line. A negative value can be used to switch to the \ts{.m} file implementation in {\tt integrules}. 

The 23 rule generates a transformation with the first axis along $N,r$, the second axis orthogonal in the plane tangent to $N,r$, $N,s$ and the third axis locally normal to the element surface. If a local material orientation is provided in columns 5 to 7 of {\tt nodeE} then the material $x$ axis is defined by projection on the surface. One recalls that columns of {\tt nodeE} are field based on the \ltt{InfoAtNode} field and the first three labels should be {\tt 'v1x','v1y','v1z'}.

With the 32 rule if a local material orientation is provided in columns 5 to 7 for $x$ and 8 to 10 for $y$ the spatial derivatives of the shape functions are given in this local frame. 

The {\tt rule} structure is described earlier in this section and {\tt node} has three columns that give the positions in the nodes of the current element. The {\tt rule.NDN} and {\tt rule.jdet} fields are modified. They must have the correct size before the call is made or severe crashes can be experienced.

If a {\tt rule.bas} field is defined ($9\times Nw$), each column is filled to contain the local basis at the integration point for {\tt 23} and {\tt 13} types. If a {\tt rule.J} field with ($4\times Nw$), each column is filled to contain the jacobian at the integration point for {\tt 23}.

%begindoc
\begin{verbatim}
model=femesh('testhexa8');  nodeE=model.Node(:,5:7);
opt=integrules('hexa8',-1);
nodeE(:,5:10)=0; nodeE(:,7)=1;  nodeE(:,8)=1; % xe=z and ye=y
integrules('buildndn',32,opt,nodeE)

model=femesh('testquad4'); nodeE=model.Node(:,5:7);
opt=integrules('q4p',-1);opt.bas=zeros(9,opt.Nw);opt.J=zeros(4,opt.Nw);
nodeE(:,5:10)=0; nodeE(:,5:6)=1;  % xe= along [1,1,0]
integrules('buildndn',23,opt,nodeE)
\end{verbatim}%enddoc



\rmain{See also} % - - - - - - - - - - - - - - - - - - - - - - - -

\elem

%------------------------------------------------------------------------------
\rtop{mass1,mass2}{mass1}

Concentrated mass elements.\index{element!property row}

\rmain{Description}

\ingraph{20}{E_Mass1}

   \mass\   places a diagonal concentrated mass and inertia at one node.


\noindent  In a model description matrix, {\bf element property rows} for \mass\   elements follow the format 

\begin{verbatim}
 [NodeID mxx myy mzz ixx iyy izz EltID]
\end{verbatim}


\noindent  where the concentrated nodal mass associated to the DOFs {\tt .01} to {\tt .06} of the indicated node is given by

\begin{verbatim}
   diag([mxx myy mzz ixx iyy izz])
\end{verbatim}


{\bf Note} \feutil\ {\tt GetDof} eliminates DOFs where the inertia is zero. You should thus use a small but non zero mass to force the use of all six DOFs.

For \massb\ elements, the {\bf element property rows} follow the format
\begin{verbatim}
 [n1 M I11 I21 I22 I31 I32 I33 EltID CID X1 X2 X3 MatId ProId]
\end{verbatim}


which, for no offset, corresponds to matrices given by

\begin{eqsvg}{mass1_1}
 \ma{\ba{cccccc} M &&& \makebox[.5cm][c]{symmetric} && \\
&M\\
&&M\\
&&& I_{11} \\
&&&-I_{21}&I_{22} \\
&&&-I_{31}&-I_{32}&I_{33} \\
\ea}=\ma{\ba{cccccc} \int \rho dV &&& \makebox[.5cm][c]{symmetric} && \\
&M\\
&&M\\
&&& \int \rho (x^2+y^2)dV \\
&&&-I_{21}&I_{22} \\
&&&-I_{31}&-I_{32}&I_{33} \\
\ea}
\end{eqsvg}

Note that local coordinates {\tt CID} are not currently supported by \massb\ elements.

\rmain{See also}

\femesh, \feplot

%------------------------------------------------------------------------------
\begin{OPENFEM}
%       Copyright (c) 2001-2015 by INRIA and SDTools, All Rights Reserved.
%       Use under OpenFEM trademark.html license and LGPL.txt library license
%       $Revision: 1.72 $  $Date: 2025/12/17 18:59:41 $

%------------------------------------------------------------------------------
\rtop{m\_elastic}{m_elastic}

 Material function for elastic solids and fluids.

\rsyntax\begin{verbatim}
 mat= m_elastic('default') 
 mat= m_elastic('database name') 
 mat= m_elastic('database -therm name') 
 pl = m_elastic('dbval MatId name');
 pl = m_elastic('dbval -unit TM MatId name');
 pl = m_elastic('dbval -punit TM MatId name');
 pl = m_elastic('dbval -therm MatId name');
\end{verbatim}

\rmain{Description}

This help starts by describing the main commands of {\tt m\_elastic} : \ts{Database} and \ts{Dbval}. 

Material formats supported by {\tt m\_elastic} are then described.

If you are not familiar with material property matrices, see section \ser{pl} before reading this help.

\ruic{m\_elastic}{Database}{,\htr{m\_elastic}{Dbval}] [-unit TY] [,MatiD]] Name} % - - - - - - - - - - - - - - 

A material property function is expected to store a number of standard materials.

{\tt m\_elastic('database Steel')} returns a the data structure describing steel.\\
{\tt m\_elastic('dbval 100 Steel')} only returns the property row. 

%begindoc
\begin{verbatim}
  % List of materials in data base
  m_elastic info
  % examples of row building and conversion
  pl=m_elastic([100 fe_mat('m_elastic','SI',1) 210e9 .3 7800], ...
    'dbval 101 aluminum', ...
    'dbval 200 lamina .27 3e9 .4 1200 0  790e9 .3 1780 0');
  pl=fe_mat('convert SITM',pl);
  pl=m_elastic(pl,'dbval -unit TM 102 steel')
\end{verbatim}%enddoc

Command option \ts{-unit} asks the output to be converted in the desired unit system.
Command option \ts{-punit} tells the function that the provided data is in a desired unit system (and generates the corresponding type).
Command option \ts{-therm} asks to keep thermal data (linear expansion coefficients and reference temperature) if existing.

You can generate orthotropic shell properties using the \ts{Dbval 100 lamina VolFrac Ef nu\_f rho\_f G\_f E\_m nu\_m Rho\_m G\_m} command which gives fiber and matrix characteristics as illustrated above (the volume fraction is that of fiber). 

The default material is steel.


To orient fully anisotropic materials, you can use the following command

%begindoc
\begin{verbatim}
 % Behavior of a material grain assumed orthotropic
 C11=168.4e9; C12=121.4e9; C44=75.4e9; % GPa
 C=[C11 C12 C12 0 0 0;C12 C11 C12 0 0 0;C12 C12 C11 0 0 0;
   0 0 0 C44 0 0;    0 0 0 0 C44 0;    0 0 0 0 0 C44]; 

 pl=[m_elastic('formulaPlAniso 1',C,basis('bunge',[5.175 1.3071 4.2012]));
     m_elastic('formulaPlAniso 2',C,basis('bunge',[2.9208 1.7377 1.3921]))];
\end{verbatim}%enddoc


\rmain{Subtypes}
{\tt m\_elastic} supports the following material subtypes\vs\vs

\ruic{m\_elastic}{1}{ : standard isotropic}

\noindent {\sl Standard isotropic materials}, see~\ser{feelas3d} and ~\ser{feelas2d}, are described by a row of the form

\begin{verbatim}
 [MatID   typ   E nu rho G Eta Alpha T0]
\end{verbatim}


\noindent with {\tt typ} an identifier generated with the {\tt fe\_mat('m\_elastic','SI',1)} command, $E$ (Young's modulus), $\nu$ (Poisson's ratio), 
$\rho$ (density), $G$ (shear modulus, set to $G=E/2(1+\nu)$ if equal to zero). $\eta$ loss factor for hysteretic damping modeling. $\alpha$ thermal expansion coefficient. $T_0$ reference temperature.
$G=E/2(1+\nu)$

By default $E$ and $G$ are interdependent through $G=E/2(1+\nu)$. One can thus define either $E$ and $G$ to use this property. If $E$ or $G$ are set to zero they are replaced on the fly by their theoretical expression. Beware that modifying only E or G, either using \feutil \ts{SetMat} or by hand, will not apply modification to the other coefficient. In case where both coefficients are defined, in thus has to modify both values accordingly.


\ruic{m\_elastic}{2}{ : acoustic fluid} % - - - - - - - - - - - - - - - - - - - - - -

\noindent {\sl Acoustic fluid} , see~\ser{feacoustics},are described by a row of the form

\begin{verbatim}
 [MatId typ rho C eta R]
\end{verbatim}


\noindent with {\tt typ} an identifier generated with the {\tt fe\_mat('m\_elastic','SI',2)} command, $\rho$ (density), $C$ (velocity) and $\eta$ (loss factor). The bulk modulus is then given by $K=\rho C^2$. 

For walls with an impedance (see~\ltr{p\_solid}{3} form 8), the real part of the impedance, which corresponds to a viscous damping on the wall is given by {\tt $Z=\rho C R$} (see~\eqr{feform_feacoustics_5} for matrix). If an imaginary part is to be present, one will use $Z=\rho C R(1+i \eta)$. In an acoustic tube the absorption factor is given by $\alpha=\frac{4R}{((R+1)^2+(R\eta)^2)}$. 

\ruic{m\_elastic}{3}{ : 3-D anisotropic solid} % - - - - - - - - - - - - - - - - - - - - - 

{\sl 3-D Anisotropic solid}, see~\ser{feelas3d}, are described by a row of the form

\begin{verbatim}
 [MatId typ Gij rho eta A1 A2 A3 A4 A5 A6 T0]
\end{verbatim}


with {\tt typ} an identifier generated with the {\tt fe\_mat('m\_elastic','SI',3)} command, $rho$ (density), $eta$ (loss factor) and $Gij$ a row containing 

\begin{verbatim}
 [G11 G12 G22 G13 G23 G33 G14 G24 G34 G44 ...
  G15 G25 G35 G45 G55 G16 G26 G36 G46 G56 G66]
\end{verbatim}

Note that shear is ordered $g_{yz}, g_{zx}, g_{xy}$ which may not be the convention of other software.

SDT supports material handling through 

\begin{itemize}
\item material bases defined for each element \htt{EltOrient} or each property \ltt{Coordm}.
\item orientation maps used for material handling are described in~\ser{VectFromDir}. It is then expected that the six components {\tt v1x,v1y,v1z,v2x,v2y,v2z} are stored sequentially in the interpolation table. It is then usual to store the MAP in the stack entry \ts{info,EltOrient}.
\end{itemize}


\ruic{m\_elastic}{4}{ : 2-D anisotropic solid} % - - - - - - - - - - - - - - - - - - - - - 

{\sl 2-D Anisotropic solid}, see~\ser{feelas2d}, are described by a row of the form

\begin{verbatim}
 [MatId typ E11 E12 E22 E13 E23 E33 rho eta a1 a2 a3 T0]
\end{verbatim}


with {\tt typ} an identifier generated with the {\tt fe\_mat('m\_elastic','SI',4)} command, $rho$ (density), $eta$ (loss factor) and $Eij$ elastic constants and $ai$ anisotropic thermal expansion coefficients.

\ruic{m\_elastic}{5}{ : shell orthotropic material} % - - - - - - - - - - - - - - - - - - - - - 

{\sl shell orthotropic material}, see~\ser{feshell} corresponding to NASTRAN MAT8,  are described by a row of the form

\begin{verbatim}
 [MatId typ E1 E2 nu12 G12 G1z G2z Rho A1 A2 T0 Xt Xc Yt Yc S Eta ...
   F12 STRN]
\end{verbatim}


with {\tt typ} an identifier generated with the {\tt fe\_mat('m\_elastic','SI',5)} command, $rho$ (density), ... See \ltr{m\_elastic}{Dbval}\ts{lamina} for building. 

\ruic{m\_elastic}{6}{ : Orthotropic material} % - - - - - - - - - - - - - - - - - - - - - - - - - -

{\sl 3-D orthotropic material}, see~\ser{feelas3d}, are described by a set of engineering constants, in a row of the form

\begin{verbatim}
 [MatId typ E1 E2 E3 Nu23 Nu31 Nu12 G23 G31 G12 rho a1 a2 a3 T0 eta]
\end{verbatim}


with {\tt typ} an identifier generated with the {\tt fe\_mat('m\_elastic','SI',6)} command, $Ei$ (Young modulus in each direction), $\nu ij$ (Poisson ratio), $Gij$ (shear modulus), $rho$ (density), $ai$ (anisotropic thermal expansion coefficient), $T_0$ (reference temperature), and $eta$ (loss factor).
Care must be taken when using these conventions, in particular, it must be noticed that

\begin{eqsvg}{m_elastic_ortho_mat}
 \nu_{ji} = \frac{E_j}{E_i} \nu_{ij}
\end{eqsvg}

\rmain{See also}

  \Ser{femp}, \ser{pl}, \femat, \pshell, \ltr{feutil}{SetMat}


%------------------------------------------------------------------------------
\rtop{m\_heat}{m_heat}

 Material function for heat problem elements.

\rsyntax\begin{verbatim}
 mat= m_heat('default') 
 mat= m_heat('database name') 
 pl = m_heat('dbval MatId name');
 pl = m_heat('dbval -unit TM MatId name');
 pl = m_heat('dbval -punit TM MatId name');
\end{verbatim}

\rmain{Description}

This help starts by describing the main commands of {\tt m\_heat} : \ts{Database} and \ts{Dbval}. Materials formats supported by {\tt m\_heat} are then described.

\ruic{m\_heat}{Database}{,Dbval] [-unit TY] [,MatiD]] Name} % - - - - - - - - - - - - - - 

A material property function is expected to store a number of standard materials. See \ser{pl} for material property interface.

{\tt m\_heat('DataBase Steel')} returns a the data structure describing steel.\\
{\tt m\_heat('DBVal 100 Steel')} only returns the property row. 

%begindoc
\begin{verbatim}
  % List of materials in data base
  m_heat info
  % examples of row building and conversion
  pl=m_heat('DBVal 5 steel');
  pl=m_heat(pl,...
    'dbval 101 aluminum', ...
    'dbval 200 steel');
  pl=fe_mat('convert SITM',pl);
  pl=m_heat(pl,'dbval -unit TM 102 steel')
\end{verbatim}%enddoc

\rmain{Subtypes}
{\tt m\_heat} supports the following material subtype\vs\vs

\ruic{m\_heat}{1}{ : Heat equation material} % - - - - - - - - - - - - - - - - - - - -

\begin{verbatim}
   [MatId fe_mat('m_heat','SI',2) k rho C Hf]
\end{verbatim}

\begin{itemize}
\item {\tt k} conductivity
\item {\tt rho} mass density
\item {\tt C}  heat capacity
\item {\tt Hf} heat exchange coefficient
\end{itemize}

\rmain{See also}

  \Ser{femp}, \ser{pl}, \femat, \pheat

%------------------------------------------------------------------------------
\rtop{m\_hyper}{m_hyper}

 Material function for hyperelastic solids.

\rsyntax\begin{verbatim}
 mat= m_hyper('default') 
 mat= m_hyper('database name') 
 pl = m_hyper('dbval MatId name');
 pl = m_hyper('dbval -unit TM MatId name');
 pl = m_hyper('dbval -punit TM MatId name');
\end{verbatim}

\rmain{Description}

Function based on {\tt m\_elastic} function adapted for hyperelastic material. Only subtype 1 is currently used:

\ruic{m\_hyper}{1}{ : Nominal hyperelastic material}

\noindent {\sl Nominal hyperelastic materials} are described by a row of the form

\begin{verbatim}
 [MatID   typ  rho Wtype C_1 C_2 K]
\end{verbatim}


\noindent with {\tt typ} an identifier generated with the {\tt fe\_mat('m\_hyper','SI',1)} command, $rho$ (density), $Wtype$ (value for Energy choice), $C_1$, $C_2$, $K$ (energy coefficients).\\
\noindent Possible values for $Wtype$ are:

$$
\begin{array}{ll}
0: & W = C_1(J_1-3) + C_2(J_2-3) + K(J_3-1)^2\\
1: & W = C_1(J_1-3) + C_2(J_2-3) + K(J_3-1) - (C_1 + 2C_2 + K)\ln(J_3)
\end{array}
$$

Other energy functions can be added by editing the {\tt hyper.c Enpassiv} function.

In RivlinCube test, m\_hyper is called in this form:
\begin{verbatim}
model.pl=m_hyper('dbval 100 Ref'); % this is where the material is defined
\end{verbatim}


the hyperelastic material called ``Ref'' is described in the database of {\tt m\_hyper.m} file:
\begin{verbatim}
  out.pl=[MatId fe_mat('type','m_hyper','SI',1) 1e-06 0 .3 .2 .3];
  out.name='Ref';
  out.type='m_hyper';
  out.unit='SI';
\end{verbatim}


Here is an example to set your material property for a given structure model:
\begin{verbatim}
model.pl = [MatID fe_mat('m_hyper','SI',1) typ rho Wtype C_1 C_2 K];
model.Elt(2:end,length(feval(ElemF,'node')+1)) = MatID;
\end{verbatim}



%       Copyright (c) 2001-2020 by SDTools and INRIA, All Rights Reserved.
%       Use under OpenFEM trademark.html license and LGPL.txt library license
%       $Revision: 1.115 $  $Date: 2020/10/26 07:58:04 $

%------------------------------------------------------------------------------
\rtop{p\_beam}{p_beam}

Element property function for beams

\rsyntax\begin{verbatim}
il = p_beam('default') 
il = p_beam('database','name') 
il = p_beam('dbval ProId','name');
il = p_beam('dbval -unit TM ProId name');
il = p_beam('dbval -punit TM ProId name');
il2= p_beam('ConvertTo1',il)
\end{verbatim}

\rmain{Description}

This help starts by describing the main commands : {\tt p\_beam} \ts{Database} and \ts{Dbval}. Supported {\tt p\_beam} subtypes and their formats are then described.

\ruic{p\_beam}{Database}{,Dbval,  ...} % - - - - - - - - - - - - - - - - - - - 

{\tt p\_beam} contains a number of defaults obtained with {\tt p\_beam('database')} or\\  
{\tt p\_beam('dbval {\ti MatId}')}. You can select a particular entry of the database with using a name matching the database entries. You can also automatically compute the properties of standard beams

\noindent\begin{tabular}{@{}p{.35\textwidth}@{}p{.65\textwidth}@{}}
%
\rz\ts{circle }\tsi{r}  & beam with full circular section of radius \tsi{r}.\\
\rz\ts{rectangle }\tsi{b h} & beam with full rectangular section of width \tsi{b} and height \tsi{h}. See \beam\ for orientation (the default reference node is 1.5, 1.5, 1.5 so that orientation MUST be defined for non-symmetric sections). \\
\rz\ts{Type }\tsi{r1 r2 ...}  & other predefined sections of subtype 3 are listed using {\tt p\_beam('info')}. 
\end{tabular}


%{\tt p\_beam('database reftube')} gives a reference property of subtype 3 for a tube.\\

For example, you will obtain the section property row with {\tt ProId} 100 associated with a circular cross section of $0.05 m$ or a rectangular $0.05 \times 0.01 m$ cross section using

%begindoc
\begin{verbatim}
 % ProId 100, rectangle 0.05 m by 0.01 m
 pro = p_beam('database 100 rectangle .05 .01')
 % ProId 101 circle radius .05
 il = p_beam(pro.il,'dbval 101 circle .05')
 p_beam('info')
 % ProId 103 tube external radius .05 internal .04
 il = p_beam(il,'dbval -unit SI 103 tube .05 .04')
 % Transform to subtype 1
 il2=p_beam('ConvertTo1',il)
 il(end+1,1:6)=[104 fe_mat('p_beam','SI',1) 0 0 0 1e-5];
 il = fe_mat('convert SITM',il);
% Generate a property in TM, providing data in SI
 il = p_beam(il,'dbval -unit TM 105 rectangle .05 .01')
% Generate a property in TM providing data in TM
  il = p_beam(il,'dbval -punit TM 105 rectangle 50 10')
\end{verbatim}%enddoc

\ruic{p\_beam}{Show3D}{,MAP  ...} % - - - - - - - - - - - - - - - - - - - 

%begindoc
\begin{verbatim}

\end{verbatim}%enddoc



\ruic{p\_beam}{format}{ description and subtypes} % - - - - - - - - - - - - - - 

Element properties are described by the row of an element property matrix or a data structure with an {\tt .il} field containing this row (see \ser{il}). Element property functions such as {\tt p\_beam} support graphical editing of properties and a database of standard properties. 

For a tutorial on material/element property handling see \ser{femp}. For a programmers reference on formats used to describe element properties see \ser{il}. 

\ruic{p\_beam}{1}{ : standard} % - - - - - - - - - - - - - - - - - - - -

%\pbeam\ currently only supports a single format (\femat\ property subtype)

\begin{verbatim}
  [ProID   type   J I1 I2 A   k1 k2 lump NSM]
\end{verbatim}


\noindent\begin{tabular}{@{}p{.25\textwidth}@{}p{.75\textwidth}@{}}
%
{\ti ProID} & element property identification number. \\
\rz{\tt type}  & identifier obtained with {\tt fe\_mat('p\_beam','SI',1)}. \\
\rz{\tt J}  & torsional stiffness parameter (often different from polar moment of inertia {\tt I1+I2}). \\
\rz{\tt I1} & moment of inertia for bending plane 1 defined by a third node {\tt nr} or the vector {\tt vx vy vz} (defined in the \beam\ element). For a case with a beam along $x$ and plane 1 the $xy$ plane {\tt I1} is equal to $Iz = \int_{S} y^2 ds$. \\
\rz{\tt I2} & moment of inertia for bending plane 2 (containing the beam and orthogonal to plane 1. \\
\rz{\tt A} & section area. \\
\rz{\tt k1} & (optional) shear factor for motion in plane 1 (when not 0, a
                     Timoshenko beam element is used). The effective
                     area of shear is given by $k_1A$.  \\
\rz{\tt k2} & (optional) shear factor for direction 2.\\
\rz{\tt lump} & (optional) request for lumped mass model. 1 for inclusion of inertia terms. 2 for simple half mass at node. \\
\rz{\tt NSM} & (optional) non structural mass (density per unit length).\\
\end{tabular}\par

\bare\   elements only use the section area. All other parameters are ignored.

\beam\ elements use all parameters.  Without correction factors ({\ti k1} {\ti k2} not given or set to 0), the \beam\ element is the standard Bernoulli-Euler 12 DOF element based on linear interpolations for traction and torsion and cubic interpolations for flexion (see Ref.  \ecite{ger3} for example). When non zero shear factors are given, the bending properties are based on a Timoshenko beam element with selective reduced integration of the shear stiffness \ecite{imb1}. No correction for rotational inertia of sections is used.

\begin{SDT}
\ruic{p\_beam}{3}{ : Cross section database } % - - - - - - - - - - - - - - - - - - - -

This subtype can be used to refer to standard cross sections defined in database. It is particularly used by \nasread\ when importing NASTRAN {\tt PBEAML} properties.

\begin{verbatim}
  [ProID   type   0  Section Dim(i) ... ]
\end{verbatim}


\noindent\begin{tabular}{@{}p{.25\textwidth}@{}p{.75\textwidth}@{}}
%
{\ti ProID} & element property identification number. \\
\rz{\tt type} & identifier obtained with {\tt fe\_mat('p\_beam','SI',3)}. \\
\rz{\tt Section} & identifier of the cross section obtained with {\tt comstr('}\tsi{SectionName}{\tt ',-32)} where \tsi{SectionName} is a string defining the section (see below).\\
\rz{\tt Dim1 ...} & dimensions of the cross section.\\
\end{tabular}

Cross section, if existing, is compatible with NASTRAN {\tt PBEAML} definition. Equivalent moment of inertia and tensional stiffness are computed at the centroid of the section.
Currently available sections are listed with {\tt p\_beam('info')}. In particular one has {\tt ROD} (1 dim), {\tt TUBE} (2 dims), {\tt T} (4 dims), {\tt T2} (4 dims), {\tt I} (6 dims), {\tt BAR} (2 dims), {\tt CHAN1} (4 dims), {\tt CHAN2} (4 dims).

For \ts{NSM} and \ts{Lump} support \ts{ConverTo1} is used during definition to obtain the equivalent {\tt subtype 1} entry. 

\end{SDT}

\rmain{See also}

  \Ser{femp}, \ser{il}, \femat 
%------------------------------------------------------------------------------
\rtop{p\_heat}{p_heat}

Formulation and material support for the heat equation.

\rsyntax\begin{verbatim}
il = p_heat('default') 
\end{verbatim}

\rmain{Description}

This help starts by describing the main commands : {\tt p\_heat} \ts{Database} and \ts{Dbval}. Supported {\tt p\_heat} subtypes and their formats are then described. For theory see \ser{fe3dth}.

\ruic{p\_heat}{Database}{,Dbval]  ...} % - - - - - - - - - - - - - - - - - - - 

Element properties are described by the row of an element property matrix or a data structure with an {\tt .il} field containing this row (see \ser{il}). Element property functions such as {\tt p\_solid} support graphical editing of properties and a database of standard properties. 

{\tt p\_heat} database

%begindoc
\begin{verbatim}
 il=p_heat('database');
\end{verbatim}%enddoc

Accepted commands for the database are 
%
\begin{itemize}
\item \ts{d3 }\tsi{Integ }\tsi{SubType} : \ltt{Integ} integration rule for 3D volumes (default -3). 
\item \ts{d2 }\tsi{Integ }\tsi{SubType} : \ltt{Integ} integration rule for 2D volumes (default -3).
\end{itemize}

For fixed values, use {\tt p\_heat('info')}.

Example of database property construction

%begindoc
\begin{verbatim}
  il=p_heat([100 fe_mat('p_heat','SI',1) 0 -3 3],...
             'dbval 101 d3 -3 2');
\end{verbatim}%enddoc


\ruic{p\_heat}{Heat}{ equation element properties} % - - - - - - - - - - - - - - 

Element properties are described by the row of an element property matrix or a data structure with an {\tt .il} field containing this row. Element property functions such as {\tt p\_beam} support graphical editing of properties and a database of standard properties. 


\ruic{p\_heat}{1}{ : Volume element for heat diffusion (dimension DIM)} % - - - - - - - - - - - - - - - - - - - -

%\pbeam\ currently only supports a single format (\femat\ property subtype)

\begin{verbatim}
  [ProId fe_mat('p_heat','SI',1) CoordM Integ DIM]
\end{verbatim}


\noindent\begin{tabular}{@{}p{.25\textwidth}@{}p{.75\textwidth}@{}}
%
{\ti ProID} & element property identification number \\
\rz{\tt type}  & identifier obtained with {\tt fe\_mat('p\_beam','SI',1)} \\
\rz{\tt Integ}  & is rule number in integrules \\
\rz{\tt DIM}  & is problem dimension 2 or 3 D \\
\end{tabular}\par

\ruic{p\_heat}{2}{ : Surface element for heat exchange (dimension DIM-1)} % - - - - - - - - - - - - - - - - - - - -

\begin{verbatim}
   [ProId fe_mat('p_heat','SI',2) CoordM Integ DIM] 
\end{verbatim}


\noindent\begin{tabular}{@{}p{.25\textwidth}@{}p{.75\textwidth}@{}}
%
{\ti ProID} & element property identification number \\
\rz{\tt type}  & identifier obtained with {\tt fe\_mat('p\_beam','SI',2)} \\
\rz{\tt Integ}  & is rule number in {\tt integrules} \\
\rz{\tt DIM}  & is problem dimension 2 or 3 D \\
\end{tabular}\par

\ruic{p\_heat}{SetFace}{} % - - - - - - - - - - - - - - - - - - - - - -
This command can be used to define a surface exchange and optionally associated load.
Surface exchange elements add a stiffness term to the stiffness matrix related to the exchange coefficient {\tt Hf} defined in corresponding material property. One then should add a load corresponding to the exchange with the source temperature at $T_0$ through a convection coefficient {\tt Hf} which is {\tt Hf.T\_0}. If not defined, the exchange is done with source at temperature equal to 0. 

{\tt  model=p\_heat('SetFace',model,SelElt,pl,il);}\\

\begin{itemize}
\item{\tt SelElt} is a findelt command string to find faces that exchange heat (use 'SelFace' to select face of a given preselected element).
\item{\tt pl} is the identifier of existing material property ({\tt MatId}), or a vector defining an {\tt m\_heat} property.
\item{\tt il} is the identifier  of existing element property ({\tt ProId}), or a vector defining an {\tt p\_heat} property.
\end{itemize}

Command option \ts{-load }\tsi{T} can be used to defined associated load, for exchange with fluid at temperature \tsi{T}. Note that if you modify {\tt Hf} in surface exchange material property you have to update the load.

Following example defines a simple cube that exchanges with thermal source at 55 deg on the bottom face.

%beginddoc
\begin{verbatim} 
model=femesh('TestHexa8'); % Build simple cube model
model.pl=m_heat('dbval 100 steel'); % define steel heat diffusion parameter
model.il=p_heat('dbval 111 d3 -3 1'); % volume heat diffusion (1)
model=p_heat('SetFace-load55',... % exchange at 55 deg
    model,...
    'SelFace & InNode{z==0}',... % on the bottom face
    100,... % keep same matid for exchange coef
    p_heat('dbval 1111 d3 -3 2')); % define 3d, integ-3, for surface exchange (2)
cf=feplot(model); fecom colordatapro
def=fe_simul('Static',model); % compute static thermal state
mean(def.def)
\end{verbatim}%enddoc

\ruic{p\_heat}{2D}{validation} % - - - - - - - - - - - - - - - - - - - 

Consider a bi-dimensional annular thick domain $\Omega$ with radii $r_e=1$ and $r_i=0.5$. The data are specified on the internal circle $\Gamma_i$ and on the external circle $\Gamma_e$. The solid is made of homogeneous isotropic material, and its conductivity tensor thus reduces to a constant $k$. The steady state temperature distribution is then given by
\begin{eqsvg}{test_ann}
- k \Delta\theta(x,y) = f(x,y) \quad in \quad \Omega.
\end{eqsvg}

The solid is subject to the following boundary conditions\\
\begin{itemize}
\item{ {$\Gamma_i \,(r=r_i)$ : Neumann condition}\\
\begin{eqsvg}{p_heat_validation_1}
\displaystyle\frac{\partial \theta}{\partial n}(x,y) = g(x,y)
\end{eqsvg}  }
\item{ {$\Gamma_e \,(r=r_e)$ : Dirichlet condition}\\
\begin{eqsvg}{p_heat_validation_2}
\theta(x,y)=\theta_{ext}(x,y)
\end{eqsvg}  }
\end{itemize}

In above expressions, $f$ is an internal heat source, \mathsvg{\theta_{ext}}{p_heat_validation_l1} an external temperature at $r=r_e$, and $g$ a function. All the variables depend on the variable $x$ and $y$. 

The OpenFEM model for this example can be found in {\tt ofdemos('AnnularHeat')}.\\
{\bf Numerical application} : assuming $k=1$, $f=0$, $Hf=1e^{-10}$, $\theta_{ext}(x,y) = \exp(x) \cos(y)$ and \mathsvg{g(x,y)= -\frac{\exp(x)} {r_i} \left ( \cos(y)  x  - \sin(y)  x \right )}{p_heat_validation_l2}, the solution of the problem is  given by
\mathsvg{\displaystyle \theta(x,y) = \exp(x) \cos(y)}{p_heat_validation_l3}



\rmain{See also}

  \ser{fe3dth}, \ser{femp}, \femat 

%------------------------------------------------------------------------------
\IfFileExists{../tex/p_pml.tex}{\input{../tex/p_pml.tex}}{}

%------------------------------------------------------------------------------
\rtop{p\_shell}{p_shell}

Element property function for shells and plates (flat shells)

\rsyntax\begin{verbatim}
il = p_shell('default');
il = p_shell('database ProId name'); 
il = p_shell('dbval ProId name');
il = p_shell('dbval -unit TM ProId name');
il = p_shell('dbval -punit TM ProId name');
il = p_shell('SetDrill 0',il);
\end{verbatim}

\rmain{Description}

This help starts by describing the main commands : {\tt p\_shell} \ts{Database} and \ts{Dbval}. Supported {\tt p\_shell} subtypes and their formats are then described.


\ruic{p\_shell}{Database}{,Dbval,  ...} % - - - - - - - - - - - - - - - - - - - 

{\tt p\_shell} contains a number of defaults obtained with the \ts{database} and \ts{dbval} commands which respectively return a structure or an element property row. You can select a particular entry of the database with using a name matching the database entries. 


You can also automatically compute the properties of standard shells with

\noindent\begin{tabular}{@{}p{.35\textwidth}@{}p{.65\textwidth}@{}}
%
\rz\ts{kirchhoff }\tsi{e}  & Kirchhoff shell of thickness \tsi{e} (is not implemented for formulation 5, see each element for available choices)\\
\rz\ts{mindlin }\tsi{e}  & Mindlin shell of thickness \tsi{e} (see each element for choices). \\
\rz\ts{laminate }\tsi{MatIdi Ti Thetai}  & Specification of a laminate property by giving the different ply {\tt MatId}, thickness and angle. By default the z values are counted from -thick/2, you can specify another value with a z0.
%
\end{tabular}

You can append a string option of the form \ts{-f }\tsi{i} to select the appropriate shell formulation. The different formulations are described under each element topology (\triaa, \quad4, ...)
For example, you will obtain the element property row with {\tt ProId} 100 associated with a .1 thick Kirchhoff shell (with formulation 5) or the corresponding Mindlin plate use

%begindoc
\begin{verbatim}
 il = p_shell('database 100 MindLin .1')
 il = p_shell('dbval 100 kirchhoff .1 -f5')
 il = p_shell('dbval 100 laminate z0=-2e-3 110 3e-3 30 110 3e-3 -30')
 il = fe_mat('convert SITM',il);
 il = p_shell(il,'dbval -unit TM 2 MindLin .1') % set in TM, provide data in SI
 il = p_shell(il,'dbval -punit TM 2 MindLin 100') % set in TM, provide data in TM
\end{verbatim}%enddoc


For laminates, you specify for each ply the {\tt MatId}, thickness and angle.

\ruic{p\_shell}{Shell}{ format description and subtypes} % - - - - - - - - - - - - - - 

Element properties are described by the row of an element property matrix or a data structure with an {\tt .il} field containing this row (see \ser{il}). Element property functions such as {\tt p\_shell} support graphical editing of properties and a database of standard properties. 

For a tutorial on material/element property handling see \ser{femp}. For a reference on formats used to describe element properties see \ser{il}. 

\pshell\ currently only supports two subtypes

\ruic{p\_shell}{1}{ : standard isotropic} % - - - - - - - - - - - - - - - - - - - -

\begin{verbatim}
  [ProID type   f d O   h   k   MID2 RatI12_T3 MID3 NSM Z1 Z2 MID4]
\end{verbatim}


\noindent\begin{tabular}{@{}p{.05\textwidth}@{}p{.05\textwidth}@{}p{.9\textwidth}@{}}
%
\rz{\tt type}  &   &  identifier obtained with {\tt fe\_mat('p\_shell','SI',1)}.\\
\rz{\tt f} &  & \rz{{\tt 0}} use default of element. For other formulations the specific help for each element (\quada, \triaa, ...), each formulation specifies integration rule. \\
\rz{\tt d} & \rz{\tt -1} & no drilling stiffness. The element DOFs are the standard translations and rotations at all nodes (DOFs {\tt .01} to {\tt .06}). The drill DOF (rotation {\tt .06} for a plate in the {\sl xy} plane) has no stiffness and is thus eliminated by \femk\ if it corresponds to a global DOF direction. The default is {\tt d=1} ({\tt d} is set to 1 for a declared value of zero). \\
& \rz{\tt d} & arbitrary drilling stiffness with value proportional to {\tt d} is added. This stiffness is often needed in shell problems but may lead to numerical conditioning problems if the stiffness value is very different from other physical stiffness values. Start with a value of 1. Use {\tt il=p\_shell('SetDrill d',il)} to set to {\tt d} the drilling stiffness of all {\tt p\_shell} subtype 1 rows of the property matrix {\tt il}. \\
\rz{\tt h} &  & plate thickness.\\
\rz{\tt k} & {\ti k} & shear correction factor (default 5/6, default used if {\tt k} is zero). This correction is not used for formulations based on triangles since \triaa\ is a thin plate element. \\
\rz{\tt RatI12\_T3} &  & Ratio of bending moment of inertia to nominal {\tt T3/I12} (default 1).\\
\rz{\tt  NSM} &  & Non structural mass per unit area.\\
\rz{\tt  MID2} &  & material property for bending. Defauts to element {\tt MatId} if equal to 0. \\
\rz{\tt  MID3} &  & material property for transverse shear. \\
\rz{\tt  z1,z2} &  & (unused) offset for fiber computations.\\
\rz{\tt  MID4} &  & material property for membrane/bending coupling.\\
\end{tabular}

Shell strain is defined by the membrane, curvature and transverse shear \texline (display with {\tt p\_shell('ConstShell')}). 
%
\begin{eqsvg}{p_shell_1}
\ve{\ba{c}\epsilon_{xx} \\\epsilon_{yy} \\ 2 \epsilon_{xy} \\ \kappa_{xx} \\\kappa_{yy} \\ 2 \kappa_{xy} \\ \gamma_{xz} \\ \gamma_{yz} \ea}=\ma{\ba{cccccccc}
 N,x & 0 & 0 & 0 & 0 \\
 0 & N,y & 0 & 0 & 0 \\
 N,y & N,x & 0 & 0 & 0 \\
 0 & 0 & 0 & 0 & N,x \\
 0 & 0 & 0 & -N,y & 0 \\
 0 & 0 & 0 & -N,x & N,y \\
 0 & 0 & N,x & 0 & -N \\
 0 & 0 & N,y & N & 0 \ea}
\ve{\ba{c} u \\ v \\ w \\ ru \\ rv \ea}
\end{eqsvg}

\ruic{p\_shell}{2}{ : composite} % - - - - - - - - - - - - - - - - - - - -

\begin{verbatim}
  [ProID type   Z0 NSM SB FT TREF GE LAM MatId1 T1 Theta1 SOUT1 ...]
\end{verbatim}


\noindent\begin{tabular}{@{}p{.15\textwidth}@{}p{.85\textwidth}@{}}
%
\rz{{\tt ProID}} &  Section property identification number. \\
\rz{{\tt type}}     &  Identifier obtained with {\tt fe\_mat('p\_shell','SI',2)}.\\
\rz{{\tt Z0}}     &  Distance from reference plate to bottom surface. \\
\rz{{\tt NSM}}     & Non structural mass per unit area. \\
\rz{{\tt SB}}     & Allowable shear stress of the bonding material. \\
\rz{{\tt FT}}     & Failure theory. \\
\rz{{\tt TREF}}     &  Reference temperature. \\
\rz{{\tt Eta}}     &  Hysteretic loss factor. \\
\rz{{\tt LAM}}     &  Laminate type. \\
\rz{{\tt MatId{\ti i}}} &  {\tt MatId} for ply {\ti i}, see \ltr{m\_elastic}{1},  \ltr{m\_elastic}{5}, ...\\
\rz{{\tt T{\ti i}}} &  Thickness of ply {\ti i}. \\
\rz{{\tt Theta{\ti i}}} &  Orientation of ply {\ti i}. \\
\rz{{\tt SOUT{\ti i}}} &  Stress output request for ply {\ti i}.
\end{tabular}

Note that this subtype is based on the format used by NASTRAN for {\tt PCOMP} and the formulation used for each topology is discussed in each element (see \quada, \triaa). You can use the \ts{DbvalLaminate} commands to generate standard entries.

\begin{eqsvg}{p_shell_2}
\ve{\ba{c}N \\ M \\ Q\ea} = \ma{\ba{ccc} A & B & 0\\ B & D & 0\\ 0 & 0 &
F\ea} \ve{\ba{c}\epsilon \\ \kappa \\ \gamma \ea}
\end{eqsvg}

\ruic{p\_shell}{setTheta}{}

When dealing with laminated plates, the classical approach uses a material orientation constant per element. OpenFEM also supports more advanced strategies with orientation defined at nodes but this is still poorly documented.

The material orientation is the reference for plies. Any angle defined in a laminate command is an additional rotation. In the example below, the element orientation is rotated 30 degrees, and the ply another 30. The fibers are thus oriented 60 degrees in the $xy$ plane. Stresses are however given in the material orientation thus with a 30 degree rotation. Per ply output is not currently implemented. 

The element-wise material angle is stored for each element. In column 7 for \triaa, 8 for \quada, ...  The \ts{setTheta} command is a utility to ease the setting of these angles. By default, the orientation is done at element center. To use the mean orientation at nodes use command option \ts{-strategy 2}.

\begin{verbatim}
model=ofdemos('composite');
model.il = p_shell('dbval 110 laminate 100 1 30'); % single ply

% Define material angle based on direction at element
MAP=feutil('getnormalElt MAP -dir1',model);
bas=basis('rotate',[],'rz=30;',1);
MAP.normal=MAP.normal*reshape(bas(7:15),3,3)';
model=p_shell('setTheta',model,MAP);

% Obtain a MAP of material orientations
MAP=feutil('getnormalElt MAP -dir1',model);
feplot(model);fecom('showmap',MAP)

% Set elementwise material angles using directions given at nodes. 
% Here a global direction
MAP=struct('normal',ones(size(model.Node,1),1)*bas(7:9), ...
    'ID',model.Node(:,1),'opt',2);
model=p_shell('setTheta',model,MAP);

% Using an analytic expression to define components of 
% material orientation vector at nodes
data=struct('sel','groupall','dir',{{'x-0','y+.01',0}},'DOF',[.01;.02;.03]);
model=p_shell('setTheta',model,data);
MAP=feutil('getnormalElt MAP -dir1',model);
feplot(model);fecom('showmap',MAP)
\end{verbatim}

{\tt model=p\_shell('setTheta',model,0)} is used to reset the material orientation to zero.


Technically, shells use the {\tt of\_mk('BuildNDN')} rule 23 which generates a basis at each integration point. The first vector {\tt v1x,v1y,v1z} is built in the direction of $r$ lines and {\tt v2x,v2y,v2z} is tangent to the surface and orthogonal to $v1$. When a \ltt{InfoAtNode} map provides {\tt v1x,v1y,v1z}, this vector is projected (NEED TO VERIFY) onto the surface and $v2$ taken to be orthogonal. 


\rmain{See also}

  \Ser{femp}, \ser{il}, \femat


%------------------------------------------------------------------------------
\rtop{p\_solid}{p_solid}

Element property function for volume elements.

\rsyntax\begin{verbatim}
il=p_solid('database ProId Value')
il=p_solid('dbval ProId Value')
il=p_solid('dbval -unit TM ProId name');
il=p_solid('dbval -punit TM ProId name');
model=p_solid('default',model)
\end{verbatim}


\rmain{Description}


This help starts by describing the main commands : {\tt p\_solid} \ts{Database} and \ts{Dbval}. Supported {\tt p\_solid} subtypes and their formats are then described.

\ruic{p\_solid}{Database}{, Dbval, Default,  ...} % - - - - - - - - - - - - - - - - - - - 

Element properties are described by the row of an element property matrix or a data structure with an {\tt .il} field containing this row (see \ser{il}). Element property functions such as {\tt p\_solid} support graphical editing of properties and a database of standard properties. 

Accepted commands for the database are 
%
\begin{itemize}
\item \ts{d3 }\tsi{Integ} : \tsi{Integ} integration rule for quadratic 3D volumes. For information on rules available see \ltr{integrules}{Gauss}. Examples are \ts{d3 2}  2x2x2 integration rule for linear volumes (hexa8 ... ); \ts{d3 -3} default integration for all 3D elements, ...
\item\ts{d2 }\tsi{Integ} :  \tsi{Integ} integration rule for quadratic 2D volumes. For example \ts{d2 2} 2x2x2 integration rule for linear volumes (q4p ... ). You can also use \ts{d2 1 0 2} for plane stress, and \ts{d2 2 0 2} for axisymmetry.
\item\ts{fsc }\tsi{Integ} : integration rule selection for fluid/structure coupling.
\end{itemize}

For fixed values, use {\tt p\_solid('info')}.

For a tutorial on material/element property handling see \ser{femp}. For a reference on formats used to describe element properties see \ser{il}. 

Examples of database property construction

%begindoc
\begin{verbatim}
  il=p_solid([100 fe_mat('p_solid','SI',1) 0 3 0 2], ...
             'dbval 101 Full 2x2x2','dbval 102 d3 -3');
  il=fe_mat('convert SITM',il);
  il=p_solid(il,'dbval -unit TM 2 Reduced shear')
  % Try a smart guess on default 
  model=femesh('TestHexa8');model.il=[]; 
  model=p_solid('default',model) 
\end{verbatim}%enddoc


\ruic{p\_solid}{1}{ : 3D volume element} % - - - - - - - - - - - - - - 

\begin{verbatim}
[ProID fe_mat('p_solid','SI',1) Coordm In Stress Isop ]
\end{verbatim}


\noindent\begin{tabular}{@{}p{.2\textwidth}@{}p{.8\textwidth}@{}}
%
\rz{{\tt ProID}}  &  Property identification number.\\
\rz{{\tt Coordm}} &  Identification number of the material coordinates system. {\bf Warning}  not implemented for all material formulations. \\
\rz{{\tt In}}     &  Integration rule selection (see \ltr{integrules}{Gauss}). 0 selects the legacy 3D mechanics element ({\tt of\_mk\_pre.c}), -3 the default rule. \\
\rz{{\tt Stress}} &  Location selection for stress output (NOT USED).\\
\rz{{\tt Isop}}   &  Integration scheme.  Used to select the generalized strain definition in \nlinout\ implementations (see~\ser{nlio3d}). May also be used to select shear protection mechanisms in the future. \\
\end{tabular}

The underlying physics for this subtype are selected through the material property. Examples are 3D mechanics with \melastic, \begin{SDT} piezo electric volumes (see {\tt m\_piezo})\end{SDT}, heat equation (\pheat).

\ruic{p\_solid}{2}{ : 2D volume element } % - - - - - - - - - - - - - - - - - - -

\begin{verbatim}
  [ProId fe_mat('p_solid','SI',2)  Form N In]
\end{verbatim}


\noindent\begin{tabular}{@{}p{.2\textwidth}@{}p{.8\textwidth}@{}}
%
\rz{{\tt ProID}}  &  Property identification number.\\
\rz{{\tt Type}}   &  Identifier obtained with {\tt fe\_mat('p\_solid,'SI',2)}.\\
\rz{{\tt Form}}   &  Formulation (0 plane strain, 1 plane stress, 2 axisymmetric), see details in \melastic. \\
\rz{{\tt N}}      &  Fourier harmonic for axisymmetric elements that support it.\\
\rz{{\tt In}}     &  Integration rule selection (see \ltr{integrules}{Gauss}). 0 selects legacy 2D element, -3 the default rule.
\end{tabular}

The underlying physics for this subtype are selected through the material property. Examples are 2D mechanics with \melastic.

\ruic{p\_solid}{3}{ : ND-1 coupling element} % - - - - - - - - - - - - - - - - - - -

\begin{verbatim}
  [ProId fe_mat('p_solid','SI',3) Integ Form Ndof1 ...]
\end{verbatim}


\noindent\begin{tabular}{@{}p{.2\textwidth}@{}p{.8\textwidth}@{}}
%
\rz{{\tt ProID}}  &  Property identification number.\\
\rz{{\tt Type}}   &  Identifier obtained with {\tt fe\_mat('p\_solid,'SI',3)}.\\
\rz{{\tt Integ}}  &  Integration rule selection (see \ltr{integrules}{Gauss}). 0 or -3 selects the default for the element.\\
\rz{{\tt Form}}  &   1 volume force, 2 volume force proportional to density, 3 pressure, 4: fluid/structure coupling, see \fsc, 5 2D volume force, 6 2D pressure. 8 Wall impedance (acoustics), then uses the $R$ parameter in fluid.\\
%
\end{tabular}

\rmain{See also}

  \Ser{femp}, \ser{il}, \femat


%------------------------------------------------------------------------------
\rtop{p\_spring}{p_spring}

Element property function for spring and rigid elements

\rsyntax\begin{verbatim}
il=p_spring('default') 
il=p_spring('database MatId Value')
il=p_spring('dbval MatId Value')
il=p_spring('dbval -unit TM ProId name');
il=p_spring('dbval -punit TM ProId name');
\end{verbatim}

\rmain{Description}

This help starts by describing the main commands : {\tt p\_spring} \ts{Database} and \ts{Dbval}. Supported {\tt p\_spring} subtypes and their formats are then described.

\ruic{p\_spring}{Database}{,Dbval]  ...} % - - - - - - - - - - - - - - - - - - - 

Element properties are described by the row of an element property matrix or a data structure with an {\tt .il} field containing this row (see \ser{il}). 

Examples of database property construction

%begindoc
\begin{verbatim}
 il=p_spring('database 100 1e12 1e4 0')
 il=p_spring('dbval 100 1e12');
 il=fe_mat('convert SITM',il);
 il=p_spring(il,'dbval 2 -unit TM 1e12') % Generate in TM, provide data in SI
 il=p_spring(il,'dbval 2 -punit TM 1e9') % Generate in TM, provide data in TM
\end{verbatim}%enddoc


\pspring\ currently supports 2 subtypes

\ruic{p\_spring}{1}{ : standard} % - - - - - - - - - - - - - - - - - - - -

\begin{verbatim}
  [ProID type  k m c Eta S]
\end{verbatim}


\noindent\begin{tabular}{@{}p{.15\textwidth}@{}p{.85\textwidth}@{}}
%
\rz{{\tt ProID}} &  property identification number.\\
\rz{\tt type}    &  identifier obtained with {\tt fe\_mat('p\_spring','SI',1)}.\\
\rz{{\tt k}}     &  stiffness value.\\
\rz{{\tt m}}     &  mass value.\\
\rz{{\tt c}}     &  viscous damping value.\\
\rz{{\tt eta}}   &  loss factor.\\
\rz{{\tt S}}     &  Stress coefficient.\\
\end{tabular}

\ruic{p\_spring}{2}{ : bush} % - - - - - - - - - - - - - - - - - - - -

Note that type 2 is only functional with \cbush\ elements.

\begin{verbatim}
  [ProId Type k1:k6 c1:c6 Eta SA ST EA ET m v]
\end{verbatim}


\noindent\begin{tabular}{@{}p{.15\textwidth}@{}p{.85\textwidth}@{}}
%
\rz{{\tt ProID}} &  property identification number. \\
\rz{\tt type}    &  identifier obtained with {\tt fe\_mat('p\_spring','SI',2)}.\\
\rz{\tt ki}      &  stiffness for each direction.\\
\rz{\tt ci}      &  viscous damping for each direction.\\
\rz{\tt SA}      &  stress recovery coef for translations.\\
\rz{\tt ST}      &  stress recovery coef for rotations.\\
\rz{\tt EA}      &  strain recovery coef for translations.\\
\rz{\tt ET}      &  strain recovery coef for rotations.\\
\rz{\tt m}       &  mass.\\
\rz{\tt v}       &  volume.\\

\end{tabular}

\rmain{See also}

  \Ser{femp}, \ser{il}, \femat, \celas, \cbush

%------------------------------------------------------------------------------
\begin{SDT}
\rtop{p\_super}{p_super}

Element property function for superelements.

\rsyntax\begin{verbatim}
il=p_super('default') 
il=p_super('database MatId Value')
il=p_super('dbval MatId Value')
il=p_super('dbval -unit TM ProId name');
il=p_super('dbval -punit TM ProId name');
\end{verbatim}


\rmain{Description}


If {\tt ProID} is not given, \fesuperb\ will see if {\tt SE.Opt(3,:)} is defined and use coefficients stored in this row instead.  If this is still not given, all coefficients are set to 1.  {\bf Element property rows} (in a standard property declaration matrix {\tt il}) for superelements take the forms described below \index{element!property row} with {\tt ProID} the property identification number and coefficients allowing the creation of a weighted sum of the superelement matrices {\tt SE}{\ti Name}{\tt .K\{i\}}. Thus, if {\tt K\{1\}} and {\tt K\{3\}} are two stiffness matrices and no other stiffness matrix is given, the superelement stiffness is given by {\tt coef1*K\{1\}+coef3*K\{3\}}.


\ruic{p\_super}{Database}{,Dbval]  ...} % - - - - - - - - - - - - - - - - - - - - - - - - 

There is no database call for {\tt p\_super} entries.

\ruic{p\_super}{1}{ : simple weighting coefficients} % - - - - - - - - - - - - - - 

\begin{verbatim}
 [ProId Type coef1 coef2 coef3 ... ]
\end{verbatim}


\noindent\begin{tabular}{@{}p{.2\textwidth}@{}p{.8\textwidth}@{}}
%
\rz{{\tt ProID}}  &  Property identification number.\\
\rz{{\tt Type}}   &  Identifier obtained with {\tt fe\_mat('p\_super','SI',1)}.\\
\rz{{\tt coef1}}  &  Multiplicative coefficient of the first matrix of the superelement ({\tt K\{1\}}). Superelement matrices used for the assembly of the global model matrices will be {\tt \{coef1*K\{1\}, coef2*K\{2\}, coef3*K\{3\}, ...\}}. Type of the matrices (stiffness, mass ...) is not changed. Note that you can define parameters for superelement using {\tt fe\_case(model,'par')}, see \fecase.\\
\end{tabular}

\ruic{p\_super}{2}{ : matrix type redefinition and weighting coefficients} % - - - 

\begin{verbatim}
 [ProId Type type1 coef1 type2 coef2 ...]
\end{verbatim}


\noindent\begin{tabular}{@{}p{.2\textwidth}@{}p{.8\textwidth}@{}}
%
\rz{{\tt ProID}}  &  Property identification number.\\
\rz{{\tt Type}}   &  Identifier obtained with {\tt fe\_mat('p\_super','SI',2)}.\\
\rz{{\tt type1}}  &  Type redefinition of the first matrix of the superelement ({\tt K\{1\}}) according to SDT standard type (1 for stiffness, 2 for mass, 3 for viscous damping... see \ltr{fe\_mknl}{MatType}).\\  
\rz{{\tt coef1}}  &  Multiplicative coefficient of the first matrix of the superelement ({\tt K\{1\}}). Superelement matrices used for the assembly of the global model matrices will be {\tt \{coef1*K\{1\}, coef2*K\{2\}, coef3*K\{3\}, ...\}}. Type of the matrices (stiffness, mass ...) is changed according to type1, type2, ... . Note that you can define parameters for superelement using {\tt fe\_case(model,'par')}, see \fecase.\\
\\
\end{tabular}


\rmain{See also}

  \fesuper, \ser{secms}

\end{SDT}










\end{OPENFEM}
\begin{SDT}
%       Copyright (c) 2001-2015 by INRIA and SDTools, All Rights Reserved.
%       Use under OpenFEM trademark.html license and LGPL.txt library license
%       $Revision: 1.72 $  $Date: 2025/12/17 18:59:41 $

%------------------------------------------------------------------------------
\rtop{m\_elastic}{m_elastic}

 Material function for elastic solids and fluids.

\rsyntax\begin{verbatim}
 mat= m_elastic('default') 
 mat= m_elastic('database name') 
 mat= m_elastic('database -therm name') 
 pl = m_elastic('dbval MatId name');
 pl = m_elastic('dbval -unit TM MatId name');
 pl = m_elastic('dbval -punit TM MatId name');
 pl = m_elastic('dbval -therm MatId name');
\end{verbatim}

\rmain{Description}

This help starts by describing the main commands of {\tt m\_elastic} : \ts{Database} and \ts{Dbval}. 

Material formats supported by {\tt m\_elastic} are then described.

If you are not familiar with material property matrices, see section \ser{pl} before reading this help.

\ruic{m\_elastic}{Database}{,\htr{m\_elastic}{Dbval}] [-unit TY] [,MatiD]] Name} % - - - - - - - - - - - - - - 

A material property function is expected to store a number of standard materials.

{\tt m\_elastic('database Steel')} returns a the data structure describing steel.\\
{\tt m\_elastic('dbval 100 Steel')} only returns the property row. 

%begindoc
\begin{verbatim}
  % List of materials in data base
  m_elastic info
  % examples of row building and conversion
  pl=m_elastic([100 fe_mat('m_elastic','SI',1) 210e9 .3 7800], ...
    'dbval 101 aluminum', ...
    'dbval 200 lamina .27 3e9 .4 1200 0  790e9 .3 1780 0');
  pl=fe_mat('convert SITM',pl);
  pl=m_elastic(pl,'dbval -unit TM 102 steel')
\end{verbatim}%enddoc

Command option \ts{-unit} asks the output to be converted in the desired unit system.
Command option \ts{-punit} tells the function that the provided data is in a desired unit system (and generates the corresponding type).
Command option \ts{-therm} asks to keep thermal data (linear expansion coefficients and reference temperature) if existing.

You can generate orthotropic shell properties using the \ts{Dbval 100 lamina VolFrac Ef nu\_f rho\_f G\_f E\_m nu\_m Rho\_m G\_m} command which gives fiber and matrix characteristics as illustrated above (the volume fraction is that of fiber). 

The default material is steel.


To orient fully anisotropic materials, you can use the following command

%begindoc
\begin{verbatim}
 % Behavior of a material grain assumed orthotropic
 C11=168.4e9; C12=121.4e9; C44=75.4e9; % GPa
 C=[C11 C12 C12 0 0 0;C12 C11 C12 0 0 0;C12 C12 C11 0 0 0;
   0 0 0 C44 0 0;    0 0 0 0 C44 0;    0 0 0 0 0 C44]; 

 pl=[m_elastic('formulaPlAniso 1',C,basis('bunge',[5.175 1.3071 4.2012]));
     m_elastic('formulaPlAniso 2',C,basis('bunge',[2.9208 1.7377 1.3921]))];
\end{verbatim}%enddoc


\rmain{Subtypes}
{\tt m\_elastic} supports the following material subtypes\vs\vs

\ruic{m\_elastic}{1}{ : standard isotropic}

\noindent {\sl Standard isotropic materials}, see~\ser{feelas3d} and ~\ser{feelas2d}, are described by a row of the form

\begin{verbatim}
 [MatID   typ   E nu rho G Eta Alpha T0]
\end{verbatim}


\noindent with {\tt typ} an identifier generated with the {\tt fe\_mat('m\_elastic','SI',1)} command, $E$ (Young's modulus), $\nu$ (Poisson's ratio), 
$\rho$ (density), $G$ (shear modulus, set to $G=E/2(1+\nu)$ if equal to zero). $\eta$ loss factor for hysteretic damping modeling. $\alpha$ thermal expansion coefficient. $T_0$ reference temperature.
$G=E/2(1+\nu)$

By default $E$ and $G$ are interdependent through $G=E/2(1+\nu)$. One can thus define either $E$ and $G$ to use this property. If $E$ or $G$ are set to zero they are replaced on the fly by their theoretical expression. Beware that modifying only E or G, either using \feutil \ts{SetMat} or by hand, will not apply modification to the other coefficient. In case where both coefficients are defined, in thus has to modify both values accordingly.


\ruic{m\_elastic}{2}{ : acoustic fluid} % - - - - - - - - - - - - - - - - - - - - - -

\noindent {\sl Acoustic fluid} , see~\ser{feacoustics},are described by a row of the form

\begin{verbatim}
 [MatId typ rho C eta R]
\end{verbatim}


\noindent with {\tt typ} an identifier generated with the {\tt fe\_mat('m\_elastic','SI',2)} command, $\rho$ (density), $C$ (velocity) and $\eta$ (loss factor). The bulk modulus is then given by $K=\rho C^2$. 

For walls with an impedance (see~\ltr{p\_solid}{3} form 8), the real part of the impedance, which corresponds to a viscous damping on the wall is given by {\tt $Z=\rho C R$} (see~\eqr{feform_feacoustics_5} for matrix). If an imaginary part is to be present, one will use $Z=\rho C R(1+i \eta)$. In an acoustic tube the absorption factor is given by $\alpha=\frac{4R}{((R+1)^2+(R\eta)^2)}$. 

\ruic{m\_elastic}{3}{ : 3-D anisotropic solid} % - - - - - - - - - - - - - - - - - - - - - 

{\sl 3-D Anisotropic solid}, see~\ser{feelas3d}, are described by a row of the form

\begin{verbatim}
 [MatId typ Gij rho eta A1 A2 A3 A4 A5 A6 T0]
\end{verbatim}


with {\tt typ} an identifier generated with the {\tt fe\_mat('m\_elastic','SI',3)} command, $rho$ (density), $eta$ (loss factor) and $Gij$ a row containing 

\begin{verbatim}
 [G11 G12 G22 G13 G23 G33 G14 G24 G34 G44 ...
  G15 G25 G35 G45 G55 G16 G26 G36 G46 G56 G66]
\end{verbatim}

Note that shear is ordered $g_{yz}, g_{zx}, g_{xy}$ which may not be the convention of other software.

SDT supports material handling through 

\begin{itemize}
\item material bases defined for each element \htt{EltOrient} or each property \ltt{Coordm}.
\item orientation maps used for material handling are described in~\ser{VectFromDir}. It is then expected that the six components {\tt v1x,v1y,v1z,v2x,v2y,v2z} are stored sequentially in the interpolation table. It is then usual to store the MAP in the stack entry \ts{info,EltOrient}.
\end{itemize}


\ruic{m\_elastic}{4}{ : 2-D anisotropic solid} % - - - - - - - - - - - - - - - - - - - - - 

{\sl 2-D Anisotropic solid}, see~\ser{feelas2d}, are described by a row of the form

\begin{verbatim}
 [MatId typ E11 E12 E22 E13 E23 E33 rho eta a1 a2 a3 T0]
\end{verbatim}


with {\tt typ} an identifier generated with the {\tt fe\_mat('m\_elastic','SI',4)} command, $rho$ (density), $eta$ (loss factor) and $Eij$ elastic constants and $ai$ anisotropic thermal expansion coefficients.

\ruic{m\_elastic}{5}{ : shell orthotropic material} % - - - - - - - - - - - - - - - - - - - - - 

{\sl shell orthotropic material}, see~\ser{feshell} corresponding to NASTRAN MAT8,  are described by a row of the form

\begin{verbatim}
 [MatId typ E1 E2 nu12 G12 G1z G2z Rho A1 A2 T0 Xt Xc Yt Yc S Eta ...
   F12 STRN]
\end{verbatim}


with {\tt typ} an identifier generated with the {\tt fe\_mat('m\_elastic','SI',5)} command, $rho$ (density), ... See \ltr{m\_elastic}{Dbval}\ts{lamina} for building. 

\ruic{m\_elastic}{6}{ : Orthotropic material} % - - - - - - - - - - - - - - - - - - - - - - - - - -

{\sl 3-D orthotropic material}, see~\ser{feelas3d}, are described by a set of engineering constants, in a row of the form

\begin{verbatim}
 [MatId typ E1 E2 E3 Nu23 Nu31 Nu12 G23 G31 G12 rho a1 a2 a3 T0 eta]
\end{verbatim}


with {\tt typ} an identifier generated with the {\tt fe\_mat('m\_elastic','SI',6)} command, $Ei$ (Young modulus in each direction), $\nu ij$ (Poisson ratio), $Gij$ (shear modulus), $rho$ (density), $ai$ (anisotropic thermal expansion coefficient), $T_0$ (reference temperature), and $eta$ (loss factor).
Care must be taken when using these conventions, in particular, it must be noticed that

\begin{eqsvg}{m_elastic_ortho_mat}
 \nu_{ji} = \frac{E_j}{E_i} \nu_{ij}
\end{eqsvg}

\rmain{See also}

  \Ser{femp}, \ser{pl}, \femat, \pshell, \ltr{feutil}{SetMat}


%------------------------------------------------------------------------------
\rtop{m\_heat}{m_heat}

 Material function for heat problem elements.

\rsyntax\begin{verbatim}
 mat= m_heat('default') 
 mat= m_heat('database name') 
 pl = m_heat('dbval MatId name');
 pl = m_heat('dbval -unit TM MatId name');
 pl = m_heat('dbval -punit TM MatId name');
\end{verbatim}

\rmain{Description}

This help starts by describing the main commands of {\tt m\_heat} : \ts{Database} and \ts{Dbval}. Materials formats supported by {\tt m\_heat} are then described.

\ruic{m\_heat}{Database}{,Dbval] [-unit TY] [,MatiD]] Name} % - - - - - - - - - - - - - - 

A material property function is expected to store a number of standard materials. See \ser{pl} for material property interface.

{\tt m\_heat('DataBase Steel')} returns a the data structure describing steel.\\
{\tt m\_heat('DBVal 100 Steel')} only returns the property row. 

%begindoc
\begin{verbatim}
  % List of materials in data base
  m_heat info
  % examples of row building and conversion
  pl=m_heat('DBVal 5 steel');
  pl=m_heat(pl,...
    'dbval 101 aluminum', ...
    'dbval 200 steel');
  pl=fe_mat('convert SITM',pl);
  pl=m_heat(pl,'dbval -unit TM 102 steel')
\end{verbatim}%enddoc

\rmain{Subtypes}
{\tt m\_heat} supports the following material subtype\vs\vs

\ruic{m\_heat}{1}{ : Heat equation material} % - - - - - - - - - - - - - - - - - - - -

\begin{verbatim}
   [MatId fe_mat('m_heat','SI',2) k rho C Hf]
\end{verbatim}

\begin{itemize}
\item {\tt k} conductivity
\item {\tt rho} mass density
\item {\tt C}  heat capacity
\item {\tt Hf} heat exchange coefficient
\end{itemize}

\rmain{See also}

  \Ser{femp}, \ser{pl}, \femat, \pheat

%------------------------------------------------------------------------------
\rtop{m\_hyper}{m_hyper}

 Material function for hyperelastic solids.

\rsyntax\begin{verbatim}
 mat= m_hyper('default') 
 mat= m_hyper('database name') 
 pl = m_hyper('dbval MatId name');
 pl = m_hyper('dbval -unit TM MatId name');
 pl = m_hyper('dbval -punit TM MatId name');
\end{verbatim}

\rmain{Description}

Function based on {\tt m\_elastic} function adapted for hyperelastic material. Only subtype 1 is currently used:

\ruic{m\_hyper}{1}{ : Nominal hyperelastic material}

\noindent {\sl Nominal hyperelastic materials} are described by a row of the form

\begin{verbatim}
 [MatID   typ  rho Wtype C_1 C_2 K]
\end{verbatim}


\noindent with {\tt typ} an identifier generated with the {\tt fe\_mat('m\_hyper','SI',1)} command, $rho$ (density), $Wtype$ (value for Energy choice), $C_1$, $C_2$, $K$ (energy coefficients).\\
\noindent Possible values for $Wtype$ are:

$$
\begin{array}{ll}
0: & W = C_1(J_1-3) + C_2(J_2-3) + K(J_3-1)^2\\
1: & W = C_1(J_1-3) + C_2(J_2-3) + K(J_3-1) - (C_1 + 2C_2 + K)\ln(J_3)
\end{array}
$$

Other energy functions can be added by editing the {\tt hyper.c Enpassiv} function.

In RivlinCube test, m\_hyper is called in this form:
\begin{verbatim}
model.pl=m_hyper('dbval 100 Ref'); % this is where the material is defined
\end{verbatim}


the hyperelastic material called ``Ref'' is described in the database of {\tt m\_hyper.m} file:
\begin{verbatim}
  out.pl=[MatId fe_mat('type','m_hyper','SI',1) 1e-06 0 .3 .2 .3];
  out.name='Ref';
  out.type='m_hyper';
  out.unit='SI';
\end{verbatim}


Here is an example to set your material property for a given structure model:
\begin{verbatim}
model.pl = [MatID fe_mat('m_hyper','SI',1) typ rho Wtype C_1 C_2 K];
model.Elt(2:end,length(feval(ElemF,'node')+1)) = MatID;
\end{verbatim}



%HEVEA \input{../../../sdt.git/piezo/tex/m_piezo.tex}
%HEVEA %       Copyright (c) 2001-2020 by SDTools and INRIA, All Rights Reserved.
%       Use under OpenFEM trademark.html license and LGPL.txt library license
%       $Revision: 1.115 $  $Date: 2020/10/26 07:58:04 $

%------------------------------------------------------------------------------
\rtop{p\_beam}{p_beam}

Element property function for beams

\rsyntax\begin{verbatim}
il = p_beam('default') 
il = p_beam('database','name') 
il = p_beam('dbval ProId','name');
il = p_beam('dbval -unit TM ProId name');
il = p_beam('dbval -punit TM ProId name');
il2= p_beam('ConvertTo1',il)
\end{verbatim}

\rmain{Description}

This help starts by describing the main commands : {\tt p\_beam} \ts{Database} and \ts{Dbval}. Supported {\tt p\_beam} subtypes and their formats are then described.

\ruic{p\_beam}{Database}{,Dbval,  ...} % - - - - - - - - - - - - - - - - - - - 

{\tt p\_beam} contains a number of defaults obtained with {\tt p\_beam('database')} or\\  
{\tt p\_beam('dbval {\ti MatId}')}. You can select a particular entry of the database with using a name matching the database entries. You can also automatically compute the properties of standard beams

\noindent\begin{tabular}{@{}p{.35\textwidth}@{}p{.65\textwidth}@{}}
%
\rz\ts{circle }\tsi{r}  & beam with full circular section of radius \tsi{r}.\\
\rz\ts{rectangle }\tsi{b h} & beam with full rectangular section of width \tsi{b} and height \tsi{h}. See \beam\ for orientation (the default reference node is 1.5, 1.5, 1.5 so that orientation MUST be defined for non-symmetric sections). \\
\rz\ts{Type }\tsi{r1 r2 ...}  & other predefined sections of subtype 3 are listed using {\tt p\_beam('info')}. 
\end{tabular}


%{\tt p\_beam('database reftube')} gives a reference property of subtype 3 for a tube.\\

For example, you will obtain the section property row with {\tt ProId} 100 associated with a circular cross section of $0.05 m$ or a rectangular $0.05 \times 0.01 m$ cross section using

%begindoc
\begin{verbatim}
 % ProId 100, rectangle 0.05 m by 0.01 m
 pro = p_beam('database 100 rectangle .05 .01')
 % ProId 101 circle radius .05
 il = p_beam(pro.il,'dbval 101 circle .05')
 p_beam('info')
 % ProId 103 tube external radius .05 internal .04
 il = p_beam(il,'dbval -unit SI 103 tube .05 .04')
 % Transform to subtype 1
 il2=p_beam('ConvertTo1',il)
 il(end+1,1:6)=[104 fe_mat('p_beam','SI',1) 0 0 0 1e-5];
 il = fe_mat('convert SITM',il);
% Generate a property in TM, providing data in SI
 il = p_beam(il,'dbval -unit TM 105 rectangle .05 .01')
% Generate a property in TM providing data in TM
  il = p_beam(il,'dbval -punit TM 105 rectangle 50 10')
\end{verbatim}%enddoc

\ruic{p\_beam}{Show3D}{,MAP  ...} % - - - - - - - - - - - - - - - - - - - 

%begindoc
\begin{verbatim}

\end{verbatim}%enddoc



\ruic{p\_beam}{format}{ description and subtypes} % - - - - - - - - - - - - - - 

Element properties are described by the row of an element property matrix or a data structure with an {\tt .il} field containing this row (see \ser{il}). Element property functions such as {\tt p\_beam} support graphical editing of properties and a database of standard properties. 

For a tutorial on material/element property handling see \ser{femp}. For a programmers reference on formats used to describe element properties see \ser{il}. 

\ruic{p\_beam}{1}{ : standard} % - - - - - - - - - - - - - - - - - - - -

%\pbeam\ currently only supports a single format (\femat\ property subtype)

\begin{verbatim}
  [ProID   type   J I1 I2 A   k1 k2 lump NSM]
\end{verbatim}


\noindent\begin{tabular}{@{}p{.25\textwidth}@{}p{.75\textwidth}@{}}
%
{\ti ProID} & element property identification number. \\
\rz{\tt type}  & identifier obtained with {\tt fe\_mat('p\_beam','SI',1)}. \\
\rz{\tt J}  & torsional stiffness parameter (often different from polar moment of inertia {\tt I1+I2}). \\
\rz{\tt I1} & moment of inertia for bending plane 1 defined by a third node {\tt nr} or the vector {\tt vx vy vz} (defined in the \beam\ element). For a case with a beam along $x$ and plane 1 the $xy$ plane {\tt I1} is equal to $Iz = \int_{S} y^2 ds$. \\
\rz{\tt I2} & moment of inertia for bending plane 2 (containing the beam and orthogonal to plane 1. \\
\rz{\tt A} & section area. \\
\rz{\tt k1} & (optional) shear factor for motion in plane 1 (when not 0, a
                     Timoshenko beam element is used). The effective
                     area of shear is given by $k_1A$.  \\
\rz{\tt k2} & (optional) shear factor for direction 2.\\
\rz{\tt lump} & (optional) request for lumped mass model. 1 for inclusion of inertia terms. 2 for simple half mass at node. \\
\rz{\tt NSM} & (optional) non structural mass (density per unit length).\\
\end{tabular}\par

\bare\   elements only use the section area. All other parameters are ignored.

\beam\ elements use all parameters.  Without correction factors ({\ti k1} {\ti k2} not given or set to 0), the \beam\ element is the standard Bernoulli-Euler 12 DOF element based on linear interpolations for traction and torsion and cubic interpolations for flexion (see Ref.  \ecite{ger3} for example). When non zero shear factors are given, the bending properties are based on a Timoshenko beam element with selective reduced integration of the shear stiffness \ecite{imb1}. No correction for rotational inertia of sections is used.

\begin{SDT}
\ruic{p\_beam}{3}{ : Cross section database } % - - - - - - - - - - - - - - - - - - - -

This subtype can be used to refer to standard cross sections defined in database. It is particularly used by \nasread\ when importing NASTRAN {\tt PBEAML} properties.

\begin{verbatim}
  [ProID   type   0  Section Dim(i) ... ]
\end{verbatim}


\noindent\begin{tabular}{@{}p{.25\textwidth}@{}p{.75\textwidth}@{}}
%
{\ti ProID} & element property identification number. \\
\rz{\tt type} & identifier obtained with {\tt fe\_mat('p\_beam','SI',3)}. \\
\rz{\tt Section} & identifier of the cross section obtained with {\tt comstr('}\tsi{SectionName}{\tt ',-32)} where \tsi{SectionName} is a string defining the section (see below).\\
\rz{\tt Dim1 ...} & dimensions of the cross section.\\
\end{tabular}

Cross section, if existing, is compatible with NASTRAN {\tt PBEAML} definition. Equivalent moment of inertia and tensional stiffness are computed at the centroid of the section.
Currently available sections are listed with {\tt p\_beam('info')}. In particular one has {\tt ROD} (1 dim), {\tt TUBE} (2 dims), {\tt T} (4 dims), {\tt T2} (4 dims), {\tt I} (6 dims), {\tt BAR} (2 dims), {\tt CHAN1} (4 dims), {\tt CHAN2} (4 dims).

For \ts{NSM} and \ts{Lump} support \ts{ConverTo1} is used during definition to obtain the equivalent {\tt subtype 1} entry. 

\end{SDT}

\rmain{See also}

  \Ser{femp}, \ser{il}, \femat 
%------------------------------------------------------------------------------
\rtop{p\_heat}{p_heat}

Formulation and material support for the heat equation.

\rsyntax\begin{verbatim}
il = p_heat('default') 
\end{verbatim}

\rmain{Description}

This help starts by describing the main commands : {\tt p\_heat} \ts{Database} and \ts{Dbval}. Supported {\tt p\_heat} subtypes and their formats are then described. For theory see \ser{fe3dth}.

\ruic{p\_heat}{Database}{,Dbval]  ...} % - - - - - - - - - - - - - - - - - - - 

Element properties are described by the row of an element property matrix or a data structure with an {\tt .il} field containing this row (see \ser{il}). Element property functions such as {\tt p\_solid} support graphical editing of properties and a database of standard properties. 

{\tt p\_heat} database

%begindoc
\begin{verbatim}
 il=p_heat('database');
\end{verbatim}%enddoc

Accepted commands for the database are 
%
\begin{itemize}
\item \ts{d3 }\tsi{Integ }\tsi{SubType} : \ltt{Integ} integration rule for 3D volumes (default -3). 
\item \ts{d2 }\tsi{Integ }\tsi{SubType} : \ltt{Integ} integration rule for 2D volumes (default -3).
\end{itemize}

For fixed values, use {\tt p\_heat('info')}.

Example of database property construction

%begindoc
\begin{verbatim}
  il=p_heat([100 fe_mat('p_heat','SI',1) 0 -3 3],...
             'dbval 101 d3 -3 2');
\end{verbatim}%enddoc


\ruic{p\_heat}{Heat}{ equation element properties} % - - - - - - - - - - - - - - 

Element properties are described by the row of an element property matrix or a data structure with an {\tt .il} field containing this row. Element property functions such as {\tt p\_beam} support graphical editing of properties and a database of standard properties. 


\ruic{p\_heat}{1}{ : Volume element for heat diffusion (dimension DIM)} % - - - - - - - - - - - - - - - - - - - -

%\pbeam\ currently only supports a single format (\femat\ property subtype)

\begin{verbatim}
  [ProId fe_mat('p_heat','SI',1) CoordM Integ DIM]
\end{verbatim}


\noindent\begin{tabular}{@{}p{.25\textwidth}@{}p{.75\textwidth}@{}}
%
{\ti ProID} & element property identification number \\
\rz{\tt type}  & identifier obtained with {\tt fe\_mat('p\_beam','SI',1)} \\
\rz{\tt Integ}  & is rule number in integrules \\
\rz{\tt DIM}  & is problem dimension 2 or 3 D \\
\end{tabular}\par

\ruic{p\_heat}{2}{ : Surface element for heat exchange (dimension DIM-1)} % - - - - - - - - - - - - - - - - - - - -

\begin{verbatim}
   [ProId fe_mat('p_heat','SI',2) CoordM Integ DIM] 
\end{verbatim}


\noindent\begin{tabular}{@{}p{.25\textwidth}@{}p{.75\textwidth}@{}}
%
{\ti ProID} & element property identification number \\
\rz{\tt type}  & identifier obtained with {\tt fe\_mat('p\_beam','SI',2)} \\
\rz{\tt Integ}  & is rule number in {\tt integrules} \\
\rz{\tt DIM}  & is problem dimension 2 or 3 D \\
\end{tabular}\par

\ruic{p\_heat}{SetFace}{} % - - - - - - - - - - - - - - - - - - - - - -
This command can be used to define a surface exchange and optionally associated load.
Surface exchange elements add a stiffness term to the stiffness matrix related to the exchange coefficient {\tt Hf} defined in corresponding material property. One then should add a load corresponding to the exchange with the source temperature at $T_0$ through a convection coefficient {\tt Hf} which is {\tt Hf.T\_0}. If not defined, the exchange is done with source at temperature equal to 0. 

{\tt  model=p\_heat('SetFace',model,SelElt,pl,il);}\\

\begin{itemize}
\item{\tt SelElt} is a findelt command string to find faces that exchange heat (use 'SelFace' to select face of a given preselected element).
\item{\tt pl} is the identifier of existing material property ({\tt MatId}), or a vector defining an {\tt m\_heat} property.
\item{\tt il} is the identifier  of existing element property ({\tt ProId}), or a vector defining an {\tt p\_heat} property.
\end{itemize}

Command option \ts{-load }\tsi{T} can be used to defined associated load, for exchange with fluid at temperature \tsi{T}. Note that if you modify {\tt Hf} in surface exchange material property you have to update the load.

Following example defines a simple cube that exchanges with thermal source at 55 deg on the bottom face.

%beginddoc
\begin{verbatim} 
model=femesh('TestHexa8'); % Build simple cube model
model.pl=m_heat('dbval 100 steel'); % define steel heat diffusion parameter
model.il=p_heat('dbval 111 d3 -3 1'); % volume heat diffusion (1)
model=p_heat('SetFace-load55',... % exchange at 55 deg
    model,...
    'SelFace & InNode{z==0}',... % on the bottom face
    100,... % keep same matid for exchange coef
    p_heat('dbval 1111 d3 -3 2')); % define 3d, integ-3, for surface exchange (2)
cf=feplot(model); fecom colordatapro
def=fe_simul('Static',model); % compute static thermal state
mean(def.def)
\end{verbatim}%enddoc

\ruic{p\_heat}{2D}{validation} % - - - - - - - - - - - - - - - - - - - 

Consider a bi-dimensional annular thick domain $\Omega$ with radii $r_e=1$ and $r_i=0.5$. The data are specified on the internal circle $\Gamma_i$ and on the external circle $\Gamma_e$. The solid is made of homogeneous isotropic material, and its conductivity tensor thus reduces to a constant $k$. The steady state temperature distribution is then given by
\begin{eqsvg}{test_ann}
- k \Delta\theta(x,y) = f(x,y) \quad in \quad \Omega.
\end{eqsvg}

The solid is subject to the following boundary conditions\\
\begin{itemize}
\item{ {$\Gamma_i \,(r=r_i)$ : Neumann condition}\\
\begin{eqsvg}{p_heat_validation_1}
\displaystyle\frac{\partial \theta}{\partial n}(x,y) = g(x,y)
\end{eqsvg}  }
\item{ {$\Gamma_e \,(r=r_e)$ : Dirichlet condition}\\
\begin{eqsvg}{p_heat_validation_2}
\theta(x,y)=\theta_{ext}(x,y)
\end{eqsvg}  }
\end{itemize}

In above expressions, $f$ is an internal heat source, \mathsvg{\theta_{ext}}{p_heat_validation_l1} an external temperature at $r=r_e$, and $g$ a function. All the variables depend on the variable $x$ and $y$. 

The OpenFEM model for this example can be found in {\tt ofdemos('AnnularHeat')}.\\
{\bf Numerical application} : assuming $k=1$, $f=0$, $Hf=1e^{-10}$, $\theta_{ext}(x,y) = \exp(x) \cos(y)$ and \mathsvg{g(x,y)= -\frac{\exp(x)} {r_i} \left ( \cos(y)  x  - \sin(y)  x \right )}{p_heat_validation_l2}, the solution of the problem is  given by
\mathsvg{\displaystyle \theta(x,y) = \exp(x) \cos(y)}{p_heat_validation_l3}



\rmain{See also}

  \ser{fe3dth}, \ser{femp}, \femat 

%------------------------------------------------------------------------------
\IfFileExists{../tex/p_pml.tex}{\input{../tex/p_pml.tex}}{}

%------------------------------------------------------------------------------
\rtop{p\_shell}{p_shell}

Element property function for shells and plates (flat shells)

\rsyntax\begin{verbatim}
il = p_shell('default');
il = p_shell('database ProId name'); 
il = p_shell('dbval ProId name');
il = p_shell('dbval -unit TM ProId name');
il = p_shell('dbval -punit TM ProId name');
il = p_shell('SetDrill 0',il);
\end{verbatim}

\rmain{Description}

This help starts by describing the main commands : {\tt p\_shell} \ts{Database} and \ts{Dbval}. Supported {\tt p\_shell} subtypes and their formats are then described.


\ruic{p\_shell}{Database}{,Dbval,  ...} % - - - - - - - - - - - - - - - - - - - 

{\tt p\_shell} contains a number of defaults obtained with the \ts{database} and \ts{dbval} commands which respectively return a structure or an element property row. You can select a particular entry of the database with using a name matching the database entries. 


You can also automatically compute the properties of standard shells with

\noindent\begin{tabular}{@{}p{.35\textwidth}@{}p{.65\textwidth}@{}}
%
\rz\ts{kirchhoff }\tsi{e}  & Kirchhoff shell of thickness \tsi{e} (is not implemented for formulation 5, see each element for available choices)\\
\rz\ts{mindlin }\tsi{e}  & Mindlin shell of thickness \tsi{e} (see each element for choices). \\
\rz\ts{laminate }\tsi{MatIdi Ti Thetai}  & Specification of a laminate property by giving the different ply {\tt MatId}, thickness and angle. By default the z values are counted from -thick/2, you can specify another value with a z0.
%
\end{tabular}

You can append a string option of the form \ts{-f }\tsi{i} to select the appropriate shell formulation. The different formulations are described under each element topology (\triaa, \quad4, ...)
For example, you will obtain the element property row with {\tt ProId} 100 associated with a .1 thick Kirchhoff shell (with formulation 5) or the corresponding Mindlin plate use

%begindoc
\begin{verbatim}
 il = p_shell('database 100 MindLin .1')
 il = p_shell('dbval 100 kirchhoff .1 -f5')
 il = p_shell('dbval 100 laminate z0=-2e-3 110 3e-3 30 110 3e-3 -30')
 il = fe_mat('convert SITM',il);
 il = p_shell(il,'dbval -unit TM 2 MindLin .1') % set in TM, provide data in SI
 il = p_shell(il,'dbval -punit TM 2 MindLin 100') % set in TM, provide data in TM
\end{verbatim}%enddoc


For laminates, you specify for each ply the {\tt MatId}, thickness and angle.

\ruic{p\_shell}{Shell}{ format description and subtypes} % - - - - - - - - - - - - - - 

Element properties are described by the row of an element property matrix or a data structure with an {\tt .il} field containing this row (see \ser{il}). Element property functions such as {\tt p\_shell} support graphical editing of properties and a database of standard properties. 

For a tutorial on material/element property handling see \ser{femp}. For a reference on formats used to describe element properties see \ser{il}. 

\pshell\ currently only supports two subtypes

\ruic{p\_shell}{1}{ : standard isotropic} % - - - - - - - - - - - - - - - - - - - -

\begin{verbatim}
  [ProID type   f d O   h   k   MID2 RatI12_T3 MID3 NSM Z1 Z2 MID4]
\end{verbatim}


\noindent\begin{tabular}{@{}p{.05\textwidth}@{}p{.05\textwidth}@{}p{.9\textwidth}@{}}
%
\rz{\tt type}  &   &  identifier obtained with {\tt fe\_mat('p\_shell','SI',1)}.\\
\rz{\tt f} &  & \rz{{\tt 0}} use default of element. For other formulations the specific help for each element (\quada, \triaa, ...), each formulation specifies integration rule. \\
\rz{\tt d} & \rz{\tt -1} & no drilling stiffness. The element DOFs are the standard translations and rotations at all nodes (DOFs {\tt .01} to {\tt .06}). The drill DOF (rotation {\tt .06} for a plate in the {\sl xy} plane) has no stiffness and is thus eliminated by \femk\ if it corresponds to a global DOF direction. The default is {\tt d=1} ({\tt d} is set to 1 for a declared value of zero). \\
& \rz{\tt d} & arbitrary drilling stiffness with value proportional to {\tt d} is added. This stiffness is often needed in shell problems but may lead to numerical conditioning problems if the stiffness value is very different from other physical stiffness values. Start with a value of 1. Use {\tt il=p\_shell('SetDrill d',il)} to set to {\tt d} the drilling stiffness of all {\tt p\_shell} subtype 1 rows of the property matrix {\tt il}. \\
\rz{\tt h} &  & plate thickness.\\
\rz{\tt k} & {\ti k} & shear correction factor (default 5/6, default used if {\tt k} is zero). This correction is not used for formulations based on triangles since \triaa\ is a thin plate element. \\
\rz{\tt RatI12\_T3} &  & Ratio of bending moment of inertia to nominal {\tt T3/I12} (default 1).\\
\rz{\tt  NSM} &  & Non structural mass per unit area.\\
\rz{\tt  MID2} &  & material property for bending. Defauts to element {\tt MatId} if equal to 0. \\
\rz{\tt  MID3} &  & material property for transverse shear. \\
\rz{\tt  z1,z2} &  & (unused) offset for fiber computations.\\
\rz{\tt  MID4} &  & material property for membrane/bending coupling.\\
\end{tabular}

Shell strain is defined by the membrane, curvature and transverse shear \texline (display with {\tt p\_shell('ConstShell')}). 
%
\begin{eqsvg}{p_shell_1}
\ve{\ba{c}\epsilon_{xx} \\\epsilon_{yy} \\ 2 \epsilon_{xy} \\ \kappa_{xx} \\\kappa_{yy} \\ 2 \kappa_{xy} \\ \gamma_{xz} \\ \gamma_{yz} \ea}=\ma{\ba{cccccccc}
 N,x & 0 & 0 & 0 & 0 \\
 0 & N,y & 0 & 0 & 0 \\
 N,y & N,x & 0 & 0 & 0 \\
 0 & 0 & 0 & 0 & N,x \\
 0 & 0 & 0 & -N,y & 0 \\
 0 & 0 & 0 & -N,x & N,y \\
 0 & 0 & N,x & 0 & -N \\
 0 & 0 & N,y & N & 0 \ea}
\ve{\ba{c} u \\ v \\ w \\ ru \\ rv \ea}
\end{eqsvg}

\ruic{p\_shell}{2}{ : composite} % - - - - - - - - - - - - - - - - - - - -

\begin{verbatim}
  [ProID type   Z0 NSM SB FT TREF GE LAM MatId1 T1 Theta1 SOUT1 ...]
\end{verbatim}


\noindent\begin{tabular}{@{}p{.15\textwidth}@{}p{.85\textwidth}@{}}
%
\rz{{\tt ProID}} &  Section property identification number. \\
\rz{{\tt type}}     &  Identifier obtained with {\tt fe\_mat('p\_shell','SI',2)}.\\
\rz{{\tt Z0}}     &  Distance from reference plate to bottom surface. \\
\rz{{\tt NSM}}     & Non structural mass per unit area. \\
\rz{{\tt SB}}     & Allowable shear stress of the bonding material. \\
\rz{{\tt FT}}     & Failure theory. \\
\rz{{\tt TREF}}     &  Reference temperature. \\
\rz{{\tt Eta}}     &  Hysteretic loss factor. \\
\rz{{\tt LAM}}     &  Laminate type. \\
\rz{{\tt MatId{\ti i}}} &  {\tt MatId} for ply {\ti i}, see \ltr{m\_elastic}{1},  \ltr{m\_elastic}{5}, ...\\
\rz{{\tt T{\ti i}}} &  Thickness of ply {\ti i}. \\
\rz{{\tt Theta{\ti i}}} &  Orientation of ply {\ti i}. \\
\rz{{\tt SOUT{\ti i}}} &  Stress output request for ply {\ti i}.
\end{tabular}

Note that this subtype is based on the format used by NASTRAN for {\tt PCOMP} and the formulation used for each topology is discussed in each element (see \quada, \triaa). You can use the \ts{DbvalLaminate} commands to generate standard entries.

\begin{eqsvg}{p_shell_2}
\ve{\ba{c}N \\ M \\ Q\ea} = \ma{\ba{ccc} A & B & 0\\ B & D & 0\\ 0 & 0 &
F\ea} \ve{\ba{c}\epsilon \\ \kappa \\ \gamma \ea}
\end{eqsvg}

\ruic{p\_shell}{setTheta}{}

When dealing with laminated plates, the classical approach uses a material orientation constant per element. OpenFEM also supports more advanced strategies with orientation defined at nodes but this is still poorly documented.

The material orientation is the reference for plies. Any angle defined in a laminate command is an additional rotation. In the example below, the element orientation is rotated 30 degrees, and the ply another 30. The fibers are thus oriented 60 degrees in the $xy$ plane. Stresses are however given in the material orientation thus with a 30 degree rotation. Per ply output is not currently implemented. 

The element-wise material angle is stored for each element. In column 7 for \triaa, 8 for \quada, ...  The \ts{setTheta} command is a utility to ease the setting of these angles. By default, the orientation is done at element center. To use the mean orientation at nodes use command option \ts{-strategy 2}.

\begin{verbatim}
model=ofdemos('composite');
model.il = p_shell('dbval 110 laminate 100 1 30'); % single ply

% Define material angle based on direction at element
MAP=feutil('getnormalElt MAP -dir1',model);
bas=basis('rotate',[],'rz=30;',1);
MAP.normal=MAP.normal*reshape(bas(7:15),3,3)';
model=p_shell('setTheta',model,MAP);

% Obtain a MAP of material orientations
MAP=feutil('getnormalElt MAP -dir1',model);
feplot(model);fecom('showmap',MAP)

% Set elementwise material angles using directions given at nodes. 
% Here a global direction
MAP=struct('normal',ones(size(model.Node,1),1)*bas(7:9), ...
    'ID',model.Node(:,1),'opt',2);
model=p_shell('setTheta',model,MAP);

% Using an analytic expression to define components of 
% material orientation vector at nodes
data=struct('sel','groupall','dir',{{'x-0','y+.01',0}},'DOF',[.01;.02;.03]);
model=p_shell('setTheta',model,data);
MAP=feutil('getnormalElt MAP -dir1',model);
feplot(model);fecom('showmap',MAP)
\end{verbatim}

{\tt model=p\_shell('setTheta',model,0)} is used to reset the material orientation to zero.


Technically, shells use the {\tt of\_mk('BuildNDN')} rule 23 which generates a basis at each integration point. The first vector {\tt v1x,v1y,v1z} is built in the direction of $r$ lines and {\tt v2x,v2y,v2z} is tangent to the surface and orthogonal to $v1$. When a \ltt{InfoAtNode} map provides {\tt v1x,v1y,v1z}, this vector is projected (NEED TO VERIFY) onto the surface and $v2$ taken to be orthogonal. 


\rmain{See also}

  \Ser{femp}, \ser{il}, \femat


%------------------------------------------------------------------------------
\rtop{p\_solid}{p_solid}

Element property function for volume elements.

\rsyntax\begin{verbatim}
il=p_solid('database ProId Value')
il=p_solid('dbval ProId Value')
il=p_solid('dbval -unit TM ProId name');
il=p_solid('dbval -punit TM ProId name');
model=p_solid('default',model)
\end{verbatim}


\rmain{Description}


This help starts by describing the main commands : {\tt p\_solid} \ts{Database} and \ts{Dbval}. Supported {\tt p\_solid} subtypes and their formats are then described.

\ruic{p\_solid}{Database}{, Dbval, Default,  ...} % - - - - - - - - - - - - - - - - - - - 

Element properties are described by the row of an element property matrix or a data structure with an {\tt .il} field containing this row (see \ser{il}). Element property functions such as {\tt p\_solid} support graphical editing of properties and a database of standard properties. 

Accepted commands for the database are 
%
\begin{itemize}
\item \ts{d3 }\tsi{Integ} : \tsi{Integ} integration rule for quadratic 3D volumes. For information on rules available see \ltr{integrules}{Gauss}. Examples are \ts{d3 2}  2x2x2 integration rule for linear volumes (hexa8 ... ); \ts{d3 -3} default integration for all 3D elements, ...
\item\ts{d2 }\tsi{Integ} :  \tsi{Integ} integration rule for quadratic 2D volumes. For example \ts{d2 2} 2x2x2 integration rule for linear volumes (q4p ... ). You can also use \ts{d2 1 0 2} for plane stress, and \ts{d2 2 0 2} for axisymmetry.
\item\ts{fsc }\tsi{Integ} : integration rule selection for fluid/structure coupling.
\end{itemize}

For fixed values, use {\tt p\_solid('info')}.

For a tutorial on material/element property handling see \ser{femp}. For a reference on formats used to describe element properties see \ser{il}. 

Examples of database property construction

%begindoc
\begin{verbatim}
  il=p_solid([100 fe_mat('p_solid','SI',1) 0 3 0 2], ...
             'dbval 101 Full 2x2x2','dbval 102 d3 -3');
  il=fe_mat('convert SITM',il);
  il=p_solid(il,'dbval -unit TM 2 Reduced shear')
  % Try a smart guess on default 
  model=femesh('TestHexa8');model.il=[]; 
  model=p_solid('default',model) 
\end{verbatim}%enddoc


\ruic{p\_solid}{1}{ : 3D volume element} % - - - - - - - - - - - - - - 

\begin{verbatim}
[ProID fe_mat('p_solid','SI',1) Coordm In Stress Isop ]
\end{verbatim}


\noindent\begin{tabular}{@{}p{.2\textwidth}@{}p{.8\textwidth}@{}}
%
\rz{{\tt ProID}}  &  Property identification number.\\
\rz{{\tt Coordm}} &  Identification number of the material coordinates system. {\bf Warning}  not implemented for all material formulations. \\
\rz{{\tt In}}     &  Integration rule selection (see \ltr{integrules}{Gauss}). 0 selects the legacy 3D mechanics element ({\tt of\_mk\_pre.c}), -3 the default rule. \\
\rz{{\tt Stress}} &  Location selection for stress output (NOT USED).\\
\rz{{\tt Isop}}   &  Integration scheme.  Used to select the generalized strain definition in \nlinout\ implementations (see~\ser{nlio3d}). May also be used to select shear protection mechanisms in the future. \\
\end{tabular}

The underlying physics for this subtype are selected through the material property. Examples are 3D mechanics with \melastic, \begin{SDT} piezo electric volumes (see {\tt m\_piezo})\end{SDT}, heat equation (\pheat).

\ruic{p\_solid}{2}{ : 2D volume element } % - - - - - - - - - - - - - - - - - - -

\begin{verbatim}
  [ProId fe_mat('p_solid','SI',2)  Form N In]
\end{verbatim}


\noindent\begin{tabular}{@{}p{.2\textwidth}@{}p{.8\textwidth}@{}}
%
\rz{{\tt ProID}}  &  Property identification number.\\
\rz{{\tt Type}}   &  Identifier obtained with {\tt fe\_mat('p\_solid,'SI',2)}.\\
\rz{{\tt Form}}   &  Formulation (0 plane strain, 1 plane stress, 2 axisymmetric), see details in \melastic. \\
\rz{{\tt N}}      &  Fourier harmonic for axisymmetric elements that support it.\\
\rz{{\tt In}}     &  Integration rule selection (see \ltr{integrules}{Gauss}). 0 selects legacy 2D element, -3 the default rule.
\end{tabular}

The underlying physics for this subtype are selected through the material property. Examples are 2D mechanics with \melastic.

\ruic{p\_solid}{3}{ : ND-1 coupling element} % - - - - - - - - - - - - - - - - - - -

\begin{verbatim}
  [ProId fe_mat('p_solid','SI',3) Integ Form Ndof1 ...]
\end{verbatim}


\noindent\begin{tabular}{@{}p{.2\textwidth}@{}p{.8\textwidth}@{}}
%
\rz{{\tt ProID}}  &  Property identification number.\\
\rz{{\tt Type}}   &  Identifier obtained with {\tt fe\_mat('p\_solid,'SI',3)}.\\
\rz{{\tt Integ}}  &  Integration rule selection (see \ltr{integrules}{Gauss}). 0 or -3 selects the default for the element.\\
\rz{{\tt Form}}  &   1 volume force, 2 volume force proportional to density, 3 pressure, 4: fluid/structure coupling, see \fsc, 5 2D volume force, 6 2D pressure. 8 Wall impedance (acoustics), then uses the $R$ parameter in fluid.\\
%
\end{tabular}

\rmain{See also}

  \Ser{femp}, \ser{il}, \femat


%------------------------------------------------------------------------------
\rtop{p\_spring}{p_spring}

Element property function for spring and rigid elements

\rsyntax\begin{verbatim}
il=p_spring('default') 
il=p_spring('database MatId Value')
il=p_spring('dbval MatId Value')
il=p_spring('dbval -unit TM ProId name');
il=p_spring('dbval -punit TM ProId name');
\end{verbatim}

\rmain{Description}

This help starts by describing the main commands : {\tt p\_spring} \ts{Database} and \ts{Dbval}. Supported {\tt p\_spring} subtypes and their formats are then described.

\ruic{p\_spring}{Database}{,Dbval]  ...} % - - - - - - - - - - - - - - - - - - - 

Element properties are described by the row of an element property matrix or a data structure with an {\tt .il} field containing this row (see \ser{il}). 

Examples of database property construction

%begindoc
\begin{verbatim}
 il=p_spring('database 100 1e12 1e4 0')
 il=p_spring('dbval 100 1e12');
 il=fe_mat('convert SITM',il);
 il=p_spring(il,'dbval 2 -unit TM 1e12') % Generate in TM, provide data in SI
 il=p_spring(il,'dbval 2 -punit TM 1e9') % Generate in TM, provide data in TM
\end{verbatim}%enddoc


\pspring\ currently supports 2 subtypes

\ruic{p\_spring}{1}{ : standard} % - - - - - - - - - - - - - - - - - - - -

\begin{verbatim}
  [ProID type  k m c Eta S]
\end{verbatim}


\noindent\begin{tabular}{@{}p{.15\textwidth}@{}p{.85\textwidth}@{}}
%
\rz{{\tt ProID}} &  property identification number.\\
\rz{\tt type}    &  identifier obtained with {\tt fe\_mat('p\_spring','SI',1)}.\\
\rz{{\tt k}}     &  stiffness value.\\
\rz{{\tt m}}     &  mass value.\\
\rz{{\tt c}}     &  viscous damping value.\\
\rz{{\tt eta}}   &  loss factor.\\
\rz{{\tt S}}     &  Stress coefficient.\\
\end{tabular}

\ruic{p\_spring}{2}{ : bush} % - - - - - - - - - - - - - - - - - - - -

Note that type 2 is only functional with \cbush\ elements.

\begin{verbatim}
  [ProId Type k1:k6 c1:c6 Eta SA ST EA ET m v]
\end{verbatim}


\noindent\begin{tabular}{@{}p{.15\textwidth}@{}p{.85\textwidth}@{}}
%
\rz{{\tt ProID}} &  property identification number. \\
\rz{\tt type}    &  identifier obtained with {\tt fe\_mat('p\_spring','SI',2)}.\\
\rz{\tt ki}      &  stiffness for each direction.\\
\rz{\tt ci}      &  viscous damping for each direction.\\
\rz{\tt SA}      &  stress recovery coef for translations.\\
\rz{\tt ST}      &  stress recovery coef for rotations.\\
\rz{\tt EA}      &  strain recovery coef for translations.\\
\rz{\tt ET}      &  strain recovery coef for rotations.\\
\rz{\tt m}       &  mass.\\
\rz{\tt v}       &  volume.\\

\end{tabular}

\rmain{See also}

  \Ser{femp}, \ser{il}, \femat, \celas, \cbush

%------------------------------------------------------------------------------
\begin{SDT}
\rtop{p\_super}{p_super}

Element property function for superelements.

\rsyntax\begin{verbatim}
il=p_super('default') 
il=p_super('database MatId Value')
il=p_super('dbval MatId Value')
il=p_super('dbval -unit TM ProId name');
il=p_super('dbval -punit TM ProId name');
\end{verbatim}


\rmain{Description}


If {\tt ProID} is not given, \fesuperb\ will see if {\tt SE.Opt(3,:)} is defined and use coefficients stored in this row instead.  If this is still not given, all coefficients are set to 1.  {\bf Element property rows} (in a standard property declaration matrix {\tt il}) for superelements take the forms described below \index{element!property row} with {\tt ProID} the property identification number and coefficients allowing the creation of a weighted sum of the superelement matrices {\tt SE}{\ti Name}{\tt .K\{i\}}. Thus, if {\tt K\{1\}} and {\tt K\{3\}} are two stiffness matrices and no other stiffness matrix is given, the superelement stiffness is given by {\tt coef1*K\{1\}+coef3*K\{3\}}.


\ruic{p\_super}{Database}{,Dbval]  ...} % - - - - - - - - - - - - - - - - - - - - - - - - 

There is no database call for {\tt p\_super} entries.

\ruic{p\_super}{1}{ : simple weighting coefficients} % - - - - - - - - - - - - - - 

\begin{verbatim}
 [ProId Type coef1 coef2 coef3 ... ]
\end{verbatim}


\noindent\begin{tabular}{@{}p{.2\textwidth}@{}p{.8\textwidth}@{}}
%
\rz{{\tt ProID}}  &  Property identification number.\\
\rz{{\tt Type}}   &  Identifier obtained with {\tt fe\_mat('p\_super','SI',1)}.\\
\rz{{\tt coef1}}  &  Multiplicative coefficient of the first matrix of the superelement ({\tt K\{1\}}). Superelement matrices used for the assembly of the global model matrices will be {\tt \{coef1*K\{1\}, coef2*K\{2\}, coef3*K\{3\}, ...\}}. Type of the matrices (stiffness, mass ...) is not changed. Note that you can define parameters for superelement using {\tt fe\_case(model,'par')}, see \fecase.\\
\end{tabular}

\ruic{p\_super}{2}{ : matrix type redefinition and weighting coefficients} % - - - 

\begin{verbatim}
 [ProId Type type1 coef1 type2 coef2 ...]
\end{verbatim}


\noindent\begin{tabular}{@{}p{.2\textwidth}@{}p{.8\textwidth}@{}}
%
\rz{{\tt ProID}}  &  Property identification number.\\
\rz{{\tt Type}}   &  Identifier obtained with {\tt fe\_mat('p\_super','SI',2)}.\\
\rz{{\tt type1}}  &  Type redefinition of the first matrix of the superelement ({\tt K\{1\}}) according to SDT standard type (1 for stiffness, 2 for mass, 3 for viscous damping... see \ltr{fe\_mknl}{MatType}).\\  
\rz{{\tt coef1}}  &  Multiplicative coefficient of the first matrix of the superelement ({\tt K\{1\}}). Superelement matrices used for the assembly of the global model matrices will be {\tt \{coef1*K\{1\}, coef2*K\{2\}, coef3*K\{3\}, ...\}}. Type of the matrices (stiffness, mass ...) is changed according to type1, type2, ... . Note that you can define parameters for superelement using {\tt fe\_case(model,'par')}, see \fecase.\\
\\
\end{tabular}


\rmain{See also}

  \fesuper, \ser{secms}

\end{SDT}










%HEVEA \input{../../../sdt.git/piezo/tex/p_piezo.tex}
\begin{latexonly}
\IfFileExists{../../../sdt.git/piezo/tex/m_piezo.tex}{\input{../../../sdt.git/piezo/tex/m_piezo.tex}}{}
%       Copyright (c) 2001-2020 by SDTools and INRIA, All Rights Reserved.
%       Use under OpenFEM trademark.html license and LGPL.txt library license
%       $Revision: 1.115 $  $Date: 2020/10/26 07:58:04 $

%------------------------------------------------------------------------------
\rtop{p\_beam}{p_beam}

Element property function for beams

\rsyntax\begin{verbatim}
il = p_beam('default') 
il = p_beam('database','name') 
il = p_beam('dbval ProId','name');
il = p_beam('dbval -unit TM ProId name');
il = p_beam('dbval -punit TM ProId name');
il2= p_beam('ConvertTo1',il)
\end{verbatim}

\rmain{Description}

This help starts by describing the main commands : {\tt p\_beam} \ts{Database} and \ts{Dbval}. Supported {\tt p\_beam} subtypes and their formats are then described.

\ruic{p\_beam}{Database}{,Dbval,  ...} % - - - - - - - - - - - - - - - - - - - 

{\tt p\_beam} contains a number of defaults obtained with {\tt p\_beam('database')} or\\  
{\tt p\_beam('dbval {\ti MatId}')}. You can select a particular entry of the database with using a name matching the database entries. You can also automatically compute the properties of standard beams

\noindent\begin{tabular}{@{}p{.35\textwidth}@{}p{.65\textwidth}@{}}
%
\rz\ts{circle }\tsi{r}  & beam with full circular section of radius \tsi{r}.\\
\rz\ts{rectangle }\tsi{b h} & beam with full rectangular section of width \tsi{b} and height \tsi{h}. See \beam\ for orientation (the default reference node is 1.5, 1.5, 1.5 so that orientation MUST be defined for non-symmetric sections). \\
\rz\ts{Type }\tsi{r1 r2 ...}  & other predefined sections of subtype 3 are listed using {\tt p\_beam('info')}. 
\end{tabular}


%{\tt p\_beam('database reftube')} gives a reference property of subtype 3 for a tube.\\

For example, you will obtain the section property row with {\tt ProId} 100 associated with a circular cross section of $0.05 m$ or a rectangular $0.05 \times 0.01 m$ cross section using

%begindoc
\begin{verbatim}
 % ProId 100, rectangle 0.05 m by 0.01 m
 pro = p_beam('database 100 rectangle .05 .01')
 % ProId 101 circle radius .05
 il = p_beam(pro.il,'dbval 101 circle .05')
 p_beam('info')
 % ProId 103 tube external radius .05 internal .04
 il = p_beam(il,'dbval -unit SI 103 tube .05 .04')
 % Transform to subtype 1
 il2=p_beam('ConvertTo1',il)
 il(end+1,1:6)=[104 fe_mat('p_beam','SI',1) 0 0 0 1e-5];
 il = fe_mat('convert SITM',il);
% Generate a property in TM, providing data in SI
 il = p_beam(il,'dbval -unit TM 105 rectangle .05 .01')
% Generate a property in TM providing data in TM
  il = p_beam(il,'dbval -punit TM 105 rectangle 50 10')
\end{verbatim}%enddoc

\ruic{p\_beam}{Show3D}{,MAP  ...} % - - - - - - - - - - - - - - - - - - - 

%begindoc
\begin{verbatim}

\end{verbatim}%enddoc



\ruic{p\_beam}{format}{ description and subtypes} % - - - - - - - - - - - - - - 

Element properties are described by the row of an element property matrix or a data structure with an {\tt .il} field containing this row (see \ser{il}). Element property functions such as {\tt p\_beam} support graphical editing of properties and a database of standard properties. 

For a tutorial on material/element property handling see \ser{femp}. For a programmers reference on formats used to describe element properties see \ser{il}. 

\ruic{p\_beam}{1}{ : standard} % - - - - - - - - - - - - - - - - - - - -

%\pbeam\ currently only supports a single format (\femat\ property subtype)

\begin{verbatim}
  [ProID   type   J I1 I2 A   k1 k2 lump NSM]
\end{verbatim}


\noindent\begin{tabular}{@{}p{.25\textwidth}@{}p{.75\textwidth}@{}}
%
{\ti ProID} & element property identification number. \\
\rz{\tt type}  & identifier obtained with {\tt fe\_mat('p\_beam','SI',1)}. \\
\rz{\tt J}  & torsional stiffness parameter (often different from polar moment of inertia {\tt I1+I2}). \\
\rz{\tt I1} & moment of inertia for bending plane 1 defined by a third node {\tt nr} or the vector {\tt vx vy vz} (defined in the \beam\ element). For a case with a beam along $x$ and plane 1 the $xy$ plane {\tt I1} is equal to $Iz = \int_{S} y^2 ds$. \\
\rz{\tt I2} & moment of inertia for bending plane 2 (containing the beam and orthogonal to plane 1. \\
\rz{\tt A} & section area. \\
\rz{\tt k1} & (optional) shear factor for motion in plane 1 (when not 0, a
                     Timoshenko beam element is used). The effective
                     area of shear is given by $k_1A$.  \\
\rz{\tt k2} & (optional) shear factor for direction 2.\\
\rz{\tt lump} & (optional) request for lumped mass model. 1 for inclusion of inertia terms. 2 for simple half mass at node. \\
\rz{\tt NSM} & (optional) non structural mass (density per unit length).\\
\end{tabular}\par

\bare\   elements only use the section area. All other parameters are ignored.

\beam\ elements use all parameters.  Without correction factors ({\ti k1} {\ti k2} not given or set to 0), the \beam\ element is the standard Bernoulli-Euler 12 DOF element based on linear interpolations for traction and torsion and cubic interpolations for flexion (see Ref.  \ecite{ger3} for example). When non zero shear factors are given, the bending properties are based on a Timoshenko beam element with selective reduced integration of the shear stiffness \ecite{imb1}. No correction for rotational inertia of sections is used.

\begin{SDT}
\ruic{p\_beam}{3}{ : Cross section database } % - - - - - - - - - - - - - - - - - - - -

This subtype can be used to refer to standard cross sections defined in database. It is particularly used by \nasread\ when importing NASTRAN {\tt PBEAML} properties.

\begin{verbatim}
  [ProID   type   0  Section Dim(i) ... ]
\end{verbatim}


\noindent\begin{tabular}{@{}p{.25\textwidth}@{}p{.75\textwidth}@{}}
%
{\ti ProID} & element property identification number. \\
\rz{\tt type} & identifier obtained with {\tt fe\_mat('p\_beam','SI',3)}. \\
\rz{\tt Section} & identifier of the cross section obtained with {\tt comstr('}\tsi{SectionName}{\tt ',-32)} where \tsi{SectionName} is a string defining the section (see below).\\
\rz{\tt Dim1 ...} & dimensions of the cross section.\\
\end{tabular}

Cross section, if existing, is compatible with NASTRAN {\tt PBEAML} definition. Equivalent moment of inertia and tensional stiffness are computed at the centroid of the section.
Currently available sections are listed with {\tt p\_beam('info')}. In particular one has {\tt ROD} (1 dim), {\tt TUBE} (2 dims), {\tt T} (4 dims), {\tt T2} (4 dims), {\tt I} (6 dims), {\tt BAR} (2 dims), {\tt CHAN1} (4 dims), {\tt CHAN2} (4 dims).

For \ts{NSM} and \ts{Lump} support \ts{ConverTo1} is used during definition to obtain the equivalent {\tt subtype 1} entry. 

\end{SDT}

\rmain{See also}

  \Ser{femp}, \ser{il}, \femat 
%------------------------------------------------------------------------------
\rtop{p\_heat}{p_heat}

Formulation and material support for the heat equation.

\rsyntax\begin{verbatim}
il = p_heat('default') 
\end{verbatim}

\rmain{Description}

This help starts by describing the main commands : {\tt p\_heat} \ts{Database} and \ts{Dbval}. Supported {\tt p\_heat} subtypes and their formats are then described. For theory see \ser{fe3dth}.

\ruic{p\_heat}{Database}{,Dbval]  ...} % - - - - - - - - - - - - - - - - - - - 

Element properties are described by the row of an element property matrix or a data structure with an {\tt .il} field containing this row (see \ser{il}). Element property functions such as {\tt p\_solid} support graphical editing of properties and a database of standard properties. 

{\tt p\_heat} database

%begindoc
\begin{verbatim}
 il=p_heat('database');
\end{verbatim}%enddoc

Accepted commands for the database are 
%
\begin{itemize}
\item \ts{d3 }\tsi{Integ }\tsi{SubType} : \ltt{Integ} integration rule for 3D volumes (default -3). 
\item \ts{d2 }\tsi{Integ }\tsi{SubType} : \ltt{Integ} integration rule for 2D volumes (default -3).
\end{itemize}

For fixed values, use {\tt p\_heat('info')}.

Example of database property construction

%begindoc
\begin{verbatim}
  il=p_heat([100 fe_mat('p_heat','SI',1) 0 -3 3],...
             'dbval 101 d3 -3 2');
\end{verbatim}%enddoc


\ruic{p\_heat}{Heat}{ equation element properties} % - - - - - - - - - - - - - - 

Element properties are described by the row of an element property matrix or a data structure with an {\tt .il} field containing this row. Element property functions such as {\tt p\_beam} support graphical editing of properties and a database of standard properties. 


\ruic{p\_heat}{1}{ : Volume element for heat diffusion (dimension DIM)} % - - - - - - - - - - - - - - - - - - - -

%\pbeam\ currently only supports a single format (\femat\ property subtype)

\begin{verbatim}
  [ProId fe_mat('p_heat','SI',1) CoordM Integ DIM]
\end{verbatim}


\noindent\begin{tabular}{@{}p{.25\textwidth}@{}p{.75\textwidth}@{}}
%
{\ti ProID} & element property identification number \\
\rz{\tt type}  & identifier obtained with {\tt fe\_mat('p\_beam','SI',1)} \\
\rz{\tt Integ}  & is rule number in integrules \\
\rz{\tt DIM}  & is problem dimension 2 or 3 D \\
\end{tabular}\par

\ruic{p\_heat}{2}{ : Surface element for heat exchange (dimension DIM-1)} % - - - - - - - - - - - - - - - - - - - -

\begin{verbatim}
   [ProId fe_mat('p_heat','SI',2) CoordM Integ DIM] 
\end{verbatim}


\noindent\begin{tabular}{@{}p{.25\textwidth}@{}p{.75\textwidth}@{}}
%
{\ti ProID} & element property identification number \\
\rz{\tt type}  & identifier obtained with {\tt fe\_mat('p\_beam','SI',2)} \\
\rz{\tt Integ}  & is rule number in {\tt integrules} \\
\rz{\tt DIM}  & is problem dimension 2 or 3 D \\
\end{tabular}\par

\ruic{p\_heat}{SetFace}{} % - - - - - - - - - - - - - - - - - - - - - -
This command can be used to define a surface exchange and optionally associated load.
Surface exchange elements add a stiffness term to the stiffness matrix related to the exchange coefficient {\tt Hf} defined in corresponding material property. One then should add a load corresponding to the exchange with the source temperature at $T_0$ through a convection coefficient {\tt Hf} which is {\tt Hf.T\_0}. If not defined, the exchange is done with source at temperature equal to 0. 

{\tt  model=p\_heat('SetFace',model,SelElt,pl,il);}\\

\begin{itemize}
\item{\tt SelElt} is a findelt command string to find faces that exchange heat (use 'SelFace' to select face of a given preselected element).
\item{\tt pl} is the identifier of existing material property ({\tt MatId}), or a vector defining an {\tt m\_heat} property.
\item{\tt il} is the identifier  of existing element property ({\tt ProId}), or a vector defining an {\tt p\_heat} property.
\end{itemize}

Command option \ts{-load }\tsi{T} can be used to defined associated load, for exchange with fluid at temperature \tsi{T}. Note that if you modify {\tt Hf} in surface exchange material property you have to update the load.

Following example defines a simple cube that exchanges with thermal source at 55 deg on the bottom face.

%beginddoc
\begin{verbatim} 
model=femesh('TestHexa8'); % Build simple cube model
model.pl=m_heat('dbval 100 steel'); % define steel heat diffusion parameter
model.il=p_heat('dbval 111 d3 -3 1'); % volume heat diffusion (1)
model=p_heat('SetFace-load55',... % exchange at 55 deg
    model,...
    'SelFace & InNode{z==0}',... % on the bottom face
    100,... % keep same matid for exchange coef
    p_heat('dbval 1111 d3 -3 2')); % define 3d, integ-3, for surface exchange (2)
cf=feplot(model); fecom colordatapro
def=fe_simul('Static',model); % compute static thermal state
mean(def.def)
\end{verbatim}%enddoc

\ruic{p\_heat}{2D}{validation} % - - - - - - - - - - - - - - - - - - - 

Consider a bi-dimensional annular thick domain $\Omega$ with radii $r_e=1$ and $r_i=0.5$. The data are specified on the internal circle $\Gamma_i$ and on the external circle $\Gamma_e$. The solid is made of homogeneous isotropic material, and its conductivity tensor thus reduces to a constant $k$. The steady state temperature distribution is then given by
\begin{eqsvg}{test_ann}
- k \Delta\theta(x,y) = f(x,y) \quad in \quad \Omega.
\end{eqsvg}

The solid is subject to the following boundary conditions\\
\begin{itemize}
\item{ {$\Gamma_i \,(r=r_i)$ : Neumann condition}\\
\begin{eqsvg}{p_heat_validation_1}
\displaystyle\frac{\partial \theta}{\partial n}(x,y) = g(x,y)
\end{eqsvg}  }
\item{ {$\Gamma_e \,(r=r_e)$ : Dirichlet condition}\\
\begin{eqsvg}{p_heat_validation_2}
\theta(x,y)=\theta_{ext}(x,y)
\end{eqsvg}  }
\end{itemize}

In above expressions, $f$ is an internal heat source, \mathsvg{\theta_{ext}}{p_heat_validation_l1} an external temperature at $r=r_e$, and $g$ a function. All the variables depend on the variable $x$ and $y$. 

The OpenFEM model for this example can be found in {\tt ofdemos('AnnularHeat')}.\\
{\bf Numerical application} : assuming $k=1$, $f=0$, $Hf=1e^{-10}$, $\theta_{ext}(x,y) = \exp(x) \cos(y)$ and \mathsvg{g(x,y)= -\frac{\exp(x)} {r_i} \left ( \cos(y)  x  - \sin(y)  x \right )}{p_heat_validation_l2}, the solution of the problem is  given by
\mathsvg{\displaystyle \theta(x,y) = \exp(x) \cos(y)}{p_heat_validation_l3}



\rmain{See also}

  \ser{fe3dth}, \ser{femp}, \femat 

%------------------------------------------------------------------------------
\IfFileExists{../tex/p_pml.tex}{\input{../tex/p_pml.tex}}{}

%------------------------------------------------------------------------------
\rtop{p\_shell}{p_shell}

Element property function for shells and plates (flat shells)

\rsyntax\begin{verbatim}
il = p_shell('default');
il = p_shell('database ProId name'); 
il = p_shell('dbval ProId name');
il = p_shell('dbval -unit TM ProId name');
il = p_shell('dbval -punit TM ProId name');
il = p_shell('SetDrill 0',il);
\end{verbatim}

\rmain{Description}

This help starts by describing the main commands : {\tt p\_shell} \ts{Database} and \ts{Dbval}. Supported {\tt p\_shell} subtypes and their formats are then described.


\ruic{p\_shell}{Database}{,Dbval,  ...} % - - - - - - - - - - - - - - - - - - - 

{\tt p\_shell} contains a number of defaults obtained with the \ts{database} and \ts{dbval} commands which respectively return a structure or an element property row. You can select a particular entry of the database with using a name matching the database entries. 


You can also automatically compute the properties of standard shells with

\noindent\begin{tabular}{@{}p{.35\textwidth}@{}p{.65\textwidth}@{}}
%
\rz\ts{kirchhoff }\tsi{e}  & Kirchhoff shell of thickness \tsi{e} (is not implemented for formulation 5, see each element for available choices)\\
\rz\ts{mindlin }\tsi{e}  & Mindlin shell of thickness \tsi{e} (see each element for choices). \\
\rz\ts{laminate }\tsi{MatIdi Ti Thetai}  & Specification of a laminate property by giving the different ply {\tt MatId}, thickness and angle. By default the z values are counted from -thick/2, you can specify another value with a z0.
%
\end{tabular}

You can append a string option of the form \ts{-f }\tsi{i} to select the appropriate shell formulation. The different formulations are described under each element topology (\triaa, \quad4, ...)
For example, you will obtain the element property row with {\tt ProId} 100 associated with a .1 thick Kirchhoff shell (with formulation 5) or the corresponding Mindlin plate use

%begindoc
\begin{verbatim}
 il = p_shell('database 100 MindLin .1')
 il = p_shell('dbval 100 kirchhoff .1 -f5')
 il = p_shell('dbval 100 laminate z0=-2e-3 110 3e-3 30 110 3e-3 -30')
 il = fe_mat('convert SITM',il);
 il = p_shell(il,'dbval -unit TM 2 MindLin .1') % set in TM, provide data in SI
 il = p_shell(il,'dbval -punit TM 2 MindLin 100') % set in TM, provide data in TM
\end{verbatim}%enddoc


For laminates, you specify for each ply the {\tt MatId}, thickness and angle.

\ruic{p\_shell}{Shell}{ format description and subtypes} % - - - - - - - - - - - - - - 

Element properties are described by the row of an element property matrix or a data structure with an {\tt .il} field containing this row (see \ser{il}). Element property functions such as {\tt p\_shell} support graphical editing of properties and a database of standard properties. 

For a tutorial on material/element property handling see \ser{femp}. For a reference on formats used to describe element properties see \ser{il}. 

\pshell\ currently only supports two subtypes

\ruic{p\_shell}{1}{ : standard isotropic} % - - - - - - - - - - - - - - - - - - - -

\begin{verbatim}
  [ProID type   f d O   h   k   MID2 RatI12_T3 MID3 NSM Z1 Z2 MID4]
\end{verbatim}


\noindent\begin{tabular}{@{}p{.05\textwidth}@{}p{.05\textwidth}@{}p{.9\textwidth}@{}}
%
\rz{\tt type}  &   &  identifier obtained with {\tt fe\_mat('p\_shell','SI',1)}.\\
\rz{\tt f} &  & \rz{{\tt 0}} use default of element. For other formulations the specific help for each element (\quada, \triaa, ...), each formulation specifies integration rule. \\
\rz{\tt d} & \rz{\tt -1} & no drilling stiffness. The element DOFs are the standard translations and rotations at all nodes (DOFs {\tt .01} to {\tt .06}). The drill DOF (rotation {\tt .06} for a plate in the {\sl xy} plane) has no stiffness and is thus eliminated by \femk\ if it corresponds to a global DOF direction. The default is {\tt d=1} ({\tt d} is set to 1 for a declared value of zero). \\
& \rz{\tt d} & arbitrary drilling stiffness with value proportional to {\tt d} is added. This stiffness is often needed in shell problems but may lead to numerical conditioning problems if the stiffness value is very different from other physical stiffness values. Start with a value of 1. Use {\tt il=p\_shell('SetDrill d',il)} to set to {\tt d} the drilling stiffness of all {\tt p\_shell} subtype 1 rows of the property matrix {\tt il}. \\
\rz{\tt h} &  & plate thickness.\\
\rz{\tt k} & {\ti k} & shear correction factor (default 5/6, default used if {\tt k} is zero). This correction is not used for formulations based on triangles since \triaa\ is a thin plate element. \\
\rz{\tt RatI12\_T3} &  & Ratio of bending moment of inertia to nominal {\tt T3/I12} (default 1).\\
\rz{\tt  NSM} &  & Non structural mass per unit area.\\
\rz{\tt  MID2} &  & material property for bending. Defauts to element {\tt MatId} if equal to 0. \\
\rz{\tt  MID3} &  & material property for transverse shear. \\
\rz{\tt  z1,z2} &  & (unused) offset for fiber computations.\\
\rz{\tt  MID4} &  & material property for membrane/bending coupling.\\
\end{tabular}

Shell strain is defined by the membrane, curvature and transverse shear \texline (display with {\tt p\_shell('ConstShell')}). 
%
\begin{eqsvg}{p_shell_1}
\ve{\ba{c}\epsilon_{xx} \\\epsilon_{yy} \\ 2 \epsilon_{xy} \\ \kappa_{xx} \\\kappa_{yy} \\ 2 \kappa_{xy} \\ \gamma_{xz} \\ \gamma_{yz} \ea}=\ma{\ba{cccccccc}
 N,x & 0 & 0 & 0 & 0 \\
 0 & N,y & 0 & 0 & 0 \\
 N,y & N,x & 0 & 0 & 0 \\
 0 & 0 & 0 & 0 & N,x \\
 0 & 0 & 0 & -N,y & 0 \\
 0 & 0 & 0 & -N,x & N,y \\
 0 & 0 & N,x & 0 & -N \\
 0 & 0 & N,y & N & 0 \ea}
\ve{\ba{c} u \\ v \\ w \\ ru \\ rv \ea}
\end{eqsvg}

\ruic{p\_shell}{2}{ : composite} % - - - - - - - - - - - - - - - - - - - -

\begin{verbatim}
  [ProID type   Z0 NSM SB FT TREF GE LAM MatId1 T1 Theta1 SOUT1 ...]
\end{verbatim}


\noindent\begin{tabular}{@{}p{.15\textwidth}@{}p{.85\textwidth}@{}}
%
\rz{{\tt ProID}} &  Section property identification number. \\
\rz{{\tt type}}     &  Identifier obtained with {\tt fe\_mat('p\_shell','SI',2)}.\\
\rz{{\tt Z0}}     &  Distance from reference plate to bottom surface. \\
\rz{{\tt NSM}}     & Non structural mass per unit area. \\
\rz{{\tt SB}}     & Allowable shear stress of the bonding material. \\
\rz{{\tt FT}}     & Failure theory. \\
\rz{{\tt TREF}}     &  Reference temperature. \\
\rz{{\tt Eta}}     &  Hysteretic loss factor. \\
\rz{{\tt LAM}}     &  Laminate type. \\
\rz{{\tt MatId{\ti i}}} &  {\tt MatId} for ply {\ti i}, see \ltr{m\_elastic}{1},  \ltr{m\_elastic}{5}, ...\\
\rz{{\tt T{\ti i}}} &  Thickness of ply {\ti i}. \\
\rz{{\tt Theta{\ti i}}} &  Orientation of ply {\ti i}. \\
\rz{{\tt SOUT{\ti i}}} &  Stress output request for ply {\ti i}.
\end{tabular}

Note that this subtype is based on the format used by NASTRAN for {\tt PCOMP} and the formulation used for each topology is discussed in each element (see \quada, \triaa). You can use the \ts{DbvalLaminate} commands to generate standard entries.

\begin{eqsvg}{p_shell_2}
\ve{\ba{c}N \\ M \\ Q\ea} = \ma{\ba{ccc} A & B & 0\\ B & D & 0\\ 0 & 0 &
F\ea} \ve{\ba{c}\epsilon \\ \kappa \\ \gamma \ea}
\end{eqsvg}

\ruic{p\_shell}{setTheta}{}

When dealing with laminated plates, the classical approach uses a material orientation constant per element. OpenFEM also supports more advanced strategies with orientation defined at nodes but this is still poorly documented.

The material orientation is the reference for plies. Any angle defined in a laminate command is an additional rotation. In the example below, the element orientation is rotated 30 degrees, and the ply another 30. The fibers are thus oriented 60 degrees in the $xy$ plane. Stresses are however given in the material orientation thus with a 30 degree rotation. Per ply output is not currently implemented. 

The element-wise material angle is stored for each element. In column 7 for \triaa, 8 for \quada, ...  The \ts{setTheta} command is a utility to ease the setting of these angles. By default, the orientation is done at element center. To use the mean orientation at nodes use command option \ts{-strategy 2}.

\begin{verbatim}
model=ofdemos('composite');
model.il = p_shell('dbval 110 laminate 100 1 30'); % single ply

% Define material angle based on direction at element
MAP=feutil('getnormalElt MAP -dir1',model);
bas=basis('rotate',[],'rz=30;',1);
MAP.normal=MAP.normal*reshape(bas(7:15),3,3)';
model=p_shell('setTheta',model,MAP);

% Obtain a MAP of material orientations
MAP=feutil('getnormalElt MAP -dir1',model);
feplot(model);fecom('showmap',MAP)

% Set elementwise material angles using directions given at nodes. 
% Here a global direction
MAP=struct('normal',ones(size(model.Node,1),1)*bas(7:9), ...
    'ID',model.Node(:,1),'opt',2);
model=p_shell('setTheta',model,MAP);

% Using an analytic expression to define components of 
% material orientation vector at nodes
data=struct('sel','groupall','dir',{{'x-0','y+.01',0}},'DOF',[.01;.02;.03]);
model=p_shell('setTheta',model,data);
MAP=feutil('getnormalElt MAP -dir1',model);
feplot(model);fecom('showmap',MAP)
\end{verbatim}

{\tt model=p\_shell('setTheta',model,0)} is used to reset the material orientation to zero.


Technically, shells use the {\tt of\_mk('BuildNDN')} rule 23 which generates a basis at each integration point. The first vector {\tt v1x,v1y,v1z} is built in the direction of $r$ lines and {\tt v2x,v2y,v2z} is tangent to the surface and orthogonal to $v1$. When a \ltt{InfoAtNode} map provides {\tt v1x,v1y,v1z}, this vector is projected (NEED TO VERIFY) onto the surface and $v2$ taken to be orthogonal. 


\rmain{See also}

  \Ser{femp}, \ser{il}, \femat


%------------------------------------------------------------------------------
\rtop{p\_solid}{p_solid}

Element property function for volume elements.

\rsyntax\begin{verbatim}
il=p_solid('database ProId Value')
il=p_solid('dbval ProId Value')
il=p_solid('dbval -unit TM ProId name');
il=p_solid('dbval -punit TM ProId name');
model=p_solid('default',model)
\end{verbatim}


\rmain{Description}


This help starts by describing the main commands : {\tt p\_solid} \ts{Database} and \ts{Dbval}. Supported {\tt p\_solid} subtypes and their formats are then described.

\ruic{p\_solid}{Database}{, Dbval, Default,  ...} % - - - - - - - - - - - - - - - - - - - 

Element properties are described by the row of an element property matrix or a data structure with an {\tt .il} field containing this row (see \ser{il}). Element property functions such as {\tt p\_solid} support graphical editing of properties and a database of standard properties. 

Accepted commands for the database are 
%
\begin{itemize}
\item \ts{d3 }\tsi{Integ} : \tsi{Integ} integration rule for quadratic 3D volumes. For information on rules available see \ltr{integrules}{Gauss}. Examples are \ts{d3 2}  2x2x2 integration rule for linear volumes (hexa8 ... ); \ts{d3 -3} default integration for all 3D elements, ...
\item\ts{d2 }\tsi{Integ} :  \tsi{Integ} integration rule for quadratic 2D volumes. For example \ts{d2 2} 2x2x2 integration rule for linear volumes (q4p ... ). You can also use \ts{d2 1 0 2} for plane stress, and \ts{d2 2 0 2} for axisymmetry.
\item\ts{fsc }\tsi{Integ} : integration rule selection for fluid/structure coupling.
\end{itemize}

For fixed values, use {\tt p\_solid('info')}.

For a tutorial on material/element property handling see \ser{femp}. For a reference on formats used to describe element properties see \ser{il}. 

Examples of database property construction

%begindoc
\begin{verbatim}
  il=p_solid([100 fe_mat('p_solid','SI',1) 0 3 0 2], ...
             'dbval 101 Full 2x2x2','dbval 102 d3 -3');
  il=fe_mat('convert SITM',il);
  il=p_solid(il,'dbval -unit TM 2 Reduced shear')
  % Try a smart guess on default 
  model=femesh('TestHexa8');model.il=[]; 
  model=p_solid('default',model) 
\end{verbatim}%enddoc


\ruic{p\_solid}{1}{ : 3D volume element} % - - - - - - - - - - - - - - 

\begin{verbatim}
[ProID fe_mat('p_solid','SI',1) Coordm In Stress Isop ]
\end{verbatim}


\noindent\begin{tabular}{@{}p{.2\textwidth}@{}p{.8\textwidth}@{}}
%
\rz{{\tt ProID}}  &  Property identification number.\\
\rz{{\tt Coordm}} &  Identification number of the material coordinates system. {\bf Warning}  not implemented for all material formulations. \\
\rz{{\tt In}}     &  Integration rule selection (see \ltr{integrules}{Gauss}). 0 selects the legacy 3D mechanics element ({\tt of\_mk\_pre.c}), -3 the default rule. \\
\rz{{\tt Stress}} &  Location selection for stress output (NOT USED).\\
\rz{{\tt Isop}}   &  Integration scheme.  Used to select the generalized strain definition in \nlinout\ implementations (see~\ser{nlio3d}). May also be used to select shear protection mechanisms in the future. \\
\end{tabular}

The underlying physics for this subtype are selected through the material property. Examples are 3D mechanics with \melastic, \begin{SDT} piezo electric volumes (see {\tt m\_piezo})\end{SDT}, heat equation (\pheat).

\ruic{p\_solid}{2}{ : 2D volume element } % - - - - - - - - - - - - - - - - - - -

\begin{verbatim}
  [ProId fe_mat('p_solid','SI',2)  Form N In]
\end{verbatim}


\noindent\begin{tabular}{@{}p{.2\textwidth}@{}p{.8\textwidth}@{}}
%
\rz{{\tt ProID}}  &  Property identification number.\\
\rz{{\tt Type}}   &  Identifier obtained with {\tt fe\_mat('p\_solid,'SI',2)}.\\
\rz{{\tt Form}}   &  Formulation (0 plane strain, 1 plane stress, 2 axisymmetric), see details in \melastic. \\
\rz{{\tt N}}      &  Fourier harmonic for axisymmetric elements that support it.\\
\rz{{\tt In}}     &  Integration rule selection (see \ltr{integrules}{Gauss}). 0 selects legacy 2D element, -3 the default rule.
\end{tabular}

The underlying physics for this subtype are selected through the material property. Examples are 2D mechanics with \melastic.

\ruic{p\_solid}{3}{ : ND-1 coupling element} % - - - - - - - - - - - - - - - - - - -

\begin{verbatim}
  [ProId fe_mat('p_solid','SI',3) Integ Form Ndof1 ...]
\end{verbatim}


\noindent\begin{tabular}{@{}p{.2\textwidth}@{}p{.8\textwidth}@{}}
%
\rz{{\tt ProID}}  &  Property identification number.\\
\rz{{\tt Type}}   &  Identifier obtained with {\tt fe\_mat('p\_solid,'SI',3)}.\\
\rz{{\tt Integ}}  &  Integration rule selection (see \ltr{integrules}{Gauss}). 0 or -3 selects the default for the element.\\
\rz{{\tt Form}}  &   1 volume force, 2 volume force proportional to density, 3 pressure, 4: fluid/structure coupling, see \fsc, 5 2D volume force, 6 2D pressure. 8 Wall impedance (acoustics), then uses the $R$ parameter in fluid.\\
%
\end{tabular}

\rmain{See also}

  \Ser{femp}, \ser{il}, \femat


%------------------------------------------------------------------------------
\rtop{p\_spring}{p_spring}

Element property function for spring and rigid elements

\rsyntax\begin{verbatim}
il=p_spring('default') 
il=p_spring('database MatId Value')
il=p_spring('dbval MatId Value')
il=p_spring('dbval -unit TM ProId name');
il=p_spring('dbval -punit TM ProId name');
\end{verbatim}

\rmain{Description}

This help starts by describing the main commands : {\tt p\_spring} \ts{Database} and \ts{Dbval}. Supported {\tt p\_spring} subtypes and their formats are then described.

\ruic{p\_spring}{Database}{,Dbval]  ...} % - - - - - - - - - - - - - - - - - - - 

Element properties are described by the row of an element property matrix or a data structure with an {\tt .il} field containing this row (see \ser{il}). 

Examples of database property construction

%begindoc
\begin{verbatim}
 il=p_spring('database 100 1e12 1e4 0')
 il=p_spring('dbval 100 1e12');
 il=fe_mat('convert SITM',il);
 il=p_spring(il,'dbval 2 -unit TM 1e12') % Generate in TM, provide data in SI
 il=p_spring(il,'dbval 2 -punit TM 1e9') % Generate in TM, provide data in TM
\end{verbatim}%enddoc


\pspring\ currently supports 2 subtypes

\ruic{p\_spring}{1}{ : standard} % - - - - - - - - - - - - - - - - - - - -

\begin{verbatim}
  [ProID type  k m c Eta S]
\end{verbatim}


\noindent\begin{tabular}{@{}p{.15\textwidth}@{}p{.85\textwidth}@{}}
%
\rz{{\tt ProID}} &  property identification number.\\
\rz{\tt type}    &  identifier obtained with {\tt fe\_mat('p\_spring','SI',1)}.\\
\rz{{\tt k}}     &  stiffness value.\\
\rz{{\tt m}}     &  mass value.\\
\rz{{\tt c}}     &  viscous damping value.\\
\rz{{\tt eta}}   &  loss factor.\\
\rz{{\tt S}}     &  Stress coefficient.\\
\end{tabular}

\ruic{p\_spring}{2}{ : bush} % - - - - - - - - - - - - - - - - - - - -

Note that type 2 is only functional with \cbush\ elements.

\begin{verbatim}
  [ProId Type k1:k6 c1:c6 Eta SA ST EA ET m v]
\end{verbatim}


\noindent\begin{tabular}{@{}p{.15\textwidth}@{}p{.85\textwidth}@{}}
%
\rz{{\tt ProID}} &  property identification number. \\
\rz{\tt type}    &  identifier obtained with {\tt fe\_mat('p\_spring','SI',2)}.\\
\rz{\tt ki}      &  stiffness for each direction.\\
\rz{\tt ci}      &  viscous damping for each direction.\\
\rz{\tt SA}      &  stress recovery coef for translations.\\
\rz{\tt ST}      &  stress recovery coef for rotations.\\
\rz{\tt EA}      &  strain recovery coef for translations.\\
\rz{\tt ET}      &  strain recovery coef for rotations.\\
\rz{\tt m}       &  mass.\\
\rz{\tt v}       &  volume.\\

\end{tabular}

\rmain{See also}

  \Ser{femp}, \ser{il}, \femat, \celas, \cbush

%------------------------------------------------------------------------------
\begin{SDT}
\rtop{p\_super}{p_super}

Element property function for superelements.

\rsyntax\begin{verbatim}
il=p_super('default') 
il=p_super('database MatId Value')
il=p_super('dbval MatId Value')
il=p_super('dbval -unit TM ProId name');
il=p_super('dbval -punit TM ProId name');
\end{verbatim}


\rmain{Description}


If {\tt ProID} is not given, \fesuperb\ will see if {\tt SE.Opt(3,:)} is defined and use coefficients stored in this row instead.  If this is still not given, all coefficients are set to 1.  {\bf Element property rows} (in a standard property declaration matrix {\tt il}) for superelements take the forms described below \index{element!property row} with {\tt ProID} the property identification number and coefficients allowing the creation of a weighted sum of the superelement matrices {\tt SE}{\ti Name}{\tt .K\{i\}}. Thus, if {\tt K\{1\}} and {\tt K\{3\}} are two stiffness matrices and no other stiffness matrix is given, the superelement stiffness is given by {\tt coef1*K\{1\}+coef3*K\{3\}}.


\ruic{p\_super}{Database}{,Dbval]  ...} % - - - - - - - - - - - - - - - - - - - - - - - - 

There is no database call for {\tt p\_super} entries.

\ruic{p\_super}{1}{ : simple weighting coefficients} % - - - - - - - - - - - - - - 

\begin{verbatim}
 [ProId Type coef1 coef2 coef3 ... ]
\end{verbatim}


\noindent\begin{tabular}{@{}p{.2\textwidth}@{}p{.8\textwidth}@{}}
%
\rz{{\tt ProID}}  &  Property identification number.\\
\rz{{\tt Type}}   &  Identifier obtained with {\tt fe\_mat('p\_super','SI',1)}.\\
\rz{{\tt coef1}}  &  Multiplicative coefficient of the first matrix of the superelement ({\tt K\{1\}}). Superelement matrices used for the assembly of the global model matrices will be {\tt \{coef1*K\{1\}, coef2*K\{2\}, coef3*K\{3\}, ...\}}. Type of the matrices (stiffness, mass ...) is not changed. Note that you can define parameters for superelement using {\tt fe\_case(model,'par')}, see \fecase.\\
\end{tabular}

\ruic{p\_super}{2}{ : matrix type redefinition and weighting coefficients} % - - - 

\begin{verbatim}
 [ProId Type type1 coef1 type2 coef2 ...]
\end{verbatim}


\noindent\begin{tabular}{@{}p{.2\textwidth}@{}p{.8\textwidth}@{}}
%
\rz{{\tt ProID}}  &  Property identification number.\\
\rz{{\tt Type}}   &  Identifier obtained with {\tt fe\_mat('p\_super','SI',2)}.\\
\rz{{\tt type1}}  &  Type redefinition of the first matrix of the superelement ({\tt K\{1\}}) according to SDT standard type (1 for stiffness, 2 for mass, 3 for viscous damping... see \ltr{fe\_mknl}{MatType}).\\  
\rz{{\tt coef1}}  &  Multiplicative coefficient of the first matrix of the superelement ({\tt K\{1\}}). Superelement matrices used for the assembly of the global model matrices will be {\tt \{coef1*K\{1\}, coef2*K\{2\}, coef3*K\{3\}, ...\}}. Type of the matrices (stiffness, mass ...) is changed according to type1, type2, ... . Note that you can define parameters for superelement using {\tt fe\_case(model,'par')}, see \fecase.\\
\\
\end{tabular}


\rmain{See also}

  \fesuper, \ser{secms}

\end{SDT}










\IfFileExists{../../../sdt.git/piezo/tex/p_piezo.tex}{\input{../../../sdt.git/piezo/tex/p_piezo.tex}}{}
\end{latexonly}
\end{SDT}

%------------------------------------------------------------------------
\rtop{quad4, quadb, mitc4}{quad4}

4 and 8 node quadrilateral plate/shell elements.
\index{plate element}\index{element!plate}

\rmain{Description}

\ingraph{60}{E_quad4}

In a model description matrix, {\bf element property rows} for \quada, \quadb\ and \quadc\ elements follow the standard format

\begin{verbatim}
 [n1 ... ni MatID ProID EltID Theta Zoff T1 ... Ti] 
\end{verbatim}


giving the node identification numbers {\tt ni} (1 to 4 or 8), material {\tt MatID}, property {\tt ProID}. Other {\bf optional} information is {\tt EltID} the element identifier, {\tt Theta} the angle between material $x$ axis and element $x$ axis, {\tt Zoff} the off-set along the element $z$ axis from the surface of the nodes to the reference plane (use \ltr{feutil}{Orient} command to check z-axis orientation), {\tt Ti} the thickness at nodes (used instead of {\tt il} entry, currently the mean of the {\tt Ti} is used). \par

If {\tt n3} and {\tt n4} are equal, the \triaa\ element is automatically used in place of the \quada.

Isotropic materials are currently the only supported (this may change soon). Their declaration follows the format described in \melastic.
Element property declarations follow the format described \pshell.

\rui{quad4}

Supported formulations ({\tt f} value stored in {\tt il(3)} \pshell\ entries for isotropic materials and element default for composites) are 

\begin{itemize}
 \item {\tt 0} element/property dependent default. This is always used for composites ({\tt p\_shell} subtype 2). 
\begin{SDT}
\item {\tt 5} Q4CS is a second implementation MITC4 elements that supports classical laminated plate theory (composites) as well as the definition of piezo-electric extension actuators. This is the default for SDT. Non flat shell geometries are supported with interpolation of normal fields.
\end{SDT}
 \item {\tt 1} 4 tria3 thin plate elements with condensation of central node. {\bf Bad} formulation implemented in {\tt quad4}. 
 \item {\tt 2} Q4WT for membrane and Q4gamma for bending (implemented in {\tt quad4}). This is only applicable if the four nodes are in a single plane. When not, formulation {\tt 1} is called.
 \item {\tt 4} MITC4 calls the MITC4 element below. This implementation has not been tested extensively, so that the element may not be used in all configurations.  It uses 5 DOFs per node with the two rotations being around orthogonal in-plane directions. This is not consistent for mixed element types assembly. Non smooth surfaces are not handled properly because this is not implemented in the \ltr{feutil}{GetNormal} command which is called for each group of {\tt mitc4} elements. 
\end{itemize}

\begin{SDT}
The definition of local coordinate systems for composite fiber orientation still needs better documentation. Currently, {\tt q4cs} the only element that supports composites, uses the local coordinate system resulting from the \ltt{BuildNDN} {\tt 23} rule. A temporary solution for uniform orientation is provided with {\tt model=feutilb('shellmap -orient dx dy dz',model)}. 
\end{SDT}

\rui{quadb}

  \ingraph{65}{nquad8}

Supported formulations (\pshell {\tt il(3)} for isotropic materials and element default for composites) are 

\begin{itemize}
 \item {\tt 1} 8 tria3 thin plate elements with condensation of central node.
 \item {\tt 2} isoparametric thick plate with reduced integration. For non-flat elements, formulation {\tt 1} is used.
\end{itemize}


\rmain{See also}

\melastic, \pshell, \femk, \feplot 

%------------------------------------------------------------------------
\rtop{q4p, q8p, t3p, t6p and other 2D volumes}{q4p}

\noindent  2-D volume elements. 
\index{plate element}\index{element!plate}

\rmain{Description} 

The \qfourp\, {\tt q5p}, {\tt q8p}, {\tt q9a}, {\tt t3p}, {\tt t6p} elements are topology references for 2D volumes and 3D surfaces.


In a model description matrix, {\bf element property rows} for this  elements follow the standard format

\begin{verbatim}
 [n1 ... ni MatID ProID EltID Theta] 
\end{verbatim}


giving the node identification numbers {\tt n1,...ni}, material {\tt MatID}, property {\tt ProID}. Other {\bf optional} information is {\tt EltID} the element identifier, {\tt Theta} the angle between material $x$ axis and element $x$ axis (material orientation maps are generally preferable).


These elements only define topologies, the nature of the problem to be solved should be specified using a property entry, see~\ser{feform} for supported problems and  \psolid, \pheat, ... for formats. 


Integration rules for various topologies are described under \integrules. Vertex coordinates of the reference element can be found using an \integrules\ command containing the name of the element such as {\tt r1=integrules('q4p');r1.xi}.


{\bf Backward compatibility note} : if no element property entry is defined, or with a {\tt p\_solid} entry with the integration rule set to zero, the element defaults to the historical 3D mechanic elements described in \ser{of_mk_subs}. 



These volume elements are used for various problem families.


\rmain{See also} % - - - - - - - - -  - - - - - - - -  - - - - 

\femat, \femk, \feplot 


%------------------------------------------------------------------------------
\rtop{rigid}{rigid}

\noindent  Linearized rigid link constraints.\index{element!rigid link}\index{rigid link}

\rmain{Description}

 Rigid links are often used to model stiff connections in finite element models. One generates a set of linear constraints that relate the 6 DOFs of master $M$ and slave $S$ nodes by

\begin{eqsvg}{rigid_1}
\ve{\ba{c} u \\ v \\ w \\ r_x \\ r_y \\ r_z \ea}_S = \ma{\ba{cccccc} 
1 & 0 & 0  &  0     & z_{MS}  & -y_{MS}\\
0 & 1 & 0  & -z_{MS} & 0      & x_{MS} \\
0 & 0 & 1  &  y_{MS} &-x_{MS} &0 \\
0 & 0 & 0  &  1     & 0      & 0 \\
0 & 0 & 0  &  0     & 1      & 0 \\
0 & 0 & 0  &  0     & 0      & 1  \ea}
 \ve{\ba{c} u \\ v \\ w \\ r_x \\ r_y \\ r_z\ea}_M
\end{eqsvg}

Resolution of linear constraints is performed using \fecase\ or model assembly (see \ser{feass}) calls. The theory is discussed in~\ser{mpc}. Note that the master node of a rigid link has 6 DOF, even if the model may only need less (3 DOF for volumes).

If coordinate systems are defined in field {\tt model.bas} (see \basis), {\tt PID} (position coordinate system) and {\tt DID} (displacement coordinate system) declarations in columns 2 and 3 of {\tt model.Node} are properly handled.

\vs Although \ts{rigid} are linear constraints rather than true elements, such connections can be declared using an element group of rigid connection with a header row of the form {\tt [Inf abs('rigid')]} followed by as many element rows as connections of the form

\begin{verbatim}
 [ n1 n2 DofSel MatId ProId EltId]
\end{verbatim}

where node {\tt n2} will be rigidly connected to node {\tt n1} which will remain free. {\tt DofSel} lets you specify which of the 3 translations and 3 rotations are connected (thus {\tt 123} connects only translations while {\tt 123456} connects both translations and rotations). The rigid elements thus defined can then be handled as standard elements.


With this strategy you can use penalized rigid links (\celas\ element) instead of truly rigid connections. This requires the selection of a stiffness constant but can be easier to manipulate. To change a group of \rigid\ elements into \celas\ elements and set a stiffness constant {\tt Kv}, one can do

\begin{verbatim}
model=feutil('SetGroup rigid name celas',model);
model.Elt(feutil('findelt group i',model),7) = Kv; % celas in group i
\end{verbatim}


\vs The other \ts{rigid} definition strategy is to store them as a {\tt case} entry. \ts{rigid} entries are rows of the {\tt Case.Stack} cell array giving {\tt \{'rigid', Name, Elt\}}. 

The syntax is 
\begin{verbatim}
model=fe_case(model,'rigid',Name,Elt);
\end{verbatim}
where {\tt Name} is a string identifying the entry. {\tt Elt} is a \hyperlink{elt}{model description matrix} containing {\tt rigid} elements. 
Command option \ts{Append} allows concatenating a new list of rigid constraints to a preexisting list in {\tt Case.Stack}. 

The call \verb+model=fe_case(model,'rigidAppend','Name',Elt1);+ would thus concatenate the previously defined list {\tt Name} with the new rigid element matrix {\tt Elt1}.


Using the \fecase\ call to implement \ts{rigid} allows an alternative rigid constraint input that can be more comprehensive in some applications. You may use a list of the form {\tt [MasterNode slaveDOF slaveNode\_1 slaveNode\_2 ... slaveNode\_i] } instead of the element matrix. Command option \ts{Append} is also valid.

\vs The following sample calls are thus equivalent, and consists in implementing a rigid link between nodes 1 and 2, and 1 and 3 (with 1 as master) for all six DOF in a sample model:

\begin{verbatim}
model=fe_case(model,'rigid','Rigid edge',...
[Inf abs('rigid'); 
1 2 123456 0 0 0;
1 3 123456 0 0 0]);
% or
model=fe_case(model,'rigid','Rigid edge',[1 123456 2 3]);
\end{verbatim}

\vs In some cases, interactions with \feplot\ visualization may transform the {\tt Elt} matrix into a structure with fields {\tt Elt} that contains the original data, and {\tt Sel} that is internally used by \feplot\ to display the rigid constraint on the mesh.

\vs The following example generates the mesh of a square plate with a rigid edge, the \ts{rigid} constraint is here declared as \ts{rigid} elements  

%begindoc
\begin{verbatim}
% generate a sample plate model
model=femesh('testquad4 divide 10 10');

% generate beam1 elements based on the edge 
% of the underlying 2D model at x=0
elt=feutil('selelt seledge & innode{x==0}',model);
% remove element header from selection, 
% we only use the node connectivity
elt=elt(2:end,:);
% assign the rigid element property 
elt(2:end,3)=123456; % all 6 DOF are slave
% remove old data from the previous element selection
elt(2:end,4:end)=0; 

% add rigid elements to the model
model=feutil('addelt',model,'rigid',elt);
% % alternative possible: define as a case entry
% model=fe_case(model,'rigid','Rigid edge',[Inf abs('rigid'); elt]); 

% Compute and display modes
def=fe_eig(model,[6 20 1e3]);
feplot(model,def);fecom(';view3;ch8;scd.1');
\end{verbatim}%enddoc

\vs 
The {\tt rigid} function itself is only used for low level access by generating the subspace {\tt T} that verifies rigid constraints 

\begin{verbatim}
[T,cdof] = rigid(node,elt,mdof)
[T,cdof] = rigid(Up)
\end{verbatim}

\rmain{See also}

\Ser{mpc}, \celas

%------------------------------------------------------------------------------
\rtop{tria3, tria6}{tria3}

Element functions for a 3 node/18 DOF and 6 nodes/36 DOF shell elements.\index{plate element}\index{element!plate}


\rmain{Description}

\ingraph{40}{ntria3}

  In a model description matrix, {\bf element property rows} for \triaa\   elements follow the standard format

\begin{verbatim}
 [n1 n2 n3 MatID ProID EltID Theta Zoff T1 T2 T3] 
\end{verbatim}


\noindent giving the node identification numbers {\tt ni}, material {\tt MatID}, property {\tt ProID}. Other {\bf optional} information is {\tt EltID} the element identifier, {\tt Theta} the angle between material $x$ axis and element $x$ axis (currently unused), {\tt Zoff} the off-set along the element $z$ axis from the surface of the nodes to the reference plane, {\tt Ti} the thickness at nodes (used instead of {\tt il} entry, currently the mean of the {\tt Ti} is used). \par

The element only supports isotropic materials with the format described in \melastic.

\noindent The supported property declaration format is described in \pshell. Note that \triaa\ only supports thin plate formulations.

\vs\noindent  \triaa\ : \pshell formulations other than 5 call a T3 triangle for membrane properties and a DKT for bending (see \ecite{bat2} for example). Formulation 5 calls {\tt q4cs} which presents significant shear locking and {\bf should thus not be used}. 

\vs\triac\ :\pshell\ formulation is not used since the currently the only implementation is a call to {\tt q4cs} (formulation {\tt 5}). 

\rmain{See also}

\noindent  \quada, \quadb, \femat, \pshell, \melastic, \femk, \feplot





