%       Copyright (c) 2001-2002 by INRIA and SDTools, All Rights Reserved.
%       Use under OpenFEM trademark.html license and LGPL.txt library license
%       $Revision: 1.5 $  $Date: 2014/12/19 14:58:17 $

\documentclass[a4paper]{article}
\usepackage[dvips]{graphicx}
\author{Claire DELFORGE}
\title{OpenFEM visualization with Medit}
\date{ }
\begin{document}
\maketitle
\section{What is Medit}
Medit is an interactive mesh visualization software, developed by the Gamma project at INRIA-Rocquencourt.\\
Medit executable is freely available at the following address :
\begin{center} http://www-rocq.inria.fr/gamma/medit \end{center}
Documentations can be obtained as well.
\section{Interfacing OpenFEM with Medit}
\subsection{Functions implemented in OpenFEM-Scilab}
The following functions have been implemented to interface Medit with OpenFEM Scilab :
\begin{itemize}
\item in directory \verb+mex/cmex/medit+ : C-functions \verb+write_mesh.c+ (writes a mesh file : file \verb+.mesh+) and \verb+write_bb.c+ (writes a data file for coloring the structure : file \verb+.bb+). Interface between these C-functions and Scilab have been implemented in the form of mex-functions. These files are exactly the same as those of the Matlab version.  
\item \verb+meditwrite.sci+ : Scilab function that creates the files \verb+.mesh+ and \verb+.bb+ in order to display a mesh, to animate modal deformations, loads or constraints, and running Medit directly from Scilab; this function uses C-routines \verb+write_mesh+ and \verb+write_bb+.
\end{itemize}

\subsection{Functions implemented in OpenFEM-Matlab}
The following functions have been implemented to interface Medit with OpenFEM Matlab :
\begin{itemize}
\item in directory \verb+mex+ : C-functions \verb+write_mesh.c+ and \verb+write_bb.c+ are the same as those in Scilab version.
\item \verb+meditwrite.m+ : this Matlab function is equivalent to the Scilab function \verb+meditwrite.sci+.
\end{itemize}



\section{Use}
A visualization example is provided in directory \verb+tests+ (file \verb+test_medit.sci+ for OpenFEM Scilab or file \verb+test_medit.m+ for OpenFEM Matlab).\\
To run this example in Scilab window, enter \verb+test_medit()+. Enter \verb+test_medit('clean')+ to run the example and to remove old files which have been created for Medit.\\
To run this example in Matlab window, enter \verb+test_medit+. Enter \verb+test_medit('clean')+ to run the example and to remove old files which have been created for Medit.
\subsection{Displaying a mesh :}
\begin{enumerate}
\item create a mesh file and execute Medit from OpenFEM :
\begin{center}  \verb+meditwrite(0,filename,node,elt,[val])+ \end{center}
\item display coloring in Medit (optional) :\\
press mouse right button in Medit window, menu \verb+Data+, \verb+Toggle metric+\\
$ $\\
with : 
\begin{itemize}
\item \verb+0+ : \verb+meditwrite+ indicator, indicates meshes display
\item \verb+filename+ : Medit file name without extension (no \verb+.mesh+)
\item \verb+node+ : node matrix, \verb+elt+ : model description matrix
\item \verb+val+ : data vector for coloring (optional).
\end{itemize}
\end{enumerate}

\subsection{Animate modal deformations}
\begin{enumerate}
\item create a mesh file and execute Medit from OpenFEM :
\begin{center} \verb+meditwrite(1,filename,node,elt,def,dof,data,[opt])+ \end{center}
\item start animation from Medit :\\
press mouse right button in Medit window, menu \verb+Animation+, \verb+Play sequence+\\
$ $\\
with : 
\begin{itemize}
\item \verb+1+ : \verb+meditwrite+ indicator, indicates modal deformations animation
\item \verb+filename+ : Medit files generic name (you'd better create a directory to put created files on, for example : \\ \verb+meditwrite(1,'files_medit/file',node,+\ldots)
\item \verb+node+ : node matrix, \verb+elt+ : model description matrix
\item \verb+def+ : deformation matrix
\item \verb+dof+ : DOF definition vector
\item \verb+data+ : frequency vector
\item \verb+opt+ : options (optional)\\\verb+opt = [num_mode scale nb_def_per_cycle]+ where \verb+num_mode+ is the number of the mode to display, \verb+scale+ is the scale for drawing deformations and \verb+nb_def_per_cycle+ is the number of images per cycle.
\end{itemize}
\end{enumerate}

\subsection{Animate loads or constraints} 
\begin{enumerate}
\item create a mesh file and execute Medit from OpenFEM :
\begin{center} \verb+meditwrite(2,filename,node,elt,def,dof,[data],[opt])+ \end{center}
\item start animation from Medit:\\
press mouse right button in Medit window, menu \verb+Animation+, \verb+Play sequence+\\
$ $\\
with : 
\begin{itemize}
\item \verb+2+ : \verb+meditwrite+ indicator, indicates animation of loads or constraints
\item \verb+filename+ : Medit files generic name (you'd better create a directory to put created files on, for example :\\ \verb+meditwrite(1,'files_medit/file',node,+\ldots)
\item  \verb+node+ : node matrix, \verb+elt+ : model description matrix
\item \verb+def+ : deformation matrix
\item \verb+dof+ : DOF definition vector
\item \verb+data+ : data vector for coloring (optional)
\item \verb+opt+ : options (optional)\\\verb+opt = [num_mode scale nb_def_per_cycle]+ where \verb+num_mode+ is the number of the mode to display, \verb+scale+ is the scale for drawing deformations and \verb+nb_def_per_cycle+ is the number of images per cycle.
\end{itemize}
\end{enumerate}

\end{document}

