%       Copyright (c) 2001-2014 by INRIA and SDTools, All Rights Reserved.
%       Use under OpenFEM trademark.html license and LGPL.txt library license
%       $Revision: 1.35 $  $Date: 2019/02/19 14:27:21 $

%-----------------------------------------------------------------------
\Tchapter{Tutorial}{fe}

NOTE : THIS TUTORIAL HAS NOT BEEN UPDATED IN A LONG TIME IT DOES NOT NECESSARILY REFLECT EXTENSIONS THAT ARE AVAILABLE IN OPENFEM.

This chapter introduces notions needed to use finite element modeling using OpenFEM. \\
All the examples presented are available for the MATLAB and Scilab versions of OpenFEM. When a difference occurs between these two versions, it is clearly reported.\\
Furthermore, all scripts mentioned are contained in the {\tt demos} directory of the distribution.\\

To begin, we explain the typical steps of a finite elements computation below.
In a modal analysis case (see the {\tt demo\_mode} script) :
\begin{itemize}
\item geometry declaration
\item handling material and element properties
\item defining boundary conditions and constraints
\item assembly of mass and stiffness matrices
\item normal modes computing
\item visualization of deformed structures
\end{itemize}

In a static analysis case, the steps are almost the same (see the {\tt demo\_static} script) :
\begin{itemize}
\item geometry declaration
\item handling material and element properties
\item defining boundary conditions and constraints
\item assembly of mass and stiffness matrices
\item loads definition
\item static response computation
\item visualization of deformed structures
\end{itemize}
The above steps will be explained in the following subsections.

All the scripts listed in this tutorial correspond to the MATLAB syntax. They can be run easily under Scilab with simple modifications : comments (\verb+%+) become \verb+\\+, cell-arrays extraction \verb+{...}+ becomes \verb+(...).entries+ and functions calls with no input need brackets (for example, a call to the \verb+fegui+ function (\verb+fegui+ in MATLAB) becomes \verb+fegui()+ in Scilab. Note that the cell-arrays definition needs modifications too : \verb+ca = {v1,v2,v3}+ becomes \verb+ca = makecell([1 3],v1,v2,v3)+ in Scilab. 

All the quoted scripts are in the \verb+demos+ directory of your OpenFEM installation (either MATLAB or Scilab).

\newpage
%-----------------------------------------%
%     Declaring finite element models     %
%-----------------------------------------%
\csection{Declaring finite element models}{fem}

Before assembly, finite element models are described by a data structure with at least five fields (for a full list of possible fields see \ser{model})

\lvs\noindent\begin{tabular}{@{}p{.2\textwidth}@{}p{.8\textwidth}@{}}
 \rz{\tt .Node}     &  \rz\hyperlink{node}{nodes} \\
 \rz{\tt .Elt}      &  \rz\hyperlink{elt}{elements}  \\
 \rz{\tt .pl}       &  \rz\hyperlink{pl}{material properties}  \\
 \rz{\tt .il}       &  \rz\hyperlink{il}{element properties}  \\
 \rz{\tt .Stack }    &  stack of entries containing additional information \hyperlink{stackref}{cases} (boundary conditions, loads, ...), material names, ...  \\
\end{tabular}\lvs


Geometry declarations are described in the sections \ref{s*fetr} ({\sl Direct declaration of geometry}), \ref{s*fesh} ({\sl Geometry declaration with femesh}) and \ref{s*fere} ({\sl Importing models from other codes}).\\
Material and element properties handling is presented in the section \ref{s*femp} and coordinate system handling in section \ref{s*febas}.


Note that, before defining a model, some particular global variables ({\tt FEnode}, {\tt FEel0}, \ldots) need initializations. This initialization must be done by a call to the {\tt fegui} function: {\tt fegui;} in MATLAB or {\tt fegui();} in Scilab.


\newpage
%- - - - - - - - - - - - - - - - - - - -%
%    Direct declaration of geometry     %
%- - - - - - - - - - - - - - - - - - - -% 
\cssection{Direct declaration of geometry}{fetr}\index{two-bay truss}

Hand declaration of a model can only be done for small models and later sections address more complex problems. This example mostly illustrates the form of the model data structure. 

\begin{figure}[H]
\centering
\ingraph{50}{tt_2bay} % [width=8.cm]
 \caption{FE model.}
  \label{fig:tt_2bay}
\end{figure}


The geometry is declared in the {\tt model.Node} matrix (see \ser{node}). In this case, one defines 6 nodes for the truss and an arbitrary reference node to distinguish principal bending axes (see \beam)\index{node}

\begin{verbatim}
 model = struct('Node',[],'Elt',[]);
 %           NodeID  unused   x y z
 model.Node=[ 1      0 0 0    0 1 0;
              2      0 0 0    0 0 0;
              3      0 0 0    1 1 0;
              4      0 0 0    1 0 0;
              5      0 0 0    2 0 0;
              6      0 0 0    2 1 0;
              7      0 0 0    1 1 1]; % reference node
\end{verbatim}

The model description matrix (see \ser{node}) describes 4 longerons, 2 diagonals and 2 battens. These can be declared using three groups of \beam\ elements

\begin{verbatim}
 model.Elt=[
            % declaration of element group for longerons
                Inf     abs('beam1') 
            %node1  node2   MatID ProID nodeR, zeros to fill the matrix 
                1       3      1    1     7       0
                3       6      1    1     7       0
                2       4      1    1     7       0
                4       5      1    1     7       0
             % declaration of element group for diagonals
                Inf     abs('beam1')
                2       3      1    2     7       0
                4       6      1    2     7       0
             % declaration of element group for battens
                Inf     abs('beam1')
                3       4      1    3     7       0
                5       6      1    3     7       0 ];
\end{verbatim}

You may view the declared geometry 

\begin{SDT}
\begin{verbatim}
 cf=feplot; cf.model=model;       % create feplot axes
 fecom(';view2;textnode;triax;'); % manipulate axes 
\end{verbatim}
\end{SDT}

\begin{OPENFEM}
in Scilab version :
\begin{verbatim}
 feplot(model);
\end{verbatim}
in MATLAB version :
\begin{verbatim}
 feplot(model);
 fecom('view2');
\end{verbatim}
\end{OPENFEM}

\begin{center}
\begin{figure}[H]
\centering
\ingraph{75}{demo_fe}
 %\caption{Simulation properties tab.}
 % \label{fig:feplot_fe_simul}
\end{figure}

This is the display result in OpenFEM for MATLAB.
\end{center}

The {\tt demo\_fe\_man} script illustrates uses of this model (part 1, {\sl Direct declaration of geometry}).

\newpage
%- - - - - - - - - - - - - - - - - - - - - %
%     geometry declaration with femesh     %
%- - - - - - - - - - - - - - - - - - - - - %
\cssection{Geometry declaration with femesh}{fesh}

Declaration by hand is clearly not the best way to proceed in general.
\femesh\ provides a number of commands for finite element model creation. The first input argument should be a string containing a single \femesh\ command or a string of chained commands starting by a \ts{;} (parsed by \commode\ which also provides a \femesh\ command mode).


To understand the examples, you should remember that \femesh\ uses the following {\sl standard global variables}\index{FEnode}\index{FEelt}\index{global variable}

\lvs\begin{tabular}{@{}p{.15\textwidth}@{}p{.85\textwidth}@{}}
%
{\tt FEnode} &  main set of nodes\\
{\tt FEn0}   &  selected set of nodes\\
{\tt FEn1}   &  alternate set of nodes\\
{\tt FEelt}  &  main finite element model description matrix\\
{\tt FEel0}  &  selected finite element model description matrix\\
{\tt FEel1}  &  alternate finite element model description matrix\\
%
\end{tabular}\lvs   

Two examples are presented below.
\begin{itemize}
\item \textbf{First example ({\tt demo\_fe} script) :}\\
In the example of the previous section, you could use \femesh\ as follows: initialize, declare the 4 nodes of a single bay by hand, declare the beams of this bay using the \ts{objectbeamline} command

\begin{verbatim}
 FEnode=[1 0 0 0  0 0 0;2 0 0 0    0 1 0;
         3 0 0 0  1 0 0;4 0 0 0    1 1 0];
 femesh('objectbeamline 1 3 0 2 4 0 3 4 0 1 4')
\end{verbatim}

The model of the first bay in is now {\sl selected} (stored in {\tt FEel0}). You can now put it in the main model, translate the selection by 1 in the $x$ direction and add the new selection to the main model

\begin{SDT}
\begin{verbatim}
 femesh(';addsel;transsel 1 0 0;addsel;info');
 % export FEnode and FEelt geometry in model
 model=femesh('model'); 
 cf=feplot; cf.model=model;
 fecom(';view2;textnode;triax;');
\end{verbatim}
\end{SDT}

\begin{OPENFEM}
\begin{verbatim}
 femesh(';addsel;transsel 1 0 0;addsel;info');
 % export FEnode and FEelt geometry in model
 model=femesh('model');  
 feplot(model);
\end{verbatim}
and in Matlab version :
\begin{verbatim}
 fecom('view2');
\end{verbatim}
\end{OPENFEM}
See the {\tt demo\_fe} script, part 1 {\sl Geometry declaration with femesh}.

\item \textbf{Second example ({\tt d\_truss} script) :}\\
You could also build more complex examples. For example, one could remove the second bay, make the diagonals a second group of \bare\ elements, repeat the cell 10 times, rotate the planar truss thus obtained twice to create a 3-D triangular section truss and show the result :

\begin{SDT}
\begin{verbatim}
 femesh('reset');
 femesh('test2bay')
 femesh('removeelt group2');
 femesh('divide group 1 InNode 1 4')
 femesh('set group1 name bar1');
 femesh(';selgroup2 1;repeatsel 10 1 0 0;addsel');
 femesh(';rotatesel 1 60 1 0 0;addsel;')
 femesh(';selgroup3:4;rotatesel 2 -60 1 0 0;addsel;')
 femesh(';selgroup3:8');
 % export FEnode and FEel0 geometry in model
 model=femesh('model0'); 
 cf=feplot; cf.model=model;
 fecom(';triaxon;view3;view y+180;view s-10');
\end{verbatim}
\end{SDT}

\begin{OPENFEM}
\begin{verbatim}
 femesh('reset');
 femesh('test2bay')
 femesh('removeelt group2');
 femesh('divide group 1 InNode 1 4')
 femesh('set group1 name bar1');
 femesh(';selgroup2 1;repeatsel 10 1 0 0;addsel');
 femesh(';rotatesel 1 60 1 0 0;addsel;')
 femesh(';selgroup3:4;rotatesel 2 -60 1 0 0;addsel;')
 femesh(';selgroup3:8');
 % export FEnode and FEel0 geometry in model
 model=femesh('model0');  
 medit('write d_truss',model);
\end{verbatim}
\end{OPENFEM}
See the {\tt d\_truss} script, part 1 {\sl Geometry declaration with femesh}.
\end{itemize}

\begin{center}
%\includegraphics[width=5cm]{plots/d_truss.ps}\\
%\epsfig{file=plots/d_truss.ps,width=5cm}\\
\begin{figure}[H]
\centering
\ingraph{50}{d_truss} % [width=5.cm]
 %\caption{Simulation properties tab.}
 % \label{fig:feplot_fe_simul}
\end{figure}

Visualization of d\_truss example with Medit.
\end{center}

\femesh\ allows many other manipulations (translation, rotation, 
symmetry, extrusion, generation by revolution, refinement by division 
of elements, selection of groups, nodes, elements, edges, etc.) which 
are detailed in the {\sl Reference} section.

\begin{SDT}
Other more complex examples are treated in the following demonstration scripts
{\tt d\_plate}, {\tt beambar}, {\tt d\_ubeam}, {\tt gartfe}.
\end{SDT}

\newpage
\begin{OPENFEM}
%- - - - - - - - - - - - - - - - - - - - - - %
%      Importing models from other codes     %
%- - - - - - - - - - - - - - - - - - - - - - %
\cssection{Importing models from other codes}{fere}\index{importing data}

As interfacing with even only the major finite element codes 
is an enormous and never ending task, such interfaces are always driven
by user demands (and supplies !). In this version the interface distributed with OpenFEM is

\lvs\noindent\begin{tabular}{@{}p{.15\textwidth}@{}p{.85\textwidth}@{}}

{\tt nopo}  & This OpenFEM function reads MODULEF models in binary format.\\

\end{tabular}

For example, you can import the model contained in the {\tt ex3d.nopo} file in the {\tt demos} directory (see the {\tt demo\_nopo} script).
\begin{verbatim}
model = nopo('read -p 3d ex3d');
medit('write ex3d',model);
\end{verbatim}

\begin{center}
\hspace{-0.75cm}
%\includegraphics[width=8cm]{plots/nopo.ps}\\
%\epsfig{file=plots/nopo.ps,width=8cm}\\
\begin{figure}[H]
\centering
\ingraph{70}{nopo} % [width=8.cm]
 %\caption{Simulation properties tab.}
 %\label{fig:feplot_fe_simul}
\end{figure}

Visualization of the {\tt demo\_nopo} example with Medit
\end{center}

Other interfaces with major FEM codes are available (for a fee) at \\\href{http://www.sdtools.com/tofromfem.html}{www.sdtools.com/tofromfem.html}.

\end{OPENFEM}

%-----------------------------------------------------------------------

%-----------------------------------------------------------------------
\csection{FEM problem formulations}{feform}

This section gives a short theoretical reminder of supported FEM problems. The selection of the formulation for each element group is done through the material and element properties.

%-----------------------------------------------------------------------
\cssection{3D elasticity}{feelas3d}



Elements with a {\tt p\_solid} property entry with a non-zero integration rule are described under \psolid. They correspond exactly to the {\tt *b} elements, which are now obsolete. These elements support 3D mechanics (DOFs  {\tt .01} to {\tt .03} at each node) with full anisotropy, geometric non-linearity, integration rule selection, ... The elements have standard limitations. In particular they do not (yet)

\begin{Eitem}
\item have any correction for shear locking found for high aspect ratios
\item have any correction for dilatation locking found for nearly incompressible materials
\end{Eitem}


With \melastic\ subtypes 1 and 3, \psolid\ deals with 3D mechanics with strain defined by
%
\begin{eqsvg}{feform_feelas3d_1}
\ve{\ba{c}\epsilon_x \\\epsilon_y \\\epsilon_z \\\gamma_{yz} \\\gamma_{zx} \\\gamma_{xy} \ea}=
\ma{\ba{cccccc}
 N,x & 0 & 0 \\
 0 & N,y & 0 \\
 0 & 0 & N,z \\
 0 & N,z & N,y \\
 N,z & 0 & N,x \\
 N,y & N,x & 0 \ea}
\ve{\ba{c} u \\ v \\ w \ea}
\end{eqsvg}
%
where the engineering notation $\gamma_{yz}=2\epsilon_{yz}$, ... is used. Stress by
%
\begin{eqsvg}{feform_feelas3d_2}
{\tiny \ve{\!\ba{c}\sigma_x \\\sigma_y \\\sigma_z \\\sigma_{yz} \\\sigma_{zx} \\\sigma_{xy} \!\ea}
\!=\!\ma{\!\ba{cccccc}
 d_{1,1} N,x\!+\!d_{1,5} N,z\!+\!d_{1,6} N,y & d_{1,2} N,y\!+\!d_{1,4} N,z\!+\!d_{1,6} N,x & d_{1,3} N,z\!+\!d_{1,4} N,y\!+\!d_{1,5} N,x \\
 d_{2,1} N,x\!+\!d_{2,5} N,z\!+\!d_{2,6} N,y & d_{2,2} N,y\!+\!d_{2,4} N,z\!+\!d_{2,6} N,x & d_{2,3} N,z\!+\!d_{2,4} N,y\!+\!d_{2,5} N,x \\
 d_{3,1} N,x\!+\!d_{3,5} N,z\!+\!d_{3,6} N,y & d_{3,2} N,y\!+\!d_{3,4} N,z\!+\!d_{3,6} N,x & d_{3,3} N,z\!+\!d_{3,4} N,y\!+\!d_{3,5} N,x \\
 d_{4,1} N,x\!+\!d_{4,5} N,z\!+\!d_{4,6} N,y & d_{4,2} N,y\!+\!d_{4,4} N,z\!+\!d_{4,6} N,x & d_{4,3} N,z\!+\!d_{4,4} N,y\!+\!d_{4,5} N,x \\
 d_{5,1} N,x\!+\!d_{5,5} N,z\!+\!d_{5,6} N,y & d_{5,2} N,y\!+\!d_{5,4} N,z\!+\!d_{5,6} N,x & d_{5,3} N,z\!+\!d_{5,4} N,y\!+\!d_{5,5} N,x \\
 d_{6,1} N,x\!+\!d_{6,5} N,z\!+\!d_{6,6} N,y & d_{6,2} N,y\!+\!d_{6,4} N,z\!+\!d_{6,6} N,x & d_{6,3} N,z\!+\!d_{6,4} N,y\!+\!d_{6,5} N,x \ea}
\ve{\ba{c} u \\ v \\ w \ea\!}}
\end{eqsvg}
%

Note that the strain states are $\ve{\epsilon_x \ \ \epsilon_y \ \ \epsilon_z \ \ \gamma_{yz} \ \ \gamma_{zx} \ \ \gamma_{xy}}$ which may not be the convention of other software. 

Note that NASTRAN, SAMCEF, ANSYS and MODULEF order shear stresses with $\sigma_{xy}, \sigma_{yz}, \sigma_{zx}$ (MODULEF elements are obtained by setting \psolid\ {\tt integ} value to zero).  Abaqus uses $\sigma_{xy}, \sigma_{xz}, \sigma_{yz}$ % see Isotropic elasticity in Abaqus Analysis User's manual

In \festress\ the stress reordering can be accounted for by the definition of the proper {\tt TensorTopology} matrix.

For isotropic materials

\begin{eqsvg}{feform_feelas3d_3}
D=\ma{\ba{cc}
 \frac{E(1-\nu)}{(1+\nu)(1-2\nu)}
  \ma{\ba{ccc}1 & \frac{\nu}{1-\nu} & \frac{\nu}{1-\nu} \\ 
 \frac{\nu}{1-\nu} & 1  & \frac{\nu}{1-\nu} \\ 
 \frac{\nu}{1-\nu} & \frac{\nu}{1-\nu} & 1 \ea}
 & 0 \\
 0 & \ma{\ba{ccc} G & 0 & 0\\ 0 & G & 0\\0 & 0 & G\ea}\ea}
\end{eqsvg}

with at nominal $G=E/(2(1+\nu))$. For isotropic materials, interpolation of $\rho,\eta,E,\nu,G,\alpha$ with temperature is supported. 

For orthotropic materials, the compliance is given by 

\begin{eqsvg}{feform_feelas3d_4}
\ve{\epsilon} = \ma{D}^{-1}\ve{\sigma}=\ma{\ba{cccccc}
1/E_1 & -\frac{\nu_{21}}{E_2} &  -\frac{\nu_{31}}{E_3} & 0 & 0 & 0 \\
-\frac{\nu_{12}}{E_1} & 1/E_2 &  -\frac{\nu_{32}}{E_3} & 0 & 0 & 0 \\
-\frac{\nu_{13}}{E_1} &  -\frac{\nu_{23}}{E_2} & 1/E_3 & 0 & 0 & 0 \\
0 & 0 & 0 & \frac{1}{G_{23}} & 0 & 0\\
0 & 0 & 0 & 0 & \frac{1}{G_{31}} \\
0 & 0 & 0 & 0 & 0 & \frac{1}{G_{12}} \ea}
{\tiny \ve{\!\ba{c}\sigma_x \\\sigma_y \\\sigma_z \\\sigma_{yz} \\\sigma_{zx} \\\sigma_{xy} \!\ea}}
\end{eqsvg}


For constitutive law building, see \psolid. Material orientation can be interpolated by defining {\tt v1 and v2} fields in the {\tt InfoAtNode}. Interpolation of non isotropic material properties was only implemented for {\tt of\_mk} >= 1.236.


%-----------------------------------------------------------------------
\cssection{2D elasticity}{feelas2d}



With \melastic\ subtype 4, \psolid\ deals with 2D mechanical volumes with strain defined by (see {\tt q4p constants})

\begin{eqsvg}{feform_feelas2d_1}
\ve{\ba{c}\epsilon_x \\\epsilon_y \\\gamma_{xy} \ea}
=\ma{\ba{ccc}
 N,x & 0 \\
 0 & N,y \\
 N,y & N,x \ea}
\ve{\ba{c} u \\ v \ea}
\end{eqsvg}
%
and stress by
%
\begin{eqsvg}{feform_feelas2d_2}
\ve{\ba{c}\sigma \epsilon_x \\\sigma \epsilon_y \\\sigma \gamma_{xy} \ea}
=\ma{\ba{ccc}
 d_{1,1} N,x+d_{1,3} N,y & d_{1,2} N,y+d_{1,3} N,x \\
 d_{2,1} N,x+d_{2,3} N,y & d_{2,2} N,y+d_{2,3} N,x \\
 d_{3,1} N,x+d_{3,3} N,y & d_{3,2} N,y+d_{3,3} N,x \ea}
\ve{\ba{c} u \\ v \ea}
\end{eqsvg}

For isotropic plane stress (\psolid\ {\tt form=1}), one has

\begin{eqsvg}{feform_feelas2d_3}
D=\frac{E}{1-\nu^2}\ma{\ba{ccc}
 1 & \nu & 0\\ \nu & 1 & 0\\ 0 &0 &\frac{1-\nu}{2} \ea}
\end{eqsvg}

For isotropic plane strain (\psolid\ {\tt form=0}), one has

\begin{eqsvg}{feform_feelas2d_4}
D=\frac{E(1-\nu}{(1+\nu)(1-2\nu)}\ma{\ba{ccc}
 1 & \frac{\nu}{1-\nu} & 0\\ \frac{\nu}{1-\nu} & 1 & 0\\ 0 & 0 & \frac{1-2\nu}{2(1-\nu)} \ea}
\end{eqsvg}



%-----------------------------------------------------------------------
\cssection{Acoustics}{feacoustics}

With \melastic\ subtype 2, \psolid\ deals with 2D and 3D acoustics (see {\tt flui4 constants}) where 3D strain is given by
%
\begin{eqsvg}{feform_feacoustics_1}
\ve{\ba{c}p,x \\p,y \\p,z \ea}
=\ma{\ba{ccc}
 N,x \\
 N,y \\
 N,z \ea}
\ve{\ba{c} p \ea}
\end{eqsvg}

This replaces the earlier {\tt flui4} ... elements.

The mass and stiffness matrices are given by
%
\begin{eqsvg}{feform_feacoustics_2}
 M_{ij}=\int_{\Omega}\frac{1}{\rho_0C^2}\ve{N_i}\ve{N_j}
\end{eqsvg}

\begin{eqsvg}{feform_feacoustics_3}
 K_{ij}=\int_{\Omega}\frac{1}{\rho_0}\ve{N_{i,k}}\ve{N_{j,k}}
\end{eqsvg}

The source associated with a enforced velocity on a surface
%
\begin{eqsvg}{feform_feacoustics_4}
 B_{i}=\int_{\partial \Omega}\ve{N_{i}}\ve{V_e}
\end{eqsvg}

When an impedance $Z=\rho C R(1+i\eta)$ is considered on a surface,
the associated viscous damping matrix is given by
%
\begin{eqsvg}{feform_feacoustics_5}
 C_{ij}=\int_{\partial \Omega_Z^e}\frac{1}{Z}\ve{N_i}\ve{N_j}
\end{eqsvg}


\begin{SDT}

%-----------------------------------------------------------------------
\cssection{Classical lamination theory}{feshell}
Both isotropic and orthotropic materials are considered. In these cases, the general form of the 3D elastic material law is
 
\begin{eqsvg}{isotropiclaw}
 \left\{ \begin{array}{c}
\sigma_{11}\\ \sigma_{22}\\ \sigma_{33}\\ \tau_{23}\\ \tau_{13}\\ \tau_{12}\\ \end{array} \right\} = 
 \left[ \begin{array}{cccccc}
C_{11} & C_{12} & C_{13}  &   0   &   0   &  0 \\
       & C_{22} & C_{23}  &   0   &   0   &  0 \\
                 &        & C_{33}      &   0   &   0   &  0 \\
                   &        &                           & C_{44}&   0   &  0 \\
             &  (s)             &             &       & C_{55}&  0 \\
                 &                              &                         &       &       & C_{66}\\ 
  \end{array} \right]
  \left\{ \begin{array}{c} \epsilon_{11}\\ \epsilon_{22}\\ \epsilon_{33}\\
\gamma_{23}\\ \gamma_{13}\\ \gamma_{12}\\ \end{array} \right\}
\end{eqsvg}

Plate formulation consists in assuming one dimension, the thickness along $x_3$, negligible compared with the surface dimensions. Thus, vertical stress $\sigma_{33}=0$ on the bottom and upper faces, and assumed to be neglected throughout the thickness,

\begin{eqsvg}{simple_eps33}
\sigma_{33}=0 \Rightarrow \epsilon_{33}=-\frac{1}{C_{33}}\left(C_{13}\epsilon_{11}+C_{23}\epsilon_{22}\right),
\end{eqsvg}
and for isotropic material,
\begin{eqsvg}{simple_eps33_isotropic}
\sigma_{33}=0 \Rightarrow \epsilon_{33}=-\frac{\nu}{1-\nu}\left(\epsilon_{11}+\epsilon_{22}\right).
\end{eqsvg}


By eliminating $\sigma_{33}$, the plate constitutive law is written, with engineering notations,

\begin{eqsvg}{PlateConstitutivelaw}
\left\{ \begin{array}{c}
\sigma_{11}\\  \sigma_{22}\\ \sigma_{12}\\ \sigma_{23}\\ \sigma_{13}\\ \end{array} \right\} =  \left[ 
\begin{array}{ccccc}
Q_{11}           & Q_{12}  & 0                  &0              & 0 \\
Q_{12}           & Q_{22}  & 0                  &0              & 0 \\
        0          & 0                   & Q_{66}       &0              & 0 \\
 0                                      & 0                      &0             & Q_{44} &0 \\
 0                                      & 0                      &0             & 0     &Q_{55} \\
  \end{array} 
 \right]
\left\{ \begin{array}{c}
\epsilon_{11}\\ \epsilon_{22}\\ \gamma_{12}\\ \gamma_{23}\\ \gamma_{13}\\ \end{array} \right\}.
\end{eqsvg}


The reduced stiffness coefficients $Q_{ij}$ (i,j = 1,2,4,5,6) are related to the 3D stiffness coefficients $C_{ij}$ by

\begin{eqsvg}{3D_planestrain}
Q_{ij}=\left\{ \begin{array}{ll}
C_{ij}-\displaystyle \frac{C_{i3}C_{j3}}{C_{33}}& \mbox{ if i,j=1,2,}\\
C_{ij}& \mbox{ if i,j=4,5,6.}\\
\end{array}\right.
\end{eqsvg}



The reduced elastic law for an isotropic plate becomes,
 
\begin{eqsvg}{isotropiclaw2}
 \small{ \left\{ \begin{array}{c}
\sigma_{11}\\
\sigma_{22}\\
\tau_{12}\\
 \end{array} \right\} = \frac{E}{(1-\nu^2)}
 \left[ \begin{array}{ccc}
1   & \nu & 0 \\
\nu &  1  & 0 \\
 0      & 0             &\frac{1-\nu}{2}\\
 \end{array} \right]
  \left\{ \begin{array}{c}
\epsilon_{11}\\
\epsilon_{22}\\
\gamma_{12}\\
 \end{array} \right\}},
\end{eqsvg}

and
 
\begin{eqsvg}{isotropiclaw3}
 \small{ \left\{ \begin{array}{c}
\tau_{23}\\
\tau_{13}\\
 \end{array} \right\} = \frac{E}{2(1+\nu)}
 \left[ \begin{array}{cc}
 1  &0 \\
0 &1 \\
  \end{array} \right]
  \left\{ \begin{array}{c}
 \gamma_{23}\\
\gamma_{13}\\
  \end{array} \right\}}.
\end{eqsvg}

Under Reissner-Mindlin's kinematic assumption the linearized strain tensor is

\begin{eqsvg}{strain}
\epsilon=
 \left[ \begin{array}{ccc}
u_{1,1}+x_3\beta_{1,1}&\frac{1}{2}(u_{1,2}+u_{2,1}+x_3(\beta_{1,2}+\beta_{2,1}))& \frac{1}{2}(\beta_1+w_{,1})\\
& u_{2,2}+x_3\beta_{2,2}&\frac{1}{2}(\beta_2+w_{,2})\\
(s)&&0 \\
  \end{array} \right].
\end{eqsvg}

So, the strain vector is written,

\begin{eqsvg}{strain_vector}
\left\{\epsilon\right\}=\left\{\begin{array}{c}
\epsilon^m_{11}+x_3\kappa_{11}\\
\epsilon^m_{22}+x_3\kappa_{22}\\
\gamma^m_{12}+x_3\kappa_{12}\\
\gamma_{23}\\
\gamma_{13}\\
 \end{array}\right\},
\end{eqsvg}
%
with $\epsilon^m$ the membrane, $\kappa$ the curvature or bending, and $\gamma$ the shear strains,
%
\begin{eqsvg}{strains}
\epsilon^m=\left\{ \begin{array}{c}
u_{1,1}\\
u_{2,2}\\
u_{1,2}+u_{2,1}\\
 \end{array} \right\},\ 
\kappa=\left\{ \begin{array}{c}
\beta_{1,1}\\
\beta_{2,2}\\
\beta_{1,2}+\beta_{2,1}\\
\end{array} \right\},\ 
\gamma=\left\{\begin{array}{c}
\beta_{2}+w_{,2}\\
\beta_{1}+w_{,1}\\
\end{array} \right\},
\end{eqsvg}\\

Note that the engineering notation with  $\gamma_{12}=u_{1,2}+u_{2,1}$ is used here rather than the tensor notation with $\epsilon_{12}=(u_{1,2}+u_{2,1})/2$ . Similarly $\kappa_{12}=\beta_{1,2}+\beta_{2,1}$, where a factor $1/2$ would be needed for the tensor.



The plate formulation links the stress resultants, membrane forces $N_{\alpha\beta}$, bending moments $M_{\alpha\beta}$ and shear forces $Q_{\alpha3}$, to the strains, membrane $\epsilon^m$, bending $\kappa$ and shearing $\gamma$,
\begin{eqsvg}{feform_feshell_1}
%\eql{ReissnerMindlinPlatelaw}
 \small{ \left\{ \begin{array}{c}
N\\
M\\
Q\\
 \end{array} \right\} =  \left[ \begin{array}{ccc}
A&B&0 \\
B&D&0 \\
0&0&F \\
\end{array} \right]
\left\{ \begin{array}{c}
\epsilon^m\\
\kappa\\
\gamma\\
\end{array} \right\}}.
\end{eqsvg}

The stress resultants are obtained by integrating the stresses through the thickness of the plate,
\begin{eqsvg}{Forces}
N_{\alpha \beta}=\displaystyle\int^{ht}_{hb}\sigma_{\alpha \beta}\:dx_3,\ \ 
M_{\alpha \beta}=\int^{ht}_{hb}x_3\:\sigma_{\alpha \beta}\:dx_3,\ \ 
Q_{\alpha 3}=\int^{ht}_{hb}\sigma_{\alpha 3}\:dx_3,
\end{eqsvg}\\ 
 
\noindent with $\alpha, \beta = 1, 2$. 


Therefore, the matrix extensional stiffness matrix $\left[A\right]$, extension/bending coupling matrix $\left[B\right]$, and the bending stiffness matrix $\left[D\right]$ are calculated by integration over the thickness interval $\ma{hb\ \ ht}$

\begin{eqsvg}{ABDF_1layer}
\begin{array}{cc}
A_{ij}=\displaystyle\int^{ht}_{hb}Q_{ij}\:dx_3,&
B_{ij}=\displaystyle\int^{ht}_{hb}x_3\:Q_{ij}\:dx_3,\\
&\\
D_{ij}=\displaystyle\int^{ht}_{hb}x^2_3\:Q_{ij}\:dx_3,&
F_{ij}=\displaystyle\int^{ht}_{hb}Q_{ij}\:dx_3. 
\end{array}
 \end{eqsvg}

An improvement of Mindlin's plate theory with transverse shear consists in modifying the shear coefficients $F_{ij}$ by
\begin{eqsvg}{correction_factor} 
H_{ij}=k_{ij}F_{ij},\end{eqsvg} 
where $k_{ij}$ are correction factors. Reddy's $3^{rd}$ order theory brings to $k_{ij}=\frac{2}{3}$. Very commonly, enriched $3^{rd}$ order theory are used, and $k_{ij}$ are equal to $\frac{5}{6}$ and give good results. For more details on the assessment of the correction factor, see~\cite{ber11}.\\

For an isotropic symmetric plate ($hb=-ht=h/2$), the in-plane normal forces $N_{11}$, $N_{22}$ and shear force $N_{12}$ become  

\begin{eqsvg}{N}
\left\{ \begin{array}{c}
N_{11}\\
N_{22}\\
N_{12}\\
\end{array} \right\} = \frac{Eh}{1-\nu^2}\left[\begin{array}{ccc}
1&\nu&0\\
&1&0\\
(s)&&\frac{1-\nu}{2}\\
\end{array}  \right]\left\{ \begin{array}{c}
u_{1,1}\\
u_{2,2}\\
u_{1,2}+u_{2,1}\\
\end{array} \right\},
\end{eqsvg}
%
the 2 bending moments $M_{11}$, $M_{22}$ and twisting moment $M_{12}$
%
\begin{eqsvg}{M}
\left\{ \begin{array}{c}
M_{11}\\
M_{22}\\
M_{12}\\
\end{array} \right\} = \frac{Eh^3}{12(1-\nu^2)}\left[\begin{array}{ccc}
1&\nu&0\\
&1&0\\
(s)&&\frac{1-\nu}{2}\\
\end{array}  \right]\left\{ \begin{array}{c}
\beta_{1,1}\\
\beta_{2,2}\\
\beta_{1,2}+\beta_{2,1}\\
\end{array} \right\},
\end{eqsvg}
%
and the out-of-plane shearing forces $Q_{23}$ and $Q_{13}$, 
%
\begin{eqsvg}{Q}
\left\{ \begin{array}{c}
Q_{23}\\
Q_{13}\\
\end{array} \right\} = \frac{Eh}{2(1+\nu)}\left[\begin{array}{cc}
1&0\\
0&1\\
\end{array}  \right]\left\{ \begin{array}{c}
\beta_{2}+w_{,2}\\
\beta_{1}+w_{,1}\\
\end{array} \right\}.
\end{eqsvg}

One can notice that because the symmetry of plate, that means the reference plane is the mid-plane of the plate ($x_3(0)=0$) the extension/bending coupling matrix $\left[B\right]$ is equal to zero. 

Using expression~\eqr{ABDF_1layer} for a constant $Q_{ij}$, one sees that for a non-zero offset, one has
%
\begin{eqsvg}{B_offset}
A_{ij}=h\ma{Q_{ij}}\ \ \ \ B_{ij}=x_3(0)h \ma{Q_{ij}} \ \ \ \ C_{ij}= (x_3(0)^2h+h^3/12) \ma{Q_{ij}} \ \ \ \ F_{ij}=h\ma{Q_{ij}}
\end{eqsvg}
%
where is clearly appears that the constitutive matrix is a polynomial function of $h$, $h^3$, $x_3(0)^2h$ and $x_3(0)h$. If the ply thickness is kept constant, the constitutive law is a polynomial function of $1,x_3(0),x_3(0)^2$.


%-----------------------------------------------------------------------
\cssection{Piezo-electric volumes}{fepiezo}

{\bf A revised version of this information is available at \url{http://www.sdtools.com/pdf/piezo.pdf}.  Missing PDF links will be found there\label{s*pz_ce3D}\label{s*pz_volume}\label{s*pz_ide}\label{s*fepiezoshell}\label{s*pztheory}\label{s*pz_tuto}}. 

The strain state associated with piezoelectric materials is described by the six classical mechanical strain components and the electrical field components. Following the IEEE standards on piezoelectricity and using matrix notations, $S$ denotes the strain vector and $E$ denotes the electric field vector ($V/m$) :

\begin{eqsvg}{feform_fepiezo_1}
\ve{\ba{c} S \\ E \ea} =
\ve{\ba{c}\epsilon_x \\\epsilon_y \\\epsilon_z \\\gamma_{yz} \\\gamma_{zx} \\\gamma_{xy} \\E_x \\E_y \\E_z \ea}
=\ma{\ba{ccccccccc}
 N,x & 0 & 0 & 0 \\
 0 & N,y & 0 & 0 \\
 0 & 0 & N,z & 0 \\
 0 & N,z & N,y & 0 \\
 N,z & 0 & N,x & 0 \\
 N,y & N,x & 0 & 0 \\
 0 & 0 & 0 & -N,x \\
 0 & 0 & 0 & -N,y \\
 0 & 0 & 0 & -N,z \ea}
\ve{\ba{c} u \\ v \\ w \\ \phi \ea}
\end{eqsvg}
where $\phi$ is the electric potential ($V$).

The constitutive law associated with this strain state is given by
%
\begin{eqsvg}{pz1}
\ve{\ba{c} T \\ D \ea} = \ma{\ba{cc} C^E & e^T \\ e &
-\varepsilon^S \ea}\ve{\ba{c} S \\ -E \ea}
\end{eqsvg}
in which $D$ is the electrical displacement vector (a density of charge in $Cb/m^2$), $T$ is the mechanical stress vector ($N/m^2$). $C^E$ is the matrix of elastic constants at zero electric field ($E=0$, short-circuited condition, see~\ser{feelas3d} for formulas (there $C^E$ is noted $D$). Note that using $-E$ rather than $E$ makes the constitutive law symmetric.


Alternatively, one can use the constitutive equations written in the following manner~:
\begin{eqsvg}{pz2}
\ve{\ba{c} S \\ D \ea} = \ma{\ba{cc} s^E & d^T \\ d &
\varepsilon^T \ea}\ve{\ba{c} T \\ E \ea}
\end{eqsvg}
In which $s^E$ is the matrix of mechanical compliances, $\ma{d}$
is the matrix of piezoelectric constants ($m/V=Cb/N$):
\begin{eqsvg}{feform_fepiezo_2}
 \ma{d} = \ma{\ba{cccccc} d_{11} & d_{12} & d_{13} & d_{14} & d_{15} & d_{16} \\
 d_{21} & d_{22} & d_{23} & d_{24} & d_{25} & d_{26} \\
 d_{31} & d_{32} & d_{33} & d_{34} & d_{35} & d_{36} \ea }
 \end{eqsvg}

Matrices $\ma{e}$ and $\ma{d}$ are related through
\begin{eqsvg}{pz3}
 \ma{e} = \ma{d} \ma{ C^E}
 \end{eqsvg}

 Due to crystal symmetries, $\ma{d}$ may have only a few non-zero elements.

Matrix $\ma{\varepsilon^S}$ is the matrix of dielectric constants
(permittivities) under zero strain (constant volume) given by
\begin{eqsvg}{feform_fepiezo_3}
 \ma{\varepsilon^S} = \ma{\ba{ccc} \varepsilon_{11}^S & \varepsilon_{12}^S & \varepsilon_{13}^S \\
 \varepsilon_{21}^S & \varepsilon_{22}^S & \varepsilon_{23}^S \\
 \varepsilon_{31}^S & \varepsilon_{32}^S & \varepsilon_{33}^S  \ea }
 \end{eqsvg}

It is more usual to find the value of $\varepsilon^T$
(Permittivity at zero stress) in the datasheet. These two values
are related through the following relationship :

\begin{eqsvg}{feform_fepiezo_4}
\ma{\varepsilon^S}= \ma{\varepsilon^T} - \ma{d} \ma{e}^T
\end{eqsvg}

For this reason, the input value for the computation should be
$\ma{\varepsilon^T}$. \\

Also notice that usually relative permittivities are given in datasheets:
\begin{eqsvg}{feform_fepiezo_5}
\varepsilon_r = \frac{\varepsilon}{\varepsilon_0}
\end{eqsvg}
$\varepsilon_0$ is the permittivity of vacuum (=8.854e-12 F/m)

The most widely used piezoelectric materials are PVDF and PZT. For
both of these, matrix $\ma{\varepsilon^T}$ takes the form
\begin{eqsvg}{feform_fepiezo_6}
 \ma{\varepsilon^T} = \ma{\ba{ccc} \varepsilon_{11}^T & 0 & 0 \\
 0 & \varepsilon_{22}^T & 0 \\
 0 & 0 & \varepsilon_{33}^T \ea }
 \end{eqsvg}

 For PVDF, the matrix of piezoelectric constants is given by

\begin{eqsvg}{feform_fepiezo_7}
 \ma{d} = \ma{\ba{cccccc}0 & 0 & 0 & 0 & 0 & 0 \\
 0 & 0 & 0 & 0 & 0 & 0 \\
 d_{31} & d_{32} & d_{33} & 0 & 0 & 0 \ea }
 \end{eqsvg}

and for PZT materials :

\begin{eqsvg}{feform_fepiezo_8}
 \ma{d} = \ma{\ba{cccccc}0 & 0 & 0 & 0 & d_{15} & 0 \\
 0 & 0 & 0 & d_{24} & 0 & 0 \\
 d_{31} & d_{32} & d_{33} & 0 & 0 & 0 \ea }
 \end{eqsvg}


%-----------------------------------------------------------------------
\cssection{Piezo-electric shells}{fepiezos}

{\bf A revised version of this information is available at \url{http://www.sdtools.com/pdf/piezo.pdf}}. 

 
Shell strain is defined by the membrane, curvature and transverse shear as well as the electric field components. It is assumed that
in each piezoelectric layer $i=1...n$, the electric field takes
the form $\vec{E}= (0 \quad 0 \quad E_{zi})$. $E_{zi}$ is assumed
to be constant over the thickness $h_i$ of the layer and is
therefore given by $E_{zi}=-\frac{\Delta \phi_i}{h_i}$ where
$\Delta \phi_i$ is the difference of potential between the
electrodes at the top and bottom of the piezoelectric layer $i$.
It is also assumed that the piezoelectric principal axes are
parallel to the structural orthotropy axes.

\begin{figure}[H]
\centering
\ingraph{60}{piezo_shell}
\end{figure}

The strain state of a piezoelectric shell takes the form

\begin{eqsvg}{feform_fepiezos_1}
\ve{\ba{c}\epsilon_{xx} \\\epsilon_{yy} \\ 2 \epsilon_{xy} \\
\kappa_{xx} \\\kappa_{yy} \\ 2 \kappa_{xy} \\ \gamma_{xz} \\
\gamma_{yz}\\ -E_{z1} \\ ... \\ -E_{zn}  \ea}=\ma{\ba{cccccccc}
 N,x & 0 & 0 & 0 & 0 & 0 & ... & 0 \\
 0 & N,y & 0 & 0 & 0 & 0 & ... & 0\\
 N,y & N,x & 0 & 0 & 0 & 0 & ... & 0 \\
 0 & 0 & 0 & 0 & -N,x & 0 & ... & 0 \\
 0 & 0 & 0 & N,y & 0 & 0 & ... & 0\\
 0 & 0 & 0 & N,x & -N,y & 0 & ... & 0 \\
 0 & 0 & N,x & 0 & N & 0 & ... & 0 \\
 0 & 0 & N,y & -N & 0 & 0 & ... & 0 \\
0 & 0 & 0 & 0 & 0 & -\frac{1}{h_1} & ... & 0 \\
... & ... & ... & ... & ... & 0 & ... & -\frac{1}{h_n} \\
  \ea}
\ve{\ba{c} u \\ v \\ w \\ ru \\ rw \\ \Delta \phi_1 \\ ... \\
\Delta \phi_n \ea}
\end{eqsvg}

There are thus $n$ additional degrees of freedom $\Delta \phi_i$,
$n$ being the number of piezoelectric layers in the laminate shell

The constitutive law associated to this strain state is given by :

\begin{eqsvg}{pzs1}
\ve{\ba{c}N \\ M \\ Q \\ D_{z1} \\ ... \\ D_{zn} \ea} = \ma{\ba{cccccc} A & B & 0 & G_1^T & ... & G_n^T \\
B & D & 0 & z_{m1} G_1^T & ... & z_{mn} G_n^T \\
0 & 0 & F & H_1^T & ... & H_n^T \\
G_1 & z_{m1} G_1 & H_1 & -{\varepsilon_1} & ... & 0 \\
... & ... & ... & 0 & ... & 0 \\
G_n & z_{mn} G_n & H_n & 0 & ... & -{\varepsilon_n}
\ea} \ve{\ba{c}\epsilon \\
\kappa
\\ \gamma \\-E_{z1} \\ ... \\ -E_{zn}  \ea}
\end{eqsvg}

where $D_{zi}$ is the electric displacement in piezoelectric layer
(assumed constant and in the $z$-direction), $z_{mi}$ is the
distance between the midplane of the shell and the midplane of
piezoelectric layer $i$, and $G_i, H_i$ are given by
\begin{eqsvg}{feform_fepiezos_2}
G_i = \ve{\ba{ccc} e_{.1} & e_{.2} & 0 \ea}_i [R_s]_i
\end{eqsvg}
\begin{eqsvg}{feform_fepiezos_3}
H_i = \ve{\ba{cc} e_{.4} & e_{.5} \ea}_i [R]_i
\end{eqsvg}
where $.$ denotes the direction of polarization. If the piezoelectric is
used in extension mode, the polarization is in the $z$-direction,
therefore $H_i =0$ and $G_i =\ve{\ba{ccc} e_{31} & e_{32} & 0
\ea}_i$ . If the piezoelectric is used in shear mode, the
polarization is in the $x$ or $y$-direction, therefore $G_i=0$,
and $H_i = \ve{0 \ e_{15} }_i$ or $H_i = \ve{e_{24} \ 0 }_i$ . It turns out however that the hypothesis of a uniform transverse shear strain distribution through the thickness is not satisfactory, a more elaborate shell element would be necessary. Shear actuation should therefore be used with caution.

$[R_s]_i$ and $[R]_i$ are rotation matrices associated to the
angle $\theta$ of the piezoelectric layer.

\begin{eqsvg}{feform_fepiezos_4}
[R_s] = \ma{\ba{ccc} \cos^2 \theta & \sin^2 \theta &\sin \theta
\cos \theta \\ \sin^2 \theta & \cos^2 \theta & - \sin \theta \cos
\theta \\
-2 \sin \theta \cos \theta & 2 \sin \theta \cos \theta & \cos^2
\theta -  \sin^2 \theta \ea}
\end{eqsvg}

\begin{eqsvg}{feform_fepiezos_5}
[R] = \ma{\ba{cc} \cos \theta & -\sin \theta \\
\sin \theta & \cos \theta \ea}
\end{eqsvg}


\end{SDT}

%-----------------------------------------------------------------------
\cssection{Geometric non-linearity}{fe3dnl}

The following gives the theory of large transformation problem implemented in OpenFEM function {\tt of\_mk\_pre.c Mecha3DInteg}.\\

The principle of virtual work in non-linear total Lagrangian formulation for an hyperelastic medium is 
%
\begin{eqsvg}{3dnl_a}
\int_{\Omega_0} (\rho_0 u'', \delta v) + \int_{\Omega_0} S : \delta e = \int_{\Omega_0} f . \delta v \ \ \forall \delta v
\end{eqsvg}
%
with  $p$ the vector of initial position, $x = p +u$ the current
position, and $u$ the displacement vector. The transformation is characterized by
%
\begin{eqsvg}{3dnl_b}
 F_{i,j} = I + u_{i,j} = \delta_{ij}+\ve{N_{,j}}^T\ve{q_i}
\end{eqsvg}
%
where the $N,j$ is the derivative of the shape functions with respect to Cartesian coordinates at the current integration point and $q_i$ corresponds to field $i$ (here translations) and element nodes. The notation is thus really valid within a single element and corresponds to the actual implementation of the element family in {\tt elem0} and {\tt of\_mk}. Note that in these functions, a reindexing vector is used to go from engineering ($\ve{e_{11}\ e_{22}\ e_{33}\ 2e_{23}\ 2e_{31}\ 2e_{12}}$) to tensor $\ma{e_{ij}}$ notations {\tt ind\_ts\_eg=[1 6 5;6 2 4;5 4 3];e\_tensor=e\_engineering(ind\_ts\_eg);}. One can also simplify a number of computations using the fact that the contraction of a symmetric and non symmetric tensor is equal to the contraction of the symmetric tensor by the symmetric part of the non symmetric tensor.

One defines the Green-Lagrange strain tensor $e=1/2(F^TF -I)$ and its variation
%
\begin{eqsvg}{3dnl_c}
 de_{ij} = \br{F^T dF}_{Sym} = \br{F_{ki} \ve{N_{,j}}^T\ve{q_k}}_{Sym}
\end{eqsvg}


Thus the virtual work of internal loads (which corresponds to the residual in non-linear iterations) is given by
%
\begin{eqsvg}{3dnl_d}
\int_{\Omega} S : \delta e = \int_{\Omega} \ve{\delta q_k}^T\ve{N_{,j}} F_{ki} S_{ij}
\end{eqsvg}
%
and the tangent stiffness matrix (its derivative with respect to the current position) can be written as 
%  
\begin{eqsvg}{3dnl_e}
K_{G}=\int_{\Omega} S_{ij} {\delta u}_{k,i} u_{l,j} + \int_{\Omega} de : \frac{\partial^2 W}{\partial e^2} : \delta e
\end{eqsvg}
% 
which using the notation $u_{i,j} = \ve{N_{,j}}^T\ve{q_i}$ leads to
%
\begin{eqsvg}{3dnl_f}
K_{G}^e=\int_{\Omega} \ve{\delta q_m} \ve{N_{,l}} \br{F_{mk}
\frac{\partial^2 W}{\partial e^2}_{ijkl} F_{ni} + S_{lj}} \ve{N_{,j}} \ve{dq_n}
\end{eqsvg}

The term associated with stress at the current point is generally called geometric stiffness or pre-stress contribution. 
For implementation, the variable names are {\tt d2wde2}, {\tt Sigma} and the large displacement computation $\br{F_{mk}
\frac{\partial^2 W}{\partial e^2}_{ijkl} F_{ni} + S_{lj}}$ has a reference implementation in {\tt elem0('LdDD')}. The result is called {\tt dd} in the code.   


In isotropic elasticity, the 2nd tensor of Piola-Kirchhoff stress is given by
%
\begin{eqsvg}{3dnl_g}
        S = D:e(u) = \frac{\partial^2 W}{\partial e^2}:e(u) =  \lambda Tr(e) I + 2\mu e 
\end{eqsvg}
%
the building of the constitutive law matrix $D$ is performed in \psolid\ \ts{BuildConstit} for isotropic, orthotropic and full anisotropic materials. {\tt of\_mk\_pre.c nonlin\_elas} then implements element level computations. For hyperelastic materials $\frac{\partial^2 W}{\partial e^2}$ is not constant and is computed at each integration point as implemented in {\tt hyper.c}.

For a geometric non-linear static computation, a Newton solver will thus iterate with

\begin{eqsvg}{3dnl_h}
 \ma{K(q^n)}\ve{q^{n+1}-q^{n}} =   R(q^n) = \int_{\Omega} f . dv - 
\int_{\Omega_0} S(q^n) : \delta e 
\end{eqsvg}
%
where external forces $f$ are assumed to be non following.

For an example see \lts{fe\_time}{staticNewton}.

%-----------------------------------------------------------------------
\cssection{Thermal pre-stress}{fe3dtherm}

Note that more recent developments are found in {\tt SDT-nlsim}, see {\tt sdtweb('hyper3D')}. The following gives the theory of the thermoelastic problem implemented in  OpenFEM function {\tt of\_mk\_pre.c nonlin\_elas}.\\

In presence of a temperature difference, the thermal strain is given by $\ma{e_T} = \ma{\alpha} (T-T_0)$, where in general the thermal expansion matrix $\alpha$ is proportional to identity (isotropic expansion). The stress is found by computing the contribution of the mechanical deformation 
%
\begin{eqsvg}{3dnl_i}
  S = C:(e - e_T) =  \lambda Tr(e) I + 2\mu e - (C:\ma{\alpha})(T-T_0) 
\end{eqsvg}

This expression of the stress is then used in the equilibrium~\eqr{3dnl_a}, the tangent matrix computation\eqr{3dnl_e}, or the Newton iteration~\eqr{3dnl_h}. Note that the fixed contribution $\int_{\Omega_0} (-C:e_T) : \delta e$ can be considered as an internal load of thermal origin.

The modes of the heated structure can be computed with the tangent matrix.

An example of static thermal computation is given in {\tt ofdemos \ts{ThermalCube}}.

%The thermoelastic equilibrium is solution of the problem
%
%\begin{eqsvg}{therm1}
%\ve{\delta v}^T \ma{K}\ve{q_T} = \int_{\Omega} \sigma_e(q_T) : \epsilon(\delta v) = \int_{\Omega} \sigma_T : \epsilon(\delta v) \ \ \forall \ \ \delta v = \ve{\delta v}^T \ve{F_T}
%\end{eqsvg}
%
%The effect of a temperature modification thus appears as an external load $F_T$ corresponding to the right hand side of ~\eqr{therm1}. Solving equation~\eqr{therm1}, with appropriate boundary conditions, leads to the static equilibrium of the structure under thermal loading $q_T$.

%For a given prestress state given by the difference between the elastic stress associated with the static equilibrium under thermal loading $\sigma_e(q_T)$ and the thermal stress $\sigma_T(\epsilon_T)$, one can compute the geometric stiffness matrix
%
%\begin{eqsvg}{therm2}
%  K_{\sigma}(q_T,\epsilon_T)=\int_{\Omega} \br{\sigma_{ij,e}(q_T)-\sigma_T} u_{k,i} v_{k,j}
%\end{eqsvg}
%
%and the modes of the heated structure are now given by
%
%\begin{eqsvg}
%  \ma{K+K_{\sigma(q_T,\epsilon_T)}-\omega_j^2 M}\ve{\phi_j} = \ve{0}
%\end{eqsvg}


%-----------------------------------------------------------------------
\cssection{Hyperelasticity}{fehyper}

The following gives the theory of the thermoelastic problem implemented in  OpenFEM function {\tt hyper.c} (called by {\tt of\_mk.c MatrixIntegration}).\\

For hyperelastic media $S=\partial W/\partial e$ with $W$ the hyperelastic energy.  {\tt hyper.c} currently supports Mooney-Rivlin materials for which the energy takes one of following forms
%
\begin{eqsvg}{feform_fehyper_1}
  W = C_1(J_1-3) + C_2(J_2-3) + K(J_3-1)^2,
\end{eqsvg}
\begin{eqsvg}{feform_fehyper_2}
 W = C_1(J_1-3) + C_2(J_2-3) + K(J_3-1) - (C_1 + 2C_2 + K)\ln(J_3),
\end{eqsvg}
%
where $(J_1,J_2,J_3)$ are the so-called reduced invariants of the Cauchy-Green tensor
%
\begin{eqsvg}{feform_fehyper_3}
  C=I+2e,
\end{eqsvg}
linked to the classical invariants $(I_1,I_2,I_3)$ by
\begin{eqsvg}{feform_fehyper_4}
  J_1=I_1 I_3^{-\frac{1}{3}},\ \ \ J_2=I_2 I_3^{-\frac{2}{3}},\ \ \  J_3=I_3^{\frac{1}{2}},
\end{eqsvg}
where one recalls that
\begin{eqsvg}{feform_fehyper_5}
  I_1={\rm tr} C,\ \ \  I_2=\frac{1}{2}\ma{({\rm tr} C)^2-{\rm tr} C^2},\ \ \  I_3={\rm det} C.
\end{eqsvg}

{\bf Note :} this definition of energy based on reduced invariants is used to have the hydrostatic pressure given directly by  $p=-K(J_3-1)$ ($K$ ``bulk modulus''), and the third term of $W$ is a penalty on incompressibility. 

Hence, computing the corresponding tangent stiffness and residual operators will require the derivatives of the above invariants with respect to $e$ (or $C$). In an orthonormal basis the first-order derivatives are given by: 
\begin{eqsvg}{feform_fehyper_6}
  \frac{\partial I_1}{\partial C_{ij}} = \delta_{ij},\ \ \ 
  \frac{\partial I_2}{\partial C_{ij}} = I_1\delta_{ij}-C_{ij},\ \ \ 
  \frac{\partial I_3}{\partial C_{ij}} = I_3 C_{ij}^{-1},
\end{eqsvg}
where $(C_{ij}^{-1})$ denotes the coefficients of the inverse matrix of $(C_{ij})$. For second-order derivatives we have:
\begin{eqsvg}{der2_inv}
  \frac{\partial^2 I_1}{\partial C_{ij}\partial C_{kl}} = 0,\ \ \ 
  \frac{\partial^2 I_2}{\partial C_{ij}\partial C_{kl}} = -\delta_{ik}\delta_{jl}+\delta_{ij}\delta_{kl},\ \ \ 
  \frac{\partial^2 I_3}{\partial C_{ij}\partial C_{kl}} = C_{mn} \epsilon_{ikm}\epsilon_{jln},
\end{eqsvg}
where the $\epsilon_{ijk}$ coefficients are defined by
\begin{eqsvg}{feform_fehyper_8}
  \left\{
  \begin{array}{lll}
    \epsilon_{ijk} &=0 &\mbox{when 2 indices coincide}\\
    &=1 &\mbox{when $(i,j,k)$ even permutation of $(1,2,3)$}\\
    &=-1 &\mbox{when $(i,j,k)$ odd permutation of $(1,2,3)$}
  \end{array}
  \right.
\end{eqsvg}
{\bf Note:} when the strain components are seen as a column vector (``engineering strains'') in the form $(e_{11},e_{22},e_{33},2e_{23},2e_{31},2e_{12})'$, the last two terms of \eqr{der2_inv} thus correspond to the following 2 matrices
\begin{eqsvg}{feform_fehyper_9}
  \left(
  \begin{array}{cccccc}
    0 & 1 & 1 & 0 & 0 & 0\\
    1 & 0 & 1 & 0 & 0 & 0\\
    1 & 1 & 0 & 0 & 0 & 0\\
    0 & 0 & 0 &-1/2& 0 & 0\\
    0 & 0 & 0 & 0 &-1/2& 0\\
    0 & 0 & 0 & 0 & 0 &-1/2
  \end{array}
  \right),
\end{eqsvg}
\begin{eqsvg}{feform_fehyper_10}
  \left(
  \begin{array}{cccccc}
    0      & C_{33}& C_{22}& -C_{23} &    0    & 0\\
    C_{33} &   0   & C_{11}&    0    & -C_{13} & 0\\
    C_{22} & C_{11}&   0   &    0    &    0    & -C_{12}\\
    -C_{23}&   0   &   0   &-C_{11}/2& C_{12}/2& C_{13}/2\\
    0      &-C_{13}&   0   & C_{12}/2&-C_{22}/2& C_{23}/2\\
    0      &   0   &-C_{12}& C_{13}/2& C_{23}/2&-C_{33}/2
  \end{array}
  \right).
\end{eqsvg}

We finally use chain-rule differentiation to compute
\begin{eqsvg}{feform_fehyper_11}
  S = \frac{\partial W}{\partial e} = 
  \sum_k \frac{\partial W}{\partial I_k} \frac{\partial I_k}{\partial e},
\end{eqsvg}
\begin{eqsvg}{feform_fehyper_12}
  \frac{\partial^2 W}{\partial e^2} =
  \sum_k \frac{\partial W}{\partial I_k} \frac{\partial^2 I_k}{\partial e^2}
  + \sum_k\sum_l \frac{\partial^2 W}{\partial I_k\partial I_l} \frac{\partial I_k}{\partial e}\frac{\partial I_l}{\partial e}.
\end{eqsvg}

Note that a factor 2 arise each time we differentiate the invariants with respect to $e$ instead of $C$.

The specification of a material is given by specification of the derivatives of the energy with respect to invariants.  The laws are implemented in the {\tt hyper.c EnPassiv} function. % xxx 

\cssection{Gyroscopic effects}{gyroef} % - - - - - - - - - - - - - - - -

{\tiny Written by Arnaud Sternchuss ECP/MSSMat.}


In the fixed reference frame which is Galilean, the Eulerian speed of the particle in ${\bf x}$ whose initial position is ${\bf p}$ is
%
\begin{eqsvg}{feform_gyroef_1}
\frac{\partial\bf x}{\partial t}= \frac{\partial\bf u}{\partial t}+\bf{\Omega}\wedge({\bf p}+{\bf u})
\end{eqsvg}
%
and its acceleration is
%
\begin{eqsvg}{feform_gyroef_2}
\frac{\partial^2\bf x}{\partial t^2} =\frac{\partial^2\bf u}{\partial t^2}+\frac{\partial\bf \Omega}{\partial t}\wedge({\bf p}+{\bf u})+2\bf{\Omega}\wedge\frac{\partial\bf u}{\partial t}+\bf{\Omega}\wedge\bf{\Omega}\wedge({\bf p}+{\bf u})
\end{eqsvg}
%
${\bf \Omega} $ is the rotation vector of the structure with 
%
\begin{eqsvg}{feform_gyroef_3}
{\bf \Omega}=\left[
\begin{array}{c}
\omega_x \\
\omega_y \\
\omega_z
\end{array}
\right]
\end{eqsvg}
%
in a $(x,y,z)$ orthonormal frame.
The skew-symmetric matrix $\ma{\Omega}$ is defined such that
%
\begin{eqsvg}{feform_gyroef_4}
\ma{\Omega} = \ma{\ba{ccc}
0 & -\omega_z & \omega_y  \\
\omega_z & 0  & -\omega_x  \\
-\omega_y  & \omega_x  & 0
\ea}
\end{eqsvg}
%
The speed can be rewritten
%
\begin{eqsvg}{feform_gyroef_5}
\frac{\partial\bf x}{\partial t}= \frac{\partial\bf u}{\partial t}+\ma{\Omega}({\bf p}+{\bf u}) 
\end{eqsvg}
%
and the acceleration becomes
%
\begin{eqsvg}{feform_gyroef_6}
\frac{\partial^2\bf x}{\partial t^2} =\frac{\partial^2\bf u}{\partial t^2}+\frac{\partial\ma{\Omega}}{\partial t}({\bf p}+{\bf u})+2\ma{\Omega}\frac{\partial\bf u}{\partial t}+\ma{\Omega}^2({\bf p}+{\bf u})
\end{eqsvg}
%
In this expression appear
\begin{itemize}
\item the acceleration in the rotating frame \mathsvg{\frac{\partial^2\bf u} {\partial t^2}}{gyroef_l1},
\item the centrifugal acceleration \mathsvg{{\bf a_g}=\ma{\Omega}^2({\bf p}+{\bf u})}{gyroef_l2},
\item the Coriolis acceleration \mathsvg{{\bf a_c}=\frac{\partial\ma{\Omega}} {\partial t}({\bf p}+{\bf u})+2\ma{\Omega}\frac{\partial\bf u} {\partial t}}{gyroef_l3}.
\end{itemize}

${\mathcal S}_0^e$ is an element of the mesh of the initial configuration ${\mathcal S}_0$ whose density is $\rho_0$. $\ma{N}$ is the matrix of shape functions on these elements, one defines the following elementary matrices
%
\begin{eqsvg}{eqn51}\ba{ll}
\ma{D_g^e} =& \int_{{\mathcal S}_0^e} 2\rho_0 \ma{N}^\top \ma{\Omega} \ma{N}\;d{\mathcal S}_0^e \quad\mbox{\emph{gyroscopic coupling}}\\
\ma{K_a^e} =& \int_{{\mathcal S}_0^e} \rho_0 \ma{N}^\top\frac{\partial\ma{\Omega}}{\partial t}\ma{N}\; d{\mathcal S}_0^e \quad\mbox{\emph{Coriolis acceleration}}\\
\ma{K_g^e} =& \int_{{\mathcal S}_0^e} \rho_0 \ma{N}^\top\ma{\Omega}^2 \ma{N}\; d{\mathcal S}_0^e \quad\mbox{\emph{centrifugal softening/stiffening}}
\ea\end{eqsvg}

The traditional \ltr{fe\_mknl}{MatType} in SDT are 7 for gyroscopic
coupling and 8 for centrifugal softening. 

\cssection{Centrifugal follower forces}{centri} % - - - - - - - - - - - - - - - -

This is the embryo of the theory for the future implementation of centrifugal follower forces.

\begin{eqsvg}{feform_centri_1}
\delta W_\omega= \int_\Omega \rho \omega^2 R({x}) \delta v_R,
\end{eqsvg}
where $\delta v_R$ designates the radial component (in deformed configuration) of $\delta{v}$. One assumes that the rotation axis is along $e_z$. Noting ${n}_R = 1/R \{x_1\; x_2 \; 0\}^T$, one then has
\begin{eqsvg}{feform_centri_2}
\delta v_R= {n}_R\cdot\delta {v}.
\end{eqsvg}

Thus the non-linear stiffness term is given by 
\begin{eqsvg}{feform_centri_3}
-d\delta W_\omega= - \int_\Omega \rho \omega^2 (dR \delta v_R + R d\delta v_R).
\end{eqsvg}
One has $dR={n}_R\cdot d{x}(= dx_R)$ and $d\delta v_R = d{n}_R\cdot\delta {v}$, with
$$
d{n}_R=-\frac{dR}{R}{n}_R + \frac{1}{R}\{dx_1\; dx_2 \; 0\}^T.
$$
Thus, finally

\begin{eqsvg}{feform_centri_4}
-d\delta W_\omega= - \int_\Omega \rho \omega^2 (du_1 \delta v_1 + du_2 \delta v_2).
\end{eqsvg}

Which gives
\begin{eqsvg}{feform_centri_5}
du_1 \delta v_1 + du_2 \delta v_2= \{\delta q_\alpha\}^T \{N\}\{N\}^T \{d q_\alpha\},
\end{eqsvg}
with $\alpha=1,2$.


%-----------------------------------------------------------------------
\cssection{Poroelastic materials}{feporous}

The poroelastic formulation comes from \cite{atalla2001enhanced}, recalled and detailed in \cite{allard2009propagation}.

Domain and variables description:

\begin{tabular}{lll}
        \mathsvg{\Omega}{feporous_l1}                         & Poroelastic domain\\
        \mathsvg{\partial\Omega}{feporous_l2}    & Bounding surface of poroelastic domain\\
        \mathsvg{n}{feporous_l3}                     & Unit external normal of \mathsvg{\partial\Omega}{feporous_l4}\\
        \mathsvg{u}{feporous_l5}                      & Solid phase displacement vector\\
        \mathsvg{u^F}{feporous_l6}                         & Fluid phase displacement vector &
        \mathsvg{u^F = \frac{\ds\phi} {\ds\tilde{\rho}_{22}\omega^2}\nabla p - \frac{\ds\tilde{\rho}_{12}} {\ds\tilde{\rho}_{22}}u}{feporous_l7}\\
        \mathsvg{p}{feporous_l8}                       & Fluid phase pressure\\
        \mathsvg{\sigma}{feporous_l9}                         & Stress tensor of solid phase\\
        \mathsvg{\sigma^t}{feporous_l10}                & Total stress tensor of porous material & 
        \mathsvg{\sigma^t=\sigma-\phi\left(1+\ds\frac{\tilde{Q}} {\tilde{R}}\right)pI}{feporous_l11}\\
\end{tabular}

Weak formulation, for harmonic time dependence at pulsation $\omega$:

\begin{eqsvg}{poro_str}
\ba{l}
\ds \int_\Omega\sigma(u) : \epsilon(\delta u) \ d\Omega - \omega^2 \int_\Omega \tilde{\rho} \ u.\delta u \ d\Omega -\int_\Omega\frac{\phi}{\tilde{\alpha}}\nabla p.\delta u \ d\Omega \\
\hspace{2cm} \ds -\int_\Omega \phi\left(1+\frac{\tilde{Q}}{\tilde{R}}\right)p\nabla.\delta u \ d\Omega -\int_{\partial\Omega}(\sigma^t(u).n).\delta u \ dS =0 \ \ \forall \delta u
\ea
\end{eqsvg}

\begin{eqsvg}{poro_fl}
\ba{l}
\ds \int_\Omega \frac{\phi^2}{\tilde{\alpha}\rho_o\omega^2}\nabla p.\nabla\delta p \ d\Omega -\int_\Omega\frac{\phi^2}{\tilde{R}}p\ \delta p \ d\Omega -\int_\Omega\frac{\phi}{\tilde{\alpha}} u.\nabla \delta p \ d\Omega \\
\hspace{2cm} \ds -\int_\Omega \phi\left(1+\frac{\tilde{Q}}{\tilde{R}}\right)\delta p\nabla. u \ d\Omega -\int_{\partial\Omega}\phi(u^F-u).n\ \delta p \ dS = 0\ \ \forall \delta p
\ea
\end{eqsvg}

Matrix formulation, for harmonic time dependence at pulsation $\omega$:

\begin{eqsvg}{syst_poro}
 \ma{\ba{cc} K-\omega^2M&-C_1-C_2\\ -C_1^T-C_2^T&\frac{1}{\omega^2}F-K_p \ea} \ve{\ba{c} u\\p\ea}
 = \ve{\ba{c}F_s^t\\F_f\ea}
\end{eqsvg}

where the frequency-dependent matrices correspond to:

\begin{eqsvg}{feform_feporous_1}\nonumber
\ba{ll}
\ds \int_{\Omega}{\sigma(u):\epsilon(\delta u)\ d\Omega}
        &       \Rightarrow\delta u^{T} K u\\
\ds \int_{\Omega}{\tilde{\rho} \ u.\delta u\ d\Omega}
        &       \Rightarrow\delta u^{T} M u\\
\ds \int_{\Omega}{\frac{\phi^2}{\tilde{\alpha}\rho_o}\nabla p.\nabla\delta p}
        &       \Rightarrow\delta p^{T} K_p p \\
\ds \int_{\Omega}{\frac{\phi^2}{\tilde{R}}p\ \delta p}
        &       \Rightarrow\delta p^{T} F p \\
\ds \int_\Omega\frac{\phi}{\tilde{\alpha}}\nabla p.\delta u \ d\Omega
        &       \Rightarrow\delta u^{T} C_1 p \\
\ds \int_\Omega \phi\left(1+\frac{\tilde{Q}}{\tilde{R}}\right)p\nabla.\delta u \ d\Omega
        &       \Rightarrow\delta u^{T} C_2 p \\ 
\ds \int_{\partial\Omega}(\sigma^t(u).n).\delta u \ dS 
        &       \Rightarrow\delta u^{T} F_s^t\\ 
\ds \int_{\partial\Omega}\phi(u^F-u).n\ \delta p \ dS 
        &       \Rightarrow\delta p^{T} F_f\\ 
\ea
\end{eqsvg}

N.B. if the material of the solid phase is homogeneous, the frequency-dependent parameters can be eventually factorized from the matrices:

\begin{eqsvg}{syst_poro_freq}
 \ma{\ba{cc} (1+i\eta_s)\bar{K}-\omega^2\tilde{\rho}\bar{M}&-\frac{\phi}{\tilde{\alpha}}\bar{C}_1- \phi\left(1+\frac{\tilde{Q}}{\tilde{R}}\right)\bar{C}_2 \\
-\frac{\phi}{\tilde{\alpha}}\bar{C}_1^T-\phi\left(1+ \frac{\tilde{Q}}{\tilde{R}}\right)\bar{C}_2^T&\frac{1}{\omega^2}\frac{\phi^2}{\tilde{R}}\bar{F}- \frac{\phi^2}{\tilde{\alpha}\rho_o}\bar{K}_p \ea} \ve{\ba{c} u\\p\ea}
 = \ve{\ba{c}F_s^t\\F_f\ea}
\end{eqsvg}

where the matrices marked with bars are frequency independent:

\begin{eqsvg}{feform_feporous_2}\nonumber
\ba{lll}
 K=(1+i\eta_s)\bar{K} & M=\tilde{\rho}\bar{M} & C_1=\frac{\phi}{\tilde{\alpha}}\bar{C}_1 \\
C_2=\phi\left(1+\frac{\tilde{Q}}{\tilde{R}}\right)\bar{C}_2 & F=\frac{\phi^2}{\tilde{R}}\bar{F} & K_p=\frac{\phi^2}{\tilde{\alpha}\rho_o}\bar{K}_p \\ 
\ea
\end{eqsvg}


Material parameters:\\

\begin{tabular}{ll}
        $\phi$                                                  & Porosity of the porous material\\
        $\bar{\sigma}$                                  & Resistivity of the porous material\\
        $\alpha_\infty$                         & Tortuosity of the porous material\\
        $\Lambda$                                               & Viscous characteristic length of the porous material\\
        $\Lambda'$                                              & Thermal characteristic length of the skeleton\\
        $\rho$                                                  & Density of the skeleton\\
        $G$                                                             & Shear modulus of the skeleton\\
        $\nu$                                                           & Poisson coefficient of the skeleton\\
        $\eta_s$                                                        & Structural loss factor of the skeleton\\
        $\rho_o$                                                        & Fluid density\\
        $\gamma$                                                        & Heat capacity ratio of fluid ($=1.4$ for air)\\
        $\eta$                                                  & Shear viscosity of fluid ($=1.84\times10^{-5}\ kg\ m^{-1} \ s^{-1}$ for air)\\
\end{tabular}

Constants:\\

\begin{tabular}{ll}
        $P_o=1,01\times 10^5\ Pa$                                                                       & Ambient pressure\\
        $Pr=0.71$                                                                                                               & Prandtl number\\
\end{tabular}

Poroelastic specific (frequency dependent) variables:\\

\begin{tabular}{lll}
        \mathsvg{{\rho}_{11}}{proelspec_l1}                                   & Apparent density of solid phase &
        \mathsvg{\rho_{11} = (1-\phi)\rho-\rho_{12}}{proelspec_l2}\\
        \mathsvg{{\rho}_{22}}{proelspec_l3}                                   & Apparent density of fluid phase &
        \mathsvg{\rho_{22} = \phi\rho_o-\rho_{12}}{proelspec_l4}\\
        \mathsvg{{\rho}_{12}}{proelspec_l5}                                   & Interaction apparent density&
        \mathsvg{\rho_{12}=-\phi\rho_o(\alpha_\infty-1)}{proelspec_l6}\\
        \mathsvg{\tilde{\rho}}{proelspec_l7}                                  & Effective density of solid phase&
        \mathsvg{\tilde{\rho} = \tilde{\rho}_{11} - \frac{\ds(\tilde{\rho}_{12})^2} {\ds\tilde{\rho}_{22}}}{proelspec_l8} \\
        \mathsvg{\tilde{\rho}_{11}}{proelspec_l9}                     & Effective density of solid phase &
        \mathsvg{\tilde{\rho}_{11} = \rho_{11} + \frac{\ds \tilde{b}} {\ds i\omega}}{proelspec_l10}\\
        \mathsvg{\tilde{\rho}_{22}}{proelspec_l11}                     & Effective density of fluid phase &
        \mathsvg{\tilde{\rho}_{22} = \rho_{22} + \frac{\ds \tilde{b}} {\ds i\omega}}{proelspec_l12}\\
        \mathsvg{\tilde{\rho}_{12}}{proelspec_l13}                     & Interaction effective density&
        \mathsvg{\tilde{\rho}_{12} = \rho_{12} - \frac{\ds \tilde{b}} {\ds i\omega}}{proelspec_l14}\\
        \mathsvg{\tilde{b}}{proelspec_l15}                                             & Viscous damping coefficient &
        \mathsvg{\tilde{b} = \phi^2\bar{\sigma}\ds\sqrt{1 + i\frac{\ds 4\alpha_\infty^2\eta\rho_o\omega} {\ds \bar{\sigma}^2\Lambda^2\phi^2}}}{proelspec_l16} \\
        \mathsvg{\tilde{\gamma}}{proelspec_l17}                                & Coupling coefficient & 
        \mathsvg{ \tilde{\gamma} = \phi\left(\frac{\ds\tilde{\rho}_{12}} {\ds\tilde{\rho}_{22}} - \frac{\ds\tilde{Q}} {\ds\tilde{R}}\right)}{proelspec_l18}\\
        \mathsvg{\tilde{Q}}{proelspec_l19}                                             & Elastic coupling coefficient \\
        & \hspace{5mm} Biot formulation &
        \mathsvg{\tilde{Q}=\frac{\ds1-\phi-\frac{K_b} {K_s}} {\ds1-\phi-\frac{K_b} {K_s}+\phi\frac{K_s} {\tilde{K}_f}}\phi K_s}{proelspec_l20}\\
        & \hspace{5mm} Approximation from \mathsvg{K_b/K_s<<1}{proelspec_l21} &
        \mathsvg{\tilde{Q}=(1-\phi)\tilde{K}_f}{proelspec_l22}\\
        \mathsvg{\tilde{R}}{proelspec_l23}                                       & Bulk modulus of air in fraction volume \\
        & \hspace{5mm} Biot formulation &
        \mathsvg{\tilde{R}=\frac{\ds\phi^2K_s} {\ds1-\phi-\frac{K_b} {K_s}+\phi\frac{K_s} {\tilde{K}_f}}}{proelspec_l24}\\
        & \hspace{5mm} Approximation from \mathsvg{K_b/K_s<<1}{proelspec_l25} &
        \mathsvg{\tilde{R}=\phi\tilde{K}_f}{proelspec_l26}\\
        \mathsvg{{K_b}}{proelspec_l27}                             & Bulk modulus of porous material in vacuo &
        \mathsvg{K_b=\frac{\ds 2G(1+\nu)} {\ds 3(1-2\nu)}}{proelspec_l28}\\
        \mathsvg{K_s}{proelspec_l29}                                           & Bulk modulus of elastic solid \\
        & \hspace{5mm} est. from Hashin-Shtrikman's upper bound &
        \mathsvg{K_s=\frac{1+2\phi} {1-\phi}K_b}{proelspec_l30}\\
        \mathsvg{\tilde{K}_f}{proelspec_l31}                         & Effective bulk modulus of air in pores &
        \mathsvg{\tilde{K}_f=\frac{\ds P_o} {\ds 1 - \frac{\ds \gamma -1} {\ds \gamma \alpha '}}}{proelspec_l32}\\
        \mathsvg{\alpha '}{proelspec_l33}      & Function in \mathsvg{\tilde{K}_f}{proelspec_l34} (Champoux-Allard model) &
        \mathsvg{\alpha ' = 1 + \frac{\ds \omega_T} {\ds 2i\omega}\left(1+\frac{\ds i\omega} {\ds \omega_T}\right)^{\frac{1} {2}}}{proelspec_l35}\\
        \mathsvg{ \omega_T}{proelspec_l36}                                     & Thermal characteristic frequency 
        & \mathsvg{\omega_T=\frac{\ds 16\eta} {\ds Pr\Lambda'^2\rho_o}}{proelspec_l37}
\end{tabular}

To add here:

\begin{itemize}
    \item coupling conditions with poroelastic medium, elastic medium, acoustic medium
        \item dissipated power in medium
\end{itemize}

%-----------------------------------------------------------------------
\cssection{Heat equation}{fe3dth}

This section is based on an OpenFEM contribution by Bourquin Fr\'{e}d\'{e}ric and Nassiopoulos Alexandre from { \it Laboratoire Central des Ponts et Chauss\'{e}es.}  


The variational form of the Heat equation is given by

\begin{eqsvg}{feform_fe3dth_1}
\ba{c}
\displaystyle\int_{\Omega} ({\bf \rho c} \dot\theta)(v)\,dx
+\displaystyle\int_{\Omega} ({\bf K} grad \,\theta)(grad \,v)\,dx
+ \int_{\partial\Omega} \alpha\theta v \,d\gamma = \\
\displaystyle \int_{\Omega} f v \, dx +
\displaystyle \int_{\partial\Omega} (g+\alpha \theta_{ext}) v \,d\gamma \\
\\
\ \ \forall v \in H^1(\Omega)\\
\ea
\end{eqsvg} 

with
\begin{itemize}
\item $\rho$ the density,  $c$ the specific heat capacity.
\item ${\bf K}$ the conductivity tensor of the material. The tensor ${\bf K}$ is symmetric, positive definite, and is often taken as diagonal. If conduction is isotropic, one can write ${\bf K}=k(x)Id$ where $k(x)$ is called the (scalar) conductivity of the material.

\item Acceptable loads and boundary conditions are 

\begin{itemize} 
\item{Internal heat source $f$}\\
\item{Prescribed temperature (Dirichlet condition, also called boundary condition of first kind)}\\
\begin{eqsvg}{feform_fe3dth_2} \theta=\theta_{ext} \quad on \quad \partial\Omega \end{eqsvg}
modeled using a \ts{DofSet} case entry.

\item{Prescribed heat flux $g$ (Neumann condition, also called boundary condition of second kind)}\\

\begin{eqsvg}{feform_fe3dth_3}({\bf K}grad \,\theta)\cdot\vec{n}=g \quad on \quad \partial\Omega \end{eqsvg}
leading to a load applied on the surface modeled using a \ts{FVol} case entry.

\item{Exchange and heat flux (Fourier-Robin condition, also called boundary condition of third kind)}\\
\begin{eqsvg}{feform_fe3dth_4}({\bf K}grad \,\theta)\cdot\vec{n}+\alpha(\theta-\theta_{ext})=g \quad on \quad \partial\Omega \end{eqsvg}

leading to a stiffness term (modeled using a group of surface elements with stiffness proportional to $\alpha$) and a load on the associated surface proportional to $g+\alpha\theta_{ext}$ (modeled using \ts{FVol} case entries).

\end{itemize}
\end{itemize}

\subsubsection{Test case}

One considers a solid square prism of dimensions $L_x,L_y, L_z$ in the three directions $(Ox)$, $(Oy)$ and $(Oz)$ respectively. The solid is made of homogeneous isotropic material, and its conductivity tensor thus reduces to a constant $k$. 

The faces, \mathsvg{\displaystyle \Gamma_i (i=1..6, \cup_{i=1}^6 \Gamma_i = \partial \Omega)}{fe3dth_l1}, are subject to the following boundary conditions and loads


\begin{itemize}
\item $f=40$ is a constant uniform internal  heat source
\item{$\Gamma_1 \,(x=0)$ : exchange \& heat flux (Fourier-Robin) given by $\alpha=1,g_1=\alpha \theta_{ext} + \frac{\alpha f L_x^2}{2k}=25$\\
\item{$\Gamma_2 \,(x=L_x)$ :  prescribed temperature : $\theta(L_x,y,z)=\theta_{ext}=20$}
\item{$\Gamma_3 \,(y=0)$, $\Gamma_4 \,(y=L_y)$, $\Gamma_5 \,(z=0)$, $\Gamma_6 \,(z=L_z)$: exchange \& heat flux $g+\alpha \theta_{ext} =\alpha\theta_{ext} +\frac{\alpha f}{2 k} (L_x^2-x^2)+g_1=25-\frac{x^2}{20}$}}\\

\end{itemize}

The problem can be solved by the method of separation of variables. It admits
the solution
$$ \displaystyle \theta(x,y,z) =-\frac{f}{2 k} x^2 + \theta_{ext} + \frac{ f L_x^2}{2k} = \frac{g(x)}{\alpha}= 25 - \frac{x^2}{20}$$

The resolution for this example can be found in {\tt demo/heat\_equation}.

\begin{figure}
%\begin{center}
\centering
\ingraph{60}{heat_eq}
%\includegraphics[width=0.5\textwidth]{plots_heat_eq/z_y_axis}
\caption{Temperature distribution along the x-axis}
\label{fig:reg_temp_fin}
%\end{center}
\end{figure}

 

\newpage
%- - - - - - - - - - - - - - - - - - - - - - - - - %
%     Handling material and element properties     %
%- - - - - - - - - - - - - - - - - - - - - - - - - %     
\cssection{Handling material and element properties}{femp}
Before assembly, one still needs to define material and element properties associated with the various elements. 

\begin{SDT}

You can edit material properties using the {\tt Materials} tab of the {\tt Model Properties} figure which lists current materials and lets you choose new ones from the database of each material type. \melastic\ is the only material function defined for the base {\sl SDT}. It supports elastic materials and linear acoustic fluids. 

\begin{figure}[H]
\centering
\ingraph{80}{matgui}
 \caption{Property tab.}
  \label{fig:matgui}
\end{figure}
%\includegraphics[width=\tw]{matgui}

Similarly the {\tt Property} tab lets you edit element properties. \pbeam\, \pshell\ and \pspring\ are supported element property functions.

\end{SDT}

The properties are stored with one property per row in {\tt pl} and {\tt il} model fields. {\tt model.pl} is a material property matrix and {\tt model.il} is a element property matrix.

A row in the material property matrix begins with a {\tt MatID} which identifies a particular material property and matches with a {\tt MatID} in the model description matrix. Then a {\tt Type} is defined and various material properties are given. See \ser{elt} and \ser{pl} for details.

A row in the element property matrix as the same shape as a row in the material property matrix. It begins with a {\tt ProID} which is an identifier of a particular element property that matches with a {\tt ProID} in the model description matrix. Then a {\tt Type} is defined and various element property are given. See \ser{il} for details.

When using scripts, it is often more convenient to use low level definitions
of the material properties. For example (see the {\tt demo\_fe} script, part 2 {\sl Handling material and element properties}) , one can define aluminum and three sets of beam properties with 

\begin{verbatim}
 ...
 %         MatId  MatType                    E       nu    rho
 model.pl=[ 1   fe_mat('m_elastic','SI',1)  7.2e+10  0.3   2700 ];
 model.il = [ ...
 %  ProId SecType                 J      I1     I2       A
 1 fe_mat('p_beam','SI',1) 5e-9   5e-9   5e-9   2e-5  0 0 % longerons
 p_beam('dbval 2','circle 4e-3') % circular section 4 mm
 p_beam('dbval 3','rectangle 4e-3 3e-3')%rectangular section 4 x 3 mm
  ];
 ...
\end{verbatim}

To assign a {\tt MatID} or a {\tt ProID} to a group of elements, you can use 

\begin{itemize}
\begin{SDT}
\item the graphical procedure (in the context menu of the material and property tabs, use the {\tt Select elements and affect ID} procedures and follow the instructions);
\end{SDT}
\item the simple {\tt femesh} set commands. For example {\tt femesh('set group1 mat1 pro3')} will set values 1 to MatID and 3 to ProID for element group 1 (see {\tt gartfe} script).

An element group is a set of elements of the same type and with the same properties. In a model description matrix ({\tt model.Elt} or {\tt FEelt} for example), an element group begins with a header row whose first element is {\tt Inf} and the following the ascii values for the name of the element. It ends by the header row of the next element group or with the end of the model description matrix.

\item more elaborate commands based on \femesh\ findelt commands. Knowing which column of the {\tt Elt} matrix you want to modify, you can use something of the form (see {\tt gartfe} script)

{\tt FEelt(femesh('find {\ti EltSelectors}'), {\ti IDColumn})={\ti ID};}

You can also get values with {\tt mpid=feutil('mpid',elt)}, modify {\tt mpid}, then set values with {\tt elt=feutil('mpid',elt,mpid)} (see the {\tt demo\_fe} script, part 3).

\end{itemize}

\newpage
%- - - - - - - - - - - - - - - - - - %
%     Coordinate system handling     %
%- - - - - - - - - - - - - - - - - - %
\cssection{Coordinate system handling}{febas}

Local coordinate systems are stored in a {\tt model.bas} field described in the \basis\ reference section. Columns 2 and 3 of \hyperlink{node}{{\tt model.Node}} define coordinate system numbers for position and displacement, respectively. 

\begin{SDT}
Use of local coordinate systems is illustrated in~\ser{corcoor} where a local basis is defined for test results.
\end{SDT}

\feplot, \femk, \rigid, ... now support local coordinates. \feutil\ does when the model is described by a data structure containing the {\tt .bas} field. \femesh\ assumes you are using global coordinate system obtained with

\begin{verbatim}
 [FEnode,bas] = basis(model.Node,model.bas)
\end{verbatim}

To write your own scripts using local coordinate systems, it is useful to know the following  calls  :

{\tt [node,bas,NNode]=feutil('getnodebas',model)} returns the nodes in global coordinate system, the bases {\tt bas} with recursive definitions resolved and the reindexing vector {\tt NNode}. 

The command
\begin{verbatim}
 cGL=basis('trans l',model.bas,model.Node,model.DOF)
\end{verbatim}

returns the local to global transformation matrix.

\newpage
%-------------------------%
%     Defining a case     %
%-------------------------%
\csection{Defining a case}{case}\index{cases}

Once the topology ({\tt .Node},{\tt .Elt}, and optionally {\tt .bas} fields) and properties ({\tt .pl},{\tt .il} fields or associated {\tt mat} and {\tt pro} entries in the {\tt .Stack} field) are defined, you still need to define boundary conditions, constraints (see~\ser{febc}) and applied loads before actually computing a response. The associated information is stored in a \hyperlink{stackref}{case} data structure. The various cases are then stored in the {\tt .Stack} field of the model data structure.

Boundary conditions and constraints are detailed in section \ref{s*febc} and load definitions in section \ref{s*loads}.


\begin{SDT}
%-----------------------------------------------------------------------
\subsection{Cases GUI}

Graphical editing of case properties is supported by the case tab of the model properties GUI (see~\ser{editmodel}).

\begin{figure}[H]
\centering
\ingraph{80}{feplot_case}
 \caption{Cases properties tab.}
  \label{fig:feplot_case}
\end{figure}
%\centre{\includegraphics[width=.5\tw]{feplot_case}}

When selecting {\tt New ...} in the case property list, as shown in the figure, you get a list of currently supported case properties. You can add a new property by clicking on the associated {\tt new} cell in the table. Once a property is opened you can typically edit it graphically. The following sections show you how to edit these properties trough command line or \ts{.m} files.
%-----------------------------------------------------------------------
\end{SDT}

%- - - - - - - - - - - - - - - - - - - - - - -%
%     Boundary conditions and constraints     %
%- - - - - - - - - - - - - - - - - - - - - - -%
\cssection{Boundary conditions and constraints}{febc}\index{boundary condition}
Boundary conditions and constraints are described in {\tt Case.Stack} using {\tt FixDof} and {\tt Rigid} case entries (see \ser{case}).

{\tt FixDof} entries are used to easily impose zero displacement on some DOFs. To treat the two bay truss example of \ser{fetr}, one will for example use (see {\tt demo\_fe} part 3 {\sl Boundary conditions and constraints})

\begin{verbatim}
 ...
 model=fe_case(model,'SetCase1', ...         % defines a new case
  'FixDof','2-D motion',[.03 .04 .05]', ...  % 2-D motion 
  'FixDof','Clamp edge',[1 2]');             % clamp edge
 ...
\end{verbatim}

When assembling the model with the specified {\tt Case} (see \ser{case}), these constraints will be used automatically.

Note that, you may obtain a similar result by building the DOF definition vector for your model using a script ({\bf such scripts are considered obsolete since \fecase\ is more compact and efficient}). \hyperlink{findnode}{Node selection} commands allow node selection and \fec\ provides additional DOF selection capabilities. In the two bay truss case, (see {\tt demo\_2bay}, part 1)
%
\begin{verbatim}
 femesh('reset');
 model=femesh('test 2bay');
 mdof = feutil('getdof group1:2',model);
 i1 = femesh('findnode x==0');
 adof1 = fe_c(mdof,i1,'dof',1);             % clamp edge
 adof2 = fe_c(mdof,[.01 .02 .06]','dof',2); % 2-D motion
 adof = [adof1;adof2];
 model=fe_case(model,'SetCase1', ...        % defines a new case
     'FixDof','fixed DOF list',adof);  
\end{verbatim}
%
finds all DOFs in element groups 1 and 2 of {\tt FEelt}, eliminates DOFs that do not correspond to 2-D motion, finds nodes in the {\tt x==0} plane and eliminates the associated DOFs from the initial {\tt mdof}.

Details on low level handling of fixed boundary conditions and constraints are given in~\ser{mpc}.

\newpage
%- - - - - - - -%
%     Loads     %
%- - - - - - - -%
\cssection{Loads}{loads}
Loads  are described in {\tt Case.Stack} using {\tt DOFLoad}, {\tt FVol} and {\tt FSurf} case entries (see \fecase).

Three examples are presented below (see the {\tt demo\_ubeam} script).

\begin{itemize}
\item To treat a 3D beam example with volume forces ($x$ direction), one will for example use
\end{itemize}
\vspace{-0.5cm}
\begin{verbatim}
 femesh('reset');
 model = femesh('test ubeam');
 data  = struct('sel','GroupAll',...
  'dir',[1 0 0]);                   % defines a force in the x direction
 model = fe_case(model,...          % defines a new case
  'FVol','Volume load',data);       % specifies the load type : FVol
 Load  = fe_load(model,'case1');    % computes the load
 feplot(model,Load);

\end{verbatim}

\begin{center}
%\includegraphics[width=5cm]{plots/loadvol.ps}\\
%\epsfig{file=plots/loadvol.ps,width=5cm}\\
\begin{figure}[H]
\centering
\ingraph{50}{loadvol}
 %\caption{Simulation properties tab.}
 % \label{fig:feplot_fe_simul}
\end{figure}

Visualization of volume forces with OpenFEM for Scilab
\end{center}

\begin{itemize}
\item To treat a 3D beam example with surface forces, one will for example use
\end{itemize}
\vspace{-0.5cm}
\begin{verbatim}
 femesh('reset');
 model = femesh('testubeam');
 data=struct('sel','x==-.5', ...      % defines a force : applied on the
    'eltsel','withnode {z>1.25}',...  % elements where x==.5 and z > 1.25 
    'def',1,'DOF',.19);
 Case1=struct('Stack',stack_cell(...  % defines a case
    stack_cell('Fsurf',...            % specifies the load type : Fsurf
    'Surface load',data))); 
 Load = fe_load(model,Case1);         % computes the load  
 feplot(model,Load);

\end{verbatim}


\begin{center}
\begin{figure}[H]
\centering
\ingraph{50}{loadsur} % [width=5.cm]
 %\caption{Simulation properties tab.}
 % \label{fig:feplot_fe_simul}
\end{figure}

Visualization of surface forces with OpenFEM for MATLAB
\end{center}

\begin{itemize}
\item To treat a 3D beam example and create two loads, a relative force between DOFs 207x and 241x and two point loads at DOFs 207z and 365z, one will for example use
\end{itemize}
\vspace{-0.5cm}
\begin{verbatim}
 femesh('reset');
 model = femesh('test ubeam');
 data  = struct('DOF',...              % defines a force applied on the
  [207.01;241.01;207.03],...           % node 207 (x and z directions)
  'def',[1 0;-1 0;0 1]);               % and node 241 (x direction)
 model = fe_case(model,  ...           % defines case
  'DOFLoad','Point load 1',data);      % specifies the load type : DOFLoad
 data  = struct('DOF',365.03,'def',1); % defines a force applied on
                                       % node 365 in the z direction
 model = fe_case(model, ...            % defines another case
  'DOFLoad','Point load 2',data);      % specifies the load type : DOFLoad
 Load  = fe_load(model,'Case1');       % computes the load
 feplot(model,Load);
\end{verbatim}

The result of \feload\ contains 3 columns corresponding to the relative force and the two point loads.
You might then combine these forces, by summing them

\begin{verbatim}
 Load.def=sum(Load.def,2);
 medit('write visu/ubeam',model,Load,'a',[1 10 0.7]);
\end{verbatim}

\begin{center}
\begin{figure}[H]
\centering
\ingraph{50}{load2} % [width=5.cm]
 %\caption{Simulation properties tab.}
 % \label{fig:feplot_fe_simul}
\end{figure}

Visualization of combined forces with Medit
\end{center}


\newpage
%-------------------------------------------%
%     Computing the response of a model     %
%-------------------------------------------%
\csection{Computing the response of a model}{response}

This section is about the computational part of OpenFEM. Assembly is detailed in section \ref{s*mk}, computing the static response of a structure in section \ref{s*static}, computing normal modes in section \ref{s*mode}. Moreover, in the section \ref{s*large}, the case of large finite element models is discussed. 

%- - - - - - - - - %
%     Assembly     %
%- - - - - - - - - %
\cssection{Assembly}{mk}
Assembly is made by the \femk\ function. See the {\sl Reference} section for details on \femk. Two examples are presented below :
\begin{itemize}
\item \textbf{First example ({\tt demo\_fe} script) :}\\
Boundary conditions have already been defined with the use of \fecase. These conditions are in the {\tt Stack} field of the {\tt model} structure. In this case, you should use \femk\ as follows :
\begin{verbatim}
model = fe_mk(model);
\end{verbatim}

or

\begin{verbatim}
model = fe_mknl(model);
\end{verbatim}

See the {\tt demo\_fe} example, part 4 {\sl Assembly}.\\
{\tt model} now contains a field {\tt K} which contains mass and stiffness matrices.

\item \textbf{Second example ({\tt demo\_2bay} script) :}\\
Boundary conditions are not defined. They can be defined directly in the call of \femk.
\begin{verbatim}
femesh('reset');
model2 = femesh('test 2bay');
model2 = fe_mk(model2,'FixDof','2-D motion',[.01 .02 .06],...
               'FixDof','clamp edge',[1 2]);
\end{verbatim}
Mass and stiffness matrices are in the {\tt K} field of {\tt model}. They can also be returned in separate matrices :
\begin{verbatim}
[m,k,mdof] = fe_mk(model2,'FixDof','2-D motion',[.01 .02 .06],...
                   'FixDof','clamp edge',[1 2]);
\end{verbatim}
See the {\tt demo\_2bay} script, part 2 {\sl Assembly}.
\end{itemize}
Note that, in OpenFEM for MATLAB (only), \femk\ renumbers matrices when the number of DOF is greater than 1000. In this case, the DOF definition vector ({\tt mdof} for example) is modified by the \femk\ function. {\tt m} and {\tt k} correspond to the output DOF definition vector.

\newpage
%- - - - - - - - - - - - -%
%     Static response     %
%- - - - - - - - - - - - -%
\cssection{Static response}{static}
The computation of the response of static loads can be done as follows. \\
We suppose that the mass and stiffness matrices have already been assembled (in field {\tt K} of the {\tt model} data structure), and that a load has already been computed in the {\tt Load} data structure. \\

\begin{verbatim}
def = struct('def',[],'DOF',model.DOF);
kd = ofact(model.K{2}); % use the factor object for large matrices
def.def = kd\Load.def;
ofact('clear',kd);  % Clear the factor when done
\end{verbatim}
You can compute the stress due to the response.
\begin{verbatim}
Stress = fe_stress('stress mises',model,def);
medit('write visu/def',model,def,Stress,[1 1e8]);
\end{verbatim}

\begin{center}
\begin{figure}[H]
\centering
\ingraph{50}{static} % [width=6.cm]
 %\caption{Simulation properties tab.}
 % \label{fig:feplot_fe_simul}
\end{figure}

Visualization of static response with Medit ({\tt demo\_static})
\end{center}
This example is described in the {\tt demo\_static} script.\\
Note that the animation of the response to static load can be run by clicking with the mouse right button and selecting ``Play sequence'' in the ``Animation'' menu.

\newpage
%- - - - - - - - - - - - - - - - - - - - - - - - - - -%
%      Normal modes (partial eigenvalues solution     %
%- - - - - - - - - - - - - - - - - - - - - - - - - - -%
\cssection{Normal modes (partial eigenvalues solution)}{mode}

The computation of normal modes is made by the \feeig\ function (see the {\sl Function reference} for more details on the use of \feeig). An example of the use of \feeig\ is shown below.

We suppose that a model has already been defined (in {\tt model} data structure) and that the mass and stiffness matrices have already been assembled (in field {\tt K} of {\tt model}).\\

\begin{verbatim}
def=struct('def',[],'DOF',model.DOF,'data',[]);
[def.def,def.data] = fe_eig(model.K{1},model.K{2},[1 4 0 11]);
\end{verbatim}
In Scilab, you need to use temporary variables for the \feeig\ call :
\begin{verbatim}
def=struct('def',[],'DOF',model.DOF,'data',[]);
[tmpdef,tmpdata] = fe_eig(model.K(1).entries,model.K(2).entries,[1 4 0 11]);
def.def = tmpdef; def.data = tmpdata;
\end{verbatim}

Normal modes are in the matrix {\tt def.def} and associated frequencies in the vector {\tt def.data}. \\
The option vector defines : the method used (in this example : 1), the number of modes to be found (4), the mass shift value needed for rigid body modes (0) and the level of printout (11). For more details, see the {\sl Function reference} section.\\
The stress due to the deformation can be computed also :\\

\begin{verbatim}
StrainEnergy = fe_stress('ener',model,def);
feplot(model.Node,model.Elt,def.def,model.DOF,1,StrainEnergy);
\end{verbatim}
and (MATLAB version) :
\begin{verbatim}
fecom(';color face flat;color edge w;view3');
\end{verbatim}

\begin{center}
\begin{figure}[H]
\centering
\ingraph{70}{mode} % [width=7.cm]
 %\caption{Simulation properties tab.}
 % \label{fig:feplot_fe_simul}
\end{figure}

Visualization of a normal mode with OpenFEM for Scilab ({\tt demo\_mode})
\end{center}
This example is described in the {\tt demo\_mode} script.

\newpage
%- - - - - - - - - - - - - - - - - - - - - - - - - %
%      Manipulating large finite element models      %
%- - - - - - - - - - - - - - - - - - - - - - - - - %
\cssection{Manipulating large finite element models}{large}
This section gives information on manipulating large finite element models.
\begin{itemize}
\item Assembly : \\
The assembly method can be changed by customizing the {\tt Opt} input (see {\sl Reference functions} section). For large models, it is recommended to use the method 2 ({\tt disk} assembly). The {\tt disk} assembly method uses temporary files and so minimizes memory usage. To change the assembly method, put {\tt opt(3)} to 2.

You also should allow DOFs with no stiffness to be eliminated ({\tt opt(2)=0}).

In OpenFEM for MATLAB, an automatic renumbering is done above 1000 DOFs.

Note that {\tt fe\_mknl} is typically much faster than \femk\ especially for repeated assembly with the same topology which are usual in non-linear problems.

\item Static response :\\
The matrix factored object ({\tt ofact}) should be used (see the {\sl Reference functions} section for more details). Furthermore you should install the {\tt umfpack} solver on your machine ({\tt umfpack} is a set of routines for solving unsymmetric sparse linear systems).

{\tt umfpack} is available at \href{http://www.cise.ufl.edu/research/sparse/umfpack}{www.cise.ufl.edu/research/sparse/umfpack}. An interface to MATLAB is directly integrated in {\tt umfpack}.

 For Scilab, {\tt umfpack} can be used with the help of the {\tt scispt} toolbox (you need to install this toolbox in your machine). This toolbox is available at  \href{http://www.scilab.org/contributions.html}{http://www.scilab.org/contributions.html}.
\end{itemize}


\newpage
%-----------------------------------------------%
%     Visualization of deformed structures      %
%-----------------------------------------------%
\csection{Visualization of deformed structures}{visu}

OpenFEM provides various post-processing tools : an OpenFEM specific visualization and an interface to {\tt Medit}. 

OpenFEM visualization tools are described in section \ref{optools} and the {\tt Medit} interface in section \ref{visumedit}.  


\subsection{OpenFEM tools \label{optools}}

Visualization specific to OpenFEM is implemented in the \feplot\ function.

\feplot\  supports a number of display types for FE results. The {\tt feplot} provided with OpenFEM (in the {\tt sdt3} directory) is \textbf{provided to let you do some post processing with no need to buy a commercial package but clearly is not developed with the same care as the rest of OpenFEM.}

Furthermore, \feplot\ is one of the main differences between Scilab and MATLAB versions of OpenFEM. Basic calls to \feplot\ are common to both but the use of the graphical window created is very different.

In this section, common calls to \feplot\ are described and then specific use of MATLAB and Scilab versions are detailed.

%- - - - - - - - - - - %
%     Common calls     %
%- - - - - - - - - - - %
\subsubsection{Common calls \label{comcalls}}\index{feplot}
As stated above, OpenFEM visualization is based on the use of \feplot\ (and \fecom\ ) function.

In the following commands, \verb+node+ represents the node matrix, \verb+elt+ the model description matrix, \verb+md+ the deformations matrix, \verb+dof+ the DOFs definition vector.\\
\verb+model+ is a data structure containing at least \verb+.Node+ and \verb+.Elt+ fields.\\
\verb+def+ is a data structure containing at least \verb+.def+ and \verb+.DOF+ fields. {\tt def.def} is a deformation matrix, as {\tt md}.\\
\verb+stres+ is a vector or a matrix defining stresses in the structure under study.

\verb+opt+ is an option vector : \\
\verb+opt(1,1)+ defines the display type : 1 for patch, 2 for lines.\\
\verb+opt(1,2)+ defines the Undef type : 00 for none, 01 for UndefDot, 02 for UndefLine (in MATLAB only).\\
\verb+opt(1,3)+ gives the number of deformations per cycle.\\
\verb+opt(1,4)+ defines the number of the node used for modeshape scaling (in MATLAB only).\\
\verb+opt(1,5)+ gives the maximum displacement. \\
To avoid specifying any option, replace \verb+opt+ by \verb+[]+.\\

Visualization commands are the following :
\begin{itemize}
\item {\tt feplot(node,elt)} : displays the mesh 
\item {\tt feplot(node,elt,md,dof,opt)} : displays and animates deformations defined by \verb+md+
\item {\tt feplot(node,elt,md,dof,opt,stres)} : displays and animates deformations defined by \verb+md+ and colors the mesh with \verb+stres+ vector.
\item {\tt feplot('initmodel',model)} or {\tt feplot('initmodel',node,elt)} : model initialization. The mesh is not displayed. This call is used to prepare the display of deformations by {\tt feplot('initdef',def)}.
\item {\tt feplot('initdef',def)} : displays and animates deformations defined by \verb+def.def+. \verb+model+ must be previously initialized by a call to another display command or by using {\tt feplot('initmodel',model)}.
\item {\tt fecom('colordatastres',stres)} : displays coloring due to \verb+stres+ vector. The associated mesh or model must already be known. This call generally follows a deformations display call.
\end{itemize}

For a full list of accepted commands, see the {\sl Reference functions} section or the online help of your OpenFEM version ({\tt help feplot} in MATLAB or Scilab).

\newpage
%- - - - - - - - - - - - - - - - - - - - - - -%
%     OpenFEM for MATLAB specific tools     %
%- - - - - - - - - - - - - - - - - - - - - - -%
\subsubsection{MATLAB specific tools \label{visumat}}

For FE analyses (connectivity specified using a model description matrix {\tt elt}) one will generally use surface plots (type {\tt 1} color-coded surface plots using {\tt patch} objects) or wire-frame plots (type {\tt 2} using {\tt line} objects).  Once the plot is created, it can be manipulated using \fecom.  Continuous animation of experimental deformations is possible although speed is strongly dependent on computer configuration and {\tt figure renderer} selection (use {\tt Feplot:Renderer} menu to switch).



You can initialize plots with

\begin{verbatim}
 feplot(node,elt,mode,mdof,1)
 fecom('view3');
\end{verbatim}

\begin{SDT}
Most demonstrations linked to finite element modeling ({\tt gartfe}, {\tt beambar}, {\tt d\_ubeam}) give examples of how to use \feplot\ and \fecom.
\end{SDT}

To get started, run the {\tt d\_ubeam} demo. Then 

\begin{Eitem}

\item At this level note how you can zoom by selecting a region of interest with your mouse (double click or press the {\tt i} key to zoom back). You can make the axis active by clicking on it and then use the any of the {\tt u}, {\tt U}, {\tt v}, {\tt V}, {\tt w}, {\tt W}, {\tt 2} keys to rotate the plot (look at the \iimouse\ help for more possibilities).

\item Initialize a set of deformations and show deformation 7 (first flexible mode)
\begin{verbatim}
 feplot(node,elt,mode,mdof,1)
 feplot('initdef',md1,mdof); fecom('ch7');
 feplot('initcdef',StrainEnergy);
\end{verbatim}


\item Scan through the various deformations using the {\tt +/-} buttons/keys. Animate the deformations by clicking on the \button{bAn} button. Notice how you can still change the current deformation, rotate, etc. while running the animation.

\item Use {\tt fecom('triax')} to display an orientation triax. 

\item Use the {\tt fecom('sub 1 2')} command  to get a plot with two views of the same mode. 

% bitmap 894 x 481
\begin{figure}[H]
\centering
\ingraph{50}{tt_strain} % [width=2.9331in]
 \caption{Strain energy.}
  \label{fig:tt_strain}
\end{figure}

\item Note that when you print the figure, you may want to use the \ts{-noui} switch so that the GUI is not printed. Example {\tt print -noui -depsc2 FileName.eps}
\end{Eitem}

\newpage
%- - - - - - - - - - - - - - - - - - - - - - -%
%     OpenFEM for Scilab specific tools     %
%- - - - - - - - - - - - - - - - - - - - - - -%
\subsubsection{Scilab specific tools \label{visusci}}
Most of the commands detailed in {\sl Common calls}  open a Scilab graphical window. This window contains specific menus. These menus are detailed below :
\begin{itemize}
\item {\bf Display} : display functionalities, patch, color \ldots \\Contains the following submenus :
\begin{itemize}
\item {\bf DefType} : defines elements display type, with use of wire-frames plots (choose {\bf Line}) or with use of surface plots (choose {\bf Patch})
\item {\bf Colors} : defines structure coloring. The user can color edges (choose {\bf Lines}), faces (choose {\bf Uniform Patch} if the user decided to represent the structure with patches). The user can also choose the type of color gradient, if he displays a structure with coloring due to constraints.
\end{itemize}
\item {\bf Parameters} : animation parameters. Used only for deformations visualization. Contains the following subdirectories :
\begin{itemize}
\item {\bf mode +} : for modal deformations, displays next mode.
\item {\bf mode -} : for modal deformations, displays previous mode.
\item {\bf mode number \ldots} : for modal deformations, allows users to choose the number of the mode to display.
\item {\bf step by step} : allows users to watch animation picture by picture. Press mouse right button to see the next picture, press mouse left button to see the previous picture, press mouse middle button to quit picture by picture animation and to return to continuous animation.
\item {\bf scale} : allows users to modify the displacement scale.
\end{itemize}
\item {\bf Draw} : for non-animated displays. Recovers the structure when it has been erased.
\item {\bf Rotate} : opens a window which requests to modify figure view angles. Click on the {\bf ok} button to visualize the new viewpoint and click on the {\bf cancel} button to close the rotation window.
\item {\bf Start/Stop} : for deformation animations. Allows users to stop or restart animation.
\end{itemize}
\newpage
%$ $\\$ $\\$ $\\$ $\\$ $\\$ $\\$ $\\$ $\\$ $\\$ $\\
\begin{center}
\begin{figure}[H]
\centering
\ingraph{80}{visusci} % [width=12.cm]{
 %\caption{Simulation properties tab.}
 % \label{fig:feplot_fe_simul}
\end{figure}

OpenFEM for Scilab graphical window
\end{center}
\emph{Remark} : To return to Scilab or to continue execution, it is necessary to close the graphical window.

\newpage
%- - - - - - - - - - - - - - - - - %
%     Visualization with Medit     %
%- - - - - - - - - - - - - - - - - %
\subsection{Visualization with {\tt Medit} \label{visumedit}}\index{medit}
Visualization with {\tt Medit} is common to MATLAB and Scilab versions of OpenFEM.\\
{\tt Medit} is a powerful interactive mesh visualization software, developed by the Gamma project at INRIA-Rocquencourt.\\Binaries are freely available at \href{http://www-rocq.inria.fr/gamma/medit}{{\tt http://www-rocq.inria.fr/gamma/medit}}.\\
An interface to {\tt Medit} is provided. Users need to install {\tt Medit} themselves if they want to use this interface. They also have to change the name of the {\tt Medit} executable as {\tt medit} if it is different.

The interface to {\tt Medit} is called {\tt medit}. It allows the same plots and continuous animations as \feplot. Details on the use of {\tt medit} are provided in the section {\sl Function reference} of this document. 

To get started, run the  {\tt test\_medit} demo : in MATLAB, type `{\tt test\_medit}' or `{\tt test\_medit clean}' to clean all the created files. In Scilab, go to the {\tt demos} directory, load the test (`{\tt getf test\_medit.sci}') and run the test (`{\tt test\_medit()}' or `{\tt test\_medit clean}').

Then 

\begin{Eitem}
\item note how you can easily move the structure by pressing the left button of the mouse and then moving the mouse. Change the background color by typing '{\tt b}'. Now run the animation : press the right button of the mouse, select the '{\bf Animation}' menu and the '{\bf Play sequence}' submenu. Close the {\tt Medit} window.
\item a second window opens. Change the background color by typing '{\tt b}'. Display energy constraints : press the right button of the mouse and select the '{\bf Data}' menu and the '{\bf Toggle metric}' submenu. You can run the animation as in  previous step. Close the {\tt Medit} window.
\item a third window opens. Change the render mode : press the right button of the mouse, select the '{\bf Render mode}' menu and the '{\bf Wireframe}' submenu. Now choose the submenu '{\bf Shading+lines}' from the menu '{\bf Render mode}' . Display the nodes numbers : press the right button of the mouse, select the '{\bf Items}' menu and the '{\bf Toggle Point num}' submenu. Close the {\tt Medit} window.
\item a final window opens. Change the background color by typing '{\tt b}'. Display energy constraints by typing '{\tt m}'. Define now a cutting section : press the right button of the mouse, choose '{\bf [F1] Toggle clip}' in menu '{\bf Clipping}'. Press the key function '{\tt F2}' : the plane is now selected, you can move it by pressing the left or the middle button of the mouse. Press '{\tt F1}' to quit the cutting section environment.
\end{Eitem}

\begin{center}
%\includegraphics[width=7cm]{plots/visumedit.ps}\\
%\epsfig{file=plots/visumedit.ps,width=7cm}\\
\begin{figure}[H]
\centering
\ingraph{70}{visumedit} % [width=7.cm]
 \caption{Simulation properties tab.}
  \label{fig:feplot_fe_simul}
\end{figure}

Cutting section with {\tt Medit}
\end{center}
You can find information about the use of {\tt Medit} in the {\tt Medit} documentation (download from the same address as the executable). 

