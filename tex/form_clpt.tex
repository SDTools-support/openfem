%-----------------------------------------------------------------------
\cssection{Classical lamination theory}{feshell}
Both isotropic and orthotropic materials are considered. In these cases, the general form of the 3D elastic material law is
 
\begin{eqsvg}{isotropiclaw}
 \left\{ \begin{array}{c}
\sigma_{11}\\ \sigma_{22}\\ \sigma_{33}\\ \tau_{23}\\ \tau_{13}\\ \tau_{12}\\ \end{array} \right\} = 
 \left[ \begin{array}{cccccc}
C_{11} & C_{12} & C_{13}  &   0   &   0   &  0 \\
       & C_{22} & C_{23}  &   0   &   0   &  0 \\
                 &        & C_{33}      &   0   &   0   &  0 \\
                   &        &                           & C_{44}&   0   &  0 \\
             &  (s)             &             &       & C_{55}&  0 \\
                 &                              &                         &       &       & C_{66}\\ 
  \end{array} \right]
  \left\{ \begin{array}{c} \epsilon_{11}\\ \epsilon_{22}\\ \epsilon_{33}\\
\gamma_{23}\\ \gamma_{13}\\ \gamma_{12}\\ \end{array} \right\}
\end{eqsvg}

Plate formulation consists in assuming one dimension, the thickness along $x_3$, negligible compared with the surface dimensions. Thus, vertical stress $\sigma_{33}=0$ on the bottom and upper faces, and assumed to be neglected throughout the thickness,

\begin{eqsvg}{simple_eps33}
\sigma_{33}=0 \Rightarrow \epsilon_{33}=-\frac{1}{C_{33}}\left(C_{13}\epsilon_{11}+C_{23}\epsilon_{22}\right),
\end{eqsvg}
and for isotropic material,
\begin{eqsvg}{simple_eps33_isotropic}
\sigma_{33}=0 \Rightarrow \epsilon_{33}=-\frac{\nu}{1-\nu}\left(\epsilon_{11}+\epsilon_{22}\right).
\end{eqsvg}


By eliminating $\sigma_{33}$, the plate constitutive law is written, with engineering notations,

\begin{eqsvg}{PlateConstitutivelaw}
\left\{ \begin{array}{c}
\sigma_{11}\\  \sigma_{22}\\ \sigma_{12}\\ \sigma_{23}\\ \sigma_{13}\\ \end{array} \right\} =  \left[ 
\begin{array}{ccccc}
Q_{11}           & Q_{12}  & 0                  &0              & 0 \\
Q_{12}           & Q_{22}  & 0                  &0              & 0 \\
        0          & 0                   & Q_{66}       &0              & 0 \\
 0                                      & 0                      &0             & Q_{44} &0 \\
 0                                      & 0                      &0             & 0     &Q_{55} \\
  \end{array} 
 \right]
\left\{ \begin{array}{c}
\epsilon_{11}\\ \epsilon_{22}\\ \gamma_{12}\\ \gamma_{23}\\ \gamma_{13}\\ \end{array} \right\}.
\end{eqsvg}


The reduced stiffness coefficients $Q_{ij}$ (i,j = 1,2,4,5,6) are related to the 3D stiffness coefficients $C_{ij}$ by

\begin{eqsvg}{3D_planestrain}
Q_{ij}=\left\{ \begin{array}{ll}
C_{ij}-\displaystyle \frac{C_{i3}C_{j3}}{C_{33}}& \mbox{ if i,j=1,2,}\\
C_{ij}& \mbox{ if i,j=4,5,6.}\\
\end{array}\right.
\end{eqsvg}



The reduced elastic law for an isotropic plate becomes,
 
\begin{eqsvg}{isotropiclaw2}
 \small{ \left\{ \begin{array}{c}
\sigma_{11}\\
\sigma_{22}\\
\tau_{12}\\
 \end{array} \right\} = \frac{E}{(1-\nu^2)}
 \left[ \begin{array}{ccc}
1   & \nu & 0 \\
\nu &  1  & 0 \\
 0      & 0             &\frac{1-\nu}{2}\\
 \end{array} \right]
  \left\{ \begin{array}{c}
\epsilon_{11}\\
\epsilon_{22}\\
\gamma_{12}\\
 \end{array} \right\}},
\end{eqsvg}

and
 
\begin{eqsvg}{isotropiclaw3}
 \small{ \left\{ \begin{array}{c}
\tau_{23}\\
\tau_{13}\\
 \end{array} \right\} = \frac{E}{2(1+\nu)}
 \left[ \begin{array}{cc}
 1  &0 \\
0 &1 \\
  \end{array} \right]
  \left\{ \begin{array}{c}
 \gamma_{23}\\
\gamma_{13}\\
  \end{array} \right\}}.
\end{eqsvg}

Under Reissner-Mindlin's kinematic assumption the linearized strain tensor is

\begin{eqsvg}{strain}
\epsilon=
 \left[ \begin{array}{ccc}
u_{1,1}+x_3\beta_{1,1}&\frac{1}{2}(u_{1,2}+u_{2,1}+x_3(\beta_{1,2}+\beta_{2,1}))& \frac{1}{2}(\beta_1+w_{,1})\\
& u_{2,2}+x_3\beta_{2,2}&\frac{1}{2}(\beta_2+w_{,2})\\
(s)&&0 \\
  \end{array} \right].
\end{eqsvg}

So, the strain vector is written,

\begin{eqsvg}{strain_vector}
\left\{\epsilon\right\}=\left\{\begin{array}{c}
\epsilon^m_{11}+x_3\kappa_{11}\\
\epsilon^m_{22}+x_3\kappa_{22}\\
\gamma^m_{12}+x_3\kappa_{12}\\
\gamma_{23}\\
\gamma_{13}\\
 \end{array}\right\},
\end{eqsvg}
%
with $\epsilon^m$ the membrane, $\kappa$ the curvature or bending, and $\gamma$ the shear strains,
%
\begin{eqsvg}{strains}
\epsilon^m=\left\{ \begin{array}{c}
u_{1,1}\\
u_{2,2}\\
u_{1,2}+u_{2,1}\\
 \end{array} \right\},\ 
\kappa=\left\{ \begin{array}{c}
\beta_{1,1}\\
\beta_{2,2}\\
\beta_{1,2}+\beta_{2,1}\\
\end{array} \right\},\ 
\gamma=\left\{\begin{array}{c}
\beta_{2}+w_{,2}\\
\beta_{1}+w_{,1}\\
\end{array} \right\},
\end{eqsvg}\\

Note that the engineering notation with  $\gamma_{12}=u_{1,2}+u_{2,1}$ is used here rather than the tensor notation with $\epsilon_{12}=(u_{1,2}+u_{2,1})/2$ . Similarly $\kappa_{12}=\beta_{1,2}+\beta_{2,1}$, where a factor $1/2$ would be needed for the tensor.



The plate formulation links the stress resultants, membrane forces $N_{\alpha\beta}$, bending moments $M_{\alpha\beta}$ and shear forces $Q_{\alpha3}$, to the strains, membrane $\epsilon^m$, bending $\kappa$ and shearing $\gamma$,
\begin{eqsvg}{feform_feshell_1}
%\eql{ReissnerMindlinPlatelaw}
 \small{ \left\{ \begin{array}{c}
N\\
M\\
Q\\
 \end{array} \right\} =  \left[ \begin{array}{ccc}
A&B&0 \\
B&D&0 \\
0&0&F \\
\end{array} \right]
\left\{ \begin{array}{c}
\epsilon^m\\
\kappa\\
\gamma\\
\end{array} \right\}}.
\end{eqsvg}

The stress resultants are obtained by integrating the stresses through the thickness of the plate,
\begin{eqsvg}{Forces}
N_{\alpha \beta}=\displaystyle\int^{ht}_{hb}\sigma_{\alpha \beta}\:dx_3,\ \ 
M_{\alpha \beta}=\int^{ht}_{hb}x_3\:\sigma_{\alpha \beta}\:dx_3,\ \ 
Q_{\alpha 3}=\int^{ht}_{hb}\sigma_{\alpha 3}\:dx_3,
\end{eqsvg}\\ 
 
\noindent with $\alpha, \beta = 1, 2$. 


Therefore, the matrix extensional stiffness matrix $\left[A\right]$, extension/bending coupling matrix $\left[B\right]$, and the bending stiffness matrix $\left[D\right]$ are calculated by integration over the thickness interval $\ma{hb\ \ ht}$

\begin{eqsvg}{ABDF_1layer}
\begin{array}{cc}
A_{ij}=\displaystyle\int^{ht}_{hb}Q_{ij}\:dx_3,&
B_{ij}=\displaystyle\int^{ht}_{hb}x_3\:Q_{ij}\:dx_3,\\
&\\
D_{ij}=\displaystyle\int^{ht}_{hb}x^2_3\:Q_{ij}\:dx_3,&
F_{ij}=\displaystyle\int^{ht}_{hb}Q_{ij}\:dx_3. 
\end{array}
 \end{eqsvg}

An improvement of Mindlin's plate theory with transverse shear consists in modifying the shear coefficients $F_{ij}$ by
\begin{eqsvg}{correction_factor} 
H_{ij}=k_{ij}F_{ij},\end{eqsvg} 
where $k_{ij}$ are correction factors. Reddy's $3^{rd}$ order theory brings to $k_{ij}=\frac{2}{3}$. Very commonly, enriched $3^{rd}$ order theory are used, and $k_{ij}$ are equal to $\frac{5}{6}$ and give good results. For more details on the assessment of the correction factor, see~\cite{ber11}.\\

For an isotropic symmetric plate ($hb=-ht=h/2$), the in-plane normal forces $N_{11}$, $N_{22}$ and shear force $N_{12}$ become  

\begin{eqsvg}{N}
\left\{ \begin{array}{c}
N_{11}\\
N_{22}\\
N_{12}\\
\end{array} \right\} = \frac{Eh}{1-\nu^2}\left[\begin{array}{ccc}
1&\nu&0\\
&1&0\\
(s)&&\frac{1-\nu}{2}\\
\end{array}  \right]\left\{ \begin{array}{c}
u_{1,1}\\
u_{2,2}\\
u_{1,2}+u_{2,1}\\
\end{array} \right\},
\end{eqsvg}
%
the 2 bending moments $M_{11}$, $M_{22}$ and twisting moment $M_{12}$
%
\begin{eqsvg}{M}
\left\{ \begin{array}{c}
M_{11}\\
M_{22}\\
M_{12}\\
\end{array} \right\} = \frac{Eh^3}{12(1-\nu^2)}\left[\begin{array}{ccc}
1&\nu&0\\
&1&0\\
(s)&&\frac{1-\nu}{2}\\
\end{array}  \right]\left\{ \begin{array}{c}
\beta_{1,1}\\
\beta_{2,2}\\
\beta_{1,2}+\beta_{2,1}\\
\end{array} \right\},
\end{eqsvg}
%
and the out-of-plane shearing forces $Q_{23}$ and $Q_{13}$, 
%
\begin{eqsvg}{Q}
\left\{ \begin{array}{c}
Q_{23}\\
Q_{13}\\
\end{array} \right\} = \frac{Eh}{2(1+\nu)}\left[\begin{array}{cc}
1&0\\
0&1\\
\end{array}  \right]\left\{ \begin{array}{c}
\beta_{2}+w_{,2}\\
\beta_{1}+w_{,1}\\
\end{array} \right\}.
\end{eqsvg}

One can notice that because the symmetry of plate, that means the reference plane is the mid-plane of the plate ($x_3(0)=0$) the extension/bending coupling matrix $\left[B\right]$ is equal to zero. 

Using expression~\eqr{ABDF_1layer} for a constant $Q_{ij}$, one sees that for a non-zero offset, one has
%
\begin{eqsvg}{B_offset}
A_{ij}=h\ma{Q_{ij}}\ \ \ \ B_{ij}=x_3(0)h \ma{Q_{ij}} \ \ \ \ C_{ij}= (x_3(0)^2h+h^3/12) \ma{Q_{ij}} \ \ \ \ F_{ij}=h\ma{Q_{ij}}
\end{eqsvg}
%
where is clearly appears that the constitutive matrix is a polynomial function of $h$, $h^3$, $x_3(0)^2h$ and $x_3(0)h$. If the ply thickness is kept constant, the constitutive law is a polynomial function of $1,x_3(0),x_3(0)^2$.
